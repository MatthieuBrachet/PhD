\documentclass[10pt,a4paper]{amsart}
\usepackage[utf8]{inputenc}
\usepackage[english]{babel}
\usepackage{amsmath}
\usepackage{amsfonts}
\usepackage{amssymb}
\usepackage{graphicx}
\usepackage[left=2cm,right=2cm,top=2cm,bottom=2cm]{geometry}

\def\gint{\displaystyle\int}

\newtheorem{proposition}{Proposition}
\newtheorem{remark}{Remark}

\author{Brachet Matthieu}
\title{Some benchmarks for Linearized Shallow Water equation with Coriolis Force}
\begin{document}
\maketitle

The purpose of this note is to present two tests for the Linear Shallow Water equation with Coriolis Force (LSWEC) :

\begin{equation}
\label{LSWEC}
\left\lbrace
\begin{array}{r @{=} l}
\partial_t \mathbf{u} & -f \mathbf{k} \wedge \mathbf{u} - g \nabla \eta + \mathbf{F} \\
\partial_t \eta & -H \nabla \cdot \mathbf{u} 
\end{array}
\right.
\end{equation}

where $\mathbf{F}$ is a forcing map. In more, we will present the analysis of convergence with a numerical scheme for this tests.
The equation is solved on a sphere $\mathbb{S}_R^2$ with radius is $R$. If $\mathbf{x}$ is a point on $\mathbb{S}_R^2$, it is localized with two coordinates : $\lambda$ is the latitude and $\theta$ the longitude. A vector $\mathbf{v} : \mathbb{S}_R^2 \rightarrow \mathbb{T}\mathbb{S}_R^2$ is :

$$\mathbf{u} = u \mathbf{e}_{\theta} + v \mathbf{e}_{\lambda}.$$ 

The Coriolis parameter :

\begin{equation}
f=2 \omega sin ( \theta )
\label{coriolis_parameter}
\end{equation}

In the numericals results, we are looking the following error :

$$e_{i} = max_n \dfrac{\| \eta^n - \eta(t^n) \|_{i}}{\| \eta(0) \|_{i}}$$

$i \in \left\lbrace 1, 2, \infty \right\rbrace$.

\section{Some conservations properties of \eqref{LSWEC}}

\begin{remark}
\label{remark_stokes}
A preliminary remark is :

If $\Omega_1$ and $\Omega_2$ are two hemisphere on $\mathbb{S}^2_R$ then :
\begin{itemize}
\item $\Omega_1 \cup \Omega_2 = \mathbb{S}^2_R $,
\item $\overbrace{\Omega_1 \cap \Omega_2}^{\circ} = \varnothing$
\end{itemize}

by the Stokes theorem :

\begin{equation}
\gint_{\Omega_i}  \nabla \cdot \mathbf{u} = \gint_{\partial \Omega_i} \mathbf{u} \cdot \mathbf{n}_i
\end{equation}

with $i \in \left\lbrace 1, 2 \right\rbrace$ and $\mathbf{n}_i$ is the exterior normal of $\Omega_i$ on $\mathbb{S}_R^2$.
Then, because $\mathbf{n_1} = -\mathbf{n_2}$ and $\partial \Omega_1 = \partial \Omega_2$ :

$$\gint_{\mathbb{S}_R^2}  \nabla \cdot \mathbf{u} = \gint_{\Omega_1}  \nabla \cdot \mathbf{u} + \gint_{\Omega_2}  \nabla \cdot \mathbf{u} = \gint_{\partial \Omega_1} \mathbf{u} \cdot \mathbf{n}_1 + \gint_{\partial \Omega_2} \mathbf{u} \cdot \mathbf{n}_2 = 0$$
\end{remark}

\begin{proposition}
(mass conservation)
The mass $\gint_{\mathbb{S}_R^2} \eta$ is constant in time.
\end{proposition}

\begin{proof}
We integrate on $\mathbb{S}_R^2$ the second equation of \eqref{LSWEC} :

$$\dfrac{\partial}{\partial t} \gint_{\mathbb{S}_R^2} \eta = \gint_{\mathbb{S}_R^2} \dfrac{\partial \eta}{\partial t} = \gint_{\mathbb{S}_R^2} \nabla \cdot \mathbf{u} = 0$$
\end{proof}

\begin{proposition}
(energy conservation)
If $\mathbf{F} = \mathbf{0}$, the energy $g  \eta^2 + H \| u \|_{L^2(\mathbb{S}_R^2)}^2$ is constant in time.
\end{proposition}

\begin{proof}
\begin{itemize}
\item First, we notice that $\mathbf{u}$ is orthogonal to $\mathbf{k} \wedge \mathbf{u}$ then :

$$\gint_{\mathbb{S}_R^2} \dfrac{\partial \mathbf{u}}{\partial t} \cdot \mathbf{u} = \dfrac{1}{2} \dfrac{\partial}{\partial t} \gint_{\mathbb{S}_R^2} \mathbf{u}^2 = -g \gint_{\mathbb{S}_R^2} \nabla \eta \cdot \mathbf{u}$$

in other words :

\begin{equation}
\dfrac{1}{2} \dfrac{\partial}{\partial t} \| u \|_{L^2(\mathbb{S}_R^2)}^2 = -g \gint_{\mathbb{S}_R^2} \nabla \eta \cdot \mathbf{u}
\label{energy_eq1}
\end{equation}

\item with a similar idea, we have :

$$\dfrac{\partial \eta}{\partial t} \times \eta = -H \eta \nabla \cdot \mathbf{u} $$

and after integration on $\mathbb{S}_R^2$ :

\begin{equation}
\dfrac{1}{2} \dfrac{\partial}{\partial t} \| \eta \|^2_{L^2(\mathbb{S}_R^2)} = -H \gint_{\mathbb{S}_R^2} \eta \nabla \cdot \mathbf{u}
\label{energy_eq2}
\end{equation}

\item It is well known that :

\begin{equation}
\nabla \cdot \left( \mathbf{A} B \right) = \left( \mathbf{A} \cdot \nabla \right) B + \left( B \nabla \cdot \mathbf{A} \right)
\label{energy_eq3}
\end{equation}

Then :

$$g \dfrac{\partial}{\partial t}  \| \eta \|^2_{L^2(\mathbb{S}_R^2)} + H \dfrac{\partial}{\partial t} \| \mathbf{u} \|^2_{L^2(\mathbb{S}_R^2)} = -2gH \gint_{\mathbb{S}_R^2} \left( \mathbf{u} \cdot \nabla \eta + \eta \nabla \cdot \mathbf{u} \right)$$

using \eqref{energy_eq3} and \ref{remark_stokes} :

\begin{equation}
\dfrac{\partial}{\partial t} \left( g  \eta^2 + H \| u \|_{L^2(\mathbb{S}_R^2)}^2 \right) = 0
\end{equation}
\end{itemize}
\end{proof}

\section{exponential solution}

The forcing $\mathbf{F}$ is adjust such that the solution is :

\begin{equation}
\left\lbrace
\begin{array}{r @{=} l}
u & \frac{\sqrt{gH}}{10} \psi ( \theta )  e^{-\sigma t} \\
v & 0 \\
\eta & \psi( \theta ) e^{-\sigma t}
\end{array}
\right.
\end{equation}

where :

\begin{equation*}
\psi ( \theta ) = 
\left\lbrace
\begin{array}{l}
0 \text{ if } \theta > \theta_1\\
\frac{1}{K}e^{\frac{1}{(\theta - \theta_0)(\theta-\theta_1)}} \text{ if } \theta_0 \leq \theta \leq \theta_1 \\
0 \text{ if } \theta < \theta_0\\
 
\end{array}
\right.
\label{galewski_fun}
\end{equation*}

$K = e^{-\dfrac{4}{(\theta_0 - \theta_1)^2}}$ is a normalization constant.

In the numerical test, the choice is $\sigma = 10^{-4}$, $\theta_0 = -\dfrac{3 \pi}{16}$ and $\theta_1 = \dfrac{3 \pi}{16}$.

For the numerical test, the CFL condition is given by :

\begin{equation}
CFL = \dfrac{c \Delta t}{R \Delta \xi}
\end{equation}

with $c = max(c_{grav}, c_{cor})$, $c_{grav} = \sqrt{gH}$ and $c_{cor} = R \omega$, $\Delta \xi = \pi / 2N$.

Tables \ref{CV_order4_hp10}, \ref{CV_order4_hp100} and \ref{CV_order4_hp1000} are obtain with a scheme of order 4 in time (RK4) and order 4 in the divergence and the gradient. For the tables Tables \ref{CV_order8_hp10}, \ref{CV_order8_hp100} and \ref{CV_order8_hp1000} the scheme is the same except for the divergence where the compact scheme is order 8.

\begin{table}[H!]
\begin{tabular}{c|c|c|c|c|c|c}
$N$ & $e_{\infty}$ & order & $e_2$ & order & $e_1$ & order \\ 
\hline 
\hline
$20$ & $1.2920 (-4)$ & - & $5.6417 (-5)$ & - & $4.7733 (-5)$ & - \\ 
\hline 
$40$ & $1.9620 (-5)$ & $2.9171$ & $5.4901 (-6)$ & $3.4823$ & $3.6578 (-6)$ & $3.8394$ \\ 
\hline 
$60$ & $3.7239 (-6)$ & $4.1827$ & $1.1867 (-6)$ & $3.8554$ & $7.3660 (-7)$ & $4.0336$ \\
\hline 
$80$ & $1.0788 (-6)$ & $4.3689$ & $3.7154 (-7)$ & $4.0951$ & $2.1765 (-7)$ & $4.2992$ \\ 
\hline 
$100$ & $4.1779(-7)$ & $4.2988$ & $1.4789 (-7)$ & $4.1745$ & $8.5641 (-8)$ & $4.2268$  \\ 
\end{tabular} 
\caption{Convergence analysis with $CFL=0.5$ and $H=10$.}
\label{CV_order4_hp10}
\end{table}

\begin{table}[H!]
\begin{tabular}{c|c|c|c|c|c|c}
$N$ & $e_{\infty}$ & order & $e_2$ & order & $e_1$ & order \\ 
\hline 
\hline
$20$ & $0.0038$ & - & $0.0015$ & - & $0.0012$ & - \\ 
\hline 
$40$ & $3.8502 (-4)$ & $3.4220$ & $1.1233 (-4)$ & $3.8738$ & $7.0664 (-5)$ & $4.2331$ \\ 
\hline 
$60$ & $6.0715 (-5)$ & $4.6491$ & $1.7546 (-5)$ & $4.6731$ & $1.1497 (-5)$ & $4.5705$ \\
\hline 
$80$ & $1.4527 (-5)$ & $5.0434$ & $4.8548 (-6)$ & $4.5309$ & $3.2599 (-6)$ & $4.4446$ \\ 
\hline 
$100$ & $5.8355(-6)$ & $4.1331$ & $1.9027 (-6)$ & $4.2447$ & $1.2637 (-6)$ & $4.2944$  \\ 
\end{tabular} 
\caption{Convergence analysis with $CFL=0.5$ and $H=100$.}
\label{CV_order4_hp100}
\end{table}

\begin{table}[H!]
\begin{tabular}{c|c|c|c|c|c|c}
$N$ & $e_{\infty}$ & order & $e_2$ & order & $e_1$ & order \\ 
\hline 
\hline
$20$ & $0.0629$ & - & $0.0283$ & - & $0.0277$ & - \\ 
\hline 
$40$ & $0.0044$ & $3.9757$ & $0.0014$ & $4.4935$ & $0.0011$ & $4.8219$ \\ 
\hline 
$60$ & $6.6608 (-4)$ & $4.7519$ & $1.9743 (-4)$ & $4.9304$ & $1.4907 (-4)$ & $5.0306$ \\
\hline 
$80$ & $1.6033 (-4)$ & $5.0222$ & $5.3660 (-5)$ & $4.5217$ & $4.2043 (-5)$ & $4.4634$ \\ 
\hline 
$100$ & $6.9510(-5)$ & $3.7874$ & $2.1324 (-5)$ & $4.1819$ & $1.6013 (-5)$ & $4.3743$  \\ 
\end{tabular} 
\caption{Convergence analysis with $CFL=0.5$ and $H=1000$.}
\label{CV_order4_hp1000}
\end{table}

\begin{table}[H!]
\begin{tabular}{c|c|c|c|c|c|c}
$N$ & $e_{\infty}$ & order & $e_2$ & order & $e_1$ & order \\ 
\hline 
\hline
$20$ & $1.0303 (-4)$ & - & $5.1192 (-5)$ & - & $5.0087 (-5)$ & - \\ 
\hline 
$40$ & $1.1221 (-5)$ & $3.3140$ & $4.7304 (-6)$ & $3.5596$ & $3.3176 (-6)$ & $4.0573$ \\ 
\hline 
$60$ & $2.8566 (-6)$ & $3.4436$ & $1.0696 (-6)$ & $3.7421$ & $6.5310 (-7)$ & $4.0908$ \\
\hline 
$80$ & $1.0837 (-6)$ & $3.4180$ & $3.2702 (-7)$ & $4.1789$ & $1.8005 (-7)$ & $4.5438$ \\ 
\hline 
$100$ & $4.2901(-7)$ & $4.1993$ & $1.2667 (-7)$ & $4.2980$ & $6.7292 (-8)$ & $4.4600$  \\ 
\end{tabular} 
\caption{Convergence analysis with $CFL=0.5$ and $H=10$.}
\label{CV_order8_hp10}
\end{table}

\begin{table}[H!]
\begin{tabular}{c|c|c|c|c|c|c}
$N$ & $e_{\infty}$ & order & $e_2$ & order & $e_1$ & order \\ 
\hline 
\hline
$20$ & $0.0025$ & - & $0.0012$ & - & $0.0011$ & - \\ 
\hline 
$40$ & $1.7414 (-4)$ & $3.9820$ & $6.5746 (-5)$ & $4.3409$ & $4.6314 (-5)$ & $4.7345$ \\ 
\hline 
$60$ & $2.7156 (-5)$ & $4.6772$ & $9.1924 (-6)$ & $4.9520$ & $5.9158 (-6)$ & $5.1795$ \\
\hline 
$80$ & $7.9310 (-6)$ & $4.3404$ & $2.3353 (-6)$ & $4.8320$ & $1.3738 (-6)$ & $5.1487$ \\ 
\hline 
$100$ & $2.7584(-6)$ & $4.7860$ & $8.5491 (-7)$ & $4.5538$ & $4.9340 (-7)$ & $4.6405$  \\ 
\end{tabular} 
\caption{Convergence analysis with $CFL=0.5$ and $H=100$.}
\label{CV_order8_hp100}
\end{table}

\begin{table}[H!]
\begin{tabular}{c|c|c|c|c|c|c}
$N$ & $e_{\infty}$ & order & $e_2$ & order & $e_1$ & order \\ 
\hline 
\hline
$20$ & $0.0376$ & - & $0.0206$ & - & $0.0241$ & - \\ 
\hline 
$40$ & $0.0019$ & $4.4618$ & $7.3624 (-4)$ & $4.9794$ & $7.3567 (-4)$ & $5.2151$ \\ 
\hline 
$60$ & $1.5134 (-4)$ & $6.3682$ & $5.3116 (-5)$ & $6.6173$ & $5.5023 (-5)$ & $6.5266$ \\
\hline 
$80$ & $2.0280 (-5)$ & $7.0877$ & $7.1205 (-6)$ & $7.0863$ & $7.3563 (-6)$ & $7.0958$ \\ 
\hline 
$100$ & $4.9820(-6)$ & $6.3615$ & $1.9414 (-6)$ & $5.8892$ & $1.7583 (-6)$ & $6.4857$  \\ 
\end{tabular} 
\caption{Convergence analysis with $CFL=0.5$ and $H=1000$.}
\label{CV_order8_hp1000}
\end{table}


\section{Test case without forcing map}

This second test is without forcing map ($\mathbf{F} = \mathbf{0}$). In this context, we construct a stationnary solution such that :

$$\mathbf{u}(\theta, \lambda) = u(\theta) \mathbf{e}_{\lambda}$$

This velocity $ \mathbf{u}$ divergence is easilly zero with this choice.

The atmosphere perturbation $\eta$ is construct such that :

\begin{equation}
f \mathbf{k} \wedge \mathbf{u} + g \nabla \eta = 0
\end{equation} 

in the basis $(\mathbf{e}_{\lambda}, \mathbf{e}_{\theta})$ :

\begin{equation}
f u \mathbf{e}_{\theta} + g \left[ \dfrac{1}{R cos \theta} \dfrac{\partial \eta}{\partial \lambda} \mathbf{e}_{\lambda} + \dfrac{1}{R}\dfrac{\partial \eta}{\partial \theta} \mathbf{e}_{\theta} \right] = 0
\end{equation}

by identification :

\begin{itemize}
\item $\dfrac{1}{R cos \theta} \dfrac{\partial \eta}{\partial \lambda} = 0$, then $\eta$ is independant of $\lambda$,

\item $f u + \dfrac{g}{R} \dfrac{\partial \eta}{\partial \theta} = 0$, then (because \eqref{coriolis_parameter}) :

\begin{equation}
\eta (\theta ) = \eta_0 - \dfrac{2 \omega R}{g} \gint_0^{\theta} sin(\tau) u(\tau) d \tau
\end{equation} 
\end{itemize}

then a stationnary solution is given. We choose $u(\theta) = u_0 \psi( \theta )$ with $\psi$ given by \eqref{galewski_fun}. The integral is numerically calculate by the trapeze rules (of course, the Simsons rules will be better but we will do this later!).

Numerically, we obtain two kinds of results :
\begin{itemize}
\item we represent the relative error at time $t^n$, $e_i^n$ with $i \in \lbrace 1, 2, \infty \rbrace$, we hope this quantity is small,
\item we reprensent too two the relative mass and the relative energy :
\begin{equation}
\dfrac{I^n}{I(0)}
\end{equation}
where $I$ is egual to $I^n = \gint_{\mathbb{S}_R^2} \eta (t^n)$ the mass or the energy $I^n = g \gint_{\mathbb{S}_R^2} \eta (t^n)^2 + H \gint_{\mathbb{S}_R^2} \|u \|_2^2$. It must be close to one.
\end{itemize} 





\end{document}