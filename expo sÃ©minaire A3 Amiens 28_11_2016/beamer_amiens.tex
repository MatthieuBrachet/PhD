\documentclass[11pt]{beamer}
\usetheme{Madrid}
\usepackage[utf8]{inputenc}
\usepackage[french]{babel}
\usepackage[T1]{fontenc}
\usepackage{amsmath}
\usepackage{amsfonts}
\usepackage{amssymb}
\usepackage{tikz}
\usetikzlibrary[patterns]
\author{\underline{M. Brachet}, J.-P. Croisille}
\title[Schémas compacts sur la sphère]{Approximation numérique d'équations aux dérivées partielles sur la sphère par un schéma compact}
%\setbeamercovered{transparent} 
%\setbeamertemplate{navigation symbols}{} 
%\logo{} 
\institute[IECL]{Institut Elie Cartan de Lorraine (Metz)} 
\date{28 novembre 2016} 
%\subject{} 
\begin{document}

\begin{frame}
\titlepage
\includegraphics[scale=0.3]{iecl.jpg}
\end{frame}

%% ***********************************************************************
\section*{Introduction}

\begin{frame}{Introduction}
\begin{block}{Physique}
Prévisions de mouvements de l'atmosphère autour d'une planète.
\end{block}

\begin{columns}
\column{0.45\textwidth}
\begin{block}{}

\end{block}

\begin{columns}
\column{0.45\textwidth}

\end{frame}


\begin{frame}
\begin{block}{Equation Shallow Water}
\begin{equation}
\left\lbrace
\begin{matrix}
\partial_t h^{\star} + \nabla \cdot \left( h^{\star} \mathbf{u} \right) & = & 0 \\
\partial_t \mathbf{u} + \nabla \left[ \dfrac{1}{2} \mathbf{u}^2 + gh \right] + \left(f + \left( \nabla \wedge \mathbf{u} \right) \cdot \mathbf{k} \right) \mathbf{k} \wedge \mathbf{u} & = & \mathbf{0} \\
\end{matrix}
\right.
\end{equation}

avec $h^{\star} = h - h_s$ et $h_s$ la topographie.
\end{block}

\begin{columns}
\column{0.45\textwidth}
\begin{block}{Géométrie sphérique :}
Opérateurs intrinsèques à la sphère :
\begin{itemize}
\item gradient,
\item divergence,
\item rotationnel.
\end{itemize}
\end{block}

\column{0.45\textwidth}
\begin{center}
\includegraphics[scale=0.25]{earth.jpg}
\end{center}

\end{columns}
\end{frame}

%% ***********************************************************************
\begin{frame}
\tableofcontents
\end{frame}

%% ***********************************************************************
\section{Discrétisation des opérateurs sphériques sur la CS}
\begin{frame}{Discrétisation des opérateurs sphériques sur la CS}





\end{frame}

%% ***********************************************************************
\section{Equation d'advection sphérique}
\begin{frame}{Equation d'advection sphérique}
\begin{itemize}
\item Equation d'advection,
\item Rotation solide : solution par caractéristiques,
\item test 1 de Williamson + résultats,
\item Besoin d'ajouter un filtrage.
\item test de Nair et Jablonowski + résultats.
\end{itemize}
\end{frame}

%% ***********************************************************************
\section{Equation Shallow Water sphérique}
\begin{frame}{Equation Shallow Water sphérique}
\begin{itemize}
\item Présentation du modèle,
\item Relations de conservations,
\item Solution Stationnaire,
\item Test 5 de Williamson, présentation,
\item Ondes parasites,
\item Filtre adaptatif,
\item Résultats numériques.
\end{itemize}
\end{frame}


\section*{Conclusion et perspectives}
\begin{frame}{Conclusion et perspectives}

\end{frame}

\begin{frame}
\begin{center}
Merci de votre attention :)
\end{center}
\end{frame}

\end{document}