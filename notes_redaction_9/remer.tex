\noindent \hrulefill
\vspace{.6cm}


Quand j'ai commencé ma thèse, on m'a dit que c'était un travail très solitaire. Je ne peux pas dire ça. J'ai eu la chance de toujours être bien entouré.

\vspace{.6cm}

Je tiens tout d'abord à exprimer ma gratitude et mes remerciements à Jean-Pierre Croisille pour avoir encadré mon travail de thèse. Il serait étrange de dire que sans lui ce travail n'aurait pas eu lieu, je le remercie pour les nombreuses discussions, ses conseils et sa patience. Il m'a permis de mener ce travail dans d'excellentes conditions pendant ces quatre années. A ses cotés, j'ai énormément appris autant en termes de sciences et de rigueur que humainement.
\vspace{.6cm}

J'adresse mes remerciements à Eric Blayo et à Thomas Dubos pour avoir accepté la lourde tâche de rapporteur. Ma gratitude est aussi adressée à Michael Ghil pour la présidence du jury de ma thèse. J'exprime ma reconnaissance à Didier Clamond, Véronique Martin et Dong Ye pour avoir accepté de faire partie du jury de ma thèse.
\vspace{.6cm}

Je souhaite remercier tout particulièrement Jean-Paul Chehab pour nos discussions, ses conseils et ses remarques. Après mon stage de master 2, j'ai eu la chance de pouvoir continuer à travailler à ses côtés et c'est toujours un plaisir pour moi. Je l'en remercie. Ma gratitude est aussi adressée à l'ensemble de l'Institut Elie Cartan de Lorraine. En particulier je remercie Xavier Antoine, David Dos Santos Ferreira et Julien Lequeurre pour nos discussions. Ma gratitude va aussi à Olivier Botella, Antoine Henrot, Vladimir Latocha et Didier Schmitt.
Je remercie également Claude Coppin, Hélène Jouve et Paola Schneider pour leur accompagnement administratif. Bien sûr je remercie Laurence Quirot et sa bonne humeur permanente ainsi que Didier Gemmerlé (sans qui j'aurais eu de sérieux problèmes informatiques) et Elodie Cunat (qui a toujours l'information recherchée). Sachez que j'ai apprécié chaque moment de nos discussions.
\vspace{.6cm}

A la fin de ma première année j'ai eu l'opportunité de participer au CEMRACS. C'était une chance et je suis très heureux d'y avoir participé. A cette occasion, je remercie l'ensemble de l'équipe Hydromorpho : Nora Aissiouene, Tarik Amtout, Romain Hild, Christophe Prud'homme, Antoine Rousseau et Stéphanie Salmon. Vous avez rendu le projet particulièrement enrichissant. Le CEMRACS fut aussi l'occasion de rencontres : Andrea, Clémentine, Guillaume, Hélène, Charlotte, Ranine, ... (la liste est bien trop longue, merci à tous! sincèrement). Je vous croise encore, nous discutons souvent et j'apprécie toujours autant ces moments.
\vspace{.6cm}

J'ai commencé ces remerciements en parlant de mon entourage. Je ne peux pas oublier de remercier mes parents. Ils m'ont permis de faire des études (longtemps, très longtemps) et m'ont toujours accompagné, que ce soit dans les bons ou les mauvais moments. Ils sont toujours à mes cotés et me soutiennent. Nous ne nous disons pas ce genre de choses mais merci beaucoup et merci d'être là. Je n'oublierai pas non plus mes frères, Samuel et Valentin, qui trouveront toujours une bonne raison de m'empêcher de travailler.
\vspace{.6cm}

Ma thèse ne se serait pas déroulée comme elle s'est déroulée sans ces gens qui m'ont supporté quotidiennement (alors qu'ils avaient le choix!). Je parle bien sûr de mes collègues et amis. Un très grand merci à mes cobureaux : Clément et Benjamin à Metz (sans qui cette thèse aurait été très différente), ainsi que Dimitry, Florian et Yang à Nancy (pour le 513). En écrivant ce texte je pense aussi évidement à Coralie (que je retrouverai à 16h je suppose), Clémence (mais tu fais des maths!), Tom, Maxime, Allan, ... Un grand merci à l'ensemble des doctorants de l'IECL. 
\vspace{.6cm}

Il y a ceux que j'ai rencontré grâce aux mathématiques et il y a ceux qui préfèrent sûrement lorsque j'en parle un peu moins. Jennifer et Mathieu, je n'aurais jamais cru que vous croiser entre deux rayons puisse aboutir à cette amitié. Magaly, je t'attend toujours pour boire un monaco. David et Hélène, ma L2 est loin mais nous discutons toujours et c'est un plaisir. Émilie, Julien et Mira, j'ai fait mes études avec vous, merci de les avoir faites avec moi ;). Un grand merci à Guillaume, Jasmine, Typhaine, Vincent et Anne-Sophie qui finalement étaient une bonne raison d'aller dans les Vosges. Un merci un peu spécial à Camille. Je remercie également Alexis, certes je t'ai rencontré grâce aux maths mais je m'en fous je te mets ici.
\vspace{.6cm}

Je remercie également tous les étudiants avec lesquels j'ai travaillé, qui ont supporté mon incontestable désorganisation et ma numérotation aléatoire des chapitres et propositions.
\vspace{.6cm}

Enfin, j'adresse mes remerciements à l'équipe du LAMFA qui, à travers les enseignements, m'a donné un goût particulier et une certaine curiosité pour les mathématiques.
\vspace{.6cm}









\noindent \hrulefill

\vspace{2cm}
\begin{flushright}
\textit{"Ce n'est qu'en essayant continuellement que l'on finit par réussir... }

\textit{En d'autres termes... Plus ça rate et plus on a de chances que ça marche..."}

Devise Shadoks, J. Rouxel.
\end{flushright}



\textcolor{white}{
\begin{flushright}
\textit{"Tout seul on va plus vite, ensemble, on va plus loin." }
, Proverbe africain.
\end{flushright}
}