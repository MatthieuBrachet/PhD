% IR.tex

L'objectif de cette partie est de présenter les invariants de Riemann comme outil pour l'étude des équations aux dérivées partielles hyperboliques.

\subsection{Cas 1D}

On considère une équation aux dérivées partielles de la forme suivante :

\begin{equation}
\dfrac{\partial u}{\partial t} + A \cdot \dfrac{\partial}{\partial x} F(u) = 0
\label{eq:edp hyperbolique 1d}
\end{equation}

où $u$ est une fonction de $\mathbb{R}$ dans $\mathbb{R}^p$ et $F$ de $\mathbb{R}^p$ dans $\mathbb{R}^n$ ($n$ entier naturel). On note $u_0$ la solution à $t=0$.

L'équation \eqref{eq:edp hyperbolique 1d} étant hyperbolique, $A  \cdot J F$, est diagonalisable, où $J F$ est la jacobienne de $F$.

Il existe $P \in \mathcal{M}_n$ inversible et $D \in \mathcal{M}_ n$ diagonale tels que :

\begin{equation}
A \cdot JF = P^{-1} D P
\end{equation}

ainsi, en mutipliant \eqref{eq:edp hyperbolique 1d} à gauche par $P$ et en notant $w = P u$, on obtient :

\begin{equation}
\dfrac{\partial w}{\partial t} + D \dfrac{\partial w}{\partial x} = 0
\label{eq:transport invariant de riemann}
\end{equation}

\begin{remarque}
\begin{itemize}
\item $w$ est appelé invariant de Riemann associé à l'équation \eqref{eq:edp hyperbolique 1d},
\item $D$ est, a priori, fonction de $u$,
\item lorsque \eqref{eq:edp hyperbolique 1d} est une équation linéaire, on peut considèrer $F(u) = u$, $D$ est indépendant de $u$ donc \eqref{eq:transport invariant de riemann} est une equation de transport que l'on sait résoudre :
\begin{equation}
w_j(x) = w_0 ( x - D_{j,j} t) = P u_0 ( x - D_{j,j} t )
\end{equation} 
et on peut déduire la solution de \eqref{eq:edp hyperbolique 1d}.
\end{itemize}
\end{remarque}

\subsection{Cas de l'équation Shallow Water}

On considère l'équation :

\begin{equation}
\left\lbrace
\begin{array}{rcl}
\partial_t \mathbf{u} + \left( \mathbf{u} \cdot \nabla_T \right) \mathbf{u} + g \nabla_T h + f \mathbf{k} \wedge \mathbf{u} & = & \mathbf{0}\\
\partial_t h^{\star} + \nabla_T \cdot \left( h^{\star} \mathbf{u} \right) & = & 0 \\ 
\end{array}
\right.
\label{eq:SWEC.annexe}
\end{equation}

avec $h^{\star} = h - h_s$, $h_s$ représente les reliefs sur le domaine de calcul.

Pour déterminer les invariants de Riemann, on met l'équation \eqref{eq:SWEC.annexe} sous la forme :

\begin{equation}
\dfrac{\partial}{\partial t} X + A_{\xi}(X) \dfrac{\partial}{\partial \xi} F_{\xi}(X) + A_{\eta}(X) \dfrac{\partial}{\partial \eta} F_{\eta}(X) + G(X) = 0
\end{equation}

$(\xi, \eta)$ est un système de coordonnées 2D, $X=(u_{\xi}, u_{\eta}, h)^T$. On note que $\mathbf{u} = u_{\xi} \mathbf{g}^{\xi}+u_{\eta} \mathbf{g}^{\eta}$. 

Ainsi, on peut montrer que $F_{\xi}(X) = F_{\eta}(X) = X$ et les matrices $A_{\xi}$ et $A_{\eta}$ sont données par :

\begin{equation}
A_{\xi} = \begin{pmatrix}
\mathbf{u} \cdot \mathbf{g}^{\xi} & 0 & g \\
0 & \mathbf{u} \cdot \mathbf{g}^{\xi} & 0 \\
h \mathbf{g}^{\xi} \cdot \mathbf{g}^{\xi} & h \mathbf{g}^{\xi} \cdot \mathbf{g}^{\eta} & \mathbf{u} \cdot \mathbf{g}^{\xi} 
\end{pmatrix}
\hspace{1cm}
A_{\eta} = \begin{pmatrix}
\mathbf{u} \cdot \mathbf{g}^{\eta} & 0 & 0 \\
0 & \mathbf{u} \cdot \mathbf{g}^{\eta}  & g \\
h \mathbf{g}^{\eta} \cdot \mathbf{g}^{\xi}  & h \mathbf{g}^{\eta} \cdot \mathbf{g}^{\eta}  & \mathbf{u} \cdot \mathbf{g}^{\eta} 
\end{pmatrix}
\end{equation}

\begin{remarque}
\begin{itemize}
\item Ce résultat est cohérent avec celui d'une base orthonormale, puisque l'on aurait $\mathbf{g}^{\xi} \cdot \mathbf{g}^{\eta} = 0$ et $\mathbf{g}^{\eta} \cdot \mathbf{g}^{\eta} = \mathbf{g}^{\xi} \cdot \mathbf{g}^{\xi} = 0$,
\item Il n'est pas utile de calculer explicitement $G(X)$ pour le calcul des invariants de Riemann.
\end{itemize}

\end{remarque}

Les valeurs propres de $A_{\xi}$ constituent l'ensemble $\left\lbrace \mathbf{u} \cdot \mathbf{g}^{\xi}, \mathbf{u} \cdot \mathbf{g}^{\xi} + \sqrt{\mathbf{g}^{\xi} \cdot \mathbf{g}^{\xi} h g} , \mathbf{u} \cdot \mathbf{g}^{\xi} - \sqrt{\mathbf{g}^{\xi} \cdot \mathbf{g}^{\xi} h g}\right\rbrace$. Les vecteurs propres associés sont :

\begin{equation*}
V_{\xi,1} = \begin{pmatrix}
- \mathbf{g}^{\xi} \cdot \mathbf{g}^{\eta} \\
\mathbf{g}^{\xi} \cdot \mathbf{g}^{\xi} \\
0
\end{pmatrix}\text{, }
V_{\xi,2} = \begin{pmatrix}
g \\
0 \\
\sqrt{\mathbf{g}^{\xi} \cdot \mathbf{g}^{\xi} h g}
\end{pmatrix}\text{ et }
V_{\xi,3} = \begin{pmatrix}
g \\
0 \\
- \sqrt{\mathbf{g}^{\xi} \cdot \mathbf{g}^{\xi} h g}
\end{pmatrix}.
\end{equation*}

De même, pour $A_{\eta}$, on a les valeurs propres $\left\lbrace \mathbf{u} \cdot \mathbf{g}^{\eta}, \mathbf{u} \cdot \mathbf{g}^{\eta} + \sqrt{\mathbf{g}^{\eta} \cdot \mathbf{g}^{\eta} h g} , \mathbf{u} \cdot \mathbf{g}^{\eta} - \sqrt{\mathbf{g}^{\eta} \cdot \mathbf{g}^{\eta} h g}\right\rbrace$ et les vecteurs propres associés sont :

\begin{equation*}
V_{\eta,1} = \begin{pmatrix}
\mathbf{g}^{\eta} \cdot \mathbf{g}^{\eta} \\
- \mathbf{g}^{\eta} \cdot \mathbf{g}^{\xi} \\
0
\end{pmatrix}\text{, }
V_{\eta,2} = \begin{pmatrix}
0 \\
\sqrt{\mathbf{g}^{\eta} \cdot \mathbf{g}^{\eta} h g}\\
g
\end{pmatrix}\text{ et }
V_{\eta,3} = \begin{pmatrix}
0 \\
-\sqrt{\mathbf{g}^{\eta} \cdot \mathbf{g}^{\eta} h g}\\
g
\end{pmatrix}.
\end{equation*}

