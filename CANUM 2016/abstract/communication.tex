%%%%%%%%%%%%%%%%%%%%%%%%%%%%%%%%%%%%%%%%%%%%%%%%%%%%%%%%%%%%%%%%%%%%%%%%%%%%
%               PRIERE DE NE RIEN MODIFIER CI-DESSOUS                      %
%       JUSQU'A LA LIGNE "PRIERE DE NE RIEN MODIFIER CI-DESSUS"            %
%   TOUT LE BLOC QUI SUIT SERA SUPPRIME LORS DE L'EDITION FINALE DU POLY   %
%   CONTENANT LES RESUMES                                                  %
%%%%%%%%%%%%%%%%%%%%%%%%%%%%%%%%%%%%%%%%%%%%%%%%%%%%%%%%%%%%%%%%%%%%%%%%%%%%
\documentclass[10pt]{article}
%===  Priere de ne pas utiliser d'autres modules
\usepackage{latexsym}
\usepackage{bbm}              % fontes doubles (pour les ensembles, par ex.)
\usepackage{graphicx}         % pour d'eventuelles figures 
\usepackage{epsfig}           % (preferer graphicx, si possible)
\usepackage{amsmath}          % AMSTEX
\usepackage{amsfonts}
%
\setlength{\paperheight}{297mm}\setlength{\paperwidth}{210mm}
\setlength{\oddsidemargin}{10mm}\setlength{\evensidemargin}{10mm}
\setlength{\topmargin}{0mm}\setlength{\headheight}{10mm}\setlength{\headsep}{8mm}
\setlength{\textheight}{240mm}\setlength{\textwidth}{160mm}
\setlength{\marginparsep}{0mm}\setlength{\marginparwidth}{0mm}
\setlength{\footskip}{10mm}
\voffset -13mm\hoffset -10mm\parindent=0cm
\def\titre#1{\begin{center}{\Large{\bf #1}}\end{center}}
\def\orateur#1#2{\begin{center}{\underline{\large{\bf #1}}}, {#2}\end{center}}
\def\auteur#1#2{\begin{center}{\large{\bf #1}}, {#2}\end{center}}
\def\auteurenbasdepage#1#2#3{\small{\bf #1}, \small{#2}\\ \small{\tt #3}\\ }
\def\motscles#1{%
	\ifx#1\IsUndefined\relax\else\noindent{\normalsize{\bf Mots-cl\'es :}} #1\\ \fi}
\renewcommand{\refname}{\normalsize R\'ef\'erences}
%
\begin{document}
\thispagestyle{empty}
%%%%%%%%%%%%%%%%%%%%%%%%%%%%%%%%%%%%%%%%%%%%%%%%%%%%%%%%%%%%%%%%%%%%%%%%%%%%
%               PRIERE DE NE RIEN MODIFIER CI-DESSUS                       %
%%%%%%%%%%%%%%%%%%%%%%%%%%%%%%%%%%%%%%%%%%%%%%%%%%%%%%%%%%%%%%%%%%%%%%%%%%%%
%
% DANS TOUTE LA SUITE NOUS PRIONS LES AUTEURS DE BIEN VOULOIR UTILISER
% LA SYNTAXE TeX STRICTE POUR LES LETTRES ACCENTUEES.
% ON PEUT AU BESOIN LES REMPLACER APR\'ES LA FRAPPE DU DOCUMENT 
% PAR LEUR \'EQUIVALENT TeX.
% DANS LE CAS CONTRAIRE LES LETTRES ACCENTU\'ES N'APPARA\^ITRONT PAS
% DANS LE DOCUMENT FINAL.
%
%                   ORATEUR ET CO-AUTEURS
%---------------------------------------------------------------
%
% LES AUTEURS SONT PRIES DE FOURNIR LES BONS ARGUMENTS AUX MACROS CI-DESSOUS :
%   \Titre         : Titre de la communication
%   \NomOrateur    : Pr\'enom(s) NOM de l'Orateur
%   \AdresseCourteOrateur : Exemple : Universit\'e de Rennes 1
%   \AdresseLongueOrateur : Exemple : IRMAR, Universit\'e de Rennes 1, 263 avenue du G\'en\'eral Leclerc, 35000 Rennes
%   \EmailOrateur : Adresse electronique
%
% ET DE MEME POUR LES EVENTUELS CO-AUTEURS :
%   \NomAuteurI ...
%   \AdresseCourteAuteurI ...
%---------------------------------------------------------------
% DEFINIR ICI LE TITRE DE VOTRE COMMUNICATION
\def\Titre{Titre de la communication}
%
% DEFINIR ICI LES NOMS, ADRESSES, ... DE l'ORATEUR OU UNIQUE AUTEUR
\def\NomOrateur{Pr\'enom NOM}
\def\AdresseCourteOrateur{Adresse Courte}
\def\AdresseLongueOrateur{Adresse Longue}
\def\EmailOrateur{email}
%
% DEFINIR ICI LES NOMS, ADRESSES, ... DES EVENTUELS CO-AUTEURS
\def\NomAuteurI{Pr\'enom NOM}
\def\AdresseCourteAuteurI{Adresse Courte}
\def\AdresseLongueAuteurI{Adresse Longue}
\def\EmailAuteurI{email}
%=== et ainsi de suite II, III, IV, V ... pour les suivants
%
%=== Liste des mots-cles separes par des virgules si besoin
% N'enlever le signe % que si necessaire
%\def\listmotcles{mot-cle-1, mot-cle-2, ...}
%
%
%                   DEBUT DE LA COMMUNICATION
%---------------------------------------------------------------
% NE PAS MODIFIER LA LIGNE SUIVANTE 
% Le titre est a definir dans la macro \Titre (23 lignes plus haut)
\titre{\Titre}% 
%---------------------------------------------------------------
% TITRE & AUTEUR(S) 
% RETIRER LES SIGNES % SI NECESSAIRE ET PLACER DANS L'ORDRE SOUHAITE
% DANS LES LIGNES SUIVANTES NE MODIFIER QUE LES SIGNES COMMENTAIRES '%'
% Les noms, adresses, email de l'orateur et des co-auteurs sont a definir
% dans les macros \NomOrateur, \AdresseCourteOrateur etc. plus haut
%---------------------------------------------------------------
\orateur{\NomOrateur}{\AdresseCourteOrateur}
% NE PAS MODIFIER LES 4 LIGNES SUIVANTES sauf a retirer le signe commentaire '%'
%\auteur{\NomAuteurI}{\AdresseCourteAuteurI}
%\auteur{\NomAuteurII}{\AdresseCourteAuteurII}
%\auteur{\NomAuteurIII}{\AdresseCourteAuteurIII}
%\auteur{\NomAuteurIV}{\AdresseCourteAuteurIV}
%
\motscles{\listmotcles}
%---------------------------------------------------------------
% TEXTE DE LA COMMUNICATION
%---------------------------------------------------------------

Texte de la communication avec ses r\'ef\'erences
bibliographiques \'even\-tuelles \cite{ref1}\cite{ref2}.\\

%---------------------------------------------------------------
% REFERENCES BIBLIOGRAPHIQUES
%---------------------------------------------------------------
% NE PAS MODIFIER LES 2 LIGNES SUIVANTES
\bibliographystyle{plain}
\begin{thebibliography}{99}

\bibitem{ref1} {\sc Auteur}, {\sl Titre}, Editeur, ann\'ee.

\bibitem{ref2} {\sc Auteur}, {\sl Titre}, Revue, r\'ef\'erences, ann\'ee.

% NE PAS MODIFIER LA LIGNE SUIVANTE
\end{thebibliography}
%
%---------------------------------------------------------------
% NOM & ADRESSE COMPLETE & EMAIL DU OU DES AUTEURS
% RETIRER LES SIGNES % SI NECESSAIRE ET PLACER DANS L'ORDRE SOUHAITE
% DANS LES LIGNES SUIVANTES NE MODIFIER QUE LES SIGNES COMMENTAIRES '%'
%---------------------------------------------------------------
\vfill
\auteurenbasdepage{\NomOrateur}{\AdresseLongueOrateur}{\EmailOrateur}
% Les noms, adresses, email de l'orateur et des co-auteurs sont a definir
% dans les macros \NomOrateur, \AdresseCourteOrateur etc. plus haut
%
%\auteurenbasdepage{\NomAuteurI}{\AdresseLongueAuteurI}{\EmailAuteurI}
%\auteurenbasdepage{\NomAuteurII}{\AdresseLongueAuteurII}{\EmailAuteurII}
%\auteurenbasdepage{\NomAuteurIII}{\AdresseLongueAuteurIII}{\EmailAuteurIII}
%\auteurenbasdepage{\NomAuteurIV}{\AdresseLongueAuteurIV}{\EmailAuteurIV}
%%%%%%%%%%%%%%%%%%%%%%%%%%%%%%%%%%%%%%%%%%%%%%%%%%%%%%%%%%%%%%%%%%%%%%%%%%%
\end{document}
