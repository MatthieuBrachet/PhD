\documentclass[portrait,final,a0paper,fontscale=0.277]{baposter}
\usepackage[utf8]{inputenc}
\usepackage{calc}
\usepackage{graphicx}
\usepackage{amsmath}
\usepackage{amssymb}
\usepackage{relsize}
\usepackage{multirow}
\usepackage{rotating}
\usepackage{bm}
\usepackage{url}
\usepackage{graphicx}
\usepackage{multicol}
\usepackage{palatino}

%\graphicspath{{images/}{../images/}}
\usetikzlibrary{calc}

\newcommand{\captionfont}{\footnotesize}
\newcommand{\SET}[1]  {\ensuremath{\mathcal{#1}}}
\newcommand{\MAT}[1]  {\ensuremath{\boldsymbol{#1}}}
\newcommand{\VEC}[1]  {\ensuremath{\boldsymbol{#1}}}
\newcommand{\Video}{\SET{V}}
\newcommand{\video}{\VEC{f}}
\newcommand{\track}{x}
\newcommand{\Track}{\SET T}
\newcommand{\LMs}{\SET L}
\newcommand{\lm}{l}
\newcommand{\PosE}{\SET P}
\newcommand{\posE}{\VEC p}
\newcommand{\negE}{\VEC n}
\newcommand{\NegE}{\SET N}
\newcommand{\Occluded}{\SET O}
\newcommand{\occluded}{o}

\def\gint{\displaystyle\int}

%%%%%%%%%%%%%%%%%%%%%%%%%%%%%%%%%%%%%%%%%%%%%%%%%%%%%%%%%%%%%%%%%%%%%%%%%%%%%%%%
%%%% Some math symbols used in the text
%%%%%%%%%%%%%%%%%%%%%%%%%%%%%%%%%%%%%%%%%%%%%%%%%%%%%%%%%%%%%%%%%%%%%%%%%%%%%%%%

%%%%%%%%%%%%%%%%%%%%%%%%%%%%%%%%%%%%%%%%%%%%%%%%%%%%%%%%%%%%%%%%%%%%%%%%%%%%%%%%
% Multicol Settings
%%%%%%%%%%%%%%%%%%%%%%%%%%%%%%%%%%%%%%%%%%%%%%%%%%%%%%%%%%%%%%%%%%%%%%%%%%%%%%%%
\setlength{\columnsep}{1.5em}
\setlength{\columnseprule}{0mm}

%%%%%%%%%%%%%%%%%%%%%%%%%%%%%%%%%%%%%%%%%%%%%%%%%%%%%%%%%%%%%%%%%%%%%%%%%%%%%%%%
% Save space in lists. Use this after the opening of the list
%%%%%%%%%%%%%%%%%%%%%%%%%%%%%%%%%%%%%%%%%%%%%%%%%%%%%%%%%%%%%%%%%%%%%%%%%%%%%%%%
\newcommand{\compresslist}{%
\setlength{\itemsep}{1pt}%
\setlength{\parskip}{0pt}%
\setlength{\parsep}{0pt}%
}

%%%%%%%%%%%%%%%%%%%%%%%%%%%%%%%%%%%%%%%%%%%%%%%%%%%%%%%%%%%%%%%%%%%%%%%%%%%%%%
%%% Begin of Document
%%%%%%%%%%%%%%%%%%%%%%%%%%%%%%%%%%%%%%%%%%%%%%%%%%%%%%%%%%%%%%%%%%%%%%%%%%%%%%

\begin{document}

%%%%%%%%%%%%%%%%%%%%%%%%%%%%%%%%%%%%%%%%%%%%%%%%%%%%%%%%%%%%%%%%%%%%%%%%%%%%%%
%%% Here starts the poster
%%%---------------------------------------------------------------------------
%%% Format it to your taste with the options
%%%%%%%%%%%%%%%%%%%%%%%%%%%%%%%%%%%%%%%%%%%%%%%%%%%%%%%%%%%%%%%%%%%%%%%%%%%%%%
% Define some colors

\definecolor{lightergreen}{rgb}{0.7843,0.9568,1}
\definecolor{lightgreen}{rgb}{0.3921,0.3921,0.7843}

\hyphenation{resolution occlusions} 

\begin{poster}%
  % Poster Options
  {
  % Show grid to help with alignment
  grid=false,
  % Column spacing
  colspacing=1em,
  % Color style
  bgColorOne=white,
  bgColorTwo=white,
  borderColor=lightgreen,
  headerColorOne=black,
  headerColorTwo=lightgreen,
  headerFontColor=white,
  boxColorOne=white,
  boxColorTwo=lightgreen,
  % Format of textbox
  textborder=roundedleft,
  % Format of text header
  eyecatcher=false,
  headerborder=closed,
  headerheight=0.1\textheight,
%  textfont=\sc, An example of changing the text font
  headershape=roundedright,
  headershade=shadelr,
  headerfont=\Large\bf\textsc, %Sans Serif
  textfont={\setlength{\parindent}{1.5em}},
  boxshade=plain,
%  background=shade-tb,
  background=plain,
  linewidth=2pt
  }
  % Eye Catcher
  {\includegraphics[height=5em]{images/graph_occluded.pdf}} 
  % Title
  {\bf\textsc{Schémas Compacts Hermitiens sur la Sphère : Applications en Climatologie et Océanographie numérique}\vspace{0.4em}}
  % Authors
  {\textsc{ M. BRACHET, J.-P. Croisille (directeur de thèse)}}
  % University logo
  {% The makebox allows the title to flow into the logo, this is a hack because of the L shaped logo.
    \includegraphics[height=6.0em]{images/logo.jpg}
  }

%%%%%%%%%%%%%%%%%%%%%%%%%%%%%%%%%%%%%%%%%%%%%%%%%%%%%%%%%%%%%%%%%%%%%%%%%%%%%%
%%% Now define the boxes that make up the poster
%%%---------------------------------------------------------------------------
%%% Each box has a name and can be placed absolutely or relatively.
%%% The only inconvenience is that you can only specify a relative position 
%%% towards an already declared box. So if you have a box attached to the 
%%% bottom, one to the top and a third one which should be in between, you 
%%% have to specify the top and bottom boxes before you specify the middle 
%%% box.
%%%%%%%%%%%%%%%%%%%%%%%%%%%%%%%%%%%%%%%%%%%%%%%%%%%%%%%%%%%%%%%%%%%%%%%%%%%%%%
    %
    % A coloured circle useful as a bullet with an adjustably strong filling
    \newcommand{\colouredcircle}{%
      \tikz{\useasboundingbox (-0.2em,-0.32em) rectangle(0.2em,0.32em); \draw[draw=black,fill=lightblue,line width=0.03em] (0,0) circle(0.18em);}}

%%%%%%%%%%%%%%%%%%%%%%%%%%%%%%%%%%%%%%%%%%%%%%%%%%%%%%%%%%%%%%%%%%%%%%%%%%%%%%
  \headerbox{Objectif :}{name=problem,column=0,row=0}{
%%%%%%%%%%%%%%%%%%%%%%%%%%%%%%%%%%%%%%%%%%%%%%%%%%%%%%%%%%%%%%%%%%%%%%%%%%%%%%
   Résolution d'équations issues de la climatologie par des méthodes numériques.
En particulier l'équation Shallow Water \eqref{SWEC} avec force de Coriolis.
   
  
 }

%%%%%%%%%%%%%%%%%%%%%%%%%%%%%%%%%%%%%%%%%%%%%%%%%%%%%%%%%%%%%%%%%%%%%%%%%%%%%%
  \headerbox{Difficultés :}{name=difficult,column=0,below=problem}{
%%%%%%%%%%%%%%%%%%%%%%%%%%%%%%%%%%%%%%%%%%%%%%%%%%%%%%%%%%%%%%%%%%%%%%%%%%%%%%
   Difficultés liées au problème physique :
   \begin{itemize}
   \item Singularités sur les maillages de la sphère,
   \item Tendre vers le problème complet.
   \end{itemize}
   
	Difficultés numériques :
	\begin{itemize}
	\item coût en calcul important : rapidité du calcul,
	\item précision spaciale et temporelle.
	\end{itemize}	   
  }
  
%%%%%%%%%%%%%%%%%%%%%%%%%%%%%%%%%%%%%%%%%%%%%%%%%%%%%%%%%%%%%%%%%%%%%%%%%%%%%%
\headerbox{Modèle mathématique :}{name=model,column=1,span=2,row=0}{
  %%%%%%%%%%%%%%%%%%%%%%%%%%%%%%%%%%%%%%%%%%%%%%%%%%%%%%%%%%%%%%%%%%%%%%%%%%%%%%
  Modèle issu de l'équation de Navier-Stokes incompressible.
  
  Equation Shallow Water avec force de Coriolis \eqref{SWEC}

  \begin{equation}
  \label{SWEC}
  \left\lbrace
  \begin{array}{rcl}
  \dfrac{\partial \mathbf{u}}{\partial t} + \mathbf{u} \cdot \nabla \mathbf{u} + g \nabla h + f \mathbf{k} \wedge \mathbf{u} & = & \mathbf{0} \\
  \dfrac{\partial h}{\partial t} + \nabla \cdot \left( h \mathbf{u} \right) & = & 0
  \end{array}
  \right.\text{ pour } \mathbf{x} \in \mathbb{S}_R^2 \text{ et } t>0.
  \end{equation}
  
  où $h$ est l'épaisseur du fluide et $\mathbf{u}$ le champ de vitesse moyenné en chaque point $\mathbf{x} \in \mathbb{S}_R^2$, $\mathbf{g}$ et $\mathbf{f}$: vecteur de gravité et terme de Coriolis.
    
}

%%%%%%%%%%%%%%%%%%%%%%%%%%%%%%%%%%%%%%%%%%%%%%%%%%%%%%%%%%%%%%%%%%%%%%%%%%%%%%
  \headerbox{Méthode numérique :}{name=method,column=0,span=1, below=difficult }{
%%%%%%%%%%%%%%%%%%%%%%%%%%%%%%%%%%%%%%%%%%%%%%%%%%%%%%%%%%%%%%%%%%%%%%%%%%%%%%
\begin{itemize}
\item \textbf{Maillage : } de type "Cubed-Sphere"

\begin{center}
\includegraphics[scale=0.4]{./images/CS_lauritzen.png}


\begin{flushright}
\tiny{\textit{Illustration P. Lauritzen}}
\end{flushright}
\end{center}



\item \textbf{Discrétisation en temps : } méthode de Runge-Kutta d'ordre 4 (RK4),
\item \textbf{Ecriture des opérateurs : } écriture des opérateurs en coordonnées locales $( \xi, \eta)$ et discrétisation (par exemple pour le gradient ) :
\begin{equation}
\nabla h = \mathbf{g}^{\xi} \dfrac{\partial h}{\partial \xi}+\mathbf{g}^{\eta} \dfrac{\partial h}{\partial \eta}
\end{equation}

avec $(\mathbf{g}^{\xi}, \mathbf{g}^{\eta})$ la base duale de $(\partial_{\xi} \mathbf{x}, \partial_{\eta} \mathbf{x})$ par rapport à la métrique $\mathbf{G}$.

\item \textbf{Discrétisation spatiale :} Utilisation de schémas compacts d'ordre 4 :
$$\dfrac{1}{6}\bar{\delta}_x h_j+\dfrac{2}{3}\bar{\delta}_x h_j + \dfrac{1}{6}\bar{\delta}_x h_{j-1} = \dfrac{h_{j+1}- h_{j-1}}{2\Delta x}
$$

\end{itemize}
  }
  
%%%%%%%%%%%%%%%%%%%%%%%%%%%%%%%%%%%%%%%%%%%%%%%%%%%%%%%%%%%%%%%%%%%%%%%%%%%%%%
  \headerbox{Résultats numériques : équation d'advection}{name=advection,column=1,span=2, below=model}{
%%%%%%%%%%%%%%%%%%%%%%%%%%%%%%%%%%%%%%%%%%%%%%%%%%%%%%%%%%%%%%%%%%%%%%%%%%%%%%
Résolution de l'équation d'advection :
\begin{equation}
  \label{advection}
  \left\lbrace
  \begin{array}{rcl}
  \dfrac{\partial h}{\partial t} + \mathbf{c}( \mathbf{x}, t ) \cdot \nabla h & = & 0 \\
  h(\mathbf{x},t) & = & h_0( \mathbf{x} )
  \end{array}
  \right. \text{ pour } \mathbf{x} \in \mathbb{S}_R^2 \text{ et } t>0
  \end{equation}
 
  
  \textbf{Test de Nair et Jablonowski \cite{Nair2008} :} rotation d'un vortex autour de la sphère.
  
\begin{center}
  \includegraphics[scale=0.3]{courbe_AE.png}
  \vspace{1cm}
  \includegraphics[scale=0.35]{erreur_AE.png}
  
  \textbf{Figure 1:} Solution à $t=12$ jours (gauche), erreur relative (droite). 
\end{center}



}
  
  %%%%%%%%%%%%%%%%%%%%%%%%%%%%%%%%%%%%%%%%%%%%%%%%%%%%%%%%%%%%%%%%%%%%%%%%%%%%%%
  \headerbox{Résultats numériques : eq. Shallow Water linéarisée }{name=SWEC,column=1,span=2, below=advection}{
%%%%%%%%%%%%%%%%%%%%%%%%%%%%%%%%%%%%%%%%%%%%%%%%%%%%%%%%%%%%%%%%%%%%%%%%%%%%%%
Résolution de l'équation d'advection :
\begin{equation}
  \label{LSWEC}
  \left\lbrace
  \begin{array}{rcl}
  \dfrac{\partial \mathbf{u}}{\partial t} + g \nabla h + f \mathbf{k} \wedge \mathbf{u} & = & 0\\
  \dfrac{\partial h}{\partial t} + H \nabla \cdot  \mathbf{u} & = & 0 \\
  h(\mathbf{x},t)  =  h_0( \mathbf{x} ) & \text{ et } & \mathbf{u}(\mathbf{x},t)  =  \mathbf{u}_0( \mathbf{x} )
  \end{array}
  \right. \text{ pour } \mathbf{x} \in \mathbb{S}_R^2 \text{ et } t>0
  \end{equation}
 
  
  \textbf{Test de conservation :} 
  \begin{itemize}
  \item Solution stationnaire : $\mathbf{u}(\mathbf{x},t) = u(\theta) \mathbf{e}_{\lambda}$ et $h(\mathbf{x},t) = h_0 - \dfrac{R}{g}\gint_0^{\theta} f(\theta)u(\theta) d \theta$
  \item Masse et énergie :
  $$\gint_{ \mathbf{S}_R^2} h(\mathbf{x})  d \mathbf{x} \text{ et }\gint_{ \mathbf{S}_R^2} gh^2 + H |\mathbf{u}|^2  d \mathbf{x}$$
  \end{itemize}
\begin{center}
  \includegraphics[height=4cm]{conservation_LSWE.png}
  \vspace{1cm}
  \includegraphics[scale=0.35]{erreur_LSWE.png}
  
  \textbf{Figure 2:} Conservation de masse et de l'énergie (gauche), erreur relative (droite). 
\end{center}
  }

  %%%%%%%%%%%%%%%%%%%%%%%%%%%%%%%%%%%%%%%%%%%%%%%%%%%%%%%%%%%%%%%%%%%%%%%%%%%%%%
  \headerbox{References :}{name=references,column=0,above=bottom}{
%%%%%%%%%%%%%%%%%%%%%%%%%%%%%%%%%%%%%%%%%%%%%%%%%%%%%%%%%%%%%%%%%%%%%%%%%%%%%%
    \smaller
    \bibliographystyle{ieee}
    \renewcommand{\section}[2]{\vskip 0.05em}
      \begin{thebibliography}{1}\itemsep=-0.01em
      \setlength{\baselineskip}{0.4em}
        
       \bibitem{Croisille2013}
        J.-P. Croisille.
        \newblock {H}ermitian compact interpolation on the Cubed-Sphere grid
        \newblock In {Jour. of Sci. Comp.}
        
       \bibitem{Galewsky2004}
        J. Galewsky, R. K. Scott, L. M. Polvani.
        \newblock {A}n initial value test case for the shallow water equations
        \newblock In {Tellus}
        
       \bibitem{Nair2008}
        R. D. Nair, C. Jablonowski.
        \newblock {M}oving Vortices on the Sphere: A Test Case for Horizontal Advection Problems
        \newblock In {American Meteorology Society}
        
       \bibitem{Williamson1994}
        D. L. Williamson, J. B. Drake, J. J. Hack.
        \newblock {A} Standard Test Set for Numerical Approximations to the Shallow Water Equation in Spherical Geometry
        \newblock In {1991}
        
      \end{thebibliography}
   \vspace{0.3em}
  }
\end{poster}

\end{document}

% J.-P. Croisille. Hermitian compact interpolation on the Cubed-Sphere grid. Jour. Sci. Comp.

