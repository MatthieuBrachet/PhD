%%%%%%%%%%%%%%%%%%%%%%%%%%%%%%%%%%%%%%%%%%%%%%%%%%%%%%%%%%%%%%%%%%%%%%%%%%%%
%               PRIERE DE NE RIEN MODIFIER CI-DESSOUS                      %
%       JUSQU'A LA LIGNE "PRIERE DE NE RIEN MODIFIER CI-DESSUS"            %
%   TOUT LE BLOC QUI SUIT SERA SUPPRIME LORS DE L'EDITION FINALE DU POLY   %
%   CONTENANT LES RESUMES                                                  %
%%%%%%%%%%%%%%%%%%%%%%%%%%%%%%%%%%%%%%%%%%%%%%%%%%%%%%%%%%%%%%%%%%%%%%%%%%%%
\documentclass[10pt]{article}
%===  Priere de ne pas utiliser d'autres modules
\usepackage{latexsym}
\usepackage{bbm}              % fontes doubles (pour les ensembles, par ex.)
\usepackage{graphicx}         % pour d'eventuelles figures 
\usepackage{epsfig}           % (preferer graphicx, si possible)
\usepackage{amsmath}          % AMSTEX
\usepackage{amsfonts}
%
\setlength{\paperheight}{297mm}\setlength{\paperwidth}{210mm}
\setlength{\oddsidemargin}{10mm}\setlength{\evensidemargin}{10mm}
\setlength{\topmargin}{0mm}\setlength{\headheight}{10mm}\setlength{\headsep}{8mm}
\setlength{\textheight}{240mm}\setlength{\textwidth}{160mm}
\setlength{\marginparsep}{0mm}\setlength{\marginparwidth}{0mm}
\setlength{\footskip}{10mm}
\voffset -13mm\hoffset -10mm\parindent=0cm
\def\titre#1{\begin{center}{\Large{\bf #1}}\end{center}}
\def\orateur#1#2{\begin{center}{\underline{\large{\bf #1}}}, {#2}\end{center}}
\def\auteur#1#2{\begin{center}{\large{\bf #1}}, {#2}\end{center}}
\def\auteurenbasdepage#1#2#3{\small{\bf #1}, \small{#2}\\ \small{\tt #3}\\ }
\def\motscles#1{%
	\ifx#1\IsUndefined\relax\else\noindent{\normalsize{\bf Mots-cl\'es :}} #1\\ \fi}
\renewcommand{\refname}{\normalsize R\'ef\'erences}
%
\begin{document}
\thispagestyle{empty}
%%%%%%%%%%%%%%%%%%%%%%%%%%%%%%%%%%%%%%%%%%%%%%%%%%%%%%%%%%%%%%%%%%%%%%%%%%%%
%               PRIERE DE NE RIEN MODIFIER CI-DESSUS                       %
%%%%%%%%%%%%%%%%%%%%%%%%%%%%%%%%%%%%%%%%%%%%%%%%%%%%%%%%%%%%%%%%%%%%%%%%%%%%
%
% DANS TOUTE LA SUITE NOUS PRIONS LES AUTEURS DE BIEN VOULOIR UTILISER
% LA SYNTAXE TeX STRICTE POUR LES LETTRES ACCENTUEES.
% ON PEUT AU BESOIN LES REMPLACER APR\'ES LA FRAPPE DU DOCUMENT 
% PAR LEUR \'EQUIVALENT TeX.
% DANS LE CAS CONTRAIRE LES LETTRES ACCENTU\'ES N'APPARA\^ITRONT PAS
% DANS LE DOCUMENT FINAL.
%
%                   ORATEUR ET CO-AUTEURS
%---------------------------------------------------------------
%
% LES AUTEURS SONT PRIES DE FOURNIR LES BONS ARGUMENTS AUX MACROS CI-DESSOUS :
%   \Titre         : Titre de la communication
%   \NomOrateur    : Pr\'enom(s) NOM de l'Orateur
%   \AdresseCourteOrateur : Exemple : Universit\'e de Rennes 1
%   \AdresseLongueOrateur : Exemple : IRMAR, Universit\'e de Rennes 1, 263 avenue du G\'en\'eral Leclerc, 35000 Rennes
%   \EmailOrateur : Adresse electronique
%
% ET DE MEME POUR LES EVENTUELS CO-AUTEURS :
%   \NomAuteurI ...
%   \AdresseCourteAuteurI ...
%---------------------------------------------------------------
% DEFINIR ICI LE TITRE DE VOTRE COMMUNICATION
\def\Titre{Approximation num\'erique de probl\`emes de convection 
sur la sph\`ere par un sch\'ema compact}
%
% DEFINIR ICI LES NOMS, ADRESSES, ... DE l'ORATEUR OU UNIQUE AUTEUR
\def\NomOrateur{Matthieu BRACHET}
\def\AdresseCourteOrateur{IECL, Univ. Lorraine, Metz}
\def\AdresseLongueOrateur{Inst. Elie Cartan de Lorraine, UMR 7502, Univ. Lorraine, Metz}
\def\EmailOrateur{email}
%
% DEFINIR ICI LES NOMS, ADRESSES, ... DES EVENTUELS CO-AUTEURS
\def\NomAuteurI{Jean-Pierre CROISILLE}
\def\AdresseCourteAuteurI{IECL, Univ. Lorraine, Metz}
\def\AdresseLongueAuteurI{Inst. Elie Cartan de Lorraine, UMR 7502, Univ. Lorraine, Metz}
\def\EmailAuteurI{jean-pierre.croisille@univ-lorraine.fr}
%=== et ainsi de suite II, III, IV, V ... pour les suivants
%
%=== Liste des mots-cles separes par des virgules si besoin
% N'enlever le signe % que si necessaire
%\def\listmotcles{mot-cle-1, mot-cle-2, ...}
%
%
%                   DEBUT DE LA COMMUNICATION
%---------------------------------------------------------------
% NE PAS MODIFIER LA LIGNE SUIVANTE 
% Le titre est a definir dans la macro \Titre (23 lignes plus haut)
\titre{\Titre}% 
%---------------------------------------------------------------
% TITRE & AUTEUR(S) 
% RETIRER LES SIGNES % SI NECESSAIRE ET PLACER DANS L'ORDRE SOUHAITE
% DANS LES LIGNES SUIVANTES NE MODIFIER QUE LES SIGNES COMMENTAIRES '%'
% Les noms, adresses, email de l'orateur et des co-auteurs sont a definir
% dans les macros \NomOrateur, \AdresseCourteOrateur etc. plus haut
%---------------------------------------------------------------
\orateur{\NomOrateur}{\AdresseCourteOrateur}
% NE PAS MODIFIER LES 4 LIGNES SUIVANTES sauf a retirer le signe commentaire '%'
\auteur{\NomAuteurI}{\AdresseCourteAuteurI}
%\auteur{\NomAuteurII}{\AdresseCourteAuteurII}
%\auteur{\NomAuteurIII}{\AdresseCourteAuteurIII}
%\auteur{\NomAuteurIV}{\AdresseCourteAuteurIV}
%
\motscles{\listmotcles}
%---------------------------------------------------------------
% TEXTE DE LA COMMUNICATION
%---------------------------------------------------------------

La r\'esolution approch\'ee d'\'equations de propagation sur la sph\`ere 
intervient de fa\c con essentielle en climatologie num\'erique.
Dans cet expos\'e, on pr\'esentera des d\'eveloppements r\'ecents 
sur une approche diff\'erences finies compacte \cite{Lele1991} sur la Cubed-Sphere \cite{Ronchi1996}.
La discr\'etisation spatiale est d'ordre 4. 
% de type sch\'emas hermitiens d'ordre \'elev\'e \cite{Lele1991}. 
La discr\'etisation temporelle est effectu\'ee \`a l'aide du sch\'ema RK4,
coupl\'e \`a un filtrage qui sert \`a att\'enuer les effets 
de dispersion et de dissipation num\'erique.

Un cas test repr\'esentant l'advection de 
% benchmark classique sera pr\'esent\'e sur l'\'equation d'advection. Il s'agit de 
deux vortex instationnaires en d\'eplacement autour de la sph\`ere \cite{Nair2008} sera pr\'esent\'e.
Des r\'esultats sur le syst\`eme Shallow Water sur la sph\`ere en rotation 
seront \'egalement pr\'esent\'es, \cite{Brachet2016}, de m\^eme que des r\'esultats 
pr\'eliminaires sur l'analyse math\'ematique
du sch\'ema.

\begin{center}
\includegraphics[scale=0.6]{ref_7363377214_solapprochee_test_2.png}
\end{center}


%---------------------------------------------------------------
% REFERENCES BIBLIOGRAPHIQUES
%---------------------------------------------------------------
% NE PAS MODIFIER LES 2 LIGNES SUIVANTES
\bibliographystyle{plain}
\begin{thebibliography}{99}

\bibitem{Brachet2016} {\sc M. Brachet, J.-P. Croisille} {\sl Numerical simulation of vortex propagation on the Cubed-Sphere using compact scheme}, Preprint, 2016.

\bibitem{Lele1991} {\sc S. K. Lele}, {\sl Compact Finite Difference Schemes with Spectral-like Resolution}, J. Comput. Phys. ,103, 1992, pp 16--42.

\bibitem{Nair2008} {\sc R. D. Nair, C. Jablonowski}, {\sl Moving Vortices on the Sphere : a test case for horizontal advection problem}, Mon. Wea. Rev. , 136, 2008, pp. 689--711.

\bibitem{Ronchi1996} {\sc C. Ronchi, R. Iacono and P. S. Paolucci}, {\sl The Cubed Sphere : A New Method for the Solution of Partial Differential Equation in Spherical Geometry}, J. Comput. Phys. , 124, 1996, pp. 93--114.

% NE PAS MODIFIER LA LIGNE SUIVANTE
\end{thebibliography}
%
%---------------------------------------------------------------
% NOM & ADRESSE COMPLETE & EMAIL DU OU DES AUTEURS
% RETIRER LES SIGNES % SI NECESSAIRE ET PLACER DANS L'ORDRE SOUHAITE
% DANS LES LIGNES SUIVANTES NE MODIFIER QUE LES SIGNES COMMENTAIRES '%'
%---------------------------------------------------------------
\vfill
\auteurenbasdepage{BRACHET Matthieu}{Institut Elie Cartan de Lorraine, Universi\'e de Lorraine,
Site de Metz, B\^at.  A Ile du Saulcy, F-57045 Metz Cedex 1}{matthieu.brachet@univ-lorraine.fr}
% Les noms, adresses, email de l'orateur et des co-auteurs sont a definir
% dans les macros \NomOrateur, \AdresseCourteOrateur etc. plus haut
%
%\auteurenbasdepage{\NomAuteurI}{\AdresseLongueAuteurI}{\EmailAuteurI}
%\auteurenbasdepage{\NomAuteurII}{\AdresseLongueAuteurII}{\EmailAuteurII}
%\auteurenbasdepage{\NomAuteurIII}{\AdresseLongueAuteurIII}{\EmailAuteurIII}
%\auteurenbasdepage{\NomAuteurIV}{\AdresseLongueAuteurIV}{\EmailAuteurIV}
%%%%%%%%%%%%%%%%%%%%%%%%%%%%%%%%%%%%%%%%%%%%%%%%%%%%%%%%%%%%%%%%%%%%%%%%%%%
\end{document}
