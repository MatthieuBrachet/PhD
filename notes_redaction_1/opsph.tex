% opsph.tex
% opérateurs sphériques sur la sphère

\chapter{Opérateurs sphériques}

\section{Opérateur sphériques discrets}

La résolution d'équations aux dérivées parties de type Shallow Water nécessite de passer par le calcul des opérateurs classiques : gradient, divergence et rotationnel.

Si $\mathbf{u} : \mathbb{S}_a^2 \mapsto \mathbb{T}\mathbb{S}_a^2$ est un champ de vecteur régulier sur la sphère et $h : \mathbb{S}_a^2 \mapsto\mathbb{R}$ une fonction régulière sur la sphère.
Chaque opérateur peut être écrit en coordonnées $(\xi, \eta)$ sur la sphère grâce aux formules suivantes :

\begin{itemize}
\item \textbf{Gradient :}
\begin{equation}
\nabla h = \mathbf{g}^{\xi} \dfrac{\partial h}{\partial \xi}_{|\eta} + \mathbf{g}^{\eta} \dfrac{\partial h}{\partial \eta}_{|\xi} 
\label{eq: gradient}
\end{equation}
\item \textbf{Divergence :}
\begin{equation}
\begin{array}{rcl}
\nabla \cdot \mathbf{u} & = & \mathbf{g}^{\xi} \cdot \dfrac{\partial \mathbf{u}}{\partial \xi}_{|\eta} + \mathbf{g}^{\eta} \cdot \dfrac{\partial \mathbf{u}}{\partial \eta}_{|\xi} \\
	& = & \dfrac{1}{\sqrt{\bar{\mathbf{G}}}} \left( \dfrac{\partial}{\partial \xi} \left( \sqrt{\bar{\mathbf{G}}} \mathbf{u} \cdot \mathbf{g}^{\xi} \right)_{|\eta} + \dfrac{\partial}{\partial \eta} \left( \sqrt{\bar{\mathbf{G}}} \mathbf{u} \cdot \mathbf{g}^{\eta} \right)_{|\xi} \right) 
\end{array}
\label{eq: divergence}
\end{equation}
\item \textbf{Rotationnel :}
\begin{equation}
\nabla \wedge \mathbf{u} = \mathbf{g}^{\xi} \wedge \dfrac{\partial \mathbf{u}}{\partial \xi}_{|\eta} + \mathbf{g}^{\eta} \wedge \dfrac{\partial \mathbf{u}}{\partial \eta}_{|\xi}  
\label{eq: rotationnel}
\end{equation}
\end{itemize}




\section{Filtrage sur la sphère}