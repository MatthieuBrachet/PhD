\documentclass[10pt,a4paper]{report}
\usepackage[utf8]{inputenc}
\usepackage[francais]{babel}
\usepackage[T1]{fontenc}
\usepackage{amsmath}
\usepackage{amsfonts}
\usepackage{amssymb}
\usepackage{graphicx}
\usepackage[left=2cm,right=2cm,top=3cm,bottom=3.5cm]{geometry}
%\author{Brachet Matthieu}
%\title{Curriculum Vitae}
\begin{document}
\begin{center}
{\fontfamily{pnc}\selectfont
\begin{LARGE}
CURRICULUM VITAE
\end{LARGE}

\begin{large}
Matthieu Brachet
\end{large}

\hrulefill
}
\end{center}

\noindent
84, avenue du Général Leclerc\\
54 000 Nancy, France
\vspace{0.2cm}\\
Téléphone : +33 6 43 50 07 86\\
E-mail : brachetmat@free.fr\\
Page Web : \verb?www.iecl.univ-lorraine.fr/~Matthieu.Brachet/?
\vspace{0.2cm}\\
Nationalité : française\\
Date de naissance : 11 février 1991 (Lille, France)


%% ******************************************************************************
\vspace{1.2cm}
\noindent
{\fontfamily{pnc}\selectfont
\begin{Large}
Scolarité
\end{Large}
\hrulefill
}

\begin{center}
\begin{tabular}{r p{12cm}}
\textbf{oct. 2014 - ...} & \textbf{Doctorant} sous la direction de Prof. J.-P. Croisille à l'Université de Lorraine (Metz, France) à l'Instut Elie Cartan de Lorraine (IECL).\newline Titre : "\textit{Schémas compacts hermitiens sur la sphère - Applications en climatologie et océanographie numérique}".\\

& \\

\textbf{2012 - 2014} & \textbf{Master en Mathématiques Appliquées et Modélisation} à l'Univ. Picardie Jules Verne (Amiens, France). \\

& \textbf{Thèse de Master} (mémoire de 2nd année) sous la direction de Prof. J.-P. Chehab. \newline 
Titre : "\textit{Préconditionnement et résolution de problèmes d'évolutions non linéaires}". \\

& \textbf{Projet de recherches} (première année) sous la direction de V. Martin. \newline 
Titre : "\textit{\'Etude de quelques méthodes de Krylov}". \\

& \\

\textbf{2010 - 2012} & \textbf{Licence de Mathématiques} à l'Univ. Picardie Jules Verne (Amiens, France) \\

& \\

\textbf{2008 - 2010} & \textbf{Classes préparatoires} (CPGE : MPSI et MP) au Lycée Pierre d'Ailly, Compiègne, France.\\

& \\

\textbf{2008} & \textbf{Baccalauréat} Scientifique option mathématiques au lycée Cassini (Clermont de l'Oise, France).\\

& \\
\end{tabular}
\end{center}

%% ******************************************************************************
\vspace{.5cm}
\noindent
{\fontfamily{pnc}\selectfont
\begin{Large}
Formations scientifiques et écoles d'étés
\end{Large}
\hrulefill
}
\vspace{0.6cm}

\begin{center}
\begin{tabular}{r p{12cm}}
\textbf{été 2015} & \textbf{CEMRACS'15 Coupling Multiphysics} au Centre International de Rencontres Mathématiques (Marseille, France) sous la direction de Pr E. Frenod, Pr. C. Prud'homme, A. Rousseau and Pr. S. Salmon.\newline
Titre : "\textit{Modélisation et résolution numérique d'un système couplant les équations de Stokes instationnaire et l'équation d'Exner (HydroMorpho)}".\\

& \\

\textbf{2014} & \textbf{Fortran 90 parties 1 et 2} à l'Institut du Développement et des Ressources en Informatique Scientifique (Orsay, France) par P. Corde.\\

& \\

\end{tabular}
\end{center}


%% ******************************************************************************
\vspace{1cm}
\noindent
{\fontfamily{pnc}\selectfont
\begin{Large}
Présentations et Posters
\end{Large}
\hrulefill
}
\vspace{0.5cm}

\noindent
{\fontfamily{pnc}\selectfont
\textbf{Séminaires :}
}

\begin{itemize}
\item \textbf{Séminaire d'Analyse Appliquée et EDP}, Amiens, France, novembre 2016 (LAMFA),
\item \textbf{Séminaire Doctorants}, Strasbourg, France, mars 2016,
\item \textbf{Séminaire Doctorants}, Rennes, France, novembre 2015,
\item \textbf{Séminaire Doctorants}, Reims, France, octobre 2015.
\end{itemize}

\vspace{0.3cm}
\noindent
{\fontfamily{pnc}\selectfont
\textbf{Conferences :}
}

\begin{itemize}
\item \textbf{PDE on the Sphere}, Paris, France, avril 2017,
\item \textbf{Journées EDP, IECL}, Metz, France, mars 2017,
\item \textbf{Journée des Doctorants, IECL}, Nancy, France, décembre 2016,
\item \textbf{CANUM}, Obernai, France, 11 mai 2016,
\item \textbf{Journées doctorantes Lebesgues}, Nantes, France, octobre 2016.
\end{itemize}

\vspace{0.3cm}
\noindent
{\fontfamily{pnc}\selectfont
\textbf{Posters :}
}

\begin{itemize}
\item \textbf{6e Colloque EDP-Normandie}, Caen, France, octobre 2017.
\end{itemize}

\vspace{0.3cm}
\noindent
{\fontfamily{pnc}\selectfont
\textbf{Vulgarisation Scientifique :}
}
\begin{itemize}
\item \textbf{Séminaire étudiant du CESI }(\'Ecole d'ingénieur), Nancy, décembre 2016,
\item \textbf{Remise de prix des olympiades de mathématiques}, Metz, mai 2016,
\item \textbf{Ma thèse en 180 secondes}, finale régionale, Nancy, avril 2016.
\end{itemize}

%% ******************************************************************************
\vspace{.5cm}
\noindent
{\fontfamily{pnc}\selectfont
\begin{Large}
Enseignements
\end{Large}
\hrulefill
}

\noindent
\begin{center}
\begin{tabular}{r p{12cm}}
\textbf{2017 - 2018} & \begin{itemize}
\item Cours et travaux dirigés (Probabilités) en L1 Biologie (Nancy),
\item Travaux dirigés (Analyse numérique) en L3 Mathématiques (Nancy),
\item Travaux pratiques (Analyse numérique - Matlab) en première année d'ingénieur aux Mines (Nancy),
\item Travaux dirigés (Probabilités) en seconde année d'ingénieur à l'ENSEM (Nancy).
\end{itemize}\\

& \\

\textbf{2015 - 2017} & \begin{itemize}
\item Travaux dirigés (Mécanique des fluides numérique), M1 mécanique (Nancy),
\item Travaux pratiques (Analyse numérique, Matlab), L3 génie civile (Nancy),
\item Travaux pratiques (Introduction à Matlab), L2 physique (Nancy).
\end{itemize}\\

& \\

\textbf{2014 - 2015} & \begin{itemize}
\item Travaux dirigés (Mécanique des fluides numérique), M1 mécanique (Nancy),
\item Travaux pratiques (Analyse numérique en Matlab), L3 génie civile (Nancy),
\item Travaux pratiques (Introduction à Matlab), L2 physique (Nancy).
\item Travaux pratiques (Analyse numérique non linéaire, Matlab), L3 mathématiques (Metz),
\item Projets, L3 SPI Nancy,
\item Projets, 2eme année d'école d'ingénieur à Telecom Nancy.
\end{itemize}\\

& \\

\textbf{2011 - 2012} & Préparation mathématiques au concours vétérinaire, L3 Biologie, Amiens.\\

& \\
\end{tabular}
\end{center}

%% ******************************************************************************
\vspace{.5cm}
\newpage
\noindent
{\fontfamily{pnc}\selectfont
\begin{Large}
Publications
\end{Large}
\hrulefill
}
\vspace{0.5cm}

[1] {\sc  M. Brachet, J.-P. Croisille}, {\sl Numerical simulations of propagation problemes on the sphere using a compact scheme}, preprint.

\vspace{0.6cm}

[2] {\sc M. Brachet, J.-P. Chehab}, {\sl Stabilized Times Schemes for High Accurate Finite Differences Solutions of Nonlinear Parabolic Equations}, J. of Sci. Comp., 69(3), 946-982, 2016.

\vspace{0.6cm}

[3] {\sc  N. Aissiouene, T. Amtout, M. Brachet, E. Frenod, R. Hild, C. Prud'homme, A. Rousseau, S. Salmon}, {\sl  Hydromorpho: A coupled model for unsteady Stokes/Exner equations and numerical results with FEEL++ library}, ESAIM: Proc and survey, December 2016, Vol. 55, p. 23-40.






%% ******************************************************************************
\vspace{1cm}
\noindent
{\fontfamily{pnc}\selectfont
\begin{Large}
Expérience professionnelle
\end{Large}
\hrulefill
}
%\vspace{0.6cm}

\noindent
\begin{center}
\begin{tabular}{p{5cm} p{10cm}}
\textbf{2010 - 2014} & Cours particuliers de mathématiques (environ 8 heures par semaines),\\

& \\

\textbf{été 2013 and 2012} & Employé de rayons Intermarché (Fitz-James, France) \\

& \\

\textbf{été 2011 and 2010} & Guichet au Crédit du Nord (Calais et Bresles, France) \\

& \\

\end{tabular}
\end{center}

%% ******************************************************************************
\vspace{0.5cm}
\noindent
{\fontfamily{pnc}\selectfont
\begin{Large}
Autres
\end{Large}
\hrulefill
}
\vspace{.5cm}

\noindent
\begin{itemize}
\item \textbf{Séminaire doctorants :} co-organisation du séminaire doctorants (2016 à 2018) ainsi que l'organisation de la rencontre annuelle des doctorants de l'IECL (2017).
\item \textbf{Languages :} Français (langue maternelle), Anglais (B2),
\item \textbf{Languages informatiques et Logiciels :} Matlab (expert), Fortran (pratiques avancées), librairies éléments finis (FreeFem, FEEL++), Maple.
\item \textbf{O.S. :} Linux (pratiques avancées), Windows, Mac.
\item \textbf{Permis de conduire :} obtenu en 2009.
\item \textbf{PESC1 :} (Formation aux premiers secours) obtenu en 2016.
\item \textbf{Loisirs :} guitare(débutant), course à pieds (trail nocturne de 15km), lecture.
\end{itemize}





\end{document}