\documentclass[10pt,a4paper]{report}
\usepackage[utf8]{inputenc}
\usepackage[francais]{babel}
\usepackage[T1]{fontenc}
\usepackage{amsmath}
\usepackage{amsfonts}
\usepackage{amssymb}
\usepackage{graphicx}
\usepackage[left=2cm,right=2cm,top=3cm,bottom=3.5cm]{geometry}
%\author{Brachet Matthieu}
%\title{Curriculum Vitae}
\begin{document}
\begin{center}
{\fontfamily{pnc}\selectfont
\begin{LARGE}
CURRICULUM VITAE
\end{LARGE}

\begin{large}
Matthieu Brachet
\end{large}

\hrulefill
}
\end{center}

\noindent
84, avenue du Général Leclerc\\
54 000 Nancy, France
\vspace{0.2cm}\\
Téléphone : +33 6 43 50 07 86\\
E-mail : brachetmat@free.fr\\
Page Web : \verb?www.iecl.univ-lorraine.fr/~Matthieu.Brachet/?
\vspace{0.2cm}\\
Nationalité : française\\
Date de naissance : 11 février 1991 (Lille, France)


%% ******************************************************************************
\vspace{1cm}
\noindent
{\fontfamily{pnc}\selectfont
\begin{Large}
Scolarité
\end{Large}
\hrulefill
}

\vspace{0.6cm}

\begin{center}
\begin{tabular}{r p{12cm}}
\textbf{oct. 2014 - ...} & \textbf{Doctorant} sous la direction de Prof. J.-P. Croisille à l'Université de Lorraine (Metz, France).\newline Titre : "\textit{Schémas compacts hermitiens sur la sphère - Applications en climatologie et océanographie numérique}".\\

& \\

\textbf{2012 - 2014} & \textbf{Master en Mathématiques Appliquées et Modélisation} à l'Univ. Picardie Jules Verne (Amiens, France). \\

& \textbf{Stage} (6 mois en 2nd année) sous la direction de Prof. J.-P. Chehab. \newline 
Titre : "\textit{Préconditionnement et résolution de problèmes d'évolutions non linéaires}". \\

& \textbf{Stage} sous la direction de V. Martin. \newline 
Titre : "\textit{Etude de quelques méthodes de Krylov}". \\

& \\

\textbf{2010 - 2012} & \textbf{Licence de Mathématiques} à l'Univ. Picardie Jules Verne (Amiens, France) \\

& \\

\textbf{2008 - 2010} & \textbf{Classes préparatoires} à Compiègne, France.\newline
MPSI et MP. \\

& \\

\textbf{2008} & \textbf{Baccalauréat} Scientifique option mathématiques au lycée Cassini (Clermont de l'Oise, France).\\
\end{tabular}
\end{center}

\vspace{1cm}
\noindent
{\fontfamily{pnc}\selectfont
\begin{Large}
Ecoles d'été
\end{Large}
\hrulefill
}

\vspace{0.6cm}

\begin{center}
\begin{tabular}{r p{12cm}}
\textbf{été 2015} & \textbf{CEMRACS'15 Coupling Multiphysics} au Centre International de Rencontres Mathématiques (Marseille, France) sous la direction de Pr E. Frenod, Pr. C. Prud'homme, A. Rousseau and Pr. S. Salmon.\newline
Titre : "\textit{Modélisation et résolution numérique d'un système couplant les équations de Stokes instationnaire et l'équation d'Exner (HydroMorpho)}".\\

& \\

\textbf{11/2014} & \textbf{Fortran 90 part 2} à l'Institut du Développement et des Ressources en Informatique Scientifique (Orsay, France) par P. Corde.\\

& \\

\textbf{09/2014} & \textbf{Fortran 90 part 1} à l'Institut du Développement et des Ressources en Informatique Scientifique (Orsay, France) par P. Corde.\\

& \\

\end{tabular}
\end{center}


%% ******************************************************************************
\vspace{1cm}
\noindent
{\fontfamily{pnc}\selectfont
\begin{Large}
Présentations
\end{Large}
\hrulefill
}


\vspace{0.4cm}
\noindent
{\fontfamily{pnc}\selectfont
\textbf{Séminaires :}
}

\begin{itemize}
\item \textbf{Séminaire d'Analyse Appliquée}, Amiens, France, 28 novembre 2016 (LAMFA),
\item \textbf{Séminaire Doctorants}, Strasbourg, France, 17 mars 2016,
\item \textbf{Séminaire Doctorants}, Rennes, France, 2 novembre 2015,
\item \textbf{Séminaire Doctorants}, Reims, France, 16 octobre 2015.
\end{itemize}

\vspace{0.4cm}
\noindent
{\fontfamily{pnc}\selectfont
\textbf{Conference :}
}

\begin{itemize}
\item \textbf{PDE on the Sphere}, Paris, France, 7 avril 2017,
\item \textbf{Journées EDP, IECL}, Metz, France, 30 mars 2017,
\item \textbf{Journée des Doctorants, IECL}, Nancy, France, 8 décembre 2016,
\item \textbf{CANUM}, Obernai, France, 11 mai 2016,
\item \textbf{PhD day Lebesgues}, Nantes, France, 28 octobre 2016.
\end{itemize}

\vspace{0.4cm}
\noindent
{\fontfamily{pnc}\selectfont
\textbf{Vulgarisation Scientifique :}
}
\begin{itemize}
\item \textbf{Séminaire étudiant du CESI }(Ecole d'ingénieur), Nancy, 6 décembre 2016,
\item \textbf{Remise de prix des olympiades de mathématiques}, Metz, 18 mai 2016,
\item \textbf{Ma thèse en 180 secondes}, Finale régionale, Nancy, 28 avril 2016.
\end{itemize}

%% ******************************************************************************
\vspace{1cm}
\noindent
{\fontfamily{pnc}\selectfont
\begin{Large}
Enseignements
\end{Large}
\hrulefill
}

\vspace{0.6cm}
\noindent
\begin{center}
\begin{tabular}{r p{12cm}}
\textbf{2017 - 2018} & \begin{itemize}
\item Cours/Travaux dirigés (Probabilités) en L1 Biologie (Nancy),
\item Travaux dirigés (Analyse numérique) en L3 Mathématiques (Nancy),
\item Travaux pratiques (Analyse numérique - Matlab) en première année d'ingénieur aux Mines (Nancy)
\item Travaux dirigés (Probabilités) en première année d'ingénieur à l'ENSEM (Nancy)
\end{itemize}\\

& \\

\textbf{2015 - 2017} & \begin{itemize}
\item Travaux dirigés (Mécanique des fluides numérique), M1 mécanique (Nancy),
\item Travaux pratiques (Analyse numérique, Matlab), L3 génie civile (Nancy),
\item Travaux pratiques (Introduction à Matlab), L2 physique (Nancy).
\end{itemize}\\

& \\

\textbf{2014 - 2015} & \begin{itemize}
\item Travaux dirigés (Mécanique des fluides numérique), M1 mécanique (Nancy),
\item Travaux pratiques (Analyse numérique en Matlab), L3 génie civile (Nancy),
\item Travaux pratiques (Introduction à Matlab), L2 physique (Nancy).
\item Travaux pratiques (Analyse numérique non linéaire, Matlab), L3 mathématiques (Metz),
\item Projets, L3 SPI Nancy,
\item Projets, 2eme année d'école d'ingénieur à Telecom Nancy.
\end{itemize}\\

& \\

\textbf{2011 - 2012} & Préparation mathématiques au concours vétérinaire, L3 Biologie, Amiens.\\

& \\
\end{tabular}
\end{center}

%% ******************************************************************************
%\vspace{1cm}
\newpage
\noindent
{\fontfamily{pnc}\selectfont
\begin{Large}
Publications
\end{Large}
\hrulefill
}

\vspace{0.6cm}

[1] {\sc  M. Brachet, J.-P. Croisille}, {\sl Numerical simulations of propagation problemes on the sphere using a compact scheme}, preprint.

\vspace{0.6cm}

[2] {\sc M. Brachet, J.-P. Chehab}, {\sl Stabilized Times Schemes for High Accurate Finite Differences Solutions of Nonlinear Parabolic Equations}, Journal of Scientific Computing, 69(3), 946-982, 2016.

\vspace{0.6cm}

[3] {\sc  N. Aissiouene, T. Amtout, M. Brachet, E. Frenod, R. Hild, C. Prud'homme, A. Rousseau, S. Salmon}, {\sl  Hydromorpho: A coupled model for unsteady Stokes/Exner equations and numerical results with FEEL++ library}, ESAIM: PROCEEDINGS AND SURVEYS, December 2016, Vol. 55, p. 23-40.






%% ******************************************************************************
\vspace{1cm}
\noindent
{\fontfamily{pnc}\selectfont
\begin{Large}
Expérience professionnelle
\end{Large}
\hrulefill
}

\vspace{0.6cm}
\noindent
\begin{center}
\begin{tabular}{p{5cm} p{10cm}}
\textbf{2010 - 2014} & Cours particuliers de mathématiques,\\

& \\

\textbf{été 2013 and été 2012} & Employé de rayons Intermarché (Fitz-James, France) \\

& \\

\textbf{été 2011 and été 2010} & Guichet au Crédit du Nord (Calais et Bresles, France) \\

& \\

\end{tabular}
\end{center}


\vspace{0.6cm}
\noindent
{\fontfamily{pnc}\selectfont
\begin{Large}
Autres
\end{Large}
\hrulefill
}

\vspace{1cm}
\noindent
\begin{itemize}
\item \textbf{Séminaire doctorants :} organisation du séminaire doctorants (2016 à 2018).
\item \textbf{Languages :} Anglais, Français (langue maternelle).
\item \textbf{Languages informatiques et Logiciels :} Matlab (expert), Fortran, FreeFem, FEEL++, Maple, \LaTeX, Pack Office, html/CSS.
\item \textbf{O.S. :} Linux, Windows, Mac.
\item \textbf{Permis de conduire :} obtenue en 2009.
\item \textbf{PESC1 :} (Formation aux premiers secours) obtenue en 2016.
\item \textbf{Loisirs :} guitare, course à pieds, lecture.

\end{itemize}





\end{document}