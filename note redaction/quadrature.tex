% quadrature.tex

\chapter{Produit Scalaire et quadrature sur le Cubed-Sphere}


\section{Produit scalaire sur la Cubed-Sphere}

\begin{itemize}
\item Harmoniques sphériques
\item Produit scalaire continu sur la sphère
\item Produit sclaire discret:
\begin{itemize}
\item Construction générale sur la CS
\item Propriété triangulaire et consèquences (en particulier pdf scalaire nul pour les HS)
\end{itemize}
\end{itemize}

\subsection{Harmoniques Sphériques sur la Sphère}

\subsubsection{Généralités sur les Harmoniques Sphériques}

\subsubsection{Construction d'une base d'Harmoniques Sphériques}

\subsection{Produit Scalaire discret sur la Cubed-Sphère}

\subsubsection{Généralités}

\subsubsection{Propriété triangulaire}













\section{Quadrature sur la Cubed-Sphere}





\subsection{Pré-requis} %% ***************************************************************************************

Si $f$ est une conction de $\mathcal{C}^{\infty}([a,b])$ alors la formule d'Euler MacLaurin s'écrit :

\begin{equation}
\dfrac{1}{b-a} \gint_a^b f(x)dx = \dfrac{f(a)+f(b)}{2} - \gsum_{j=1}^k (b-a)^{2j-1} \dfrac{b_{2j}}{(2j)!} \left( f^{(2j-1)}(b) - f^{(2j-1)}(a) \right) - \mathcal{O}\left( (b-a)^{(2k+2)} \right)
\label{eq:euler maclaurin}
\end{equation}

avec $b_{2j}$ les nombres de Bernoulli ($b_2=1/6$, $b_4=1/30$, ...).

Si on souhaite calculer $\gint_{-\pi/4}^{\pi/4} f(\xi,\eta)d\xi$ en utilisant la formule d'Euler MacLaurin \eqref{eq:euler maclaurin} sur les sous intervalles $[\xi_i, \xi_{i+1}]$ (formule d'Euler MacLaurin composite) avec $\xi_i= i \times \Delta \xi$, $\Delta \xi = \dfrac{\pi}{2(N+1)}$ on obtient :

\begin{equation}
\begin{array}{rl}
\gint_{-\pi/4}^{\pi/4} f(\xi,\eta)d\xi & = \dfrac{\Delta \xi}{2} f(\xi_{-\frac{N}{2}},\eta) +  \Delta \xi \gsum_{i=-\frac{N}{2}+1}^{\frac{N}{2}-1} f(\xi_i,\eta) + \frac{\Delta \xi}{2} f(\xi_{\frac{N}{2}},\eta)- ...\\
                                       & ... - \dfrac{b_2}{2!} \Delta \xi^2 \left( \partial_\xi f(\xi_{\frac{N}{2}},\eta) - \partial_\xi f(\xi_{-\frac{N}{2}},\eta) \right) + \mathcal{O} \left( \Delta \xi ^4 \right)
\end{array}
\end{equation}

On pose : $$I = \left\lbrace i\in\mathbb{N} \text{ tels que } -\frac{N}{2}+1 \leq i \leq \frac{N}{2}-1 \right\rbrace$$ et on applique la même méthode dans la direction de $\eta$  :

\begin{equation}
\begin{array}{rl}
\gint_{-\pi/4}^{\pi/4} \gint_{-\pi/4}^{\pi/4} f(\xi,\eta)d\xi d\eta = & \Delta \xi \Delta \eta \gsum_{i \in I} \gsum_{j \in I} f_{i,j} + ... \\
 & ... + \dfrac{\Delta \xi \Delta \eta}{2} \left[ \gsum_{i \in I} \left(  f_{i,\frac{N}{2}} + f_{i,-\frac{N}{2}})  \right) + \gsum_{j \in I} \left(  f_{\frac{N}{2},j} + f_{-\frac{N}{2},j}  \right) \right] - ... \\
 &  ... - \dfrac{b_2}{2!} \Delta \xi \Delta \eta \left[ \Delta \eta \gsum_{i \in I} \left(  \partial_{\eta }f_{i,\frac{N}{2}} -  \partial_{\eta }f_{i,-\frac{N}{2}}  \right) + \Delta \xi \gsum_{j \in I} \left(   \partial_{\xi }f_{\frac{N}{2},j} - \partial_{\xi }f_{-\frac{N}{2},j}  \right) \right] + ... \\
 & ... + \dfrac{\Delta \xi \Delta \eta}{4} \left[ f_{\frac{N}{2},\frac{N}{2}}+f_{-\frac{N}{2},\frac{N}{2}}+f_{\frac{N}{2},-\frac{N}{2}}+f_{-\frac{N}{2},-\frac{N}{2}} \right] + ... \\
 & ... - b_2 \dfrac{\Delta \xi \Delta \eta^2}{4} \left[ \partial_{\eta} f_{-\frac{N}{2},\frac{N}{2}} - \partial_{\eta} f_{-\frac{N}{2},-\frac{N}{2}} + \partial_{\eta} f_{\frac{N}{2},\frac{N}{2}} -\partial_{\eta} f_{\frac{N}{2},-\frac{N}{2}}  \right] - ... \\
  & ... - b_2 \dfrac{\Delta \xi^2 \Delta \eta}{4} \left[ \partial_{\xi} f_{\frac{N}{2},\frac{N}{2}} + \partial_{\xi} f_{\frac{N}{2},-\frac{N}{2}} - \partial_{\xi} f_{-\frac{N}{2},\frac{N}{2}} -\partial_{\xi} f_{-\frac{N}{2},-\frac{N}{2}}  \right] + ... \\
  & ... + b_2^2 \dfrac{\Delta \xi^2 \Delta \eta^2}{4} \left[ \partial_{\xi,\eta} f_{\frac{N}{2},\frac{N}{2}} - \partial_{\xi,\eta} f_{\frac{N}{2},-\frac{N}{2}} - \partial_{\xi,\eta} f_{-\frac{N}{2},\frac{N}{2}} + \partial_{\xi,\eta} f_{-\frac{N}{2},-\frac{N}{2}} \right] + ... \\
  & ... + \mathcal{O} \left( \Delta \xi^6, \Delta \eta^6 \right)
  \end{array}
\label{eq:euler maclaurin 2d composite}
\end{equation}

avec l'abus de notation $f_{i,j} = f(\xi_i, \eta_j)$ pour tous $i$ et $j$.

On note l'intégrale sur le panel $(K)$ (avec $K \in \left\lbrace I, II, III, IV, V, VI \right\rbrace$) de $h$ :
$$I^{(K)}(h) = \gint_{(K)} h(\mathbf{x}) d \sigma (\mathbf{x}).$$

Après changement de variable $\psi^{(K)} : \mathbf{x} \in \mathbb{S}_a^2 \mapsto (\xi, \eta) \in [-\pi/4, \pi/4]^2$, on obtient :

\begin{equation}
I^{(K)}(h) = \gint_{-\pi/4}^{\pi/4}\gint_{-\pi/4}^{\pi/4} h(\xi, \eta) \sqrt{\mathbf{\bar{G}}^{(K)}(\xi, \eta)} d \xi d \eta.
\end{equation}

La formule d'Euler MacLaurin \eqref{eq:euler maclaurin 2d composite} peut être appliquée sur chaque panel de la Cubed-Sphère en posant $f(\xi,\eta)=h(\xi,\eta)\sqrt{\mathbf{\bar{G}}^{(K)}(\xi, \eta)}$. On dispose alors d'une intègrale sur la sphère complète :

\begin{equation}
I(h)=\gint_{\mathbb{S}_a^2} h(\mathbf{x})d\sigma(\mathbf{x}) = \gsum_{K = I}^{VI} I^{(K)}(h)
\end{equation}





\subsection{Quadrature de type Trapèzes}  %% ***************************************************************************************

On utilise une formule de quadrature $Q_{tpz}(h)$ visant à approcher $I^{(K)}(h)$ en utilisant la formule des trapèzes composites :

\begin{equation}
\gint_{a}^b f(x)dx \approx \dfrac{b-a}{N} \left[ \dfrac{f(a)+f(b)}{2} + \gsum_{k=1}^{N-1} f\left( a+k \dfrac{b-a}{N} \right) \right]
\label{eq:trapezes composites}
\end{equation}

Il s'agit en fait de l'intégration de l'interpolation de $f$ par une fonction affine par morceaux. On note immédiatement que la formule des trapèzes est exacte par construction pour $f$ affine.

En appliquant \eqref{eq:trapezes composites} sur un carré, on obtient :

\begin{equation}
\begin{split}
\gint_{-\pi/4}^{\pi/4} \gint_{-\pi/4}^{\pi/4} f(\xi,\eta)d\xi d\eta \approx \Delta \xi \Delta \eta \gsum_{i \in I} \gsum_{j \in I} f_{i,j} +  \dfrac{\Delta \xi \Delta \eta}{2} \left[ \gsum_{i \in I} \left(  f_{i,\frac{N}{2}} + f_{i,-\frac{N}{2}}  \right) + \gsum_{j \in I} \left(  f_{\frac{N}{2},j} + f_{-\frac{N}{2},j}  \right) \right] + ... \\
...+ \dfrac{\Delta \xi \Delta \eta}{4} \left[ f_{\frac{N}{2},\frac{N}{2}}+f_{-\frac{N}{2},\frac{N}{2}}+f_{\frac{N}{2},-\frac{N}{2}}+f_{-\frac{N}{2},-\frac{N}{2}} \right]- b_2 \dfrac{\Delta \xi \Delta \eta^2}{4} \left[ \partial_{\eta} f_{-\frac{N}{2},\frac{N}{2}} - \partial_{\eta} f_{-\frac{N}{2},-\frac{N}{2}} \right]
\end{split}
\label{eq:trapezes 2d composite}
\end{equation}

La formule de quadrature sur le panel $(K)$ est alors la suivante :

\begin{equation}
Q_{tpz}^{(K)}(h)=\gsum_{i = -\frac{N}{2}}^{\frac{N}{2}}\gsum_{j = -\frac{N}{2}}^{\frac{N}{2}} \Delta \xi \Delta \eta \omega_{i,j} h(\xi_i, \eta_j) \sqrt{\mathbf{\bar{G}}^{(K)}(\xi_i, \eta_j)}
\label{eq:trapezes par panel}
\end{equation}

avec :

\begin{itemize}
\item $\omega_{-\frac{N}{2},-\frac{N}{2}}=\omega_{\frac{N}{2},-\frac{N}{2}}=\omega_{-\frac{N}{2},\frac{N}{2}}=\omega_{\frac{N}{2},\frac{N}{2}}=\frac{1}{4}$,
\item $\omega_{i,\frac{N}{2}}=\omega_{i,-\frac{N}{2}}=\frac{1}{2}$ pour $-\frac{N}{2}+1 \leq i \leq \frac{N}{2}-1$,
\item $\omega_{\frac{N}{2},j}=\omega_{-\frac{N}{2},j}=\frac{1}{2}$ pour $-\frac{N}{2}+1 \leq j \leq \frac{N}{2}-1$,
\item $\omega_{i,j}=1$ dans tous les autres cas.
\end{itemize}

\begin{proposition}
On suppose que $h$ est une fonction régulière sur la sphère. Alors :
\begin{equation}
I^{(K)}(h) - Q_{tpz}^{(K)}(h) = \mathcal{O} \left( \Delta \xi^2 \right)
\end{equation}
pour tout $K \in \lbrace I, II, III, IV, V, VI \rbrace$.
\label{prop:consistance tpz panel}
\end{proposition}

\begin{proof}
On remarque que par contruction sur la Cubed-Sphere, on a $\Delta \xi = \Delta \eta$.
On applique la formule d'Euler MacLaurin \eqref{eq:euler maclaurin 2d composite} à \eqref{eq:trapezes par panel}) avec :
\begin{equation}
h(\xi_i, \eta_j) \sqrt{\mathbf{\bar{G}}^{(K)}(\xi_i, \eta_j)} = f(\xi_i,\eta_j)
\end{equation}
pour tous $-\dfrac{N}{2} \leq i,j \leq \dfrac{N}{2}$ et le résultat est immédiatement obtenu.
\end{proof}

\begin{remarque}
La formule de quadrature $Q_{tpz}^{(K)}(h)$ n'est pas exacte pour $h$ affine.

En effet, $(\xi,\eta) \mapsto h(\xi, \eta) \sqrt{\mathbf{\bar{G}}^{(K)}(\xi, \eta)}$ n'est pas affine.
En revanche, lorsque $h(\xi,\eta)=\dfrac{1}{\sqrt{\mathbf{\bar{G}}^{(K)}(\xi, \eta)}}$, la formule de quadrature $Q_{tpz}^{(K)}(h)$ est exacte.
\end{remarque}

\begin{proposition}
On note :
\begin{equation}
Q_{tpz}=\gsum_{K=I}^{VI} Q_{tpz}^{(K)}
\end{equation}
Si $h$ est une fonction régulière sur la sphère alors :
\begin{equation}
I(h) - Q_{tpz}(h) = \mathcal{O} \left( \Delta \xi^2 \right)
\end{equation}
\label{prop:consistance tpz}
\end{proposition}

\begin{proof}
Consèquence directe de la proposition \ref{prop:consistance tpz panel}.
\end{proof}




\subsection{Quadrature de type Simpson}

La formule de quadrature de Simpson est de la forme :

\begin{equation}
\gint_a^b f(x)dx \approx \dfrac{b-a}{6} \left[ f(a)+f\left(\dfrac{a+b}{2}\right) + f(b) \right]
\end{equation}

elle est basée sur une intégration d'un polynôme d'ordre 2 interpolant $f$ en $a$, $\frac{a+b}{2}$ et $b$.
L'erreur effectuée est de la forme :

\begin{equation}
\gint_a^b f(x)dx - \dfrac{b-a}{6} \left[ f(a)+f\left(\dfrac{a+b}{2}\right) + f(b) \right] = - \dfrac{(b-a)^5}{2880}f^{(4)}\left( \chi \right) \text{ pour un certain } \chi \in [a,b],
\end{equation}

Ainsi, la formule de Simpson composite appliquée sur l'intervalle $\left[ - \frac{\pi}{4}, \frac{\pi}{4} \right]$ donne :

\begin{equation}
\gint_{-\frac{\pi}{4}}^{\frac{\pi}{4}} f(\xi, \eta) d\xi = \dfrac{\Delta \xi}{3} \left[ f\left( - \dfrac{\pi}{4}, \eta \right) + 2 \gsum_{i=-n/4-1}^{n/4} f\left( \xi_{2i}, \eta \right) + 4\gsum_{i=-n/4}^{n/4} f\left( \xi_{2i-1}, \eta \right) + f \left( \dfrac{\pi}{4}, \eta \right) \right] + \mathcal{O}\left( \Delta \xi^4 \right)
\end{equation}

et en 2 dimensions :

\begin{equation}
\gint_{-\frac{\pi}{4}}^{\frac{\pi}{4}} \gint_{-\frac{\pi}{4}}^{\frac{\pi}{4}} f(\xi, \eta) d\xi  d\eta= \Delta\xi \Delta \eta \gsum_{i=-N/2}^{N/2}\gsum_{j=-N/2}^{N/2} \omega_{i,j} f(\xi_i, \eta_j) + \mathcal{O}\left( \Delta \xi^4, \Delta \xi^4 \right)
\label{eq:simpson 2d}
\end{equation}

où les coefficients $\omega_{i,j}$ sont donnés par :
\begin{itemize}
\item $\omega_{\frac{N}{2},\frac{N}{2}}=\omega_{\frac{N}{2},-\frac{N}{2}}=\omega_{-\frac{N}{2},\frac{N}{2}}=\omega_{-\frac{N}{2},-\frac{N}{2}}=1/9$,
\item $\omega_{\frac{N}{2},i}=\omega_{-\frac{N}{2},i}=\omega_{i,\frac{N}{2}}=\omega_{i,-\frac{N}{2}}=4/9$ si $i$ est pair,
\item $\omega_{\frac{N}{2},i}=\omega_{-\frac{N}{2},i}=\omega_{i,\frac{N}{2}}=\omega_{i,-\frac{N}{2}}=2/9$ si $i$ est impair,
\item $\omega_{i,j}=16/9$ si $i$ et $j$ sont pairs,
\item $\omega_{i,j}=4/9$ si $i$ et $j$ sont impairs,
\item $\omega_{i,j}=8/9$ dans les autres cas.
\end{itemize}


Ainsi, la formule de quadrature par panel issue de la méthode de Simpson est donnée par :

\begin{equation}
Q_{sps}^{(K)}(h)=\gsum_{i = -\frac{N}{2}}^{\frac{N}{2}}\gsum_{j = -\frac{N}{2}}^{\frac{N}{2}} \Delta \xi \Delta \eta \omega_{i,j} h(\xi_i, \eta_j) \sqrt{\mathbf{\bar{G}}^{(K)}(\xi_i, \eta_j)}
\label{eq:simpson par panel}
\end{equation}

avec les coefficients donnés dans la remarque précédente.

\begin{proposition}
Si $h$ est une fonction suffisament régulière sur la sphère et $\Delta \xi = \Delta \eta$ alors :
\begin{equation}
I(h) - Q_{sps}(h) = \mathcal{O} \left( \Delta \xi^4 \right)
\end{equation}
pour tout $K \in \lbrace I, II, III, IV, V, VI \rbrace$.
\label{prop:consistance sps panel}
\end{proposition}

\begin{proof}
Consèquence directe de la construction.
\end{proof}

\begin{remarque}
\begin{itemize}
\item Comme pour la quadrature de type trapèzes $Q_{tpz}^{(K)}$, la formule de quadrature est exacte lorsque pour tous $(\xi,\eta)$, on a :
\begin{equation}
h(\xi,\eta)=\dfrac{1}{\sqrt{\overline{\mathbf{G}}(\xi,\eta)}}
\end{equation}
\item Par construction, il est nécéssaire d'avoir $N$ pair pour que la méthode soit d'ordre 4. Si $N$ est impair, une erreur s'ajoute est la méthode est d'ordre 1.
\end{itemize}
\end{remarque}

\begin{proposition}
La formule de quadrature sur la Cubed-Sphère est d'ordre 4 par recouvrement :
\begin{equation}
Q_{sps(h)}=\gsum_{K=I}^{VI} Q_{sps(h)}^{(K)}(h) = I(h) + \mathcal{O} \left( \Delta \xi^4 \right)
\end{equation}
si $h$ est suffisament régulière.
\end{proposition}

\begin{proof}
Consèquence de la proposition \ref{prop:consistance sps panel}.
\end{proof}

