% opsph.tex
% opérateurs sphériques sur la sphère

\chapter{Opérateurs sphériques}


\section{Gradient sphèrique}

Soit $h : \mathbb{S}^2_a \rightarrow \mathbb{R}$ une fonction sur la sphère à valeurs réelles. On peut calculer le gradient de cette fonction grâce à la formule :

\begin{equation}
grad(h) = \nabla h = \mathbf{g}^{\xi} \dfrac{\partial h}{\partial \xi} + \mathbf{g}^{\eta} \dfrac{\partial h}{\partial \eta}
\label{eq:gradient}
\end{equation}

avec $(\mathbf{g}^{\xi}, \mathbf{g}^{\eta})$ la base duale construite précédemment.

\begin{remarque}
Le gradient de $h$ est, par construction, un champ de vecteur de $\mathbb{S}^2_a$ dans $\mathbb{T}\mathbb{S}^2_a$ :

\begin{equation}
\nabla h : \mathbb{S}^2_a \rightarrow \mathbb{T}\mathbb{S}^2_a
\end{equation}
\end{remarque}


\section{Rotationnel et vorticité sur la sphère}

De manière similaire aux constructions précédenntes, si $\mathbf{u} : \mathbb{S}^2_a \rightarrow \mathbb{T} \mathbb{S}^2_a$ est une fonction à valeur dans le champ de vecteurs tangents à la sphère, le rotationnel de $\mathbf{u}$ est donné par :

\begin{equation}
rot( \mathbf{u} ) = \nabla \wedge \mathbf{u} = \mathbf{g}^{\xi} \wedge \dfrac{\partial \mathbf{u}}{\partial \xi} + \mathbf{g}^{\eta} \wedge \dfrac{\partial \mathbf{u}}{\partial \eta}
\label{eq:rotationnel}
\end{equation}

On note que $rot( \mathbf{u} ) : \mathbb{S}^2_a \rightarrow \left[ \mathbf{T} \mathbf{S}^2_a \right]^{\bot}$. En effet ... 

ADD PROOF!!!


\section{Divergence sur la sphère}

On peut calculer la divergence de $\mathbf{u} : \mathbb{S}^2_a \rightarrow \mathbb{T} \mathbb{S}^2_a$ grâce à la formule :

\begin{equation}
div( \mathbf{u} ) = \nabla \cdot \mathbf{u} = \mathbf{g}^{\xi} \cdot \dfrac{\partial \mathbf{u}}{\partial \xi} + \mathbf{g}^{\eta} \cdot \dfrac{\partial \mathbf{u}}{\partial \eta}
\label{eq:divergence_old}
\end{equation}

Cette formule nécessite de calculer la dérivée en $\xi$ et en $\eta$ de chaque composante de $\mathbf{u}$. Il est possible de diminuer le coût en calcul.

METTRE LA PREUVE

\begin{equation}
div( \mathbf{u} ) = \nabla \cdot \mathbf{u} = \dfrac{1}{\sqrt{\overline{\mathbf{G}}}} \left[ \dfrac{\partial}{\partial \xi} \left( \sqrt{\overline{\mathbf{G}}} \mathbf{u} \cdot \mathbf{g}^{\xi} \right) + \dfrac{\partial}{\partial \eta} \left( \sqrt{\overline{\mathbf{G}}} \mathbf{u} \cdot \mathbf{g}^{\xi} \right) \right]
\label{eq:divergence}
\end{equation}
