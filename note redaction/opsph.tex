% opsph.tex
% opérateurs sphériques sur la sphère

\chapter{Opérateurs sphériques}

\section{Opérateur sphériques discrets}

La résolution d'équations aux dérivées parties de type Shallow Water nécessite de passer par le calcul des opérateurs classiques : gradient, divergence et rotationnel.

Si $\mathbf{u} : \mathbb{S}_a^2 \mapsto \mathbb{T}\mathbb{S}_a^2$ est un champ de vecteur régulier sur la sphère et $h : \mathbb{S}_a^2 \mapsto\mathbb{R}$ une fonction régulière sur la sphère.
Chaque opérateur peut être écrit en coordonnées $(\xi, \eta)$ sur la sphère grâce aux formules suivantes :

\begin{itemize}
\item \textbf{Gradient :}
\begin{equation}
\nabla h = \mathbf{g}^{\xi} \dfrac{\partial h}{\partial \xi}_{|\eta} + \mathbf{g}^{\eta} \dfrac{\partial h}{\partial \eta}_{|\xi} 
\label{eq: gradient}
\end{equation}
\item \textbf{Divergence :}
\begin{equation}
\begin{array}{rcl}
\nabla \cdot \mathbf{u} & = & \mathbf{g}^{\xi} \cdot \dfrac{\partial \mathbf{u}}{\partial \xi}_{|\eta} + \mathbf{g}^{\eta} \cdot \dfrac{\partial \mathbf{u}}{\partial \eta}_{|\xi} \\
	& = & \dfrac{1}{\sqrt{\bar{\mathbf{G}}}} \left( \dfrac{\partial}{\partial \xi} \left( \sqrt{\bar{\mathbf{G}}} \mathbf{u} \cdot \mathbf{g}^{\xi} \right)_{|\eta} + \dfrac{\partial}{\partial \eta} \left( \sqrt{\bar{\mathbf{G}}} \mathbf{u} \cdot \mathbf{g}^{\eta} \right)_{|\xi} \right) 
\end{array}
\label{eq: divergence}
\end{equation}
\item \textbf{Rotationnel :}
\begin{equation}
\nabla \wedge \mathbf{u} = \mathbf{g}^{\xi} \wedge \dfrac{\partial \mathbf{u}}{\partial \xi}_{|\eta} + \mathbf{g}^{\eta} \wedge \dfrac{\partial \mathbf{u}}{\partial \eta}_{|\xi}  
\label{eq: rotationnel}
\end{equation}
\end{itemize}


\section{Produit scalaire sur la Cubed-Sphere}

\begin{itemize}
\item Harmoniques sphériques
\item Produit scalaire continu sur la sphère
\item Produit sclaire discret:
\begin{itemize}
\item Construction générale sur la CS
\item Propriété triangulaire et consèquences (en particulier pdf scalaire nul pour les HS)
\end{itemize}
\end{itemize}

\subsection{Harmoniques Sphériques sur la Sphère}

\subsubsection{Généralités sur les Harmoniques Sphériques}

\subsubsection{Construction d'une base d'Harmoniques Sphériques}

\subsection{Produit Scalaire discret sur la Cubed-Sphère}

\subsubsection{Généralités}

\subsubsection{Propriété triangulaire}













\section{Quadrature sur la Cubed-Sphere}

La quadrature numérique sur la sphère permet d'avoir une approximation numérique de l'intégrale. 






\subsection{Quadrature naturelle}

Si $f$ est une conction de $\mathcal{C}^{\infty}([a,b])$ alors la formule d'Euler MacLaurin s'écrit ainsi :

\begin{equation}
\dfrac{1}{b-a} \gint_a^b f(x)dx = \dfrac{f(a)+f(b)}{2} - \gsum_{j=1}^k (b-a)^{2j-1} \dfrac{b_{2j}}{(2j)!} \left( f^{(2j-1)}(b) - f^{(2j-1)}(a) \right) - \mathcal{O}\left( (b-a)^{(2k+2)} \right)
\label{eq:euler maclaurin}
\end{equation}

avec $b_{2j}$ les nombres de Bernoulli ($b_2=1/6$, $b_4=1/30$, ...).

Si on souhaite calculer $\gint_{-\pi/4}^{\pi/4} f(\xi,\eta)d\xi$ en utilisant la formule d'Euler MacLaurin \eqref{eq:euler maclaurin} sur les sous intervalles $[\xi_i, \xi_{i+1}]$ avec $\xi_i= i \times \Delta \xi$, $\Delta \xi = \dfrac{\pi}{2(N+1)}$ on obtient :

\begin{equation}
\begin{array}{rl}
\gint_{-\pi/4}^{\pi/4} f(\xi,\eta)d\xi & = \dfrac{\Delta \xi}{2} f(\xi_{-\frac{N}{2}},\eta) +  \Delta \xi \gsum_{i=-\frac{N}{2}+1}^{\frac{N}{2}-1} f(\xi_i,\eta) + \frac{\Delta \xi}{2} f(\xi_{\frac{N}{2}},\eta)- ...\\
                                       & ... - \dfrac{b_2}{2!} \Delta \xi^2 \left( \partial_\xi f(\xi_{\frac{N}{2}},\eta) - \partial_\xi f(\xi_{-\frac{N}{2}},\eta) \right) + \mathcal{O} \left( \Delta \xi ^4 \right)
\end{array}
\end{equation}

On pose : $$I = \left\lbrace i\in\mathbb{N} \text{ tels que } -\frac{N}{2}+1 \leq i \leq \frac{N}{2}-1 \right\rbrace$$ et on applique la même méthode dans la direction de $\eta$  :

\begin{equation}
\begin{array}{rl}
\gint_{-\pi/4}^{\pi/4} \gint_{-\pi/4}^{\pi/4} f(\xi,\eta)d\xi d\eta = & \Delta \xi \Delta \eta \gsum_{i \in I} \gsum_{j \in I} f_{i,j} + ... \\
 & ... + \dfrac{\Delta \xi \Delta \eta}{2} \left[ \gsum_{i \in I} \left(  f_{i,\frac{N}{2}} + f_{i,-\frac{N}{2}})  \right) + \gsum_{j \in I} \left(  f_{\frac{N}{2},j} + f_{-\frac{N}{2},j}  \right) \right] - ... \\
 &  ... - \dfrac{b_2}{2!} \Delta \xi \Delta \eta \left[ \Delta \eta \gsum_{i \in I} \left(  \partial_{\eta }f_{i,\frac{N}{2}} -  \partial_{\eta }f_{i,-\frac{N}{2}}  \right) + \Delta \xi \gsum_{j \in I} \left(   \partial_{\xi }f_{\frac{N}{2},j} - \partial_{\xi }f_{-\frac{N}{2},j}  \right) \right] + ... \\
 & ... + \dfrac{\Delta \xi \Delta \eta}{4} \left[ f_{\frac{N}{2},\frac{N}{2}}+f_{-\frac{N}{2},\frac{N}{2}}+f_{\frac{N}{2},-\frac{N}{2}}+f_{-\frac{N}{2},-\frac{N}{2}} \right] + ... \\
 & ... - b_2 \dfrac{\Delta \xi \Delta \eta^2}{4} \left[ \partial_{\eta} f_{-\frac{N}{2},\frac{N}{2}} - \partial_{\eta} f_{-\frac{N}{2},-\frac{N}{2}} + \partial_{\eta} f_{\frac{N}{2},\frac{N}{2}} -\partial_{\eta} f_{\frac{N}{2},-\frac{N}{2}}  \right] - ... \\
  & ... - b_2 \dfrac{\Delta \xi^2 \Delta \eta}{4} \left[ \partial_{\xi} f_{\frac{N}{2},\frac{N}{2}} + \partial_{\xi} f_{\frac{N}{2},-\frac{N}{2}} - \partial_{\xi} f_{-\frac{N}{2},\frac{N}{2}} -\partial_{\xi} f_{-\frac{N}{2},-\frac{N}{2}}  \right] + ... \\
  & ... + b_2^2 \dfrac{\Delta \xi^2 \Delta \eta^2}{4} \left[ \partial_{\xi,\eta} f_{\frac{N}{2},\frac{N}{2}} - \partial_{\xi,\eta} f_{\frac{N}{2},-\frac{N}{2}} - \partial_{\xi,\eta} f_{-\frac{N}{2},\frac{N}{2}} + \partial_{\xi,\eta} f_{-\frac{N}{2},-\frac{N}{2}} \right] + ... \\
  & ... + \mathcal{O} \left( \Delta \xi^6, \Delta \eta^6 \right)
  \end{array}
\label{eq:euler maclaurin 2d composite}
\end{equation}

avec l'abus de notation $f_{i,j} = f(\xi_i, \eta_j)$ pour tous $i$ et $j$.

On note l'intégrale sur le panel $(K)$ (avec $K \in \left\lbrace I, II, III, IV, V, VI \right\rbrace$) de $h$ :
$$I^{(K)}(h) = \gint_{(K)} h(\mathbf{x}) d \sigma (\mathbf{x}).$$

Après changement de variable $\mathbf{x} \in \mathbb{S}_a^2 \mapsto (\xi, \eta) \in [-\pi/4, \pi/4]^2$, on obtient :

\begin{equation}
I^{(K)}(h) = \gint_{-\pi/4}^{\pi/4}\gint_{-\pi/4}^{\pi/4} h(\xi, \eta) \sqrt{\mathbf{\bar{G}}(\xi, \eta)} d \xi d \eta.
\end{equation}

La formule d'Euler MacLaurin \eqref{eq:euler maclaurin 2d composite} peut être appliquée sur chaque panel de la Cubed-Sphère et donner une formule de quadrature naturelle à l'ordre 2.

\begin{remarque}
Le changement de variable $\mathbf{x} \in \mathbb{S}_a^2 \mapsto (\xi, \eta) \in [-\pi/4, \pi/4]^2$ est propre au panel sur lequel on se situe. $\xi$ au sens du panel $(I)$ est différent de $\xi$ au sens du panel $(II)$ par exemple.
\end{remarque}

$\gint_{\mathbb{S}_a^2}h(\mathbf{x}) d \sigma (\mathbf{x}) = \gsum_{K = I}^{VI} I^{(K)}(h)$. On cherche une formule de quadrature $Q(h) = \gsum_{K = I}^{VI} Q^{(K)}(h) $ où $Q^{(K)}(h) \approx I^{(K)}(h)$ et $Q^{(K)}(h) = \gsum_{-N/2 \leq i \leq N/2} \gsum_{-N/2 \leq i \leq N/2} \omega_{i,j} h_{i,j} \sqrt{\bar{\mathbf{G}_{i,j}}}$.

Les poids $\omega_{i,j}$ sont donnés par :
\begin{itemize}
\item $\omega_{N/2,N/2}=\omega_{-N/2,N/2}=\omega_{N/2,-N/2}=\omega_{-N/2,-N/2}=1/4$,
\item $\omega_{i,N/2}=\omega_{i,-N/2}=\omega_{N/2,i}=\omega_{-N/2,i}=1/2$ pour tout $i \in I$,
\item $\omega_{i,j} = 1$ pour tous $(i,j) \in I^2$.
\end{itemize}

On notera $Q_T$ la formule de quadrature associée à ces poids.

\begin{theoreme}
Si $h : \mathbf{S}_a^2 \mapsto \mathbb{R}$ est une fonction suffisament régulière de la sphère alors :
\begin{equation}
I(h)-Q_T(h)=
\end{equation}
\end{theoreme}























\subsection{Quadrature de type Simpson}

Au sens continu, le théorème de Stokes est vérifiée sur la surface de la sphère:

\begin{theoreme}
\textbf{Théorème de Stokes}
Si $\Omega$ est une surface sur la sphère $\mathbb{S}_a^2$ (i.e. $\Omega \subset \mathbb{S}_a^2$) régulier. Alors :
\begin{equation}
\gint_\Omega \nabla \cdot \mathbf{u} d \sigma (\mathbf{x}) = \gint_{\partial \Omega} \mathbf{u} \cdot \mathbf{n} dl(\mathbf{x})
\end{equation}
où $n$ est la normale extérieure unitaire pour le produit scalaire associé à la métrique $\mathbf{G}$.
\label{th: Stokes}
\end{theoreme}

Ce théorème permet de vérifier la conservation de la matière sur la Sphère :

\begin{equation}
\gint_{\mathbb{S}_a^2}\nabla \cdot \mathbf{u} d \sigma (\mathbf{x}) = 0
\label{eq: Stokes global}
\end{equation}

De plus, via le changement de variable $\mathbf{x} \in \Omega \subset \mathbb{S}_a^2 \mapsto (\xi, \eta) \in [-a a] \subset [- \pi/4, \pi/4]^2 = \hat{\Omega}$ on a, si $h$ est une fonction de $\Omega$ dans $\mathbb{R}$:
\begin{equation}
\gint_{\Omega} h(\mathbf{x}) d \sigma ( \mathbf{x} ) = \gint_{-a}^a \gint_{-a}^a h(\xi, \eta) \sqrt{\bar{\mathbf{G}}}d\xi d\eta
\end{equation}

La formule de Simpson appliquée dans chaque direction donne l'approximation suivante :

\begin{equation}
\begin{array}{rcl}
\gint_{-a}^a \gint_{-a}^a f(\xi, \eta)d\xi d\eta & \approx & a^2 \left[ \dfrac{1}{9} \lbrace  f(-a,-a) + f(a,a) + f(a,-a) + f(-a,a) \rbrace \right. + ...\\
& & ...+ \dfrac{4}{9} \lbrace  f(0,-a) + f(0,a) + f(a,0) + f(-a,0) \rbrace +  \left. \dfrac{16}{9} f(0,0) \right]
\end{array}
\end{equation}

ainsi si $f(\xi, \eta) \approx \nabla \cdot \mathbf{u} \sqrt{\bar{\mathbf{G}}}$ on a une méthode permettant d'approché le terme de gauche dans le théorème de Stokes \ref{th: Stokes}

Pour le terme de droite de \ref{th: Stokes}, on a :

\begin{equation}
\begin{array}{rcl}
\gint_{\partial \Omega} \mathbf{u} \cdot \mathbf{n} dl(\mathbf{x}) & = & a \left[ \gint_{-a}^a \mathbf{u}(a,\eta) \cdot \mathbf{g}^{\xi} (a, \eta) \sqrt{\bar{\mathbf{G}(a,\eta)}}- \mathbf{u}(-a,\eta) \cdot \mathbf{g}^{\xi} (-a, \eta) \sqrt{\bar{\mathbf{G}(-a,\eta)}} d\eta \right]+ ...\\
&  & ... + a \left[ \gint_{-a}^a \mathbf{u}(\xi,a) \cdot \mathbf{g}^{\xi} (\xi a) \sqrt{\bar{\mathbf{G}(\xi,a)}}- \mathbf{u}(\xi,-a) \cdot \mathbf{g}^{\xi} (\xi, -a) \sqrt{\bar{\mathbf{G}(\xi,-a)}}\right]
\end{array}
\label{eq: quadrature simpson L}
\end{equation}

A ce niveau, il n'y a pas eut d'approximation pour le terme de droite. On utilise la règle de Simpson pour approcher les intégrales. On pose pour tout $(\xi, \eta) \in \lbrace -a, 0, a \rbrace^2$

\begin{itemize}
\item $\psi_1(\xi, \eta) = \mathbf{u}(\xi, \eta) \cdot \mathbf{g}^{\xi}(\xi, \eta) \sqrt{\bar{\mathbf{G}(\xi, \eta)}}$,
\item $\psi_2(\xi, \eta) = \mathbf{u}(\xi, \eta) \cdot \mathbf{g}^{\eta}(\xi, \eta) \sqrt{\bar{\mathbf{G}(\xi, \eta)}}$,
\end{itemize}

De plus, $\psi_{1,\xi}$ et $\xi_{2,\eta}$ sont donnés en chaque point par un schéma compact d'ordre 4:

\begin{itemize}
\item $\dfrac{1}{6}\psi_{1,\xi}(a, \eta)+\dfrac{2}{3}\psi_{1,\xi}(0, \eta)+\dfrac{1}{6}\psi_{1,\xi}(-a, \eta) = \dfrac{\psi_1(a,\eta)-\psi_1(-a,\eta)}{2 \Delta \xi}$,
\item $\dfrac{1}{6}\psi_{2,\eta}(\xi,a)+\dfrac{2}{3}\psi_{2,\eta}(\xi,0)+\dfrac{1}{6}\psi_{2,\eta}(\xi,-a) = \dfrac{\psi_2(\xi,a)-\psi_2(\xi,-a)}{2 \Delta \eta}$,
\end{itemize}

D'où :

\begin{equation}
\begin{array}{rcl}
\gint_{\partial \Omega} \mathbf{u} \cdot \mathbf{n} dl(\mathbf{x}) & \approx &  a^2 \left[ \dfrac{1}{9} \lbrace  \psi_{1,\xi}(-a,-a)+\psi_{2,\eta}(-a,-a)+\psi_{1,\xi}(a,a)+\psi_{2,\eta}(a,a) \right. \\
& & \left. +\psi_{1,\xi}(a,-a)+\psi_{2,\eta}(a,-a)+\psi_{1,\xi}(-a,a)+\psi_{2,\eta}(-a,a)  \rbrace \right. + ...\\
& & ...+ \dfrac{4}{9} \lbrace  \psi_{1,\xi}(-a,0)+\psi_{2,\eta}(-a,0)+\psi_{1,\xi}(a,0)+\psi_{2,\eta}(a,0) \\
& & \left. +\psi_{1,\xi}(0,-a)+\psi_{2,\eta}(0,-a)+\psi_{1,\xi}(0,a)+\psi_{2,\eta}(0,a)   \rbrace + \dfrac{16}{9} \lbrace \psi_{1,\xi}(0,0)+\psi_{2,\eta}(0,0) \rbrace \right]
\end{array}
\label{eq: quadrature simpson R}
\end{equation}

Dès lors, le théorème de Stokes discret, i.e \eqref{eq: quadrature simpson L}=\eqref{eq: quadrature simpson R} si et seulement si :

\begin{equation}
f(\xi,\eta) = \psi_{1,\xi}(\xi,\eta)+\psi_{2,\eta}(\xi,\eta)
\end{equation}

ce qui revient à approcher la divergence par :

\begin{equation}
\nabla_{\Delta} \cdot \mathbf{u} =\dfrac{1}{\sqrt{\bar{\mathbf{G}}}} \left[ \delta_{\xi,4} \left( \sqrt{\bar{\mathbf{G}}} \mathbf{u} \cdot \mathbf{g}^{\xi} \right) +\delta_{\eta,4} \left( \sqrt{\bar{\mathbf{G}}} \mathbf{u} \cdot \mathbf{g}^{\eta} \right)  \right]
\label{eq:approx divergence1}
\end{equation}

où $\delta_{\xi,4}$ et $\delta_{\eta,4}$ sont les opérateurs de schémas compacts d'ordre 4.

De plus, la méthode de Simpson étant d'ordre 4, d'où le théorème suivant :

\begin{theoreme}
La formule de quadrature sur la sphère :
\begin{equation}
\gint_{\mathbb{S}_a^2} h(\mathbf{x}) d\sigma(\mathbf{x}) \approx I_s(h)=\gsum_{-\frac{N}{2} \leq i,j \leq \frac{N}{2}} \gsum_{k=(I)}^{(VI)} \omega_{i,j} h(x_{i,j})\sqrt{\bar{\mathbf{G}}(x_{i,j}^k}
\label{eq:quadrature simpson}
\end{equation}
est d'ordre 4, c'est à dire :
\begin{equation}
I(h)- \gint_{\mathbb{S}_a^2} h(\mathbf{x}) d\sigma(\mathbf{x}) = \mathcal{O}\left( \Delta \xi^4,\Delta \eta^4\right)
\end{equation}
De plus, si $h$ est l'approximation de la divergence \eqref{eq:approx divergence1}, alors la formule de Stokes globale \eqref{eq: Stokes global} est vérifiée au sens discret :
\begin{equation}
I_s(\nabla_{\Delta} \cdot \mathbf{u}) = 0
\label{eq: Stokes global I_s}
\end{equation}
\end{theoreme}

\begin{remarque}
Les coefficients $\omega_{i,j}$ sont donnés par la règme suivante :
\begin{itemize}
\item $\omega_{1,1}=\omega_{1,N+1}=\omega_{N+1,1}=\omega_{N+1,N+1}=1/9$,
\item $\omega_{1,i}=\omega_{N+1,i}=\omega_{i,1}=\omega_{i,N+1}=4/9$ si $i$ est pair,
\item $\omega_{1,i}=\omega_{N+1,i}=\omega_{i,1}=\omega_{i,N+1}=2/9$ si $i$ est impair,
\item $\omega_{i,j}=16/9$ si $i$ et $j$ sont pairs,
\item $\omega_{i,j}=4/9$ si $i$ et $j$ sont impairs,
\item $\omega_{i,j}=8/9$ dans les autres cas.
\end{itemize}
\end{remarque}

\begin{proof}
La formule $I_s$ est d'ordre 4 par construction car la méthode de Simpson est d'ordre 4.
La formule \eqref{eq: Stokes global I_s} est vérifiée en sommant \eqref{eq: quadrature simpson L} pour recouvrir la sphère et en utilisant \eqref{eq: quadrature simpson L}=\eqref{eq: quadrature simpson R}.
\end{proof}



\section{Filtrage sur la sphère}