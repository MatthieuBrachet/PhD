%% SWE.tex

\chapter{Modèle mathématique}

\section{Système d'équation Shallow Water}

\subsection{Equations de Navier-Stokes}

\subsection{Modèle a-dimension}

\subsection{Modèle Shallow Water}

\section{Etude théorique}

\subsection{Equation d'advection}

Dans cette partie, on considère l'équation d'advection suivante :

\begin{equation}
\left\lbrace
\begin{array}{rcl}
\dfrac{\partial h}{\partial t} + \mathbf{c}_T \cdot \nabla_T h & = & 0 \\
h(t=0,\mathbf{x}) & = & h_0(\mathbf{x})
\end{array}
\right. t>0 \text{ et } \mathbf{x} \in \mathbb{S}_a^2
\label{eq:advection}
\end{equation}

$\mathbf{c}:\mathbb{R}^+ \times \mathbb{S}_a^2 \mapsto \mathbb{T}\mathbb{S}_a^2$ est un champ de vecteur donné tangent à la sphère et $h_0$ est donnée.

Dans cette partie, on souhaite montrer l'existence et l'unicité d'une solution de \eqref{eq:advection}.

\begin{proposition}
L'équation :
\begin{equation}
\left\lbrace
\begin{array}{rcl}
\dfrac{\partial h}{\partial t} + \mathbf{c} \cdot \nabla h & = & 0 \\
h(t=0,\mathbf{x}) & = & h_0(\mathbf{x})
\end{array}
\right. t>0 \text{ et } \mathbf{x} \in \mathbb{R}^3
\end{equation}
admet une unique solution.
POUR TOUT $t>0$? REGULARITE DE $h_0$ ?
\label{prop:existence et unicité advection R^3}
\end{proposition}

\begin{proof}
TBA
\end{proof}

\begin{proposition}
L'équation \eqref{eq:advection} admet une unique solution $h$. De plus cette solution $h$ vérifie :
\begin{equation}
\|h\|_{L^2} \leq \|h(t=0,\cdot) \|_{L^2} \exp \left[ \dfrac{1}{2} \gint_0^t \| \nabla_T \cdot \mathbf{c}_T \|_{\infty}  \right]
\end{equation}
\end{proposition}

\begin{proof}
\begin{itemize}
\item \textbf{Unicité :} Soient $h_1$ et $h_2$ deux solutions distinctes de \eqref{eq:advection}. Alors $h=h_1-h_2$ est solution de :

$$
\left\lbrace
\begin{array}{rcl}
\dfrac{\partial h}{\partial t} + \mathbf{c}_T \cdot \nabla_T h & = & 0 \\
h(t=0,\mathbf{x}) & = & 0
\end{array}
\right. t>0 \text{ et } \mathbf{x} \in \mathbb{S}_a^2
$$
Montrons que $h \equiv 0$ en tout temps $t$.

D'après l'équation précédente, en multipliant par $h$ et en intégrant sur $\mathbb{S}_a^2$ :

$$\dfrac{1}{2} \dfrac{d}{dt} \gint_{\mathbb{S}_a^2} h(\mathbf{x}) d \sigma(\mathbf{x}) = - \gint_{\mathbb{S}_a^2} h(\mathbf{x}) \mathbf{c}_T(t,\mathbf{x}) \cdot \nabla_T h(\mathbf{x}) d \sigma (\mathbf{x})$$

d'où :

$$
\begin{array}{rcl}
\dfrac{1}{2} \dfrac{d}{dt} \| h \|^2_{L^2} & = & - \dfrac{1}{2} \gint_{\mathbb{S}_a^2} \mathbf{c}_T(t,\mathbf{x}) \cdot \nabla_T h(t,\mathbf{x})^2 d \sigma(\mathbf{x})\\
& = & \dfrac{1}{2} \gint_{\mathbb{S}_a^2} h(t,\mathbf{x})^2 \nabla_T \cdot \mathbf{c}_T (t, \mathbf{x}) d \sigma(\mathbf{x}) \\
& \leq & \dfrac{1}{2} \| h \|^2_{L^2} \| \nabla_T \cdot \mathbf{c}_T \|_{\infty}
\end{array}
$$

Par le lemme de Gronwall, on en déduit :

$$
\|h\|^2_{L^2} \leq \|h(t=0,\cdot) \|^2_{L^2} \exp \left[ \gint_0^t \| \nabla_T \cdot \mathbf{c}_T \|_{\infty}  \right] = 0
$$

dont $\|h\|_{L^2} = 0$ d'où $h \equiv 0$.

\item \textbf{Existence :}

On pose :

\begin{equation}
\psi(r) = 
\left\lbrace
\begin{array}{rcl}
C \exp \left[ \dfrac{1}{(r-a/2)(r-3a/2)} \right] \text{ si } a/2 \leq r \leq 3a/2 \\
0 \text{ sinon. }
\end{array}
\right.
\end{equation}

la constante $C$ est telle que $\psi(a)=1$ et $\psi \in \mathcal{C}^{\infty}_c$.
Ainsi on peut prolonger $\mathbf{c}_T$ à $\mathbf{R}^3$ grâce au changement suivant (en coordonnées sphériques, voir annexe) :

\begin{equation}
\mathbf{c}(\lambda, \theta, r) = \mathbf{c}_T(\lambda, \theta) \psi(r)
\end{equation}

La fonction $\mathbf{c}$ est régulière et bornée par construction.
 $h_0$ est prolongée dans $\mathbb{R}^3$ en la remplaçant de même par $(\lambda, \theta, r) \mapsto h_0(\lambda, \theta) \psi(r)$. D'après la proposition \ref{prop:existence et unicité advection R^3} il existe une unique solution au problème sur $\mathbb{R}^3$ notée $q$.
 
 Montrons que $h=q_{|x\in\mathbb{S}_a^2}$ est solution de \eqref{eq:advection}.
 
 Pour tout $r<0$, $\mathbf{c}(\lambda, \theta, r) \in \mathbb{T}\mathbb{S}_r^2$ donc $\mathbf{c} \cdot \nabla q = \mathbf{c}_T \cdot \nabla h$ sur $\mathbb{S}_a^2$.
 
 Ainsi sur $\mathbb{S}_a^2$ on a :
 \begin{equation}
 0 = \dfrac{\partial q}{\partial t} + \mathbf{c} \cdot \nabla q = \dfrac{\partial h}{\partial t} + \mathbf{c}_T \cdot \nabla h
 \end{equation}
 
 de plus $q_{|t=0}=h_0$ sur $\mathbb{S}_a^2$.
 
 L'existence est démontrée.


\end{itemize}

\end{proof}


\subsection{Equation Shallow Water linéarisée}

\subsection{Equation Shallow Water}




