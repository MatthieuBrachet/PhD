% testSWEC.tex
\chapter{Benchmarks sur l'équation SWEC}

\section{Forme de SWEC}

On a montré dans le Chapitre 1 que l'équation Shallow Water était issue de l'équation de Navier-Stokes en tenant compte d'une faible profondeur de fluide et de la faible viscosité.

L'équation Shallow Water obtenue était la suivante :

\begin{equation}
\label{eq:SWEC_new}
\left\lbrace
\begin{array}{rcl}
\dfrac{\partial \mathbf{u}}{\partial t} + \left( \mathbf{u} \cdot \nabla \right) \mathbf{u} + f \mathbf{k} \wedge \mathbf{u} + g \nabla h & = & \mathbf{0} \\
\dfrac{\partial h^{\star}}{\partial t} + \nabla \cdot \left( h^{\star} \mathbf{u} \right) & = & 0
\end{array}
\right.
\end{equation}

avec $h^{\star} = h - h_s$, $\eta_s$ représentant les reliefs sur la sphère. On suppose $h_s$ indépendant du temps $t$.

Il peut être délicat de travailler avec le terme $\left( \mathbf{u} \cdot \nabla \right) \mathbf{u}$. Pour éviter de discrétiser directement ce terme, on se repose sur la formule suivante :

\begin{equation}
\left( \mathbf{u} \cdot \nabla \right) \mathbf{u} = \nabla \left( \dfrac{1}{2} \mathbf{u}^2 \right) + \zeta \mathbf{k} \wedge \mathbf{u}
\end{equation}

avec $\zeta = \mathbf{k} \cdot \left( \nabla \wedge \mathbf{u} \right)$ la vorticité relative. On note la présence de $\nabla \wedge \mathbf{u}$ le rotationnel de $\mathbf{u}$.

L'équation \eqref{eq:SWEC_new} s'écrit alors :

\begin{equation}
\label{eq:SWEC_vectform}
\left\lbrace
\begin{array}{rcl}
\dfrac{\partial \mathbf{u}}{\partial t} + \nabla \left( g h + \dfrac{1}{2} \mathbf{u}^2  \right) + \left( \zeta + f \right) \mathbf{k} \wedge \mathbf{u} & = & \mathbf{0} \\
\dfrac{\partial h^{\star}}{\partial t} + \nabla \cdot \left( h^{\star} \mathbf{u} \right) & = & 0
\end{array}
\right.
\end{equation}

Dans la suite, nous travaillerons avec cette forme de l'équation \eqref{eq:SWEC_new}.


\section{Relations de Conservations}

L'équation \eqref{eq:SWEC_vectform} vérifie certaines relations de conservations au sens continu.

\begin{proposition}
Les relations de conservations suivantes sont vérifiées si $(\mathbf{u},h)$ est solution de \eqref{eq:SWEC_vectform} :
\begin{itemize}
\item Conservation de la matière :
\begin{equation}
\dfrac{d}{dt} \gint_{\mathbb{S}_a^2} h^{\star} = 0
\label{eq:mass}
\end{equation}
 
\item Conservation de l'énergie :
\begin{equation}
\dfrac{d}{dt} \gint_{\mathbb{S}_a^2} \dfrac{1}{2} h^{\star} \mathbf{u}.^2 + \dfrac{1}{2} g \left( h^2 - h_s^2 \right) = 0
\label{eq:energy}
\end{equation}

\item Conservation de l'enstrophie potentielle :
\begin{equation}
\dfrac{d}{dt} \gint_{\mathbb{S}_a^2} \dfrac{\left( \zeta + f \right)^2}{h^{\star}} = 0
\label{eq:enstrophie}
\end{equation}

\item Conservation de la vorticité :
\begin{equation}
\dfrac{d}{dt} \gint_{\mathbb{S}_a^2} \zeta = 0
\label{eq:vorticity}
\end{equation}

\item Conservation de la divergence
\begin{equation}
\dfrac{d}{dt} \gint_{\mathbb{S}_a^2} \nabla \cdot \mathbf{u} = 0
\label{eq:divergence}
\end{equation}
\end{itemize}

avec $\zeta = \left( \nabla \wedge \mathbf{u} \right) \cdot \mathbf{k}$.
\end{proposition}

\begin{remarque}
Pour prouver que $\delta$ est conservée, il suffit de montrer qu'il existe $\mathbf{F} \mathbb{T} \mathbb{S}_a$ tel que :
$$
\dfrac{\partial \delta}{\partial t} = \nabla \cdot \mathbf{F}
$$
puis d'intégrer et de conclure avec la remarque \ref{remark_stokes}.
\label{rmq:int diverg nulle}
\end{remarque}

\begin{proof}
Pour la conservation de la masse \eqref{eq:mass} et de la divergence \eqref{eq:divergence}, le résultat est immédiat.

De plus, en posant $q = \dfrac{\zeta+f}{h^{\star}}$ et en appliquant à \eqref{eq:SWEC_vectform} l'opérateur de vorticité "$\mathbf{k} \cdot \left( \nabla \wedge \cdot \right)$", on obtient :

$$
\dfrac{\partial \zeta}{\partial t}+\nabla \wedge \left( q h^{\star} \mathbf{k} \wedge \mathbf{u} \right) \cdot\mathbf{k} + \underbrace{\nabla \wedge \nabla \left( gh + \dfrac{1}{2}\mathbf{u}^2 \right) \cdot \mathbf{k}}_{=0} = 0 
$$

Or on sait que pour tout $\mathbf{X}$ et $\mathbf{F}$, on a :
\begin{equation}
\nabla \cdot \left( \mathbf{X} \wedge \mathbf{F} \right) = - \left( \nabla \wedge \mathbf{F} \right) \cdot \mathbf{X}.
\end{equation}

donc :

$$
\dfrac{ \partial \zeta}{\partial t} + \nabla \cdot \left( q h^{\star} \mathbf{u} \right) = 0
$$

En utilisant la remarque \ref{rmq:int diverg nulle}, on en déduit la conservation de la vorticité \eqref{eq:vorticity}.

On note, que $f$ est indépendant du temps : $\dfrac{\partial q h^{\star}}{\partial t
} = \dfrac{\partial \zeta}{\partial t}$.

$$
\dfrac{\partial}{\partial t} \left( q h^{\star} \right) + \nabla \cdot \left( q h^{\star} \mathbf{u} \right) = 0
$$

et on a démontré la conservation de l'enstrophie potentielle \eqref{eq:enstrophie}.

En ce qui concerne la conservation de l'énergie, on pose :
\begin{equation}
\begin{array}{rcl}
E_1 & = & \dfrac{1}{2} h^{\star} \mathbf{u}^2 \\
E_2 & = & \dfrac{1}{2} g \left( h^2 - h_s^2 \right)
\end{array}
\end{equation}

par dérivation :
\begin{equation}
\begin{array}{rcl}
\dfrac{\partial}{\partial t} E_1 & = & -\dfrac{1}{2} \mathbf{u}^2 \nabla \cdot \left( h^{\star} \mathbf{u} \right) - h^{\star} \mathbf{u} \cdot \nabla \left( \dfrac{1}{2} \mathbf{u}^2 + gh \right) \\
\dfrac{\partial}{\partial t} E_2 & = & - gh \nabla \cdot \left( h^{\star} \mathbf{u} \right) - g h_s \dfrac{\partial h_s}{\partial t} 
\end{array}
\end{equation}

par somme :

$$
\dfrac{\partial E_1 + E_2}{\partial t} = - \nabla \cdot \left( \dfrac{1}{2} \mathbf{u}^2 + gh \right) - g h_s \dfrac{\partial h_s}{\partial t} 
$$

$h_s$ est indépendant du temps donc en appliquant la remarque \ref{rmq:int diverg nulle}, on obtient la conservation de l'énergie.
\end{proof}

\section{Solution stationnaire zonale -  test 2 de Williamson}

Dans cette section, on cherche $(\mathbf{u},h)$ une solution stationnaire (indépendante de $t$) de \eqref{eq:SWEC_vectform} avec $\mathbf{u}$ zonale, c'est à dire $\mathbf{u}(\lambda, \theta) = u(\theta) \mathbf{e}_{\lambda}$.

\begin{proposition}
Les solutions stationnaires zonales $(\mathbf{u},h)$ de \eqref{eq:SWEC_vectform} sont données par :
\begin{equation}
\left\lbrace
\begin{array}{rcl}
\mathbf{u}(\theta) & = & u(\theta) \mathbf{e}_{\lambda}\\
h(\theta) & = & h_0 - \dfrac{a}{g} \gint^{\theta} u(\tau) \left( u(\tau) \dfrac{\tan (\tau)}{a} + f \right) d\tau\\
\end{array}
\right.
\label{eq:sol. stationaire zonale de swe}
\end{equation}
avec $f \equiv f(\theta) = 2 \omega \sin \theta$, $u$ une fonction de classe $\mathcal{C}^1$ et $h_s \equiv 0$.
\end{proposition}

\begin{proof}
On utilise les expression du gradient et du rotationnel \eqref{divergence_lonlat} et \eqref{rotationnel_lonlat} sur la première équation de \eqref{eq:SWEC_vectform}. On a alors :
\begin{equation}
\xi + f = u(\theta) \dfrac{\tan \theta}{a} - \dfrac{1}{a} u'(\theta) + f.
\end{equation}
Ainsi :
\begin{equation}
\left( \xi + f \right) \mathbf{k} \wedge \mathbf{u} = \left( u^2 (\theta) \dfrac{\tan \theta}{a} - \dfrac{1}{a} u(\theta) u'(\theta) + f(\theta) u(\theta) \right) \mathbf{e}_{\theta}
\end{equation}

De même, avec l'expression du gradient, on obtient :

\begin{equation}
\nabla \left( gh + \dfrac{1}{2} |\mathbf{u}|^2 \right) = \dfrac{g}{a \cos \theta} \dfrac{\partial h}{\partial \lambda} \mathbf{e}_{\lambda} + \left[ \dfrac{g}{a} \dfrac{\partial h}{\partial \theta} + \dfrac{1}{a} u'(\theta) u(\theta) \right] \mathbf{e}_{\theta}
\end{equation}

Comme la solution recherchée est stationnaire, on a $h$ et $\mathbf{u}$ indépendants de $t$. D'où :

\begin{equation}
\left( \xi + f \right) \mathbf{k} \wedge \mathbf{u} + \nabla \left( gh + \dfrac{1}{2} |\mathbf{u}|^2 \right) = 0
\end{equation}

En traitant cette équation composante par composante, on peut en déduire des informations sur $h$.

\begin{itemize}
\item \textbf{Composante en} $\mathbf{e}_{\lambda}$ : 

\begin{equation}
\dfrac{g}{a \cos \theta} \dfrac{\partial h}{\partial \lambda} = 0
\end{equation}

donc $h$ est indépendant de $\lambda$.

\item \textbf{Composante en} $\mathbf{e}_{\theta}$ :

\begin{equation}
u^2 (\theta) \dfrac{\tan \theta}{a}  + f(\theta) u(\theta) + \dfrac{g}{a} h'(\theta) = 0
\end{equation}

d'où l'on déduit facilement :

\begin{equation}
h'(\theta) = - u(\theta) \dfrac{a}{g} \left( u(\theta) \dfrac{\tan \theta}{a} + f(\theta) \right)
\end{equation}

que l'on intègre pour obtenir la formule de la proposition :

\begin{equation}
h(\theta) = h_0 - \dfrac{a}{g} \gint^{\theta} u(\tau) \left( u(\tau) \dfrac{\tan (\tau)}{a} + f(\tau) \right) d\tau
\end{equation}

Enfin, comme $u$ et $h$ ne dépendent que de $\theta$, il est facile de vérifier que $\nabla \cdot \left( h \mathbf{u} \right)=0$
\end{itemize}
\end{proof}

Ainsi si on cherche une solution stationnaire zonale avec $\mathbf{u}(\theta) = u_0 \cos \theta \mathbf{e}_{\lambda}$, on obtient :

\begin{equation}
h(\theta) = h_0 - \dfrac{1}{g}\left( 2 \omega a + u_0^2 \right) \gint^{\theta} \cos \tau \sin \tau d \tau
\end{equation}

d'où :

\begin{equation}
h(\theta) = h_0 - \dfrac{1}{g} \left( 2 \omega a + u_0^2 \right) \sin^2 \theta
\end{equation}




