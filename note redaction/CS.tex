% CS.tex
\chapter{Cubed-Sphere}

\section{Construction}

\section{Représentation en coordonnées}

\subsection{Coordonnées et base locale}

On décrit dans cette partie les coordonnées d'un point de la Cube-Sphere. Un point $\mathbf{x} \in \mathbb{S}_a^2$ est localisée exactement sur un panel $K \in \left\lbrace I, II, III, IV, V, VI \right\rbrace$ et à l'intersection de deux cercles : $C_{\xi}$ et $C_{\eta}$. $(\xi,\eta)$ correspond aux coordonnées gnonomiques, en effet si $\mathbf{x} \in (x,y,z)^T \in \mathbb{R}^3 \cap \mathbb{S}_a^2$ alors :

\begin{equation}
\left\lbrace
\begin{array}{rcccl}
\tan \xi & = & \dfrac{y}{x} & = & X\\
\tan \eta & = & \dfrac{z}{x}& = & Y
\end{array}
\right.
\end{equation}

Le point $\mathbf{x}$ est aussi localisé grâce à ses coordonnées $(\alpha, \beta)$ avec $\alpha$ l'absisse curviligne entre $\mathbf{x}$ et l'équateur Nord-Sud le long du cercle $C_{\eta}$ et $\beta$ entre $\mathbf{x}$ et l'équateur Est-Ouest.

Dans cette partie, on se concentre sur le panel $I$ pour le calcul.
Entre les différents systèmes de localication, les relations suivantes sont vérifiées :

\begin{equation}
\left\lbrace
\begin{array}{rcccl}
x & = & a \cos \alpha \cos \eta & = & a \cos \beta \cos \xi \\
y & = & a \sin \alpha & = & a \cos \beta \cos \xi \\
z & = & a \cos \alpha \sin \eta & = & a \sin \beta \\
\end{array}
\right.
\end{equation}

On en déduit que :

\begin{equation}
\alpha = \arctan \left[ \dfrac{\tan \xi}{\sqrt{1 + \tan^2 \eta}} \right]
\end{equation}

ainsi que :

\begin{equation}
\beta = \arctan \left[ \dfrac{\tan \eta}{\sqrt{1 + \tan^2 \xi}} \right]
\end{equation}

et en dérivant :

\begin{equation}
\dfrac{\partial \alpha}{\partial \xi} = \cos \eta \dfrac{1+X^2}{1+\cos^2 \eta X^2} \hspace{1cm} \dfrac{\partial \beta}{\partial \eta} = \cos \xi \dfrac{1+Y^2}{1+\cos^2 \xi Y^2}
\end{equation}

La base locale $(\mathbf{g}_{\xi}, \mathbf{g}_{\eta})$ associée à $(\xi,\eta)$ est donnée par :

$$\mathbf{g}_{\xi} = \dfrac{\partial \mathbf{x}}{\partial \xi} \hspace{1cm} \mathbf{g}_{\eta} = \dfrac{\partial \mathbf{x}}{\partial \eta}.$$

Ainsi, en posant $\delta = \sqrt{1+X^2+Y^2}$, $\mathbf{g}_{\xi}$ et $\mathbf{g}_{\eta}$ sont donnés par :

\begin{equation}
\mathbf{g}_{\xi} = \dfrac{1+X^2}{\delta^2} \begin{pmatrix}
-y \\ x(1+Y^2) \\ -yY
\end{pmatrix} \text{ et } \mathbf{g}_{\eta} = \dfrac{1+Y^2}{\delta^2} \begin{pmatrix}
-z \\ -zX \\ x(1+X^2)
\end{pmatrix}
\label{base locale}
\end{equation}

\subsection{Tenseur métrique et base duale}

Pour tenir compte de la non-orthogonalité des coordonnées, il faut tenir compte de la métrique pour le calcul des opérateurs.

Le tenseur métrique est donné par :

\begin{equation}
\mathbf{G} = \begin{pmatrix}
\mathbf{g}_{\xi} \cdot \mathbf{g}_{\xi} & \mathbf{g}_{\xi} \cdot \mathbf{g}_{\eta} \\
\mathbf{g}_{\eta} \cdot \mathbf{g}_{\xi} & \mathbf{g}_{\eta} \cdot \mathbf{g}_{\eta}
\end{pmatrix}
\end{equation}

Ainsi, après calcul, on a :

\begin{equation}
\mathbf{G} = \begin{pmatrix}
G_{1,1} & G_{1,2} \\ G_{2,1} & G_{2,2}
\end{pmatrix} = a^2 \dfrac{(1+X^2)(1+Y^2)}{\delta^4} \begin{pmatrix}
1+X^2 & -XY \\ -XY & 1+Y^2
\end{pmatrix}
\end{equation}

que l'on peut inverser en :

\begin{equation}
\mathbf{G}^{-1} = \begin{pmatrix}
G^{1,1} & G^{1,2} \\ G^{2,1} & G^{2,2}
\end{pmatrix} = \dfrac{\delta^2}{a^2 (1+X^2)(1+Y^2)} \begin{pmatrix}
1+Y^2 & XY \\ XY & 1+X^2
\end{pmatrix}
\end{equation}

Il est important de connaitre la base duale par rapport à la métrique $\mathbf{G}$. On cherche $\mathbf{g}_{\xi}$ et $\mathbf{g}_{\eta}$ tels que :

\begin{equation}
\left\lbrace
\begin{array}{rcccl}
\mathbf{g}_{\xi} \cdot \mathbf{g}^{\xi} & = & \mathbf{g}_{\eta} \cdot \mathbf{g}^{\eta} & = & 1 \\
\mathbf{g}_{\xi} \cdot \mathbf{g}^{\eta} & = & \mathbf{g}_{\eta} \cdot \mathbf{g}^{\xi} & = & 0 \\
\end{array}
\right.
\end{equation}

d'où la relation suivante :

\begin{equation}
\left\lbrace
\begin{array}{rcl}
\mathbf{g}^{\xi} & = & G^{1,1} \mathbf{g}_{\xi} + G^{1,2} \mathbf{g}_{eta} \\
\mathbf{g}^{\eta} & = & G^{2,1} \mathbf{g}_{\xi} + G^{2,2} \mathbf{g}_{eta} \\
\end{array}
\right.
\end{equation}

d'où :

\begin{equation}
\mathbf{g}^{\xi} = \dfrac{1}{x(1+X^2)}\begin{pmatrix}
-X \\ 1 \\ 0
\end{pmatrix} \text{ et } \mathbf{g}^{\eta} = \dfrac{1}{x(1+Y^2)}\begin{pmatrix}
-Y \\ 0 \\ 1
\end{pmatrix}
\end{equation}

$( \mathbf{g}^{\xi}, \mathbf{g}^{\eta})$ est la base duale de $(\mathbf{g}_{\xi}, \mathbf{g}_{\eta})$ par rapport à la métrique $\mathbf{G}$.

\subsection{Symboles de Christophel}

Les champs de vecteur $( \mathbf{g}^{\xi}, \mathbf{g}^{\eta})$ et $( \mathbf{g}_ {\xi}, \mathbf{g}_{\eta})$ sont tangents à la sphère. En conséquence, ils varient (continuement) en fonction de $\xi$ et $\eta$. On définit les symboles de Christoffel par :

\begin{equation}
\left\lbrace
\begin{array}{rcl}
\dfrac{\partial \mathbf{g}_{\xi}}{\partial \xi} & = & \Gamma_{\xi,\xi}^{\xi} \mathbf{g}_{\xi} + \Gamma_{\xi,\xi}^{\eta} \mathbf{g}_{\eta}\\

\dfrac{\partial \mathbf{g}_{\xi}}{\partial \eta} & = & \Gamma_{\eta,\xi}^{\xi} \mathbf{g}_{\xi} + \Gamma_{\eta,\xi}^{\eta} \mathbf{g}_{\eta}\\

\dfrac{\partial \mathbf{g}_{\eta}}{\partial \xi} & = & \Gamma_{\xi,\eta}^{\xi} \mathbf{g}_{\xi} + \Gamma_{\xi,\eta}^{\eta} \mathbf{g}_{\eta}\\

\dfrac{\partial \mathbf{g}_{\eta}}{\partial \eta} & = & \Gamma_{\eta,\eta}^{\xi} \mathbf{g}_{\xi} + \Gamma_{\eta,\eta}^{\eta} \mathbf{g}_{\eta}\\
\end{array}
\right.
\end{equation}

\begin{remarque}
On note que $\Gamma_{\eta,\xi}^{\xi}=\Gamma_{\xi,\eta}^{\xi}$ et $\Gamma_{\eta,\xi}^{\eta}=\Gamma_{\xi,\eta}^{\eta}$.

En effet :

$$\Gamma_{\xi, \eta}^{\eta} = \left( \dfrac{\partial \mathbf{g}_{\eta}}{\partial \xi} \right) \cdot \mathbf{g}^{\eta} = \left( \dfrac{\partial}{\partial \xi} \dfrac{\partial \mathbf{x}}{\partial \eta} \right) \cdot \mathbf{g}^{\eta} = \left( \dfrac{\partial}{\partial \eta} \dfrac{\partial \mathbf{x}}{\partial \xi} \right) \cdot \mathbf{g}^{\eta} = \left( \dfrac{\partial \mathbf{g}_{\xi}}{\partial \eta} \right) \cdot \mathbf{g}^{\eta} = \Gamma_{\eta, \xi}^{\eta}$$

de même pour $\Gamma_{\eta,\xi}^{\xi}=\Gamma_{\xi,\eta}^{\xi}$.
\end{remarque}

\begin{proposition}
Les relations suivantes sont vérifiées :

\begin{equation}
\left\lbrace
\begin{array}{rcccl}
\Gamma_{\xi,\xi}^{\xi} & = & \left[ \dfrac{\partial \mathbf{g}_{\xi}}{\partial \xi} \right] \cdot \mathbf{g}^{\xi} & = & - \left[ \dfrac{\partial \mathbf{g}^{\xi}}{\partial \xi} \right] \cdot \mathbf{g}_ {\xi}\\

\Gamma_{\xi,\xi}^{\eta} & = & \left[ \dfrac{\partial \mathbf{g}_{\xi}}{\partial \xi} \right] \cdot \mathbf{g}^{\eta} & = & - \left[ \dfrac{\partial \mathbf{g}^{\eta}}{\partial \xi} \right] \cdot \mathbf{g}_ {\xi}\\

\Gamma_{\xi,\eta}^{\xi} & = & \left[ \dfrac{\partial \mathbf{g}_{\eta}}{\partial \xi} \right] \cdot \mathbf{g}^{\xi} & = & - \left[ \dfrac{\partial \mathbf{g}^{\xi}}{\partial \xi} \right] \cdot \mathbf{g}_ {\eta}\\

\Gamma_{\xi,\eta}^{\eta} & = & \left[ \dfrac{\partial \mathbf{g}_{\eta}}{\partial \xi} \right] \cdot \mathbf{g}^{\eta} & = & - \left[ \dfrac{\partial \mathbf{g}^{\eta}}{\partial \xi} \right] \cdot \mathbf{g}_ {\eta}\\

\Gamma_{\eta,\eta}^{\eta} & = & \left[ \dfrac{\partial \mathbf{g}_{\eta}}{\partial \eta} \right] \cdot \mathbf{g}^{\eta} & = & - \left[ \dfrac{\partial \mathbf{g}^{\eta}}{\partial \eta} \right] \cdot \mathbf{g}_ {\eta}\\

\Gamma_{\eta,\eta}^{\xi} & = & \left[ \dfrac{\partial \mathbf{g}_{\eta}}{\partial \eta} \right] \cdot \mathbf{g}^{\xi} & = & - \left[ \dfrac{\partial \mathbf{g}^{\xi}}{\partial \eta} \right] \cdot \mathbf{g}_ {\eta}\\
\end{array}
\right.
\end{equation}
\end{proposition}

\begin{proof}
Nous ne démontrons que la première équalité, les autres égualités se retrouvent de manière extrêmement similaire.

$$\left[ \dfrac{\partial \mathbf{g}_{\xi}}{\partial \xi} \right] \cdot \mathbf{g}^{\xi} = \left[ \Gamma_{\xi,\xi}^{\xi} \mathbf{g}_{\xi} + \Gamma_{\xi,\xi}^{\eta} \mathbf{g}_{\eta} \right] \cdot \mathbf{g}^{\xi}$$

Or $\mathbf{g}_{\xi} \cdot \mathbf{g}^{\eta} = 1$ et $\mathbf{g}_{\eta} \cdot \mathbf{g}^{\eta} = 0$, d'où la première partie :

$$\Gamma_{\xi,\xi}^{\xi} = \left[ \dfrac{\partial \mathbf{g}_{\xi}}{\partial \xi} \right] \cdot \mathbf{g}^{\xi}.$$

D'autres part, on a :

$$\Gamma_{\xi,\xi}^{\xi} = \left[ \dfrac{\partial \mathbf{g}_{\xi}}{\partial \xi} \right] \cdot \mathbf{g}^{\xi} = \dfrac{\partial}{\partial \xi}  \underbrace{\left(\mathbf{g}_{\xi} \cdot \mathbf{g}^{\xi}\right)}_{=1}  - \left[ \dfrac{\partial \mathbf{g}^{\xi}}{\partial \xi}  \right] \cdot \mathbf{g}_{\xi} = - \left[ \dfrac{\partial \mathbf{g}^{\xi}}{\partial \xi}  \right] \cdot \mathbf{g}_{\xi}$$

et la relation est démontrée.
\end{proof}

On peut calculer les dérivées suivantes :

\begin{equation}
\dfrac{\partial \mathbf{g}^{\xi}}{\partial \xi} = \dfrac{1}{x} \begin{pmatrix}
-\dfrac{X^2}{\delta^2}+2X \cos \xi \sin \xi -1 \\ \dfrac{X}{\delta^2}-2 \cos \xi \sin \xi \\ 0
\end{pmatrix}
\text{ et }
\dfrac{\partial \mathbf{g}^{\xi}}{\partial \eta} = \dfrac{1+Y^2}{\delta^2 x^2 (1+X^2)} \begin{pmatrix}
-X \\ 1 \\ 0
\end{pmatrix}
\end{equation}

de même :

\begin{equation}
\dfrac{\partial \mathbf{g}^{\eta}}{\partial \xi} = \dfrac{X(1+X^2)}{x \delta^2 (1+Y^2)} \begin{pmatrix}
-Y \\ 0 \\ 1
\end{pmatrix}
\text{ et }
\dfrac{\partial \mathbf{g}^{\eta}}{\partial \eta} = \dfrac{1}{x} \begin{pmatrix}
- \dfrac{Y^2}{\delta^2} + 2 Y \cos \eta \sin \eta -1 \\ 0 \\ \dfrac{Y}{\delta^2}- 2 \cos \eta \sin \eta
\end{pmatrix}
\end{equation}

d'où les symboles de Christoffel :

\begin{equation}
\left\lbrace
\begin{array}{rcl}
\Gamma_{\xi,\eta}^{\xi} & = & - \dfrac{Y ( 1+Y^2)}{\delta^2}\\
\Gamma_{\xi,\eta}^{\eta} & = & - \dfrac{X(1+X^2)}{\delta^2}\\
\Gamma_{\eta,\eta}^{\xi} & = & 0 \\
\Gamma_{\xi,\xi}^{\eta} & = & 0 \\
\Gamma_{\eta,\eta}^{\eta} & = & \dfrac{1+Y^2}{\delta^2} \left[ 2 \delta^2 \cos \eta \sin \eta - Y \right]\\
\Gamma_{\xi,\xi}^{\xi} & = & \dfrac{1+X^2}{\delta^2} \left[ 2 \delta^2 \cos \xi \sin \xi - X \right]
\end{array}
\right.
\end{equation}

\begin{proposition}
Les symboles de Christoffel sont invariants par changement de face
\end{proposition}

\begin{proof}
Soit $R_f$ la rotation permettant de passer d'une face à une autre. On note en particulier que $R_f$ est indépendant de $\xi$ et $\eta$.

Alors soit $K$ et $L$ dans $\left\lbrace I, II, III, IV, V, VI \right\rbrace$ et $\tau$, $\upsilon \in \left\lbrace \xi, \eta \right\rbrace$. Alors par propriété du produit scalaire :

$$\left( \dfrac{\partial}{\partial \tau}  R_f \mathbf{g}_{\tau} \right) \cdot \left( R_f \mathbf{g}_{\upsilon} \right) = \left( R_f \dfrac{\partial}{\partial \tau}   \mathbf{g}_{\tau} \right) \cdot \left( R_f \mathbf{g}_{\upsilon} \right) =  \left( \dfrac{\partial}{\partial \tau}  \mathbf{g}_{\tau} \right) \cdot \left( R_f^{-1} R_f \mathbf{g}_{\upsilon} \right) =  \left( \dfrac{\partial}{\partial \tau}  \mathbf{g}_{\tau} \right) \cdot \left( \mathbf{g}_{\upsilon} \right)$$

de même :

$$\left( \dfrac{\partial}{\partial \upsilon}  R_f \mathbf{g}_{\tau} \right) \cdot \left( R_f \mathbf{g}_{\upsilon} \right) =  \left( \dfrac{\partial}{\partial \upsilon} \mathbf{g}_{\tau} \right) \cdot \left( \mathbf{g}_{\upsilon} \right)$$

et on retrouve l'ensemble des symboles de Christoffel.
\end{proof}































