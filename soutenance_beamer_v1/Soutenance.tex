\documentclass[11pt]{beamer}
\usetheme{Boadilla}
\usepackage[utf8]{inputenc}
\usepackage[francais]{babel}
\usepackage[T1]{fontenc}
\usepackage{amsmath}
\usepackage{amsfonts}
\usepackage{amssymb}
\usepackage{graphicx}
\usepackage{tikz}

\def\CS{\text{CS}}
\def\gint{\displaystyle\int}
\def\gsum{\displaystyle\sum\limits}
\def\dxi{\tilde{\delta}^H_{\xi}}
\def\deta{\tilde{\delta}^H_{\eta}}
\def\fxi{\tilde{\mathcal{F}}_{\xi}}
\def\feta{\tilde{\mathcal{F}}_{\eta}}

\author[M. Brachet]{Matthieu Brachet}
\title[Soutenance de thèse]{Schémas compacts hermitiens sur la Sphère - Applications en climatologie et océanographie numérique}
%\setbeamercovered{transparent} 
%\setbeamertemplate{navigation symbols}{} 
%\logo{} 
%\institute{} 
\date[3-7-2018]{Mardi 3 Juillet 2018} 
%\subject{} 
\begin{document}

\begin{frame}
\titlepage
\begin{flushright}
\includegraphics[scale=.21]{ul.png}
\includegraphics[scale=.25]{iecl.jpg}
\end{flushright}
\end{frame}

%% ***************************************************************

\begin{frame}
\tableofcontents
\end{frame}

%% ***************************************************************
\section{Introduction}
\begin{frame}{Introduction}
\begin{block}{Enjeux}
Prévisions de la dynamique atmosphérique.
\end{block}

\begin{alertblock}{}
Problème difficile :
\begin{itemize}
\item Equations complexes,
\item Contexte sphérique,
\item Nombreuses échelles à prendre en compte.
\end{itemize}
\end{alertblock}

\begin{exampleblock}{Objectif :}
Conception d'un schéma numérique pour un modèle simplifié : les équations \textbf{Shallow Water}.
\end{exampleblock}
\end{frame}



\begin{frame}{}
\begin{block}{Système d'équations Shallow Water}
\begin{equation*}
\left\lbrace
\begin{array}{rcl}
\dfrac{\partial h^{\star}}{\partial t} + \nabla_T \cdot \left( h^{\star} \mathbf{u} \right) & = & 0\\
\dfrac{\partial \mathbf{u}}{\partial t} + \nabla_T \left( gh + \dfrac{1}{2} |\mathbf{u}|^2 \right) + \left( \zeta + f \right) \mathbf{n} \wedge \mathbf{u} & = & 0.
\end{array}
\right.
\end{equation*}
pour $\mathbb{S}_a^2$ pour $t>0$.

$g$ : gravité, $f$ : Coriolis, $\zeta$ : vorticité relative, $h^{\star} = h - h_s$ où $h_s$ représente les reliefs.
\end{block}

\begin{itemize}
\item Résolution plus complexe que dans le cadre plan.
\item Problème représentatif des difficultés sur la sphère $\mathbb{S}_a^2$,
\item Besoin d'un algorithme de résolution numérique.
\end{itemize}
\end{frame}







\begin{frame}{Grille Yin-Yang}
\begin{columns}
\column{0.45\textwidth}
\begin{center}
\includegraphics[scale=.5]{yinyang.png}
\end{center}

\column{0.45\textwidth}
Combinaison de deux grilles Longitude-Latitude tronquées.

\begin{itemize}
\item Volumes finis : Li \textit{et al.}, 2008.
\item Éléments finis de Galerkin discontinu : Hall \textit{et al.}, 2013.
\end{itemize}
\end{columns}
\end{frame}



\begin{frame}{Grille Icosaèdrale}
\begin{columns}
\column{0.45\textwidth}
\begin{center}
\includegraphics[scale=.45]{ico.png}
Image non satisfaisante...
\end{center}

\column{0.45\textwidth}
Combinaison de deux grilles Longitude-Latitude tronquées.

\begin{itemize}
\item Volumes finis : Thuburn \textit{et al.}, 2014.
\item Éléments finis de Galerkin discontinu : Giraldo \textit{et al.}, 2002.
\end{itemize}
\end{columns}
\end{frame}



\begin{frame}{Objectif}
\begin{columns}
\column{0.45\textwidth}
\begin{center}
\includegraphics[scale=.2]{cs.png}
\end{center}

\column{0.45\textwidth}

\begin{block}{}
Conception et étude d'un schéma compact sur la \textbf{Cubed-Sphere}. Applications aux \textbf{équations Shallow Water}.
\end{block}


\end{columns}
\end{frame}

%% ***************************************************************

\section{La Cubed-Sphere}
\begin{frame}{La grille Cubed-Sphere}
\begin{columns}
\column{0.45\textwidth}
\begin{figure}[htbp]
\begin{center}
\includegraphics[scale=.27]{cs_cercles.png}
\end{center}
\caption{Le panel $(I)$ est constitué des points d'intersections d'un ensemble de grands cercles.}
\label{fig: panel I}
\end{figure}  

\column{0.5\textwidth}
\begin{itemize}
\item 2 ensembles de grands cercles $C_i^{(1)}$ et $C_j^{(2)}$, $-N/2 \leq i,j \leq N/2$,

\item $C_0^{(1)}$ et $C_0^{(2)}$ s'intersectent avec un angle de $90$ degrés

\item Les points $\mathbf{x}_{i,j}=C_i^{(1)} \cap C_j^{(2)}$ sont les points de maillage sur un panel de la Cubed-Sphere.
\end{itemize}
\end{columns}
\end{frame}













\begin{frame}{La grille Cubed-Sphere: 6 panels}
\begin{figure}
\begin{center}
\hspace{-1.cm}
\includegraphics[scale=0.28]{plot_CS.png}
\caption{La Cubed-Sphere avec $N=16$ ($16^2$ cellules par panels).
}
\end{center}
\end{figure}
\end{frame}










\begin{frame}{Localisation d'un point sur un panel}
\begin{columns}
\column{0.45\textwidth}
\begin{figure}[htbp]
\begin{center}
\includegraphics[scale=.27]{cs_angles.png}
\end{center}
\caption{Sur un panel, un point $\mathbf{x}$ est localisé par $\xi$ et $\eta$.}
\label{fig: panel I xi eta}
\end{figure}
\column{0.5\textwidth}
\begin{itemize}
\item Une base en $\mathbf{x}$ est $$(\mathbf{g}_{\xi}, \mathbf{g}_{\eta}) = (\partial_{\xi} \mathbf{x}, \partial_{\eta} \mathbf{x}).$$

\item Le tenseur métrique en $\mathbf{x} \in \mathbb{S}_R^2$ est
\begin{equation*}
G=\begin{bmatrix}
\mathbf{g}_{\xi} \cdot \mathbf{g}_{\xi} & \mathbf{g}_{\xi} \cdot \mathbf{g}_{\eta} \\
\mathbf{g}_{\eta} \cdot \mathbf{g}_{\xi} & \mathbf{g}_{\eta} \cdot \mathbf{g}_{\eta} 
\end{bmatrix}
\end{equation*}

\item La base duale en $\mathbf{x} \in \mathbb{S}^2_a $ est $(\mathbf{g}^\xi(\mathbf{x}),\mathbf{g}^\eta(\mathbf{x}))$ donnée par
$$
\begin{bmatrix}
\mathbf{g}^{\xi} \\ \mathbf{g}^{\eta} 
\end{bmatrix}
=
\mathbf{G}^{-1} \cdot
\begin{bmatrix}
\mathbf{g}_{\xi} \\ \mathbf{g}_{\eta} 
\end{bmatrix}.
$$
\end{itemize}
\end{columns}
\end{frame}





\begin{frame}{Bilan}
\begin{itemize}
\item La Cubed-Sphere est déduite de 6 réseaux de $N+1$ grands cercles chacun.

\item Elle est composée de $6N^2+2$ points :
\begin{itemize}
\item $8$ coins de la Cubed-Sphere,
\item $12(N-1)$ points appartenant à deux panels exactement,
\item $6(N-1)^2$ points intérieurs des panels.
\end{itemize}

\item On pose $\Delta \xi = \Delta \eta = \dfrac{\pi}{2N}$ le pas de discrétisation.
\end{itemize}
\end{frame}















\begin{frame}{Produit scalaire sur la Cubed-Sphere}

On pose $\bar{\mathbf{G}}_{i,j} = \det ( \mathbf{G}(\mathbf{x}_{i,j}^{(k)}))$.

\begin{block}{}
Soient $\mathfrak{u}$ et $\mathfrak{v}$ deux fonctions de grilles sur la Cubed-Sphere. On définit
$$<\mathfrak{u},\mathfrak{v}>_{\CS} = \gsum_{(k) = (I)}^{(VI)} \gsum_{-N/2 \leq i,j \leq N/2} \omega_{i,j} \sqrt{\bar{\mathbf{G}}_{i,j}} \mathfrak{u}_{i,j}^{(k)} \bar{\mathfrak{v}}_{i,j}^{(k)}$$
alors $<\cdot, \cdot>_{\CS}$ est un \textbf{produit scalaire} si les coefficients $\omega_{i,j}$ sont positifs.
\end{block}
\end{frame}




\begin{frame}
Soient $m, m' \in \mathbb{Z}$, $l \leq |m|$ et $l' \leq |m'|$, alors pour $\mathbf{Y}_m^l$ et $\mathbf{Y}_{m'}^{l'}$ des harmoniques sphériques, on a 
$$
<\mathbf{Y}_m^l, \mathbf{Y}_{m'}^{l'}>_{L^2} = \delta_{m,m'}\delta_{l,l'}.
$$
Cette propriété est partiellement vérifiée par $<\cdot, \cdot>_{\CS}$ :

\begin{block}{}
Si on a
$$
\omega_{i,j} = \omega_{-i,-j} = \omega_{-i,j} = \omega_{i,-j}>0
$$
pour tous $i$ et $j$. Alors on a
$$
<\mathbf{Y}_m^{l,*}, \mathbf{Y}_{m'}^{l',*}>_{L^2} = 0
$$
lorsque
\begin{itemize}
\item $m+m'$ est impair ou,
\item $l+l'$ est impair.
\end{itemize}
\end{block}
\end{frame}








\begin{frame}{Quadrature sur la Cubed-Sphere}
Un choix judicieux de $\omega_{i,j}$ permet d'avoir
$$
\gint_{\mathbb{S}_a^2} f(\mathbf{x}) d \sigma(\mathbf{x}) = <f, 1>_{L^2} \approx <f^*, \mathfrak{1}>_{\CS} = Q(f^*)
$$
pour $f : \mathbf{x} \in \mathbb{S}_a^2 \mapsto f(\mathbf{x}) \in \mathbb{C}$ régulière. 

\begin{block}{}
Si on a :
\begin{itemize}
\item $\omega_{i,j} = \alpha$ si $|i| = |j| = N/2$,
\item $\omega_{i,j} = 1/2$ si $|i| = N/2$ ou $|j| = N/2$,
\item $\omega_{i,j} = 1$ sinon,
\end{itemize}
alors
$$
 Q_{\alpha}(f^*) =  <f^*, \mathfrak{1}>_{\CS} = \gint_{\mathbb{S}_a^2} f(\mathbf{x}) d \sigma(\mathbf{x}) + \mathcal{O} \left( \Delta \xi^2 \right).
$$
De plus, si $\alpha = 1/3$, on a
$$
<\mathfrak{1}, \mathfrak{1}>_{\CS} = 4 \pi a^2 + \mathcal{O} \left( \Delta \xi^4 \right).
$$ 
\end{block}
\end{frame}





\begin{frame}{Vérification numérique}
On pose $f_1 : (x,y,z) \in \mathbb{S}_a^2 \mapsto
f_1(x,y,z)$ avec
$$
f_1(x,y,z) = 1+x+y^2 + yx^2 + x^4 + y^5 + x^2 y^2 z^2
$$
\begin{columns}
\column{0.45\textwidth}
alors on a :
$$
\gint_{\mathbb{S}_a^2}f_1(\mathbf{x}) d \sigma(\mathbf{x}) = \dfrac{213 \pi}{35} a^2
$$

\column{0.5\textwidth}
\begin{figure}
\begin{center}
\includegraphics[scale=.4]{quadf1.png}
\caption{Convergence de différentes formules de quadrature $Q$ sur $f_1$.}
\end{center}
\end{figure}
\end{columns}
\end{frame}












\begin{frame}{Vérification numérique}
On pose $f_3 : (x,y,z) \in \mathbb{S}_a^2 \mapsto
f_3(x,y,z)$ avec
$$
f_3(x,y,z) = \frac{1 + \tanh(-9x-9y+9z)}{9}
$$
\begin{columns}
\column{0.45\textwidth}
alors on a :
$$
\gint_{\mathbb{S}_a^2}f_3(\mathbf{x}) d \sigma(\mathbf{x}) = \dfrac{4 \pi}{9} a^2
$$

\column{0.5\textwidth}
\begin{figure}
\begin{center}
\includegraphics[scale=.4]{quadf3.png}
\caption{Convergence de différentes formules de quadrature $Q$ sur $f_3$.}
\end{center}
\end{figure}
\end{columns}
\end{frame}











%% ***************************************************************
\section{Opérateurs discrets sur la Cubed-Sphere}
\begin{frame}{Opérateurs discrets sur la Cubed-Sphere}
Les opérateurs sont définis en coordonnées $(\xi,\eta)$ par
\begin{block}{}
\begin{itemize}
\item \textbf{Gradient :}
$$
\nabla_T h = \mathbf{g}^{\xi} \dfrac{\partial h}{\partial \xi} + \mathbf{g}^{\eta} \dfrac{\partial h}{\partial \eta},
$$
\item \textbf{Divergence :}
$$
\nabla_T \cdot \mathbf{u} = \mathbf{g}^{\xi} \cdot \dfrac{\partial \mathbf{u}}{\partial \xi} + \mathbf{g}^{\eta} \cdot \dfrac{\partial \mathbf{u}}{\partial \eta},
$$
\item \textbf{Vorticité :}
$$
\text{vort}(\mathbf{u}) = \left( \mathbf{g}^{\xi} \wedge \dfrac{\partial \mathbf{u}}{\partial \xi} + \mathbf{g}^{\eta} \wedge \dfrac{\partial \mathbf{u}}{\partial \eta} \right) \cdot \mathbf{n}.
$$
\end{itemize}
\end{block}
Il s'agit d'approcher $\dfrac{\partial}{\partial \xi}$ et $\dfrac{\partial}{\partial \eta}$ sur la Cubed-Sphere.
\end{frame}







\begin{frame}
\begin{columns}
\column{0.45\textwidth}
\textbf{Calcul de $\dxi$ :}
\begin{enumerate}
\item Complétion des données sur un grand cercle complet par \textbf{spline cubique},
\item Calcul de la dérivée approchée en utilisant $\delta_{\xi}^H$.
\end{enumerate}
On procède de même pour le calcul de $\deta$.

\column{0.5\textwidth}
\begin{center}
\includegraphics[scale=.28]{CS_interp.png}
\end{center}
\end{columns}

$\Rightarrow$ On remplace $\dfrac{\partial}{\partial \xi}$ et $\dfrac{\partial}{\partial \eta}$ par $\dxi$ et $\deta$ calculés sur la Cubed-Sphere.
\end{frame}


















\begin{frame}{Opérateurs discrets sur la Cubed-Sphere}
Les opérateurs discrets sont définis par
\begin{block}{}
\begin{itemize}
\item \textbf{Gradient discret:}
$$
\nabla_{T,\Delta} \mathfrak{h} = \mathbf{g}^{\xi} \dxi \mathfrak{h} + \mathbf{g}^{\eta} \deta \mathfrak{h},
$$
\item \textbf{Divergence discrète :}
$$
\nabla_{T,\Delta} \cdot \mathfrak{u} = \mathbf{g}^{\xi} \cdot \dxi \mathfrak{u} + \mathbf{g}^{\eta} \cdot \deta \mathfrak{u},
$$
\item \textbf{Vorticité discrète :}
$$
\text{vort}_{\Delta} (\mathfrak{u}) = \left( \mathbf{g}^{\xi} \wedge \dxi \mathfrak{u} + \mathbf{g}^{\eta} \wedge \deta \mathfrak{u} \right) \cdot \mathbf{n}.
$$
\end{itemize}
\end{block}
Il s'agit d'approcher $\dfrac{\partial}{\partial \xi}$ et $\dfrac{\partial}{\partial \eta}$ sur la Cubed-Sphere.
\end{frame}




\begin{frame}{Consistance des opérateurs discrets}
Soit $h$ une fonction scalaire sur $\mathbb{S}_a^2$ et $\mathbf{u}$ un champ vectoriel sur $\mathbb{S}_a^2$. Si $h$ et $\mathbf{u}$ sont réguliers, alors on a :
\begin{block}{}
\begin{itemize}
\item \textbf{Consistance du gradient :}
$$
(\nabla_T h)^* - \nabla_{T,\Delta} h^* = \mathcal{O}(\Delta \xi^3),
$$
\item \textbf{Consistance de la divergence :}
$$
(\nabla_T \cdot \mathbf{u})^* - \nabla_{T,\Delta} \cdot \mathbf{u}^* = \mathcal{O}(\Delta \xi^3),
$$
\item \textbf{Consistance de la vorticité :}
$$
\text{vort}(\mathbf{u})^* - \text{vort}_{\Delta}(\mathbf{u}^*) = \mathcal{O}(\Delta^3).
$$
\end{itemize}
\end{block}

\begin{alertblock}{}
\textbf{Ordre 4} observé dans les expériences numériques.
\end{alertblock}
\end{frame}








\begin{frame}{Opérateur de filtrage}

En suivant le même procédé on construit les opérateurs de filtrage $\fxi$ et $\feta$. On définit alors :
$$
\mathcal{F} = \dfrac{1}{2} \left( \fxi \circ \feta + \feta \circ \fxi \right),
$$

\begin{block}{}
Si $h$ est une fonction régulière sur la sphère, on a:
$$
\mathcal{F}(h^*) - h^* = \mathcal{O} \left( \Delta \xi^4 \right).
$$
\end{block}
\end{frame}






























%% *****************************************************************
\section{Résultats numériques}
\begin{frame}{Méthode des lignes}
On considère le problème :
$$
\left\lbrace
\begin{array}{rcl}
\dfrac{\partial q}{\partial t} & = & F(q,t) \\
q(t=0,\mathbf{x}) & = & q_0(\mathbf{x})
\end{array}
\right. \text{ avec } \mathbf{x} \in \mathbb{S}_a^2.
$$
\begin{block}{}
\textbf{Première étape : } discrétisation en espace.
\end{block}
$$
\left\lbrace
\begin{array}{rcl}
\dfrac{\partial q}{\partial t} & = & F_{\Delta}(q,t) \\
q(t=0,\mathbf{x}) & = & q_0(\mathbf{x})
\end{array}
\right. \text{ avec } \mathbf{x} \in \mathbb{S}_a^2.
$$
\begin{block}{}
\textbf{Seconde étape : } discrétisation en temps : RK4 avec étape de filtrage.
\end{block}
\end{frame}






\begin{frame}{Schéma de référence}
\begin{block}{Runge-Kutta d'ordre 4 + Filtre spatial
}
\begin{enumerate}
\item $K_1 = F_{\Delta}(\mathfrak{q}^n, t^n)$,
\item $K_2 = F_{\Delta}(\mathfrak{q}^n + \frac{\Delta t}{2} K_1, t^n + \frac{\Delta t}{2})$,
\item $K_3 = F_{\Delta}(\mathfrak{q}^n + \frac{\Delta t}{2} K_2, t^n + \frac{\Delta t}{2})$,
\item $K_4 = F_{\Delta}(\mathfrak{q}^n + \Delta t K_3, t^n + \Delta t)$
\item $\hat{\mathfrak{q}}^{n+1} = \mathfrak{q}^n + \frac{\Delta t}{6} \left( K_1 + 2 K_2 + 2 K_3 + K_4 \right)$
\item $\mathfrak{q}^{n+1} = \mathcal{F}(\hat{\mathfrak{q}}^{n+1})$
\end{enumerate}
\end{block}
\end{frame}









\begin{frame}{Equation d'advection linéaire}

On considère l'équation
$$
\left\lbrace
\begin{array}{rcl}
\dfrac{\partial h}{\partial t} + c(\mathbf{x},t) \cdot \nabla_T h & = & 0  \\
h(t=0,\mathbf{x}) & = & h_0(\mathbf{x})
\end{array}
\right. \text{ avec } \mathbf{x} \in \mathbb{S}_a^2.
$$

\begin{exampleblock}{Test de R. Nair et C. Jablonowski (2008)}
\begin{itemize}
\item Déplacement d'un vortex autour de la sphère,
\item Une solution analytique est disponible $\Rightarrow$ mesure de l'erreur relative.
\end{itemize}
\end{exampleblock}
\end{frame}


\begin{frame}
\begin{figure}
\begin{center}
\includegraphics[scale=.5]{rate_NJ1.png}
\end{center}
\caption{Erreur relative pour l'équation d'advection linéaire. On pose CFL=0.7. Le test est celui de Nair-Jablonowski.}
\end{figure}
\end{frame}

















\begin{frame}{Equation d'advection non linéaire}

On considère l'équation
$$
\left\lbrace
\begin{array}{rcl}
\dfrac{\partial h}{\partial t} + \cdot \nabla_T F(h) & = & 0  \\
h(t=0,\mathbf{x}) & = & h_0(\mathbf{x})
\end{array}
\right. \text{ avec } \mathbf{x} \in \mathbb{S}_a^2.
$$

avec $F(h) = \dfrac{1}{2} h^2 \mathbf{n} \wedge (\mathbf{i} + \mathbf{j} + \mathbf{k})$ 

\begin{exampleblock}{Test de M. Ben-Artzi \text{et al.} (2013)}
\begin{itemize}
\item Conservation de la solution stationnaire 
$$
h_0(x,y,z) = \dfrac{x+y+z}{\sqrt{3}}.
$$
\item La masse doit être conservée.
\end{itemize}
\end{exampleblock}
\end{frame}



\begin{frame}
\begin{figure}[htbp]
\begin{center}
\includegraphics[height=5cm]{rateBA_test3.png}
\end{center}
\caption{Erreur et taux de convergence pour le test stationnaire de l'équation d'advection non linéaire en fonction de $\Delta = a \Delta \xi$. Le pas de temps est donné par $\Delta t = 0.96 \Delta \xi / \pi$. Le temps final est $t=6$.}
\label{fig:benartzi_test3}
\end{figure}
\end{frame}





\begin{frame}
\begin{figure}[htbp]
\begin{center}
\includegraphics[height=4cm]{erreur_test3.png}
\includegraphics[height=4cm]{cons_test3.png}
\end{center}
\caption{ L'erreur en norme et l'erreur de conservation est représentée pour le test stationnaire de l'équation d'advection non linéaire. Le pas de temps est donné par $\Delta t = 0.96 \Delta \xi / \pi$. Le temps final est $t=6$. Le paramètre de la Cubed-Sphere est $N=32$.}
\label{fig:benartzi_test3_hist}
\end{figure}
\end{frame}














%% ****************************************************************************************************************************************

\begin{frame}{Equation Shallow Water}
\begin{block}{}
\begin{equation}
\left\lbrace
\begin{array}{rcl}
\dfrac{\partial h^{\star}}{\partial t} + \nabla_T \cdot \left( h^{\star} \mathbf{v} \right) & = & 0 \\
\dfrac{\partial \mathbf{v}}{\partial t} + \nabla_T \left( \dfrac{1}{2}|\mathbf{v}|^2 + gh \right) + \left( f + \zeta \right) \mathbf{n} \times \mathbf{v} & = & 0
\end{array}
\right.
\end{equation}
où  
\begin{itemize}
\item $h$ est l'épaisseur de fluide et $\mathbf{v}$ le champ de vitesse tangent,
\item $h^{\star}=h-h_s$ avec $h_s$ représentant les reliefs, en général on aura $h_s=0$,
\item $\mathbf{n}$ la normale extérieure, 
\item $\zeta = \left( \nabla_T \times \mathbf{v} \right) \cdot \mathbf{n}$ est la vorticité,
\item $f$ est le paramètre de Coriolis.
\end{itemize}
\end{block}
\end{frame}

%% ***************************************************************************************************************************************

\begin{frame}{Propriétés de conservation}
\begin{block}{}
Si $(h, \mathbf{v})$ est solution des équations Shallow water, les quantités suivantes sont conservées

\begin{itemize}
\item \textbf{masse :} 
$\gint_{\mathbb{S}^2_a} h(t, \mathbf{x}) d \sigma(\mathbf{x})$
\item \textbf{energie :}
$ \gint_{\mathbb{S}^2_a} \left( \dfrac{1}{2}g(h^2 - h_s^2) + \dfrac{1}{2} h | \mathbf{v} |^2 \right) d \sigma(\mathbf{x})$
\item \textbf{enstrophie potentielle :}
$\gint_{\mathbb{S}^2_a} \dfrac{\left( f + \zeta \right)^2}{2gh} d \sigma(\mathbf{x})$
\end{itemize}
\end{block}
\end{frame}

%% ***************************************************************************************************************************************

\begin{frame}{Solution stationnaire pour l'équation Shallow Water [Williamson \textit{et al.}, 1992]}

On a la solution stationnaire suivante :
\begin{exampleblock}{}
\begin{itemize}
\item $h = h_0 - \dfrac{1}{g} \left( a \Omega u_0 + \dfrac{u_0^2}{2} \right)\left( - \cos \lambda \cos \theta \sin \alpha + \sin \theta \cos \alpha \right)^2$
\item $\mathbf{v} = u \mathbf{e}_{\lambda}+ v \mathbf{e}_{\theta}$ avec :
\begin{equation*}
\left\lbrace \begin{array}{rcl}
 u & = & u_0 ( \cos \theta \cos \alpha + \cos \lambda \sin \theta \sin \alpha)\\
 v & = & -u_0 \sin \lambda \sin \alpha
 \end{array} \right.
\end{equation*}
\end{itemize}
\end{exampleblock}
\end{frame}

%% ***************************************************************************************************************************************

\begin{frame}{SW time independent test case [Williamson and al., 1992]}
\begin{figure}
\includegraphics[scale=0.24]{ref_7369088270_snapshot_err_color.png}
\includegraphics[scale=0.24]{ref_7369088270_snapshot_intermediaire598.png}
\end{figure}
\begin{itemize}
\item Solution stationnaire avec $\alpha=\pi/4$, la taille de grille est $6 \times 32 \times 32$, le temps final est $T=6$ jours.
\item Erreur relative sur $h$ au temps finale (gauche).
\item $h$ au temps final (droite),
\item $\Delta t \approx 10$ minutes.
\end{itemize}
\end{frame}

%% ***************************************************************************************************************************************

\begin{frame}{Analyse de l'erreur}
\begin{figure}
\includegraphics[scale=0.55]{rate_W2_pi4.png}
\end{figure}
\begin{itemize}
\item  On mesure l'erreur relative à $T=6$ jours  :
$$
\dfrac{\| \mathfrak{h}_0 - \mathfrak{h}^n \|}{\| \mathfrak{h}_0 \|}
$$
\item Convergence à l'ordre 4.
\end{itemize}
\end{frame}

%% ***************************************************************************************************************************************

\begin{frame}{Conservation des quantités}
\begin{figure}
\includegraphics[scale=0.5]{ref_7369088270_conservationA.png}
\end{figure}
\begin{itemize}
\item Erreur relative de conservation pour une grille : $6 \times 32 \times 32$ avec $\alpha=\pi/4$.
\end{itemize}
\end{frame}





























%% **************************************************************************************************************************************

\begin{frame}{Montagne isolée [Williamson and al., 1992]}

\begin{exampleblock}{}
Test similaire au précédent avec $\alpha = 0$ :

\begin{itemize}
\item $h = h_0 - \dfrac{1}{g} \left( a \Omega u_0 + \dfrac{u_0^2}{2} \right)(\sin \theta)^2$
\item $\mathbf{v} = u_0 \cos \theta \mathbf{e}_{\lambda}$.
\end{itemize}

Cette solution stationnaire est perturbée par une montagne isolée:

$$h_s = h_{s_0} \left( 1 - \dfrac{r}{R} \right)$$

avec $h_{s_0}=2000m$ and $r^2=min(R^2, (\lambda + \pi/2)^2+(\theta - \pi/6)^2)$, $R=\pi/9$.
\end{exampleblock}
\end{frame}

%% ***************************************************************************************************************************************


\begin{frame}{}
\begin{figure}
\includegraphics[scale=0.4]{ref_7372184434_snapshot_intermediaire499.png}
\end{figure}
\begin{itemize}
\item $h$ au bout de 5 jours sur une grille $6 \times 32 \times 32$.
\item $\Delta t \approx 10min$.
\end{itemize}
\end{frame}

%% ***************************************************************************************************************************************


\begin{frame}{}
\begin{figure}
\includegraphics[scale=0.4]{ref_7372184434_snapshot_intermediaire999.png}
\end{figure}
\begin{itemize}
\item $h$ au bout de 10 jours. La grille est $6 \times 32 \times 32$.
\item $\Delta t \approx 10min$.
\end{itemize}
\end{frame}

%% ***************************************************************************************************************************************


\begin{frame}{}
\begin{figure}
\includegraphics[scale=0.4]{ref_7372184434_snapshot_intermediaire1499.png}
\end{figure}
\begin{itemize}
\item $h$ au bout de 15 jours. La grille a pour paramètres : $6 \times 32 \times 32$.
\item $\Delta t \approx 10min$, $2140$ itérations en temps, $6146$ points sur la grille.
\item Les résultats sont très similaires à ceux obtenus par volumes finis ou par Galerkin discontinu d'ordre élevé.
\end{itemize}
\end{frame}

%% ***************************************************************************************************************************************

\begin{frame}{Conservation}
\begin{figure}
\includegraphics[scale=0.4]{ref_7369088145_conservationA.png}
\end{figure}
\begin{itemize}
\item La grille est $6 \times 32 \times 32$.
\item La masse et l'énergie sont conservées avec une erreur relative proche de $10^{-5}$.
\item L'enstrophie potentielle est conservée avec une erreur relative proche de $10^{-4}$.
\end{itemize}
\end{frame}






























%% ***************************************************************************************************************

%\begin{frame}{Instabilité Barotrope [J. Galewsky \textit{et al.}, 2004]}
%
%\begin{exampleblock}
%Une solution stationnaire (instableà est donnée par
%\begin{equation}
%\begin{array}{rcl}
%\bar{h}(\theta) & = & h_0 + \dfrac{1}{g}\gint^{\theta}_{-\pi/2} a u(\tau) \left[ f + \dfrac{\tan(\tau)}{a} u(\tau) \right] d \tau \\
%\mathbf{v}(\lambda,\theta) & = & u(\theta) \mathbf{e}_{\lambda}
%\end{array}
%\end{equation}
%avec :
%\begin{itemize}
%\item le paramètre de Coriolis $f = 2 \Omega \sin \theta$,
%\item $u(\theta)=\left\lbrace
%\begin{array}{ll}
%\dfrac{u_{max}}{e_n} \exp\left( \dfrac{1}{(\theta-\theta_0)(\theta-\theta_1)} \right) & \text{ if } \theta_0 \leq \theta \leq \theta_1 \\
%0 & \text{ else}
%\end{array}\right.$\\
% avec $e_n=C^{ste}$, $u_{max} = 80 ms^{-1}$, $\theta_0 = \pi/7$ and $\theta_1 = \pi/2 - \theta_0$.
%\end{itemize}
%\end{exampleblock}
%\end{frame}


%% ***************************************************************************************************************************************

%\begin{frame}{}
%\begin{exampleblock}{Perturbation}
%La fonction précédente est perturbée avec :
%\begin{itemize}
%\item zonal velocity:
%$$\mathbf{v}(\lambda,\theta) = u(\theta) \mathbf{e}_{\lambda}$$
%\item perturbation de $h$ par $\bar{h}$:
%$$h(\lambda,\theta) = \bar{h}(\lambda,\theta) + \hat{h} \cos \theta \exp \left[ - \left( \dfrac{\lambda}{\alpha} \right)^2 - \left( \dfrac{\theta_2 - \theta}{\beta} \right)^2 \right] \text{, } \hat{h}/\bar{h} \approx 1 \%$$
%avec $\theta_2 = \pi/4$, $\alpha = 1/3$ and $\beta = 1/15$.
%\end{itemize}
%\end{exampleblock}
%
%\begin{block}{}
%Ce test est difficile pour la Cubed-Sphere Cubed-Sphere:
%
%\begin{itemize}
%\item $h$ varie le long du panel (V)
%\item the initial perturbation is located at the boundary between panel (I) and panel (V).
%\end{itemize}
%\end{block}
%\end{frame}

%% ***************************************************************************************************************************************
%
%\begin{frame}{J. Galewsky and al. test case}
%\begin{figure}
%\href{run:ref_7367787500.avi}{\includegraphics[scale=0.4]{ref_7367767680_snapshot.png}} 
%\end{figure}
%\begin{itemize}
%\item Vorticity (6 days), grid: $6 \times 128 \times 128 \Rightarrow $ correct number of vortices!
%\item Results similar to high order conservative methods such as FV or DG.
%\end{itemize}
%\end{frame}
%
%%% ***************************************************************************************************************************************
%
%\begin{frame}{Conservation : mass, energy and potential enstrophy}
%\begin{figure}
%\includegraphics[scale=0.3]{ref_7367767680_massenergy.png}
%\includegraphics[scale=0.3]{ref_7367767680_enstrophy.png}
%\end{figure}
%\begin{itemize}
%\item Conservation of mass and energy, grid: $6 \times 128 \times 128$.
%\item Mass and energy are conserved up to $10^{-7}$ (relative error).
%\item Potential enstrophy is conserved up to $10^{-3}$ (relative error).
%\end{itemize}
%\end{frame}
%
%%% ***************************************************************************************************************************************
%
%\begin{frame}{Computational complexity}
%\begin{itemize}
%\item All the computations are performed with a Matlab sequential code on a work station (Intel(R) Xeon(R) CPU E5-2620 v2 @ 2.10GHz).
%\item Computational cost of $\nabla_T h$ $\approx 96N^2$ for $12N^2$ unknowns (due to the tridiagonal matrices).
%\item Typical CPU time : $1.5$ hours for $6$ days with $N=80$ ($2140$ it.).
%\end{itemize}
%\end{frame}
%
%%% ***************************************************************************************************************************************
%
%\begin{frame}{Conclusion and future works}
%\begin{itemize}
%\item Fully centered scheme in space.
%\item Explicit RK4 time scheme with filtering.
%\item Results similar to high order conservative schemes in terms of accuracy.
%\item Current work : mathematical convergence analysis for gradient, divergence, curl and the time discretization.
%\item Discretization of the Laplacian and Biharmonic to design more advanced artificial viscosity operators.
%\item Implicit time stepping required for slow wave problems (Rossby).
%\end{itemize}
%\end{frame}
%
%%% ***************************************************************************************************************************************
%
%\begin{frame}
%\begin{center}
%Thank you for your attention.
%\end{center}
%\end{frame}
%
%
%
%
%
%
%
%




\end{document}