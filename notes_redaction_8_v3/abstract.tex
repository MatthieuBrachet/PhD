 \begin{FrenchAbstract}
L'enjeu de la simulation de la dynamique atmosphérique et/ou océanographique a pris une importance accrue avec la question du réchauffement climatique.
Le modèle mathématique complet à simuler s'obtient en couplant les équations de la mécanique des fluides avec les équations de la thermodynamique.

Au 19ième siècle, le mathématicien Adhémar Barré de Saint-Venant formule un système d'équations aux dérivées partielles décrivant les mouvements d'un fluide soumis à la gravité et de faible épaisseur. Cette simplification des équations de Navier-Stokes permet de transformer un problème 3D en un problème 2D. Dans le contexte de la sphère en rotation, elles décrivent la réponse d'une couche mince de fluide soumise aux forces de gravité et de Coriolis. Elles permettent de décrire de nombreux phénomènes (ondes de Kelvin, ondes de Rossby, ...). Bien que représentant un problème simplifié, ces équations sont complexes et leur résolution nécessite des méthodes numériques adaptées.

L'objectif de cette thèse est d'étudier une méthode numérique de type différences finies pour résoudre ces équations, les équations Shallow Water, grâce à la grille Cubed-Sphere.

Dans la première partie, on introduit les notations et les schémas utiles à la résolution d'équations aux dérivées partielles sur la Cubed-Sphere. On étudie les schémas aux différences finies dans le contexte périodique pour approcher la dérivée première à différents ordres. Le schéma utilisé sur la sphère est un schéma hermitien d'ordre 4. Nous introduisons aussi les schémas de filtrage. Ces derniers sont consistants avec l'identité et permettent de supprimer les modes oscillants. Ces schémas permettent des approximations en espace. La discrétisation en temps est faite par un algorithme de Runge-Kutta d'ordre 4 explicite couplé à l'opérateur de filtrage. Nous étudions l'algorithme pour la résolution de l'équation de transport et de l'équation des ondes. 

La seconde partie est dédiée au maillage sur la sphère ainsi qu'à la construction des opérateurs approchés. Le maillage utilisé est la Cubed-Sphere, introduit en 1972 par Robert Sadourny. Il s'agit du maillage des faces d'un cube projeté sur la sphère. Chaque face est appelé panel. Il y a de nombreuses symétries entre les panels, ce qui permet de construire un produit scalaire vérifiant l'orthogonalité d'un grand nombre d'harmoniques sphériques. De plus, on construit des formules de quadrature précises sur ce maillage. Les points sur un panel sont des portions de grands cercles. En complétant les données sur les grands cercles à l'aide de splines cubiques, on construit des opérateurs approchés de la divergence, du gradient et de la vorticité. Ces opérateurs utilisent les schémas aux différences finis périodiques et sont analysés.

La troisième partie concerne la résolution numérique d'équations sur la sphère. Les expériences numériques concernent l'équation d'advection, l'équation Shallow Water linéarisée et l'équation Shallow Water. La résolution se fait par la méthode des lignes en couplant les opérateurs différentiels discrets avec un schéma de RK4 muni d'un opérateur de filtrage. Les tests sont issus de la littérature classique. Sur certains, une solution analytique est disponible. On compare la solution exacte et la solution donnée par l'algorithme. Les erreurs mesurées confirment la précision attendue. Lorsqu'il n'y a pas de solution analytique connue, nous comparons nos résultats numériques avec ceux obtenus par d’autres méthodes. Nous vérifions la conservation de la masse et de l'énergie. 

Dans notre contexte, les simulations en temps long jouent un rôle important. Les résultats obtenus sur temps longs sont ceux attendus. Il serait intéressant d'utiliser un algorithme de résolution en temps implicite pour effectuer ce type de simulations. C'est l'une des perspectives de ce travail.

A plus long terme, l'objectif est de simuler un modèle en 3D. Il faudra coupler un tel modèle avec les équations de la thermodynamique (modèle GCM).


    \KeyWords{Equation Shallow Water, Cubed-Sphere, schémas compacts, discrétisation en temps.}
  \end{FrenchAbstract}
  
  
  
  
  
  
  
  
  \begin{EnglishAbstract}
The challenge to simulate the atmospheric and/or oceanic fluid dynamics has become crucial with the climate change problems. The full mathematical model to simulate consists in the coupling of fluid dynamics with thermodynamics.
  
In the 19-th century, Adhémar Barré de Saint-Venant first formulated the equations describing the dynamic of a fluid subject to gravity and bottom topography. This equation can be considered as a bidimensional simplification of the 3D Navier-Stokes system. When expressed in the context of the rotating sphere, this equation describes the reaction of a fluid thin layer subject to the gravitational and Coriolis forces. It permits to describe many wave phenomena (Kelvin waves, Rossby waves, ...). Although representing a simplified problem, this equation is difficult to solve and its numerical resolution requires suitable numerical schemes.

The goal of this thesis is to study a particular finite difference scheme to solve this equations, (also called  Shallow Water equation) with the Cubed-Sphere grid. 

In the first part, we introduce notations and schemes used to solve partial differential equation on the Cubed-Sphere. We study finite difference schemes in the periodic context. They allow us to approximate the first derivative with different order of accuracy. The scheme used on the sphere is the hermitian scheme of order 4. We introduce filtering schemes. These ones are consistent with the identity with high order of accuracy and allow us to remove oscillating modes. These schemes allow us space approximation. The time discretization is made with explicit 4-th order Runge Kutta algorithm coupled to a filtering operator. We study this algorithm to solve the advection equation and the wave equation. The accuracy is proved and the stability is studied. The filtering operator allows us to reduce parasitic oscillations which can appear during the resolution of hyperbolic equations by a centered scheme.

The second part is devoted to the spherical grid and to approximation operators. The mesh used is the Cubed-Sphere. This mesh was introduced in 1972 by Robert Sadourny. It is the mesh of faces of a cube projected on the sphere. Each face of this mesh, on the sphere, is named panel. There are multiple symetries between panels, that allow us to build a scalar product. This product permits us to check the orthogonality for a large number of spherical harmonics on the mesh. Further more, we build accurate quadrature formulas on this grid. Panel points are portions of great circles. We build approximated operators of the divergence, the gradient and the vorticity by completing the data on great circles with cubic spline. These operators use periodic finite difference schemes and are analyzed.

The third part is devoted to the numerical resolution of partial differential equations on the sphere. Experiments concerne advection equation, linearized Shallow Water equation and Shallow Water equation. The resolution is made with the lines method by coupling discrete differential operators, fourth order Runge-Kutta algorithm and filtering operator. Numerical tests are from the classical litterature. For some, an analytical solution is available. Errors are weak and allow us to confirm the expected accuracy. For numerical tests without analytical solution, we compare our numerical results with those obtained by other methods. Furthermore, we check the conservation of mass and energy.

Our context is the one of simulations over particularly long time. This is why we perform tests to predict a solution in a distant future. We are satisfied by the behaviours observed. It would be interesting to use implicit time algorithm. It is one of the perspectives of this work.

In the longer term, the goal is to simulate a model in dimension 3 coupled with the equations of thermodynamic (GCM model).
  
  
    \KeyWords{Shallow Water equation, Cubed-Sphere, compact scheme, time discretisation.}
  \end{EnglishAbstract}