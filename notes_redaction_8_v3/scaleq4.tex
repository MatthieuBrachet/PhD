% equation scalaire

\chapter{Equations d'advection sphériques}
\label{chap:advection}

\section{Equations d'advection linéaires sur la sphère}

Dans cette section, on s'intéresse à l'équation d'advection \eqref{eq:advection_sphere} :
\begin{equation}
\left\lbrace
\begin{array}{rcl}
\dfrac{\partial h}{\partial t} + \mathbf{c}(t,\mathbf{x}) \cdot \nabla_T h & = & 0 \\
h(0,\mathbf{x}) & = & h_0(\mathbf{x})
\end{array}
\right. \text{ pour tout } \mathbf{x} \in \mathbb{S}_a^2 \text{ et } t \geq 0,
\label{eq:advection_sphere}
\end{equation}
sur la sphère $\mathbb{S}_a^2$. Le rayon terrestre est $a = 6 371 220 \si{m}$.
La fonction $\mathbf{c} : (t,\mathbf{x}) \in \mathbb{R}^+ \times \mathbb{S}_a^2 \mapsto \mathbf{c}(t,\mathbf{x}) \in \mathbb{T}_{\mathbf{x}} \mathbb{S}_a^2$ désigne un champ de vecteurs tangents à la sphère $\mathbb{S}_a^2$.

\subsection{Résolution numérique}

L'équation \eqref{eq:advection_sphere} est résolue par la méthode des lignes en utilisant l'opérateur gradient discret $\nabla_{T,\Delta}$ (définition \ref{def:gradient_disc}). On pose $J_{\Delta}$ l'application agissant sur une fonction de grille $\bh$ donnée au temps $t$ par
\begin{equation}
J_{\Delta}(t, \bh) = - \mathbf{c}(t,\cdot)^* \cdot \nabla_{T,\Delta} \bh.
\end{equation}
En chaque point $\mathbf{x}_{i,j}^{(k)}$ de la Cubed-Sphere, on a
\begin{equation}
J_{\Delta}(t, \bh)_{i,j}^{(k)} = - \mathbf{c}(t,\mathbf{x}_{i,j}^{(k)}) \cdot \left(\nabla_{T,\Delta} \bh\right)_{i,j}^{(k)}
\end{equation}
pour tous $-N/2 \leq i,j \leq N/2$ et $(k) = (I), ..., (VI)$.
La résolution en temps se fait par un algorithme de type $\RK4$ couplé à un opérateur de filtrage $\mathcal{F}$ donné par \eqref{eq:operateur_filtrage}. Cet algorithme est analogue à l'algorithme \ref{alg:RK4_transport1d}. Il est donné par :

\begin{center}
\begin{minipage}[H]{12cm}
  \begin{algorithm}[H]
    \caption{: Schéma en temps RK4 avec étape de filtrage pour l'équation \eqref{eq:advection_sphere} }\label{alg:RK4_transportSa}
    \begin{algorithmic}[1]
    \State $h^0 = h_0^*$ connu,
    \For{$n=0,1, \ldots$}
             \State  $K^{(1)} = J_{\Delta}(t^n, \bh^n)$,
             \State  $K^{(2)} = J_{\Delta}\left(t^n + \frac{\Delta t}{2}, \bh^n + \frac{\Delta t}{2} K^{(1)} \right)$,
             \State  $K^{(3)} = J_{\Delta}\left(t^n + \frac{\Delta t}{2}, \bh^n + \frac{\Delta t}{2} K^{(2)} \right)$,
             \State  $K^{(4)} = J_{\Delta}\left(t^n + \Delta t \bh^n + \Delta t K^{(3)} \right)$,  
             \State  $\bh^{n+1} = \mathcal{F}\left( \bh^n  + \dfrac{\Delta t}{6} \left( K^{(1)} + 2 K^{(2)} + 2 K^{(3)} + K^{(4)} \right) \right)$.
            \EndFor
    \end{algorithmic}
    \end{algorithm}
\end{minipage}
\end{center}

La donnée $\bh^n$ désigne une approximation de $(h(t^n,\cdot))^*$ solution au temps $t^n = n \Delta t$ de \eqref{eq:advection_sphere}.
La solution de l'équation \eqref{eq:advection_sphere} étant connue dans les tests effectués, nous mesurons l'erreur relative au temps $t^n$ par
\begin{equation}
e_l^n = \dfrac{\| \bh^n - h(t^n,\cdot)^* \|_l}{\| h(t^n,\cdot)^* \|_l}
\end{equation}
où $l \in \left\lbrace 1, 2, \infty \right\rbrace$ et $\| \cdot \|_l$ désigne la norme $1$, $2$ ou $\infty$ calculée par
\begin{equation}
\| \mathfrak{q} \|_l = \left( Q(|\mathfrak{q}|^l ) \right)^{1/l} \text{ avec } l=1,2
\end{equation}
et $Q$ un opérateur de quadrature numérique introduit dans \cite{Portelenelle2018}, $Q= Q_{1/3}$. Pour la norme $\| \cdot \|_{\infty}$, on note
\begin{equation}
\|  \bq \|_{\infty} = \max_{-N/2 \leq i,j \leq N/2} \max_{(k) = (I) ... (VI)} |\mathfrak{q}_{i,j}^{(k)}|,
\end{equation}
pour $\mathfrak{q}$ fonction de grille.

Dans les simulations numériques effectuées, le pas de temps $\Delta t$ est proportionnel à $\Delta \xi$. La relation de proportionnalité est donnée par la condition
\begin{equation}
\CFL = \dfrac{u_0 \Delta t}{a \Delta \xi} =C^{\text{ste}},
\end{equation}
la valeur de $u_0$ est donnée par le contexte lors du test effectué.















\subsection{Test de rotation solide}

On considère d'abord le test numéro 1 de \cite{Williamson1992}. Il s'agit d'une rotation sans déformation de la condition initiale au cours du temps autour d'un axe incliné.

On considère $(\lambda, \theta)$ les coordonnées longitude-latitude associées au pôle Nord noté $\mathbf{N}$ et $(\lambda', \theta')$ les coordonnées longitude-latitude associées à un pôle Nord déplacé en $\mathbf{P}$ de coordonnées $(\lambda_P, \theta_P)$. La proposition suivante énonce le lien entre $(\lambda, \theta)$ et $(\lambda', \theta')$ :

\begin{proposition}
Le changement de variables $(\lambda, \theta) \mapsto (\lambda', \theta')$ est donné par :
\begin{equation}
\label{from classic to prime}
\left\lbrace 
\begin{array}{rcl}
\theta' & = & \arcsin \left[ \sin( \theta) \sin(\theta_P) + \cos( \theta ) \cos( \theta_P) \cos( \lambda - \lambda_P ) \right] \\
\lambda' & = & \arctan \left[ \dfrac{\cos ( \theta) \sin( \lambda - \lambda_P)}{\cos( \theta) \cos( \lambda - \lambda_P) \sin( \theta_P) - \sin( \theta) \cos( \theta_P)} \right].
\end{array}
\right.
\end{equation}
Le changement de variables inverse $(\lambda', \theta')\mapsto (\lambda, \theta)$ est donné par :
\begin{equation}
\label{from prime to classic}
\left\lbrace 
\begin{array}{rcl}
\theta & = & \arcsin \left[ \sin( \theta') \sin(\theta_P) - \cos( \theta' ) \cos( \theta_P) \cos( \lambda' ) \right] \\
\lambda & = & \lambda_P + \arctan \left[ \dfrac{\cos ( \theta') \sin( \lambda ')}{\sin( \theta') \cos( \theta_P) + \cos ( \theta') \cos( \lambda') \sin ( \theta_P)} \right].
\end{array}
\right.
\end{equation}
\end{proposition}

\begin{proof}
Un point $\mathbf{x} \in \mathbb{S}_a^2$ a pour coordonnées $(\lambda, \theta)$ en longitude-latitude associées au pôle Nord et $(\lambda', \theta')$ en coordonnées latitude-longitude associées à un pôle déplacé en $P$. Le lien entre ces coordonnées se fait par rotations successives.
En considérant les rotations liées au changement de pôle Nord, on a
\begin{align}
\begin{bmatrix}
\cos \theta' \cos \lambda' \\ \cos \theta' \sin \lambda' \\ \sin \theta'
\end{bmatrix} & = 
\begin{bmatrix}
\cos (\theta_P - \pi/2) & 0 & \sin (\theta_p - \pi/2) \\
0 & 1 & 0 \\
- \sin (\theta_P - \pi/2 ) & 0 & \cos (\theta_P - \pi/2)
\end{bmatrix}
\begin{bmatrix}
\cos \lambda_P & \sin \lambda_P & 0 \\
- \sin \lambda_P & \cos \lambda_P & 0 \\
0 & 0 & 1 
\end{bmatrix}
\begin{bmatrix}
\cos \theta \cos \lambda \\ \cos \theta \sin \lambda \\ \sin \theta
\end{bmatrix} \\
& = \begin{bmatrix}
\sin \theta_P \cos \lambda_P & \sin \theta_P \sin \lambda_P & - \cos \theta_P \\
- \sin \lambda_P & \cos \lambda_P & 0 \\
\cos \theta_P \cos \lambda_P & \cos \theta_P \sin \lambda_P & \sin \theta_P
\end{bmatrix}
\begin{bmatrix}
\cos \theta \cos \lambda \\ \cos \theta \sin \lambda \\ \sin \theta
\end{bmatrix}.
\label{eq:chgmt_coord_proof1}
\end{align}
La seconde ligne de \eqref{eq:chgmt_coord_proof1} donne
\begin{align*}
\cos \theta' \in \lambda'  & = - \sin \lambda_P \cos \theta \cos \lambda + \cos \lambda_P \cos \theta \sin \lambda \\
	& = \cos \theta ( \cos \lambda_P \sin \lambda - \sin \lambda_P \cos \lambda) \\
	& = \cos \theta \sin (\lambda - \lambda_P)
\end{align*}
ce qui donne l'équation :
\begin{equation}
\cos \theta' \sin \lambda' = \cos \theta \sin (\lambda - \lambda_P).
\label{eq:chgmt_coord_proof2}
\end{equation}
De la même manière, la troisième ligne de \eqref{eq:chgmt_coord_proof1} permet d'obtenir
\begin{equation}
\sin \theta' = \cos \theta_P \cos \theta \cos (\lambda - \lambda_P) + \sin \theta_P \sin \theta .
\label{eq:chgmt_coord_proof3}
\end{equation}
D'après la première ligne de \eqref{eq:chgmt_coord_proof1} :
\begin{align*}
\cos \theta' \cos \lambda' & = \sin \theta_P \cos \lambda_P \cos \theta \cos \lambda + \sin \theta_P \sin \lambda_P \cos \theta \sin \lambda - \cos \theta_P \sin \theta \\
	& = \sin \theta_P \cos \theta \cos ( \lambda - \lambda_P) - \cos \theta_P \sin \theta \\
	& = \sin \theta_P \dfrac{\sin \theta' - \sin \theta_P \sin \theta}{\cos \theta_P} - \cos \theta_P \sin \theta \text{ d'après \eqref{eq:chgmt_coord_proof3} } \\
	& = \dfrac{\sin \theta_P \sin \theta' - \sin \theta}{\cos \theta_P}.
\end{align*}
D'où une troisième équation
\begin{equation}
\sin \theta = \sin \theta_P \sin \theta' - \cos \theta' \cos \lambda' \cos \theta_P .
\label{eq:chgmt_coord_proof4}
\end{equation}
Les équations démontrées sont les suivantes :
\begin{equation}
\label{eq:coord_rot}
\left\lbrace 
\begin{array}{rcl}
\sin ( \theta ) & = & \sin( \theta_P) \sin( \theta') - \cos( \theta_P) \cos( \theta') \cos( \lambda' ) \\
\sin( \theta' ) & = & \sin( \theta) \sin(\theta_P) + \cos( \theta ) \cos( \theta_P) \cos( \lambda - \lambda_P ) \\
\cos( \theta ) \sin( \lambda - \lambda_P) & = & \cos( \theta' ) \sin( \lambda' ).
\end{array}
\right.
\end{equation}
En utilisant (\ref{eq:coord_rot}.b), la formule suivante est immédiate :
\begin{equation}
\theta' = \arcsin \left[ \sin( \theta) \sin(\theta_P) + cos( \theta ) cos( \theta_P) cos( \lambda - \lambda_P ) \right] .
\end{equation}
De plus, (\ref{eq:coord_rot}.a) et (\ref{eq:coord_rot}.c) donnent :
\begin{equation}
\left\lbrace 
\begin{array}{rcl}
\cos( \theta ) \sin( \lambda - \lambda_P) & = & \cos( \theta' ) \sin( \lambda' ) \\
\cos( \theta ) \cos( \lambda - \lambda_P) & = & \frac{\sin( \theta' ) \sin ( \theta_P ) - \sin( \theta )}{\cos( \theta_P)}.
\end{array}
\right.
\end{equation}
Or :
\begin{equation*}
\begin{array}{rcl}
\cos( \theta ) \cos( \lambda - \lambda_P) & = & \dfrac{\sin( \theta' ) \sin ( \theta_P ) - \sin( \theta )}{\cos( \theta_P)} \\
 & = & \dfrac{\sin( \theta) (\sin^2 ( \theta_P) -1 )}{\cos( \theta_P)} + \cos( \theta ) \cos( \lambda- \lambda_P) \sin( \theta_P)\\
 & = & \cos ( \theta) \cos( \lambda - \lambda_P) \sin( \theta_P ) - \sin( \theta) \cos ( \theta_P ).
\end{array}
\end{equation*}
En utilisant 
\begin{equation}
\tan ( \lambda' ) =  \dfrac{\cos( \theta' ) \sin(  \lambda' ) }{\cos( \theta' ) \cos(  \lambda' )}
\end{equation}
on obtient le changement de coordonnées \eqref{from classic to prime} :
\begin{equation}
\left\lbrace 
\begin{array}{rcl}
\theta' & = & \arcsin \left[ \sin( \theta) \sin(\theta_P) + \cos( \theta ) \cos( \theta_P) \cos( \lambda - \lambda_P ) \right] \\
\lambda' & = & \arctan \left[ \dfrac{\cos ( \theta) \sin( \lambda - \lambda_P)}{\cos( \theta) \cos( \lambda - \lambda_P) \sin( \theta_P) - \sin( \theta) \cos( \theta_P)} \right].
\end{array}
\right.
\end{equation}
Inversement et par une démonstration similaire, on obtient
\begin{equation}
\left\lbrace 
\begin{array}{rcl}
\theta & = & \arcsin \left[ \sin( \theta') \sin(\theta_P) - \cos( \theta' ) \cos( \theta_P) \cos( \lambda' ) \right] \\
\lambda & = & \lambda_P + \arctan \left[ \dfrac{\cos ( \theta') \sin( \lambda ')}{\sin( \theta') \cos( \theta_P) + \cos ( \theta') \cos( \lambda') \sin ( \theta_P)} \right].
\end{array}
\right.
\end{equation}
\end{proof}

\begin{proposition}
La solution de l'équation d'advection sur la sphère \eqref{eq:advection_sphere} avec
\begin{equation}
\mathbf{c}(\mathbf{x}) = \mathbf{c}(\lambda,\theta) = u_0 \cos \theta \mathbf{e}_{\lambda}
\end{equation}
est donnée pour $t \geq 0$ par
\begin{equation}
h(\mathbf{x}, t ) = h(\lambda, \theta, t ) = h_0(\lambda- \omega_s t , \theta)
\end{equation}
avec $u_0 = a \omega_s$ et $\mathbf{x}$ un point de la sphère $\mathbb{S}_a^2$ de coordonnées longitude-latitude $(\lambda, \theta)$.
On a également
\begin{equation}
h(\mathbf{x},t) = h_0(R_{-t} \mathbf{x})
\end{equation}
où $R_{-t}$ est la matrice de rotation
\begin{equation}
R_{-t} = \begin{pmatrix}
\cos (- \omega_s t) & - \sin (- \omega_s t) & 0 \\
\sin (- \omega_s t) & \cos (- \omega_s t)   & 0 \\
0 & 0 & 1
\end{pmatrix} .
\end{equation}
\label{prop:rotsolide1}
\end{proposition}

\begin{proof}
On résout cette équation par la méthode des caractéristiques. Soit $X : t \in \mathbb{R}^+ \mapsto X(t)=(\lambda(t), \theta(t)) \in \mathbb{S}_a^2$ solution de 
\begin{equation}
\left\lbrace
\begin{array}{rcl}
\dfrac{dX}{dt} & = & \mathbf{c}(X(t)) \\
X(0) & = & \mathbf{x}_0 = (\lambda_0, \theta_0).
\end{array}
\right.
\label{eq:cauchy_sphere1}
\end{equation}
D'après le théorème de Cauchy-Lipschitz, il existe une telle courbe $X$ solution maximale.

Si $h$ est solution de \eqref{eq:advection_sphere}, $h$ est constante le long de $X$. En effet
\begin{align*}
\dfrac{dh}{dt}(X(t),t) & = \dfrac{\partial h}{\partial t} (X(t),t) + \dfrac{d X}{dt}(t) \cdot \nabla_T h(X(t),t) \\
& = \dfrac{\partial h}{\partial t} (X(t),t) + \mathbf{c}(X(t)) \cdot \nabla_T h(X(t),t) \\
& = 0.
\end{align*}
En exprimant $X$ en coordonnée latitude-longitude, on obtient la formule
\begin{equation}
\dfrac{dX}{dt} = a \cos \theta \dfrac{d \lambda}{dt} \mathbf{e}_{\lambda} + a \dfrac{d \theta}{d t} \mathbf{e}_{\theta}.
\end{equation}
En identifiant les termes dans le problème de Cauchy, on obtient
\begin{equation}
\left\lbrace
\begin{array}{rcl}
\dfrac{d \lambda}{d t} & = & \omega_s \\
\dfrac{d \theta}{d t}  & = & 0
\end{array}
\right.
\end{equation}
d'où
\begin{equation}
\left\lbrace
\begin{array}{rcl}
\lambda(t) & = & \omega_s t + \lambda_0 \\
\theta(t)  & = & \theta_0.
\end{array}
\right.
\end{equation}
La fonction $h$ est constante le long de la caractéristique $t \mapsto (\lambda(t), \theta(t))$, donc
\begin{equation}
h(\lambda(t), \theta(t), t) = h_0(\lambda_0 , \theta_0 ) = h_0(\lambda - \omega_s t, \theta ),
\end{equation}
ce qui termine la preuve.
\end{proof}

Si $(\lambda_P, \theta_P) = (\pi, \pi / 2 - \alpha)$, alors la matrice de rotation pour passer d'un système de coordonnées à l'autre est
\begin{equation}
P_{\alpha} = \begin{bmatrix}
- \cos \alpha & 0 & - \sin \alpha \\
0 & -1 & 0 \\
- \sin \alpha & 0 & \cos \alpha
\end{bmatrix}
\end{equation}
et on a le théorème suivant :

\begin{theoreme}
La solution de l'équation \eqref{eq:advection_sphere} avec
\begin{equation}
\mathbf{c}_s(\mathbf{x}) = \mathbf{c}(\mathbf{x}) = \mathbf{c}(\lambda, \theta) = u_0 \left( \cos \theta \cos \alpha + \sin \theta \cos \lambda \sin \alpha \right) \mathbf{e}_{\lambda} - u_0 \sin \lambda \sin \alpha \mathbf{e}_{\theta}
\label{eq:rot_solide_1}
\end{equation}
est donnée pour $t \geq 0$ par
\begin{equation}
h(\mathbf{x}, t ) = h_0(P_{\alpha}^{-1}R_{-t}P_{\alpha} \mathbf{x})
\end{equation}
où $R_{-t}$ est la matrice de rotation
\begin{equation}
R_{-t} = \begin{bmatrix}
\cos (- \omega_s t) & - \sin (- \omega_s t) & 0 \\
\sin (- \omega_s t) & \cos (- \omega_s t)   & 0 \\
0 & 0 & 1
\end{bmatrix}
\end{equation}
, où $\omega_s = u_0/a$ et $\mathbf{x}$ est un point de la sphère $\mathbb{S}_a^2$.
\end{theoreme}

\begin{proof}
La rotation $P_{\alpha}$ est inversible donc $\mathbf{x} \mapsto P_{\alpha} \mathbf{x}$ réalise une bijection de $\mathbb{S}_a^2$ dans $\mathbb{S}_a^2$.

Soit $g : (\mathbf{x},t) \in \mathbb{S}_a^2 \times \mathbb{R}^+ \mapsto g(\mathbf{x},t)$ la solution de 
\begin{equation}
\left\lbrace
\begin{array}{rcl}
\dfrac{\partial g}{\partial t} + \mathbf{c}_g \cdot \nabla_T g & = & 0\\
g(\mathbf{x},0) & = & h_0(P_{\alpha}^{-1}\mathbf{x}).
\end{array}
\right.
\label{eq:sol_g_proof}
\end{equation}
D'après la proposition \ref{prop:rotsolide1}, en tout point $\mathbf{x} \in \mathbb{S}_a^2$, on a
\begin{equation}
g(\mathbf{x},t) = h_0(R_{-t} P_{\alpha}^{-1}\mathbf{x}).
\end{equation}

En posant $h(\mathbf{x},t) = g(P_{\alpha} \mathbf{x} ,t)$, alors $h$ est solution de \eqref{eq:advection_sphere} avec \eqref{eq:rot_solide_1}, en effet
\begin{align*}
\dfrac{\partial h}{\partial t} (\mathbf{x},t) + \mathbf{c}(\mathbf{x}) \cdot \nabla_T h(\mathbf{x},t) & = \dfrac{\partial g}{\partial t} (P_{\alpha}\mathbf{x},t) + P_{\alpha} \mathbf{c}(\mathbf{x}) \cdot \nabla_T g (P_{\alpha}\mathbf{x},t) \\
	& = \dfrac{\partial g}{\partial t} (P_{\alpha}\mathbf{x},t) + P\mathbf{c}_g(\mathbf{x}) \cdot \nabla_T g (P_{\alpha}\mathbf{x},t) \\
	& = \dfrac{\partial g}{\partial t} (P_{\alpha}\mathbf{x},t) + u_0 \cos \theta \mathbf{e}_{\lambda} \cdot \nabla_T g (P_{\alpha}\mathbf{x},t) \\
	& = 0,
\end{align*}
car $g$ est solution du problème \eqref{eq:sol_g_proof}.
De plus, on a bien $h(\mathbf{x},0) = g(P_{\alpha} \mathbf{x} ,0) = h_0(\mathbf{x})$, donc en tout point $\mathbf{x} \in \mathbb{S}_a^2$, on a
\begin{align*}
h(\mathbf{x},t) & = g(P_{\alpha} \mathbf{x}, t) \\
	& = g_0(R_{-t} P_{\alpha} \mathbf{x} ) \\
	& = h_0(P_{\alpha}^{-1}R_{-t}P_{\alpha} \mathbf{x})
\end{align*}
ce qui conclut la preuve.
\end{proof}

Le test numéro 1 présenté dans \cite{Williamson1992} consiste à comparer la solution numérique obtenue pour la résolution de \eqref{eq:advection_sphere} avec le champ de vitesse $\mathbf{c}$ donné par \eqref{eq:rot_solide_1} et la donnée initiale donnée par la fonction localisée :
\begin{equation}
h_0(\lambda, \theta) = \left\lbrace
\begin{array}{ccl}
(h_0/2) (1 + \cos (\pi r/R) ) & \text{ si } & r<R \\
0 & \text{ si } & r \geq R
\end{array}
\right.
\label{eq:initial_solid_body}
\end{equation}
avec $h_0 = 1000 \si{m}$ et $R=a/3$. La fonction $r$ représente la distance sur la sphère entre le point de coordonnées $(\lambda, \theta)$ et le point de coordonnées $(\lambda_C, \theta_C)$. Elle est donnée par
\begin{equation}
r = a \arccos \left( \sin \theta_C \sin \theta + \cos \theta_C \cos \theta \cos (\lambda - \lambda_C) \right),
\end{equation}
$(\lambda_C, \theta_C) = (3 \pi / 2 , 0)$ est la position initiale du Bump. Dans ce test, on a $\omega_s = u_0/a = 2 \pi/12$ jours$^{-1}$. Il s'agit d'une condition initiale de classe $\mathcal{C}^1$, elle n'est pas de classe $\mathcal{C}^2$. Dans les sections \ref{sec:propag_vortex}, les tests sont consacrés à des solutions plus régulières. Des tests existent avec des solutions initiales moins régulières (voir \cite{Nair2010}), mais ici, nous nous limitons à \eqref{eq:initial_solid_body}.

Les Tables \ref{tab:rate1_bump} et \ref{tab:rate2_bump} donnent l'erreur obtenue sur 12 jours avec différentes tailles de grilles avec $\alpha=0$ et $\alpha = \pi/4$. Le taux de convergence est compris entre $1.5$ et $2.5$. La convergence d'ordre $3$ au moins était attendue en supposant la solution suffisamment régulière ce qui n'est pas le cas ici puisque la solution est seulement de classe $\mathcal{C}^1$.

Dans la Figure \ref{fig:erreur_bump}, on observe la localisation spatiale de l'erreur $h(t^n, \cdot)^* - \bh^n$ après une rotation complète de la solution initiale, au temps $t=12$ jours ainsi que l'erreur relative au cours du temps pour $N=40$. L'erreur est principalement localisée là où la fonction $h$ est la moins régulière.

\begin{table}[htbp]
\begin{center}
\begin{tabular}{|c||c|c|c|}
\hline
\textbf{N}  & $\mathbf{e_1}$ & $\mathbf{e_2}$ & $\mathbf{e_{\infty}}$\\
\hline
\hline
$40$  & $4.3043 (-2)$ & $2.4784 (-2)$ & $2.0921 (-2)$ \\
$50$  & $2.4403 (-2)$ & $1.4917 (-2)$ & $1.3748 (-2)$ \\
$60$  & $1.5367 (-2)$ & $9.9131 (-3)$ & $1.0476 (-3)$ \\
$80$  & $7.5508 (-3)$ & $5.3960 (-3)$ & $6.2646 (-3)$ \\
$100$  & $4.3709 (-3)$ & $3.3958 (-3)$ & $4.4360 (-3)$ \\
$150$  & $1.6538 (-3)$ & $1.4917 (-3)$ & $2.2885 (-3)$ \\
\hline 
\hline
\textbf{Ordre estimé}& $2.47$ & $2.12$ & $1.67$\\
\hline
\end{tabular}
\end{center}
\caption{Erreur et taux de convergence pour la rotation solide sur l'équation \eqref{eq:advection_sphere} en norme $1$, $2$ et $\infty$ pour $\alpha = 0$ et $\CFL = 0.7$. Le filtre utilisé est le filtre d'ordre 10.}
\label{tab:rate1_bump}
\end{table} 

\begin{table}[htbp]
\begin{center}
\begin{tabular}{|c||c|c|c|}
\hline
\textbf{N}  & $\mathbf{e_1}$ & $\mathbf{e_2}$ & $\mathbf{e_{\infty}}$\\
\hline
\hline
$40$  & $3.7638 (-2)$ & $2.0633 (-2)$ & $1.4639 (-2)$ \\
$50$  & $2.1323 (-2)$ & $1.2496 (-2)$ & $1.0056 (-2)$ \\
$60$  & $1.3546 (-2)$ & $8.4518 (-3)$ & $7.3167 (-3)$ \\
$80$  & $6.6905 (-3)$ & $4.6505 (-3)$ & $4.6060 (-3)$ \\
$100$  & $3.9119 (-3)$ & $2.9448 (-3)$ & $3.1809 (-3)$ \\
$150$  & $1.4922 (-3)$ & $1.3019 (-3)$ & $1.6341 (-3)$ \\
\hline 
\hline
\textbf{Ordre estimé}& $2.44$ & $2.08$ & $1.66$\\
\hline
\end{tabular}
\end{center}
\caption{Erreur et taux de convergence pour la rotation solide sur l'équation \eqref{eq:advection_sphere} en norme $1$, $2$ et $\infty$ pour $\alpha = \pi / 4$ et $\CFL = 0.7$. Le filtre utilisé est le filtre d'ordre 10.}
\label{tab:rate2_bump}
\end{table} 

\begin{figure}[htbp]
\begin{center}
\includegraphics[height=5cm]{rate_bump_0.png}
\includegraphics[height=5cm]{rate_bump_pi4.png}
\end{center}
\caption{Taux de convergence pour la rotation solide sur l'équation \eqref{eq:advection_sphere} en normes $\| \cdot\|_1$, $\| \cdot\|_2$ et $\| \cdot\|_{\infty}$ en fonction de $\Delta = a \Delta \xi$ pour $\alpha = 0$ (gauche) et $\alpha = \pi / 4$ (droite) et $\CFL = 0.7$. Le taux de convergence est quasiment identique pour $\alpha = \pi/4$ et $\alpha=0$. Le filtre utilisé est le filtre d'ordre 10.}
\label{fig:rate_bump}
\end{figure}

\begin{figure}[htbp]
\begin{center}
\includegraphics[height=5cm]{ref_7371377001_normerreur_test_0.png}
\includegraphics[height=5cm]{ref_7371377001_erreur_test_0.png}
\end{center}
\caption{Erreur relative pour l'équation \eqref{eq:advection_sphere} en norme $1$, $2$ et $\infty$ pour $\alpha = \pi/4$ (gauche) et localisation spatiale de l'erreur au temps $t=12$ jours (droite) avec $\CFL = 0.7$ et $N=40$. Le filtre utilisé est le filtre d'ordre 10.}
\label{fig:erreur_bump}
\end{figure}

Les valeurs obtenues par le schéma sont comparables à celles obtenues par \cite{Ullrich2010, Ullrich2011} à l'aide d'un schéma volumes finis d'ordre 4. La comparaison est donnée dans la Table \ref{tab:comp_ullrich_bump}. On constate que les valeurs des erreurs sont tout à fait comparables.

\begin{table}[htbp]
\begin{center}
\begin{tabular}{|cc||cc||cc||cc|}
\hline 
 & & $\mathbf{e}_1$ &   & $\mathbf{e}_2$ &   & $\mathbf{e}_{\infty}$ &   \\ 
\hline 
$\CFL$ & $\alpha$ & \cite{Ullrich2010} & Algo. \ref{alg:RK4_transportSa} & \cite{Ullrich2010} & Algo. \ref{alg:RK4_transportSa} & \cite{Ullrich2010} & Algo. \ref{alg:RK4_transportSa} \\ 
\hline 
1.0 & $\alpha = 0$ & $4.4262(-2)$ & $5.4173(-2)$ & $2.6982(-2)$ & $3.2511(-2)$ & $2.3012(-2)$ & $2.6469(-2)$ \\ 

  & $\alpha = \pi / 4$ & $4.2173(-2)$ & $5.1187(-2)$ & $2.3674(-2)$ & $2.9114(-2)$ & $1.8696(-2)$ & $2.2722(-2)$ \\ 
\hline 
0.5 & $\alpha = 0$ & $3.8326(-2)$ & $4.0429(-2)$ & $2.3194(-2)$ & $2.2452(-2)$ & $1.9969(-2)$ & $1.8989(-2)$ \\ 

  & $\alpha = \pi/4$ & $3.5096(-2)$ & $3.4451(-2)$ & $1.9601(-2)$ & $1.8444(-2)$ & $1.4171(-2)$ & $1.4138(-2)$ \\ 
\hline 
\end{tabular} 
\end{center}
\caption{Erreur relative pour la rotation solide sur l'équation \eqref{eq:advection_sphere} en norme $1$, $2$ et $\infty$ pour $\alpha = \pi / 4$ ainsi que $\alpha=0$. Le pas de temps est issu de la condition $\CFL = u_0 \Delta t / a \Delta \xi$, le filtre est d'ordre 10. Les résultats obtenus sont pratiquement identiques à ceux obtenus par volumes finis d'ordre 4 dans \cite{Ullrich2010}. Le paramètre de grille est $N=40$.}
\label{tab:comp_ullrich_bump}
\end{table} 

Pour analyser l'effet dissipatif de l'opérateur de filtrage, on compare la valeur du maximum de $h_0$ pour différents opérateurs de filtrage. Les résultats sont donnés sur la Table \ref{tab:max_bump} et sur la Figure \ref{fig:max_bump}. On constate qu'un filtre d'ordre 2 est trop dissipatif et ne permet pas de conserver correctement la hauteur de $h$. Le filtre d'ordre 10 donne de bons résultats. On note sur la Figure \ref{fig:parasite_bump} que sans filtrage, des oscillations parasites apparaissent et perturbent le calcul.

\begin{table}[htbp]
\begin{center}
\begin{tabular}{|c||ccc|}
\hline 
$\mathbf{N}$ & $20$ & $40$ & $80$ \\ 
\hline 
\hline 
Maxi. théorique & $1000$ & $1000$ & $1000$ \\ 
Filtre d'ordre $10$ & $968.87$ & $990.68$ & $997.45$ \\ 
Filtre d'ordre $8$ & $915.22$ & $995.34$ & $996.77$ \\ 
Filtre d'ordre $6$ & $767.00$ & $996.30$ & $996.86$ \\  
Filtre d'ordre $4$ & $436.37$ & $795.56$ & $969.92$ \\ 
Filtre d'ordre $2$ & $47.52$ & $91.45$ & $170.70$ \\ 
\hline 
\end{tabular} 
\end{center}
\caption{Maximum au temps $t=12$ jours de $\bh^n$ pour l'équation \eqref{eq:advection_sphere} discrétisée par l'algorithme \ref{alg:RK4_transportSa} avec $\alpha = \pi / 4$ et $\CFL = 0.7$.}
\label{tab:max_bump}
\end{table} 

\begin{figure}[htbp]
\begin{center}
\includegraphics[height=5cm]{N20BUMP.png}
\includegraphics[height=5cm]{N40BUMP.png}
\includegraphics[height=5cm]{N80BUMP.png}
\end{center}
\caption{Coupe au niveau de l'équateur du test 1 de \cite{Williamson1992} pour l'équation \eqref{eq:advection_sphere} au temps $t=12$ jours avec $\alpha = \pi / 4$ et $\CFL = 0.7$. Les tailles de maillage sont $N=20$ (haut, gauche), $N=40$ (haut, droite) et $N=80$ (bas). Plus l'ordre du filtre est bas, plus la solution est dissipée.}
\label{fig:max_bump}
\end{figure}

\begin{figure}[htbp]
\begin{center}
\includegraphics[height=5cm]{ref_7371377627_solapprochee_test_0.png}
\includegraphics[height=5cm]{ref_7371377627_erreur_test_0.png}
\end{center}
\caption{Calcul de la solution au temps $t=12$ jours sans opérateur de filtrage avec $N=40$ et $\CFL = 0.7$. Solution obtenue (gauche), erreur $\bh^n - h(t^n,\cdot)^*$ (droite) lorsque $\alpha = \pi/4$. On observe que sans filtrage, des oscillations sont présentes alors qu'elles ne le sont pas lorsque le filtrage est présent (voir Fig. \ref{fig:erreur_bump}).}
\label{fig:parasite_bump}
\end{figure}








\subsection{Propagation d'un vortex}
\label{sec:propag_vortex}

\subsubsection{Vortex statique}

Dorénavant, nous utilisons le filtrage d'ordre 10 lors de la résolution. Le test précédent est un test de déplacement sans déformation de la solution initiale. Dans \cite{Nair2002}, un autre test est construit pour que la condition initiale soit déformée au fil du temps.
On considère $(\lambda_C, \theta_C) \in \mathbb{S}_a^2$ un point de la sphère. Le test consiste à suivre la propagation de deux vortex diamétralement opposés dont l'un est situé en $(\lambda_C, \theta_C)$. Les deux vortex s'enroulent autour de leurs centres respectifs au fil du temps, rendant la solution de plus en plus difficile à représenter sur un maillage fixe.

L'objectif est de résoudre l'équation \eqref{eq:advection_sphere} avec le champ $\mathbf{c}$ donné par l'équation :
\begin{equation}
\mathbf{c}_r(\mathbf{x}) = \mathbf{c}(\mathbf{x}) = u_r \mathbf{e}_{\lambda} + v_r \mathbf{e}_{\theta}
\label{eq:rotation_vortex}
\end{equation}
où $u_r, v_r : (\mathbf{x}) \in \mathbb{S}_a^2 \mapsto u_r(\mathbf{x}), v_r(\mathbf{x}) \in \mathbb{R}$ sont les fonctions définies par
\begin{equation}
\left\lbrace
\begin{array}{rcl}
u_r(\lambda, \theta) & = & a \omega_r(\theta') \left[ \sin \theta_C \cos \theta - \cos \theta_C \cos (\lambda - \lambda_C) \sin \theta \right] \\
v_r(\lambda, \theta) & = & a \omega_r (\theta') \left[\cos \theta_C \sin (\lambda - \lambda_C)  \right]
\end{array}
\right.
\end{equation}
où $(\lambda', \theta')$ sont les coordonnées longitude-latitude associées au pôle Nord placé en $(\lambda_C, \theta_C)$. On déduit ces valeurs grâce aux équations \eqref{from classic to prime}.
La vitesse de rotation du vortex est définie par $a \omega_r(\theta')$ avec la fonction $\omega_r$ définie par
\begin{equation}
\omega_r(\theta') = \left\lbrace
\begin{array}{cl}
V/a\rho & \text{ si } \rho \neq 0 \\
0 & \text{ sinon.}
\end{array}
\right.
\end{equation}
où $\rho = \rho_0 \cos (\theta')$ est une pseudo-distance au centre du vortex et $V=u_0 \dfrac{3 \sqrt{3}}{2} \sech^2 (\rho) \tanh (\rho)$. Noter que $\dfrac{3 \sqrt{3}}{2}$ est une constante de normalisation. On choisit $u_0 = 2 \pi a / (12 \text{jours})$ et $\rho_0 = 3$.

Une solution exacte de \eqref{eq:advection_sphere} est donnée par
\begin{equation}
h(t, \lambda, \theta) = 1 - \tanh \left[ \dfrac{\rho}{\gamma} \sin (\lambda' - \omega_r (\theta') t) \right]
\label{eq:NM_solexacte}
\end{equation}
où $\gamma$ est une constante influençant le gradient de la solution. Comme dans l'article \cite{Nair2002}, on choisit $\gamma = 5$.

Sur la Table \ref{tab:rate1_NM} et la Figure \ref{fig:rate1_NM}, on donne la convergence pour ce test avec $(\lambda_C, \theta_C)=(\pi /4 , \pi /4)$. Les centres des vortex sont alors placés proches des coins de la Cubed-Sphere. La convergence se fait à un ordre supérieur ou égal à $5$.

\begin{table}[htbp]
\begin{center}
\begin{tabular}{|c||c|c|c|}
\hline
\textbf{N}  & $\mathbf{e_1}$ & $\mathbf{e_2}$ & $\mathbf{e_{\infty}}$\\
\hline
\hline
$40$  & $1.2170 (-3)$ & $5.2773 (-3)$ & $3.8615 (-2)$ \\
$50$  & $4.4810 (-4)$ & $2.1100 (-3)$ & $1.6306 (-2)$ \\
$60$  & $1.6313 (-4)$ & $8.2236 (-4)$ & $6.5687 (-3)$ \\
$80$  & $2.8658 (-5)$ & $1.4710 (-4)$ & $1.4042 (-3)$ \\
$100$  & $8.7526 (-6)$ & $4.1919 (-5)$ & $4.1512 (-4)$ \\
$150$  & $1.1105 (-6)$ & $5.7646 (-6)$ & $6.2903 (-5)$ \\
\hline 
\hline
\textbf{Ordre estimé}& $5.40$ & $5.30$ & $4.98$\\
\hline
\end{tabular}
\end{center}
\caption{Erreur et taux de convergence pour le test du vortex stationnaire sur 12 jours pour l'équation \eqref{eq:advection_sphere} en normes $\| \cdot\|_1$, $\| \cdot\|_2$ et $\| \cdot\|_{\infty}$, $\CFL = 0.7$ \cite{Nair2002}, le filtre est d'ordre 10. Le vortex est localisé en $(\lambda_C, \theta_C)=(\pi /4 , \pi /4)$. La convergence se fait à un ordre supérieur ou égal à $5$.}
\label{tab:rate1_NM}
\end{table} 

\begin{figure}[htbp]
\begin{center}
\includegraphics[height=5cm]{rate_NM1.png}
\end{center}
\caption{Erreur et taux de convergence pour le test du vortex stationnaire sur l'équation \eqref{eq:advection_sphere} en normes $\| \cdot\|_1$, $\| \cdot\|_2$ et $\| \cdot\|_{\infty}$ et en fonction de $\Delta = a \Delta \xi$, avec $\CFL = 0.7$ \cite{Nair2002}, le filtre est d'ordre 10. Le vortex est localisé en $(\lambda_C, \theta_C)=(\pi /4 , \pi /4)$. L'ordre de convergence est d'environ $5$ pour la norme $\| \cdot \|_{\infty}$ et supérieur pour les normes $\| \cdot \|_{1}$ et $\| \cdot \|_{2}$.}
\label{fig:rate1_NM}
\end{figure} 

Sur une grille grossière ($N=36$ correspondant à l'équateur à $\Delta \lambda = 2.5$deg.), on compare l'erreur au cours du temps pour deux valeurs différentes du pas de temps $\Delta t$. Les résultats sont donnés sur la Figure \ref{fig:cfl_NM}. Lorsque $\CFL = u_0 \Delta t / a \Delta \xi = 0.5$, il y a 288 pas de temps pour arriver au temps final. Lorsque $\CFL = u_0 \Delta t / a \Delta \xi = 0.05$, il y a 2880 pas de temps. Les erreurs obtenues sont tout a fait comparables à celles obtenues par la méthode de Galerkin Discontinu \cite{Nair2008}. Avec 288 pas de temps, les erreurs spatiales et temporelles sont observées simultanément. L'erreur est sensiblement meilleure lorsque 2880 pas de temps sont utilisés.

\begin{figure}[htbp]
\begin{center}
\includegraphics[height=5cm]{ref_7367656360_normerreur_test_1.png}
\includegraphics[height=5cm]{ref_7367656531_normerreur_test_1.png}
\end{center}
\caption{Evolution de l'erreur sur $t=12$ jours pour le cas test du vortex \cite{Nair2002} avec $(\lambda_C, \theta_C) = (\pi/4, \pi/4)$. Les paramètres numériques sont $N=40$, le filtrage utilisé est d'ordre $10$. Le pas de temps est déduit des relations $\CFL = 0.5$ (gauche), et $\CFL = u_0 \Delta t / \Delta \xi = 0.05$ (droite).}
\label{fig:cfl_NM}
\end{figure}

Sur la Figure \ref{fig:space_NM}, on représente la solution $\bh^n$ au temps $t=12$ jours ainsi que l'erreur spatiale $\bh^n-h(t^n,\cdot)^*$ sur un maillage de paramètre $N=40$. L'erreur est localisée au centre du vortex. Ce résultat était attendu, le vortex devient de plus en plus fin lorsque $t$ grandit et la solution devient sous résolue. 
On vérifie cela sur la Figure \ref{fig:coupe_NM}: on représente, au temps $t=12$ jours, une coupe le long de l'équateur de la solution lorsque $(\lambda_C, \theta_C)=(3 \pi/4,0)$. On observe qu'avec une grille de paramètre $N=25$, la solution est sous représentée en comparaison à une grille de paramètre $N=50$.

\begin{figure}[htbp]
\begin{center}
\includegraphics[height=5cm]{ref_7371383883_solapprochee_test_1.png}
\includegraphics[height=5cm]{ref_7371383883_erreur_test_1.png}
\end{center}
\caption{Solution au temps $t=12$ jours pour le vortex statique \cite{Nair2002} avec $(\lambda_C, \theta_C) = (\pi/4, \pi/4)$. Les paramètres numériques sont $N=40$ et $\CFL = 0.7$, le filtrage utilisé est d'ordre $10$. La solution $\bh^n$ (gauche), erreur spatiale $\bh^n - h(t^n,\cdot)^*$ (droite).}
\label{fig:space_NM}
\end{figure}

\begin{figure}[htbp]
\begin{center}
\includegraphics[height=5cm]{coupe_NM.png}
\end{center}
\caption{Coupe le long de l'équateur de la solution au temps $t=12$ jours pour le cas test du vortex \cite{Nair2002} avec $(\lambda_C, \theta_C) = (3 \pi / 4,0)$. Le pas de temps est issu de $\CFL = 0.7$ et filtrage d'ordre 10. La solution sur grille grossière est moins bien représentée que celle sur grille fine.}
\label{fig:coupe_NM}
\end{figure}









\subsubsection{Vortex avec rotation solide}

Une variante du test \cite{Nair2002} consiste à combiner la vitesse de rotation solide $\mathbf{c}_s$ \eqref{eq:rot_solide_1} avec la vitesse de rotation du vortex $\mathbf{c}_r$ \eqref{eq:rotation_vortex}. Il s'agit du test présenté dans \cite{Nair2008}.
On considère l'équation d'advection \eqref{eq:advection_sphere} munie du champ de vitesse
\begin{equation}
\mathbf{c}(t,\mathbf{x}) = u \mathbf{e}_{\lambda} + v \mathbf{e}_{\theta}
\end{equation}
où les fonctions $u$ et $v$ dépendent à présent du temps par 
\begin{equation}
\left\lbrace
\begin{array}{rcl}
u(t,\lambda, \theta) & = & u_0 \left( \cos \theta \cos \alpha + \sin \theta \cos \lambda \sin \alpha \right) + a \omega_r \left( \sin \theta_C(t) \cos \theta - \cos \theta_C(t) \cos (\lambda - \lambda_C(t)) \sin \theta \right) \\
v(t,\lambda, \theta) & = & - u_0 \sin \lambda \sin \alpha + a \omega_r \left( \cos \theta_C(t) \sin (\lambda - \lambda_C(t)) \right),
\end{array}
\right.
\label{eq:vitesse_NJ}
\end{equation}
La donnée $(\lambda_C(t), \theta_C(t)) \in \mathbb{S}_a^2$ correspond à la position du vortex au fil du temps. Cette dernière est donnée dans la base "tournée" d'un angle $\alpha$ par
\begin{equation}
\label{eq:vortexcenter1}
\left\lbrace
\begin{array}{rcl}
\lambda_C'(t) & = & \lambda_0' + \omega_s t \\
\theta_C'(t) & = & \theta_0'
\end{array}
\right.
\end{equation}
avec $(\lambda_0', \theta_0')$ la position initiale du vortex statique dans la base associée à $(\lambda_P, \theta_P)=(\pi, \pi/2-\alpha)$. Dans le système de coordonnées longitude latitude associé au pôle Nord $\mathbf{N}$, on a $(\lambda_0,\theta_0)=(3 \pi / 2, 0)$. $\omega_s = u_0/a$ est la vitesse de rotation solide du vortex. 

La solution exacte est alors donnée par \eqref{eq:NM_solexacte} en déplaçant la position du vortex au fil du temps $t$. On note $(\lambda, \theta)$ les coordonnées longitude-latitude de $\mathbf{x} \in \mathbb{S}_a^2$.
La solution exacte en $\mathbf{x}$ et au temps $t>0$, notée $h(t,\mathbf{x})$, est calculée de la façon suivante :
\begin{enumerate}
\item Calculer $(\lambda', \theta')$ les coordonnées longitude-latitude associées au pôle de coordonnées $(\lambda_P, \theta_P) = (\pi, \pi/2 - \alpha)$. Pour cela, on utilise la formule \eqref{from classic to prime}.

\item Déplacer $(\lambda', \theta')$ sur la position du vortex par
\begin{equation}
\label{eq:vortexcenter}
\left\lbrace
\begin{array}{rcl}
\lambda'_s & = & \lambda' - \omega_s t \\
\theta_s' & = & \theta'.
\end{array}
\right.
\end{equation}
Il s'agit de l'action de la rotation solide.

\item Calcul de $(\lambda_s, \theta_s)$ en revenant dans le système de coordonnées longitude-latitude grâce à la formule \eqref{from prime to classic} avec $(\lambda_P, \theta_P) = (\pi, \pi/2 - \alpha)$.

\item Calcul de $(\lambda_s'', \theta_s'')$ déduit de $(\lambda_s, \theta_s)$. Le point de coordonnées longitude latitude $(\lambda_s, \theta_s)$ a pour coordonnées longitude latitude $(\lambda_s'', \theta_s'')$ (pour le pôle de coordonnées $(\lambda_C, \theta_C)$ donné par \eqref{eq:vortexcenter1}) grâce à la formule \eqref{from classic to prime}.

\item Calculer la solution exacte $h(t,\lambda, \theta)$ par
\begin{equation}
h(t,\lambda, \theta) = 1 - \tanh \left[ \dfrac{\rho}{\gamma} \sin (\lambda_s'' - \omega_r(\theta_s'')t) \right],
\label{eq:NJ_solexacte}
\end{equation}
avec $\omega_r$ donné par
\begin{equation}
\omega_r = \left\lbrace
\begin{array}{cl}
V/(a \rho) & \text{ si } \rho \neq 0 \\
0 & \text{ sinon,}
\end{array}
\right.
\end{equation}
et $\rho = \rho_0 \cos (\theta_s'')$ ainsi que $V = u_0 \dfrac{3\sqrt{3}}{2} \sech^2(\rho) \tanh(\rho)$.
\end{enumerate}
La solution exacte représente un vortex s'enroulant sur lui même. Les détails sont de plus en plus fins et à grille fixée, elle devient difficile à représenter. De plus, le centre des vortex se déplace sur un grand cercle de la sphère. En fonction de la valeur de $\alpha$, les vortex passent plus ou moins loin des coins de la Cubed-Sphere.

Sur la Figure \ref{fig:NJ_difftps}, on représente la solution aux temps $t=3$, $t=6$, $t=9$ et $t=12$ jours lorsque $\alpha = \pi/4$. On y observe le déplacement des vortex le long d'un grand cercle longeant les panels $(V)$ et $(VI)$.

\begin{figure}[htbp]
\begin{center}
\includegraphics[height=5cm]{ref_7371598704_snapshot_test_2_nday_3.png}
\includegraphics[height=5.2cm]{ref_7371598715_snapshot_test_2_nday_6.png}
\includegraphics[height=5cm]{ref_7371598724_snapshot_test_2_nday_9.png}
\includegraphics[height=5cm]{ref_7371598877_snapshot_test_2_nday_12.png}
\end{center}
\caption{Vortex avec rotation solide de \cite{Nair2008}. On représente la solution \eqref{eq:NJ_solexacte} de l'équation de transport \eqref{eq:advection_sphere} avec le champ de vitesse \eqref{eq:vitesse_NJ} avec une grille de paramètre $N=40$. On représente la solutions aux temps $t=3$, $t=6$, $t=9$ et $t=12$ jours (dans cet ordre, de haut en bas). En plus du déplacement des tourbillons, on observe que lors de la formation du vortex, la solution devient difficile à représenter.}
\label{fig:NJ_difftps} 
\end{figure}

Sur la Table \ref{tab:rate1_NJ} et la Figure \ref{fig:rate1_NJ}, on représente le taux de convergence en utilisant différentes tailles de grilles et en conservant $\CFL = 0.7$. On choisit $\alpha = \pi/4$ de manière à ce que les vortex longent les bords des panels comme c'est visible sur la Figure \ref{fig:NJ_difftps}. Un tel choix vise à mettre en difficulté la méthode de résolution en faisant passer les détails fins de la solution sur les bords des panels. Les résultats permettent d'observer un ordre de convergence proche de $4$ en norme $\| \cdot \|_1$ et $\| \cdot \|_2$. L'ordre de convergence est proche de $3.5$ pour la norme $\| \cdot \|_{\infty}$. 

\begin{table}[htbp]
\begin{center}
\begin{tabular}{|c||c|c|c|}
\hline
\textbf{N}  & $\mathbf{e_1}$ & $\mathbf{e_2}$ & $\mathbf{e_{\infty}}$\\
\hline
\hline
$40$  & $2.9241 (-3)$ & $1.0646 (-2)$ & $5.7267 (-2)$ \\
$50$  & $1.3634 (-3)$ & $5.3187 (-3)$ & $3.3187 (-2)$ \\
$60$  & $6.6453 (-4)$ & $2.7522 (-3)$ & $1.8792 (-2)$ \\
$80$  & $2.0635 (-4)$ & $8.8170 (-4)$ & $6.3350 (-3)$ \\
$100$ & $8.2353 (-5)$ & $3.5454 (-4)$ & $2.7479 (-3)$ \\
$150$ & $1.6044 (-5)$ & $7.0918 (-5)$ & $5.9455 (-4)$ \\
\hline 
\hline
\textbf{Ordre estimé}& $3.98$ & $3.84$ & $3.52$\\
\hline
\end{tabular}
\end{center}
\caption{Erreur pour le test vortex avec rotation solide. On donne différentes erreurs et taux de convergence pour l'équation \eqref{eq:advection_sphere} avec le champ de vitesse \eqref{eq:vitesse_NJ} en norme $1$, $2$ et $\infty$, $\CFL = 0.7$. On choisit $\alpha = \pi/4$ et le temps final $t=12$ jours. Nous utilisons l'opérateur de filtrage d'ordre 10.}
\label{tab:rate1_NJ}
\end{table} 

\begin{figure}[htbp]
\begin{center}
\includegraphics[height=5cm]{rate_NJ1.png}
\end{center}
\caption{Erreur pour le vortex avec rotation solide en fonction de $\Delta = a \Delta \xi$. On représente l'erreur et le taux de convergence pour l'équation \eqref{eq:advection_sphere} avec le champ de vitesse \eqref{eq:vitesse_NJ} en norme $1$, $2$ et $\infty$, $\CFL = 0.7$, l'opérateur de filtrage est d'ordre 10. L'angle $\alpha$ est $\alpha = \pi/4$ et le temps final $t=12$ jours. L'odre de convergence est proche de 4 pour les normes $\| \cdot \|_1$ et $\| \cdot \|_2$. Il est proche de 3.5 pour la norme $\| \cdot \|_{\infty}$.}
\label{fig:rate1_NJ}
\end{figure} 

On a vu que le vortex statique est difficile à représenter lorsque $t$ augmente. Le vortex en rotation représente la même fonction se déplaçant sur la sphère. La solution du vortex en rotation est aussi difficile à représenter lorsque $t$ croît. Sur la Figure \ref{fig:NJ24jours}, on représente l'historique de l'erreur relative jusqu'à $t=24$ jours. Le temps final pour ce test est usuellement $t=12$ jours, mais on observe le comportement du schéma sur un temps plus long. Les résultats sont obtenus avec $\CFL = 0.7$, l'obtention de résultats pour 24 jours sont obtenus après 457 pas de temps. Pour la grille de taille $40 \times 40 \times 6$, l'erreur finale est de $15.95\%$ en norme $\| \cdot \|_{\infty}$, $3.67\%$ pour la norme $\| \cdot \|_{2}$ et $1.36\%$ en norme $\| \cdot \|_{1}$. Pour la grille $80 \times 80 \times 6$, on effectue 914 pas de temps pour arriver au temps $t=24$ jours. L'erreur finale est de $9.63\%$ en norme $\| \cdot \|_{\infty}$, $1.69\%$ pour la norme $\| \cdot \|_{2}$ et $0.45\%$ en norme $\| \cdot \|_{1}$.

\begin{figure}[htbp]
\begin{center}
\includegraphics[height=5cm]{ref_7371598961_normerreur_test_2.png}
\includegraphics[height=5cm]{ref_7371598976_normerreur_test_2.png}
\end{center}
\caption{Historique de l'erreur pour le test du vortex avec rotation solide. On représente l'historique de l'erreur pour l'équation \eqref{eq:advection_sphere} avec le champ de vitesse \eqref{eq:vitesse_NJ} en norme $\| \cdot \|_1$, $\| \cdot \|_2$ et $\| \cdot \|_{\infty}$ avec $\CFL = 0.7$, le filtrage est d'ordre 10. On choisit $\alpha = \pi/4$. Le temps final est $t=24$ jours. La grille est $40 \times 40  \times 6$, 457 pas de temps (gauche), la grille est $80 \times 80  \times 6$, l'algorithme effectue 914 pas de temps (droite). Sur la grille $80 \times 80 \times 6$, l'erreur est en dessous de $10 \%$ ce qui demeure acceptable.}
\label{fig:NJ24jours}
\end{figure} 

Les erreurs au temps $t=12$ jours sont comparables à celles obtenues par la méthode de Galerkin discontinue \cite{Nair2008}. Nous comparons aussi notre schéma à des schémas de type volumes finis d'ordre élevé \cite{Katta2015}. Nous comparons les résultats au temps $t=12$ jours. Les schémas volumes finis sont nommés WENO5 et KL4 dans \cite{Katta2015}. Nous utilisons toujours $\alpha = \pi/4$. Sur la grille $80 \times 80 \times 6$ et après 750 pas de temps, le schéma WENO5 donne les erreurs relatives suivantes : $e_1 = 0.0021$, $e_2 = 0.0043$ et $e_{\infty} = 0.0191$. Le schéma KL4 obtient, dans le même contexte, les erreurs $e_1 = 0.0021$, $e_2 = 0.0043$ et $e_{\infty} = 0.0194$. Avec notre schéma, on obtient $e_1 = 1.67(-4)$, $e_2=7.23(-4)$ et $e_{\infty} = 5.75(-3)$. Les niveaux d'erreurs obtenus sont plus faibles que ceux obtenus par les méthodes de volumes finis WENO5 et KL4.


























\section{Equations de conservation non linéaire}

L'équation d'advection \eqref{eq:advection_sphere} est un problème linéaire. Dans cette section, on s'intéresse à une équation non linéaire de type "Burgers" sphérique \cite{BenArtzi2009}. Les tests effectués concernent l'équation
\begin{equation}
\left\lbrace
\begin{array}{rcl}
\dfrac{\partial h}{\partial t} + \nabla_T \cdot F(h) & = & 0 \\
h(0,\mathbf{x}) & = & h_0(\mathbf{x})
\end{array}
\right. \text{ avec } \mathbf{x} \in \mathbb{S}^2_a \text{ et } t \geq 0.
\label{eq:advection_sphere_NL}
\end{equation}
Nous choisissons $a = 1$. L'application $F : h \mapsto F(h) \in \mathbb{T}\mathbb{S}^2$ transforme une fonction en un champ de vecteurs tangent à la sphère.

En particulier, on note que \eqref{eq:advection_sphere_NL} est une loi de conservation. La relation suivante est vérifiée :
\begin{equation}
\dfrac{d}{dt} \gint_{\mathbb{S}^2} h(t,\mathbf{x}) d \sigma(\mathbf{x}) = 0.
\end{equation}

On a vu dans le lemme \ref{lem:n_vect_w} que si $\mathbf{w} : \mathbf{x} \in \mathbb{S}^2 \mapsto \mathbf{w} \in \mathbb{R}^3$ est un champ de vecteurs de $\mathbb{R}^3$ et si $\mathbf{n}$ est la normale extérieure à la sphère, alors $\mathbf{F} = \mathbf{n} \wedge \mathbf{w}$ est un champ de vecteurs tangent à la sphère. On considère ici des champs de vecteurs de cette forme avec 
\begin{equation}
\mathbf{w} = f_1 \mathbf{i} + f_2 \mathbf{j}+ f_3 \mathbf{k} = \begin{bmatrix}
f_1 \\ f_2 \\ f_3
\end{bmatrix},
\end{equation}
où $f_p : h \in L^2(\mathbb{S}_a^2, \mathbb{R}) \mapsto f_p(h)$  pour $1 \leq p \leq 3$.


Dans ce qui suit, on s'intéresse à deux tests pour cette équation, introduits dans \cite{BenArtzi2009}. Le premier permet d'analyser le comportement d'un schéma numérique lors de l'apparition d'un choc. Le second permet d'étudier la conservation d'une solution stationnaire.










\subsection{Résolution numérique}

Pour résoudre l'équation \eqref{eq:advection_sphere_NL}, on considère l'application $J_{\Delta}$ définie pour toute fonction de grille $\bh$ sur la sphère par
\begin{equation}
J_{\Delta}(t,\bh) = - \nabla_{T,\Delta} F(\bh).
\end{equation}
Nous couplons cet opérateur d'approximation spatiale à un algorithme de résolution en temps. L'algorithme permettant la résolution de \eqref{eq:advection_sphere_NL} est analogue à l'algorithme \ref{alg:RK4_transportSa}. Il s'écrit :

\begin{center}
\begin{minipage}[H]{12cm}
  \begin{algorithm}[H]
    \caption{: Equation d'advection sphérique non linéaire \eqref{eq:advection_sphere_NL} }\label{alg:RK4_transportSa_NL}
    \begin{algorithmic}[1]
    \State $\bh^0 = h_0^*$ connu,
    \For{$n=0,1, \ldots$}
             \State  $K^{(1)} = J_{\Delta}(t^n, \bh^n)$,
             \State  $K^{(2)} = J_{\Delta}\left(t^n + \frac{\Delta t}{2}, \bh^n + \frac{\Delta t}{2} K^{(1)} \right)$,
             \State  $K^{(3)} = J_{\Delta}\left(t^n + \frac{\Delta t}{2}, \bh^n + \frac{\Delta t}{2} K^{(2)} \right)$,
             \State  $K^{(4)} = J_{\Delta}\left(t^n + \Delta t \bh^n + \Delta t K^{(3)} \right)$,  
             \State  $\bh^{n+1} = \mathcal{F}\left( \bh^n  + \dfrac{\Delta t}{6} \left( K^{(1)} + 2 K^{(2)} + 2 K^{(3)} + K^{(4)} \right) \right)$.
            \EndFor
    \end{algorithmic}
    \end{algorithm}
\end{minipage}
\end{center}

L'opérateur $\mathcal{F}$ est de la forme \eqref{eq:operateur_filtrage} où $\fxi$ et $\feta$ utilisent l'opérateur de filtrage en dimension 1 d'ordre 10 : $\mathcal{F}_{10,x}$ \eqref{eq:ftr}. Les opérateurs $\fxi$ et $\feta$ sont calculés par les algorithmes \ref{alg:ftrxi} et \ref{alg:ftreta}.


























\subsection{Solution équatoriale périodique}

Pour ce test, inspiré du premier test de \cite{BenArtzi2009}, on considère les fonctions $f_1$ et $f_2$ nulles :
\begin{equation}
f_1 = f_2 \equiv 0.
\end{equation}
Le champ de vecteurs $F$ est alors donné pour toute fonction $h$ par
\begin{equation}
F(h) = \begin{bmatrix}
x \\ y \\z
\end{bmatrix}
\wedge
\begin{bmatrix}
f_1(h) \\ f_2(h) \\ f_3(h)
\end{bmatrix} = \begin{bmatrix}
y f_3 (h) \\ -x f_3(h) \\ 0
\end{bmatrix} = -f_3(h) \cos (\theta) \mathbf{e}_{\lambda}.
\end{equation}
On a en coordonnées longitude-latitude
\begin{equation}
\nabla_T \cdot F(h) = - \dfrac{\partial}{\partial \lambda} f_3(h).
\end{equation}
L'équation \eqref{eq:advection_sphere_NL} s'écrit alors en coordonnées longitude-latitude :
\begin{equation}
\dfrac{\partial h}{\partial t} - \dfrac{\partial}{\partial \lambda}f_3(h) = 0 \text{ en tout } \mathbf{x} \in \mathbb{S}^2_a \text{ et } t \geq 0.
\end{equation}
Il découle la proposition suivante :

\begin{proposition}
Soit $\tilde{h}$ la solution du problème périodique en dimension 1
\begin{equation}
\left\lbrace
\begin{array}{rcl}
\dfrac{\partial \tilde{h}}{\partial t} - \dfrac{\partial}{\partial \lambda}f_3(\tilde{h}) = 0 \\
\tilde{h}(0,\lambda) = \tilde{h}_0(\lambda) 
\end{array}
\right. \text{ pour } \lambda \in [0, 2 \pi[ \text{ et } t > 0,
\label{eq:conservation_sph_burgers}
\end{equation}
et soit $\hat{h} : \theta \in ]- \pi/2, \pi/2[ \mapsto \hat{h}(\theta) \in \mathbb{R}$ tel que
\begin{equation}
h_0(\mathbf{x}) = \tilde{h}_0(\lambda) \hat{h}(\theta).
\end{equation}
Alors la solution du problème \eqref{eq:advection_sphere_NL} est donnée par
\begin{equation}
h(t,\mathbf{x}) = h(t,\lambda, \theta) = \tilde{h}(t,\lambda) \hat{h}(\theta),
\end{equation}
pour $t>0$, $\mathbf{x} \in \mathbb{S}^2$ un point de la sphère de coordonnées longitude-latitude $(\lambda, \theta)$.
\end{proposition}

On pose $f_3(h) = - \pi h^2$. L'équation \eqref{eq:conservation_sph_burgers} est identique à l'équation \eqref{eq:Burgers_1d}. On compare une coupe le long de l'équateur de la solution calculée par l'algorithme \ref{alg:RK4_transportSa_NL} avec la solution calculée par l'algorithme \ref{alg:RK4_burgers1d} lorsque
\begin{equation}
\left\lbrace
\begin{array}{rcl}
\tilde{h}_0(\lambda) & = & \sin \lambda, \\
\hat{h}_0(\theta) & = & \mathbf{1}_{[-\pi/12, \pi/12]}(\theta).
\end{array}
\right.
\end{equation}
On rappelle que dans ce contexte, la solution de \eqref{eq:conservation_sph_burgers} est de classe $C^1$ pour $t \leq 1/(2 \pi)$. Sur la Figure \ref{fig:BenArtzi_equatorial1}, on représente la solution aux temps $t=1/(2 \pi)$ et $t=10/(2\pi)$ pour un paramètre de grille $N=32$ pour la Cubed-Sphere ($128$ points de discrétisation sur l'équateur) et $128$ points de discrétisation pour le problème 1D. Le pas de temps est $\Delta t=0.005$. Les résultats des deux algorithmes sont très semblables et donnent des résultats satisfaisants même au delà du temps $1/(2 \pi)$ au delà duquel la solution est moins régulière. De plus, le filtre d'ordre 10 est suffisant pour que les oscillations ne dégradent pas excessivement le résultat.

\begin{figure}[htbp]
\begin{center}
\includegraphics[height=5cm]{BenArtzi_equatorial1.png}
\includegraphics[height=5cm]{BenArtzi_equatorial2.png}
\end{center}
\caption{Coupe de la solution équatoriale. On représente une coupe équatoriale de la solution de \eqref{eq:advection_sphere_NL} et la solution de \eqref{eq:conservation_sph_burgers} pour le test périodique. On compare la solution au temps $t=1/(2\pi)$ (gauche) et $t=10/(2\pi)$ (droite). La grille Cubed-Sphere a pour paramètre $N=32$ ($128$ points de discrétisation sur l'équateur). Le problème en dimension 1 est résolu avec $128$ points de discrétisation. Le pas de temps est $\Delta t = 0.005$.}
\label{fig:BenArtzi_equatorial1}
\end{figure} 

Sur la Figure \ref{fig:BenArtzi_equatorial3}, on représente l'historique de l'erreur de conservation au cours du temps :
\begin{equation}
|Q(\mathfrak{h}^n) - Q(h(t^n, \cdot)^*|.
\end{equation}
Nous n'utilisons pas l'erreur relative car pour les conditions initiales utilisées, on a
\begin{equation}
\gint_{\mathbb{S}_a^2} h_0(\mathbf{x}) d \sigma(\mathbf{x}) = 0.
\end{equation}
L'erreur de conservation est proche de $6.0 (-5)$ lorsque $N=32$ et proche de $1.5 (-5)$ lorsque $N=64$. L'erreur est cependant nettement plus importante au temps d'apparition du choc $t=1/(2\pi) \approx 0.1592$. 

\begin{figure}[htbp]
\begin{center}
\includegraphics[height=5cm]{BenArtzi_equatorial3.png}
\includegraphics[height=5cm]{BenArtzi_equatorial4.png}
\end{center}
\caption{Historique de l'erreur de conservation pour la solution équatoriale périodique. On représente l'erreur de conservation en fonction du temps $t$ pour le test équatorial périodique \eqref{eq:advection_sphere_NL}. La grille Cubed-Sphere a pour paramètres $N=32$ (gauche) et $N=64$ (droite). Le pas de temps est le même dans les deux cas. On a $\Delta t = 0.005$, la simulation est faite en 318 pas de temps pour le temps final $t=10/(2\pi)$. L'erreur n'est pas relative car la masse totale initiale est nulle. La masse totale mesurée est très bien conservée.}
\label{fig:BenArtzi_equatorial3}
\end{figure} 

Ce test permet d'analyser le comportement du schéma numérique de l'algorithme \ref{alg:RK4_transportSa_NL} en présence d'un choc. Les résultats sont satisfaisants. Les oscillations qui apparaissent au delà du temps d'apparition du choc ne dégradent pas excessivement le calcul. Le filtrage d'ordre 10 symétrique est suffisant pour assurer un bon déroulement de la simulation. De plus, l'erreur sur la conservation de la masse reste faible au cours du temps.

Nous insistons sur le fait que le calcul effectué n'utilise aucun décentrement et aucun traitement de type reconstruction ou limitation de pente. Il n'est donc pas surprenant d'observer des oscillations non linéaires (Fig. \ref{fig:BenArtzi_equatorial1}, droite). On constate que ces oscillations demeurent très limitées et très localisées. Une amélioration de ces résultats fera l'objet d'études ultérieures.





















\subsection{Solution stationnaire}

Dans cette section, on construit une solution stationnaire de \eqref{eq:advection_sphere_NL}. Si la condition initiale $h_0$ est telle que
\begin{equation}
\nabla_T \cdot F(h_0) = 0,
\end{equation}
alors $h_0$ est une solution stationnaire de \eqref{eq:advection_sphere_NL}.

On considère les fonctions $f_1$, $f_2$ et $f_3$ identiques, c'est à dire :
\begin{equation}
f_1(h) = f_2(h) = f_3(h) = f(h)
\end{equation}
avec $h : \mathbb{S}_a^2 \rightarrow \mathbb{R}$ donnée.
Alors, $F$ est donnée par
\begin{equation}
F(h) = \mathbf{n} \wedge \left( f(h) (\mathbf{i}+\mathbf{j}+\mathbf{k}) \right).
\end{equation}
Le calcul de $\nabla_T \cdot F(h)$ en fonction de $h$ et $f$ donne
\begin{equation}
\nabla_T \cdot F(h) = f'(h) \left( (y-z)\dfrac{\partial h}{\partial x} + (z-x)\dfrac{\partial h}{\partial y} + (x-y)\dfrac{\partial h}{\partial z} \right).
\end{equation}
On en déduit que indépendamment du choix de $f$, si $h_0(x,y,z) = \alpha (x+y+z)$, avec $\alpha \in \mathbb{R}$, on a
\begin{equation}
\nabla_T \cdot F(h_0) = 0,
\end{equation}
et $h(t,\mathbf{x}) = h_0(\mathbf{x})$ est une solution stationnaire.

Dans \cite{BenArtzi2009}, le test numéro 3 consiste à choisir
\begin{equation}
f_1(h) = f_2(h) = f_3(h) = \dfrac{h^2}{2},
\end{equation}
et la condition initiale 
\begin{equation}
h_0(x,y,z) = \dfrac{x+y+z}{\sqrt{3}}.
\end{equation}
Cette condition initiale est une solution stationnaire. On compare la solution calculée par l'algorithme \ref{alg:RK4_transportSa_NL} avec la solution initiale jusqu'au temps $t=6$. On mesure l'erreur relative à chaque itération. Les résultats de convergence sont donnés sur la Table \ref{tab:benartzi_test3} et sur la Figure \ref{fig:benartzi_test3}. L'ordre de convergence est proche de 4 en normes $\| \cdot \|_1$, $\| \cdot \|_2$ et $\| \cdot \|_{\infty}$. La conservation de la masse est vérifiée à un ordre supérieur à 5.

\begin{table}[htbp]
\begin{center}
\begin{tabular}{|c||c|c|c||c|}
\hline 
$\mathbf{N}$ & $\mathbf{e}_1$ & $\mathbf{e}_2$ & $\mathbf{e}_{\infty}$ & \textbf{Conservation} \\ 
\hline 
\hline 
$\mathbf{16}$ & $2.1446(-5)$ & $1.5759(-5)$ & $1.4251(-5)$ & $3.6380(-7)$ \\ 
$\mathbf{32}$ & $2.2000(-6)$ & $1.1752(-6)$ & $1.0776(-6)$ & $8.3644(-9)$ \\ 
$\mathbf{64}$ & $1.4092(-7)$ & $7.9823(-8)$ & $7.7308(-8)$ & $9.6391(-11)$ \\ 
$\mathbf{128}$ & $8.7856(-9)$ & $5.0291(-9)$ & $4.5510(-9)$ & $8.4850(-12)$ \\ 
\hline 
\textbf{Ordre estimé} & $3.94$ & $3.87$ & $3.86$ & $5.18$ \\ 
\hline 
\end{tabular} 
\end{center}
\caption{Erreur pour la solution stationnaire. Table de convergence pour le test stationnaire de l'équation \eqref{eq:advection_sphere_NL}. Le pas de temps est donné par $\Delta t = 0.96 \Delta \xi / \pi$. Le temps final est $t=6$. On mesure aussi l'erreur sur la conservation de la masse. Le taux de convergence est proche de 4. L'erreur de conservation mesurée n'est pas relative car la masse totale initiale est nulle. La conservation est excellente.}
\label{tab:benartzi_test3}
\end{table}

\begin{figure}[htbp]
\begin{center}
\includegraphics[height=5cm]{rateBA_test3.png}
\end{center}
\caption{Erreur et taux de convergence pour le test stationnaire de l'équation \eqref{eq:advection_sphere_NL} en fonction de $\Delta = a \Delta \xi$. Le pas de temps est donné par $\Delta t = 0.96 \Delta \xi / \pi$. Le temps final est $t=6$.}
\label{fig:benartzi_test3}
\end{figure}

Sur la Figure \ref{fig:benartzi_test3_hist}, on représente l'historique de l'erreur lorsque $N=31$ et $\Delta t = 0.96 \Delta \xi / \pi = 0.015$. On observe sur ces figures que l'erreur sur la fonction $h$ est proche de $3 \times 10^{-6}$. L'erreur sur la conservation de la masse est proche de $10^{-8}$. Les résultats sont très satisfaisants pour la conservation d'une solution stationnaire sur une équation non-linéaire. Sur la Figure \ref{fig:benartzi_test3_sol}, on représente la localisation spatiale de l'erreur ainsi que la solution calculée au temps $t=6$.
Les oscillations observées s'apparentent à des oscillations dispersives. Leur amplitude demeure très limitée.

\begin{figure}[htbp]
\begin{center}
\includegraphics[height=5cm]{erreur_test3.png}
\includegraphics[height=5cm]{cons_test3.png}
\end{center}
\caption{Courbe d'erreur pour la solution stationnaire. L'erreur en norme et l'erreur de conservation est représentée pour le test stationnaire de l'équation \eqref{eq:advection_sphere_NL}. Le pas de temps est donné par $\Delta t = 0.96 \Delta \xi / \pi$. Le temps final est $t=6$. Le paramètre de la Cubed-Sphere est $N=32$. L'erreur de conservation mesurée n'est pas relative car la masse totale initiale est nulle. L'erreur sur la conservation est très faible.}
\label{fig:benartzi_test3_hist}
\end{figure}

\begin{figure}[htbp]
\begin{center}
\includegraphics[scale=0.5]{solexacte_BA3.png}
\includegraphics[scale=0.5]{erreur_BA3.png}
\end{center}
\caption{Solution exacte (haut) et erreur (bas) pour le test stationnaire sur l'équation \eqref{eq:advection_sphere_NL}. Le paramètre de la Cubed-Sphere est $N=32$. Le pas de temps est donné par $\Delta t = 0.96 \Delta \xi / \pi = 0.015$. On représente les fonctions au temps $t=6$.}
\label{fig:benartzi_test3_sol}
\end{figure}


















