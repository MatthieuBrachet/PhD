\documentclass[11pt]{beamer}
\usetheme{Madrid}
\usepackage[utf8]{inputenc}
\usepackage[francais]{babel}
\usepackage[T1]{fontenc}
\usepackage{amsmath}
\usepackage{amsfonts}
\usepackage{amssymb}
\usepackage{graphicx}
\usepackage{tikz}
\title[Résolution numérique d'EDP sur $\mathbb{S}^2$]{Schémas Compact Hermitiens sur la Sphère\\
Rencontres Doctorales Lebesgue}
\setbeamercovered{transparent} 
%\setbeamertemplate{navigation symbols}{} 
\author{Brachet Matthieu}
\logo{IECL.jpg} 
\date[29.10.2015]{Jeudi 29 Octobre 2015}
\institute[IECL]{Institut Elie Cartan de Lorraine}
%\subject{} 
\begin{document}

\begin{frame}
\titlepage
\begin{center}
\includegraphics[width=2.5cm]{IECL.jpg}
\end{center}
\end{frame}

% ********************************************************************************
\section{Introduction}
\begin{frame}{Introduction}

\begin{exampleblock}{But :}
Résoudre des EDP sur la sphère.
\end{exampleblock}

\begin{exampleblock}{Pourquoi :}
Applications numériques en océanographie, climatologie, ...
\end{exampleblock}

\begin{block}{Dans cet exposé :}
Résolution numérique de :
$$\left\{
\begin{array}{rcl}
\dfrac{\partial h}{\partial t} + \mathbf{c} \cdot \nabla_S h & = & 0 \\
h(\mathbf{x},t=0) & = & h_0 ( \mathbf{x} )
\end{array}
\right. \hspace{1cm} \mathbf{x} \text{ sur la sphère, } t>0$$
\end{block}
\end{frame}

\begin{frame}
\tableofcontents
\end{frame}

% ********************************************************************************

\section{Calcul du gradient sphèrique}
\subsection{Maillage "Cube-Sphere"}
\begin{frame}{Maillage "Cube-Sphere"}

\begin{columns}
\column{0.45\textwidth}
Poles :
\begin{itemize}
\item Nord : $N (0,0,R)$,
\item Sud : $S (0,0,-R)$.
\end{itemize}

\begin{center}
\begin{tikzpicture}[scale=1.2]
    \draw (0,1) node {$\bullet$} ;
	\draw (0,1) node[above]{$N$} ;
	\draw (0,-1) node {$\bullet$} ;
	\draw (0,-1) node[below]{$S$} ;
    \draw (0,1) arc (90:270:0.75cm and 1cm);
    \draw (0,1) arc (90:270:0.50cm and 1cm);
    \draw (0,1) arc (90:270:0.25cm and 1cm);
    \draw (0,1) arc (90:270:0cm and 1cm);
    \draw (0,1) arc (90:-90:0.25cm and 1cm);
    \draw (0,1) arc (90:-90:0.50cm and 1cm);
    \draw (0,1) arc (90:-90:0.75cm and 1cm);    
    \draw (0,0) circle (1cm);
    \shade[ball color=blue!10!white,opacity=0.20] (0,0) circle (1cm);
\end{tikzpicture}
\end{center}
\pause

\column{0.45\textwidth}
Poles :
\begin{itemize}
\item East : $E (0,R,0)$,
\item West : $W (0,-R,0)$.
\end{itemize}

\begin{center}
\begin{tikzpicture}[scale=1.2]
	\draw (0,1) node {$\bullet$} ;
	\draw (0,1) node[above]{$N$} ;
	\draw (0,-1) node {$\bullet$} ;
	\draw (0,-1) node[below]{$S$} ;
	\draw (1,0) node {$\bullet$} ;
	\draw (1,0) node[right]{$E$} ;
	\draw (-1,0) node {$\bullet$} ;
	\draw (-1,0) node[left]{$W$} ;   
    \draw (-1,0) arc (180:360:1cm and 0.75cm);
    \draw (-1,0) arc (180:360:1cm and 0.25cm);
    \draw (-1,0) arc (180:360:1cm and 0.5cm);
    \draw (-1,0) arc (180:360:1cm and 0cm);
    \draw (-1,0) arc (180:360:1cm and -0.25cm);
    \draw (-1,0) arc (180:360:1cm and -0.5cm);
    \draw (-1,0) arc (180:360:1cm and -0.75cm);    
    \draw (0,0) circle (1cm);
    \shade[ball color=blue!10!white,opacity=0.20] (0,0) circle (1cm);
\end{tikzpicture}
\end{center}
\end{columns}
\end{frame}

\begin{frame}
\begin{exampleblock}{}
Création d'un maillage couvrant une partie de la sphère.
\end{exampleblock}
\begin{figure}
\begin{columns}
\column{0.45\textwidth}
\begin{center}
\begin{tikzpicture}[scale=1.2]
    \draw (-0.60,-0.60) arc (225:315:0.85cm and 0.50cm);
    \draw (-0.74,-0.17) arc (210:331:0.85cm and 0.16cm);
    \draw (-0.705,-0.35) arc (225:314:1cm and 0.5cm);
    \draw (-0.75,0) arc (180:360:0.75cm and 0cm);    
    \draw (0.74,0.17) arc (30:151:0.85cm and 0.16cm);
    \draw (0.705,0.35) arc (45:135:1cm and 0.5cm);
    \draw (0.60,0.60) arc (45:135:0.85cm and 0.50cm);
    \draw (0.60,-0.60) arc (-45:45:0.51cm and 0.85cm);
    \draw (0.17,-0.74) arc (-60:60:0.16cm and 0.85cm);
    \draw (0.35,-0.705) arc (-58:60:0.3cm and 0.82cm);
    \draw (0,0.75) arc (90:270:0cm and 0.75cm);
    \draw (-0.60,0.60) arc (135:225:0.51cm and 0.85cm);
    \draw (-0.17,0.74) arc (120:240:0.16cm and 0.85cm);
    \draw (-0.35,0.705) arc (122:240:0.3cm and 0.82cm);    
    \draw (0,0) circle (1cm);
    \draw (0.60,0.60) -- (0.707,0.707) ;
    \draw (-0.60,0.60) -- (-0.707,0.707) ;
    \draw (0.60,-0.60) -- (0.707,-0.707) ;
    \draw (-0.60,-0.60) -- (-0.707,-0.707) ;
    \shade[ball color=blue!10!white,opacity=0.20] (0,0) circle (1cm);
\end{tikzpicture}
\end{center}
\column{0.45\textwidth}
\begin{center}
\includegraphics[scale=0.055]{CS_grid.jpg}
\end{center}
\end{columns}
\caption{Cube-Sphere - Front (gauche) et Complet (droite)}
\end{figure}
\end{frame}

% ***

\subsection{Coordonnées}
\begin{frame}{Coordonnées}
\begin{columns}
\begin{column}{5cm}
\begin{figure}
\begin{tikzpicture}[scale=1.8]
	\draw (0,0) node {$\bullet$} ;
	\draw [>=stealth, ->] (0,0) -- (1.5,0) ; 
	\draw (1.5,0) node[above right]{$y$} ;   
	\draw [>=stealth, ->] (0,0) -- (1.1,-0.8) ; 
	\draw (1.1,-0.8) node[above right]{$x$} ; 
	\draw [>=stealth, ->] (0,0) -- (0,1.5) ; 
	\draw (0,1.5) node[above right]{$z$} ; 

    \draw[color=green] (0.065,0.5) arc (85:68:1cm and 0.5cm);
    \draw[color=green] (0.2,0.55) node{$\alpha$} ;
    \draw[color=red] (0.4,-0.05) arc (-5:25:0.41cm and 1cm);
    \draw[color=red] (0.48,0.2) node{$\beta$} ;
    \draw (-1,0)[dotted, color=black!40] arc (180:360:1cm and -0.5cm);
    \draw (0,1)[dotted, color=black!40] arc (90:270:-0.4cm and 1cm);
    \draw (-1,0)[dotted, color=black!20] arc (180:360:1cm and 0.5cm);
    \draw (0,1)[dotted, color=black!20] arc (90:270:0.4cm and 1cm);
    
    \draw (-1,0) arc (180:360:1cm and 0.06cm);
    \draw (0,1) arc (90:270:-0.08cm and 1cm);
    \draw[dashed] (-1,0) arc (180:360:1cm and -0.06cm);
    \draw[dashed] (0,1) arc (90:270:0.08cm and 1cm);
    
    \draw (0.36,0.46) node {$\bullet$} ;
	\draw (0.36,0.46) node[above right]{$M$} ;
    \draw (0,0) circle (1cm);
    \shade[ball color=blue!10!white,opacity=0.20] (0,0) circle (1cm);
\end{tikzpicture}
\caption{Coordonnées sur la CS}
\end{figure}
\end{column}
\pause
\begin{column}{5cm}
Dans ce système de coordonnées $M$ est donné par :

\begin{itemize}
\item $\alpha$ abscisse curviligne le long du grand cercle "horizontal".

\item $\eta$ angle longitudinal pour le panel I.

\item $\beta$ abscisse curviligne le long du grand cercle "vertical".

\item $\xi$ angle latitudinal pour le panel I.
\end{itemize}
\end{column}
\end{columns}
\end{frame}

\subsection{Gradient sphérique}
\begin{frame}{Calcul du gradient sphèrique}
Définition naturelle du gradient :
$$\nabla_s u = \dfrac{\partial u}{\partial \xi}_{|\eta} g^\xi + \dfrac{\partial u}{\partial \eta}_{|\xi} g^\eta $$

$\rightarrow$ $\dfrac{\partial u}{\partial \xi}_{|\eta}$ et $\dfrac{\partial u}{\partial \eta}_{|\xi}$ : pas le long des grands cercles. Exprimer dans $(\alpha, \beta)$.

$\rightarrow$ $g^\xi$ et $g^\eta$ vecteurs contravariants obtenus par inversion du tenseur métrique.
\pause

\begin{block}{Gradient sur la CS}
$$\nabla_s u = \frac{\partial u}{\partial \alpha}_{|\eta} \left( \cos \eta \dfrac{1+ \tan^2 \xi}{1 + \cos^2 \eta \tan^2 \xi} \right) g^\xi + \frac{\partial u}{\partial \beta}_{|\xi} \left( \cos \xi \dfrac{1+ \tan^2 \eta}{1 + \cos^2 \xi \tan^2 \eta} \right) g^\eta $$
\end{block}

$\rightarrow$ Évaluer $\dfrac{\partial u}{\partial \alpha}_{|\eta}$ et $\dfrac{\partial u}{\partial \beta}_{|\xi}$ en chaque point du maillage.
\end{frame}


% ***

\subsection{Dérivées Hermitiennes}
\begin{frame}{Dérivées Hermitiennes}
$u$ assez régulière et périodique. Alors :

$$\dfrac{u(x+h)-u(x-h)}{2h} = u'(x) + \dfrac{h^2}{6} \underbrace{u^{(3)}(x)}_{= \left( u'(x) \right)''} + \mathcal{O}\left( h^4 \right)$$

\pause

$$\dfrac{u(x+h)-u(x-h)}{2h} = u'(x) + \dfrac{h^2}{6} \mathbf{ \dfrac{u'(x-h) - 2 u'(x) + u'(x+h) }{h^2}} + \mathcal{O}\left( h^4 \right)$$

\begin{block}{}
Après simplification :
$$\dfrac{u(x+h)-u(x-h)}{2h} = \dfrac{1}{6} \left[ u'(x-h) + 4 u'(x) + u'(x+h) \right] + \mathcal{O}\left( h^4 \right)$$
\end{block}

$\rightarrow$ Résolution d'un système linéaire sur chaque grand cercle.
\end{frame}

% ***

\subsection{Solveur Rapide}
\begin{frame}{Solveur Rapide}
\begin{block}{Schéma compact}
\begin{itemize}
\item $U' = \left[ u'_1, \hdots ,u'_N \right]^T \approx \left[ u'(x_1), \hdots ,u'(x_N) \right]^T$

\item $U' = \left[ u_1, \hdots ,u_N \right]^T$
\end{itemize}

\pause

Alors :
$$\dfrac{1}{2h}\begin{pmatrix}
0 & 1 &   &   & -1 \\ 
-1 & 0 & 1 & (0) &   \\ 
  & \ddots & \ddots & \ddots &   \\ 
  & (0) & -1 & 0 & 1 \\ 
1 &   &   & -1 & 0
\end{pmatrix} U = \dfrac{1}{6} \begin{pmatrix}
4 & 1 &   &   & 1 \\ 
1 & 4 & 1 & (0) &   \\ 
  & \ddots & \ddots & \ddots &   \\ 
  & (0) & 1 & 4 & 1 \\ 
1 &   &   & 1 & 4
\end{pmatrix} U'$$
\end{block}
\pause
$\rightarrow$ Résolution d'un système linéaire sur chaque grand cercle.
\end{frame}

\begin{frame}
Résolution d'un système "quasi" tridiagonal en : 
$$\dfrac{1}{6} \begin{pmatrix}
4 & 1 &   &   & 1 \\ 
1 & 4 & 1 & (0) &   \\ 
  & \ddots & \ddots & \ddots &   \\ 
  & (0) & 1 & 4 & 1 \\ 
1 &   &   & 1 & 4
\end{pmatrix}$$

Utilisation de la formule de Shermann-Morisson-Woodbury :

\begin{block}{Formule de Shermann-Morisson-Woodbury :}
Soit $N \geq 1$. Supposons que $A\text{, }B \in \mathcal{M}_N \left( \mathbb{R} \right)$ soient deux matrices inversibles telles que :
$$A = B + RS^T$$
avec $R$ et $S \in \mathcal{M}_{N,n} \left( \mathbb{R} \right)$. Alors :

$$A^{-1} = B^{-1} - B^{-1}R\left( I + S^T B^{-1} R\right)^{-1} S^T B^{-1}$$
\end{block}
\end{frame}

\begin{frame}{Solveur Rapide}
Alors pour résoudre $Ax=b$ où A est la matrice précédente :

\begin{enumerate}
\item Calcul de $V_1=B^{-1}b$ grâce à l'algorithme de Thomas,

\item $V_2 = S^T V_1$,

\item Calcul de $V_3 = \left( I + S^T B^{-1} R \right)^{-1} V_2$ (système $2 \times 2$ dans notre cas),

\item $V_4 = R V_3$,

\item Calcul de $V_5 = B^{-1} V_4$ par l'algorithme de Thomas,

\item Solution : $x = V_1 - V_5$.
\end{enumerate}
\end{frame}

% ***

\subsection{Interpolation}
\begin{frame}

  \begin{alertblock}{Problème...}
  Les points des cercles ne coïncident pas avec les points du maillage  
  \end{alertblock}
  
\begin{columns}
\column{0.45\textwidth}

\begin{figure}
\begin{center}
\includegraphics[height=3cm]{fig22.jpg}
\end{center}
\caption{grand cercle}
\end{figure}

\pause
\column{0.45\textwidth}

\begin{exampleblock}{Solution : Interpolation}
  \begin{itemize}
  \item Interpolation d'Hermite

    $\rightarrow$ Faible coût en calcul. \\
    $\rightarrow$ Phénomène de Runge... Grosse erreur due au grand nombre de points.

  \item Choix du spline cubique

    $\rightarrow$ Systèmes linéaires à résoudre.\\
    $\rightarrow$ Erreur en $\mathcal{O}\left( h^4 \right)$.
  \end{itemize}
\end{exampleblock}
\end{columns}
\end{frame}

\begin{frame}
\tableofcontents
\end{frame}

% ********************************************************************************

\section{Schéma en temps}
\subsection{Runge-Kutta d'ordre 4}
\begin{frame}
\begin{block}{}
Résolution de :
$$\left\{
\begin{array}{rcl}
\dfrac{\partial h}{\partial t} & = & f(t,h) = - \mathbf{c} \cdot \nabla_S h\\
u(t=0) & = & u_0
\end{array}
\right.$$
\end{block}

\pause
\begin{block}{Méthode de Runge-Kutta d'ordre 4 (RK4 pour les intimes)}
$u(t)$ donné, 
\begin{itemize}
\item $K_1 = f(t, h(t))$
\item $K_2 = f(t+\Delta t/2, h(t)+\Delta t K_1 /2)$
\item $K_3 = f(t+\Delta t/2, h(t)+\Delta t K_2 /2)$
\item $K_4 = f(t+\Delta t, h(t)+\Delta t K_3)$
\item $h(t+\Delta t) = h(t)+\dfrac{\Delta t}{6}\left( K_1 + 2 K_2 + 2 K_3 + K_4\right)$
\end{itemize}
\end{block}
\end{frame}

\subsection{Quelques propriétés}
\begin{frame}
On cherche à résoudre :
$$\left\{
\begin{array}{rcl}
\dfrac{\partial h}{\partial t} + \mathbf{c} \cdot \nabla_S h & = & 0 \\
h(\mathbf{x},t=0) & = & h_0 ( \mathbf{x} )
\end{array}
\right. \hspace{1cm} \mathbf{x} \text{ sur la sphère, } t>0$$

\pause
\begin{block}{Condition Courant–Friedrichs–Lewy (1928)}
Pour que le schéma soit stable, il suffit que :
$$cfl := \dfrac{\|c\|_{\infty} \times \Delta t}{\Delta x} \leq 1$$
\end{block}

\pause
\begin{block}{Dissipation et dispersion numérique}
Schéma en espace centré $\Rightarrow$ Dissipation et dispersion numérique.
\end{block}
\end{frame}

\subsection{Filtrage en espace}
\begin{frame}
Augmenter la régulatité de la solution pour limiter la dissipation et la dispersion numérique par un filtre à l'ordre 10 :

\begin{block}{}
$$h_{i} \leftarrow \sum_{k} f_{k} h_{i+k}$$
\end{block}

avec : 

$$\begin{pmatrix}
f_0 \\ 
f_1 = f_{-1} \\ 
f_2 = f_{-2} \\ 
f_3 = f_{-3} \\ 
f_4 = f_{-4} \\ 
f_5 = f_{-5}
\end{pmatrix} = \begin{pmatrix}
772/1024 \\ 
210/1024 \\ 
-120/1024 \\ 
45/1024 \\ 
-10/1024 \\ 
1/1024
\end{pmatrix} $$

\begin{block}{Remarque}
Convolution discréte par une fonction adaptée.
\end{block}
\end{frame}

\begin{frame}
\tableofcontents
\end{frame}

% ********************************************************************************

\section{Benchmark}

\begin{frame}
On résout :
$$\left\{
\begin{array}{rcl}
\dfrac{\partial h}{\partial t} + \mathbf{c} \cdot \nabla_S h & = & 0 \\
h(\mathbf{x},t=0) & = & h_0 ( \mathbf{x} )
\end{array}
\right. \hspace{1cm} \mathbf{x} \text{ sur la sphère, } t>0$$

Choix de $h_0$ et $\mathbf{c}$ fixés tels que la solution exacte est \textbf{connue}.

$\rightarrow$ Comparaison de la solution exacte et approchée par le schéma numérique.
$$e^n = max_{\text{face}} \dfrac{ \| h^n_{i,j} - h(\mathbf{x}_{i,j}, t^n ) \| }{\| h(\mathbf{x}_{i,j}, t^n ) \|}$$

3 tests : 
\begin{itemize}
\item Test de déplacement,\\
\item Test de déformation,\\
\item Test mixte.
\end{itemize}

\end{frame}


% ***

\subsection{Bump en rotation}
\begin{frame}{Test 1 : BUMP en rotation}
\begin{columns}
\column{0.45\textwidth}
Idée :

\begin{itemize}
\item $\mathbf{c}$ et $h_0$ donnés,

\item faire tourner une "cloche" (le BUMP) autour de la sphère.
\end{itemize}


\column{0.45\textwidth}

\begin{figure}
\href{run:CSapprox_test0.avi}{\includegraphics[scale=0.25]{normerreur_test0_100cfl90.png}} 
\caption{Test 1 : Bump en rotation - 40 mailles par face - CFL=0.9}
\end{figure}

\end{columns}
\end{frame}

% ***
\subsection{Vortex Stationnaire}
\begin{frame}{Test 2 : Vortex stationnaire (Nair et Machenhauer - 2001)}
\begin{columns}
\column{0.45\textwidth}
Idée :

\begin{itemize}
\item $\mathbf{c}$ et $h_0$ donnés,

\item Formation d'une "tempête" (vortex) sur une zone localisée de la sphère.
\end{itemize}


\column{0.45\textwidth}

\begin{figure}
\href{run:CSapprox_test1.avi}{\includegraphics[scale=0.25]{normerreur_test1_100cfl90.png}} 
\caption{Test 2 : Vortex stationnaire - 40 mailles par face - CFL=0.9}
\end{figure}
\end{columns}
\end{frame}

% ***

\subsection{Vortex Instationnaire}
\begin{frame}{Test 3 : Vortex instationnaire (Nair et Jablonowski - 2007)}
\begin{columns}
\column{0.45\textwidth}
Idée :

Combiner les deux tests précédents : test de déformation et de déplacement.
\begin{itemize}

\item $\mathbf{c}$ et $h_0$ donnés et issus des tests 1 et 2,

\item Formation d'une "tempête" (vortex) se déplacant le long d'un cercle sur la sphère.
\end{itemize}


\column{0.45\textwidth}

\begin{figure}
\href{run:CSapprox_test2.avi}{\includegraphics[scale=0.25]{14-Oct-2015_normerreur_test_2.png}} 
\caption{Test 3 : Vortex instationnaire - 40 mailles par face - CFL=0.9}
\end{figure}
\end{columns}
\end{frame}

\begin{frame}

Estimation de l'ordre de convergence sur le test 3 :

\begin{figure}
\begin{tabular}{ccc}
Nombre de mailles par coté de face & $max_n |e_2^n|$ & estimation de l'ordre \\
\hline
\hline
$40$ & $9.7386 (-2)$ & - \\
\hline 
$50$ & $5.0160 (-2)$ & $3.0399$ \\
\hline
$60$ & $2.6157 (-3)$ & $3.6365$ \\
\hline
$80$ & $8.3722 (-4)$ & $4.0173$ \\
\hline
$100$ & $3.4524 (-4)$ & $4.0143$\\
\hline
$150$ & $6.9199 (-5)$ & $3.9966$
\end{tabular}
\caption{Analyse de convergence du test 3 - $cfl = 0.7$}
\end{figure}

\end{frame}


% ********************************************************************************
\section{Conclusion et perspectives}
\begin{frame}{Conclusion et perspectives}
\begin{block}{Bilan}
\begin{itemize}
\item Méthode rapide (solveurs rapides) et précise (ordre 4),

\item Bons résultats sur les tests.
\end{itemize}
\end{block}

\pause

\begin{block}{Avenir...}
\begin{itemize}
\item Etude sur les harmoniques sphériques,

\item Mise en place d'un zoom \textit{Local Defect Correction},

\item Résolution de l'équation de Saint-Venant (Shallow Water Equation).
\end{itemize}
\end{block}
\end{frame}

% ********************************************************************************

\begin{frame}
\frametitle{Bibliographie}

\begin{thebibliography}{9}
        

\scriptsize{

\bibitem{Lele}
         Sanjiva K. Lele
         \emph{: Compact Finite Difference Schemes with spectral-like resolution}.
         1991.

\bibitem{Croisille}
         J.-P. Croisille
         \emph{: Hermitian compact interpolation on the cubed-sphere grid}.
         2013.

\bibitem{Nair Jablonowski}
         Ramachandran D. Nair et Christiane Jablonowski
         \emph{: Moving Vortices on the Sphere : A test case for Horizontal Advection Problems}.
         1991.

\bibitem{Redonnet}
         Stéphane Redonnet
         \emph{: Simulation de la propagation acoustique en présence d'écoulements quelconques et de structures solides par résolution numérique des équations d'Euler}.
         Thèse, 2001.
         
\bibitem{Williamson}
         David L. Williamson, John B. Drake, James J. Hack, Rüdiger Jakob et Paul N.Swarztrauber
         \emph{: A strandard test set for Numerical Approximations to the Shallow Water Equations in Spherical Geometry}.
         1994.
    
    }     
         
         
         
\end{thebibliography}
\end{frame}

% ********************************************************************************
\begin{frame}
\begin{center}
Merci de votre attention :)
\end{center}
\end{frame}


\end{document}