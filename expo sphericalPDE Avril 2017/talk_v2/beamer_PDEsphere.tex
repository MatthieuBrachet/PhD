\documentclass[11pt]{beamer}
\usetheme{Madrid}
\usepackage[utf8]{inputenc}
\usepackage[english]{babel}
\usepackage{amsmath}
\usepackage{amsfonts}
\usepackage{amssymb}
\usepackage{graphicx}
\def\gint{\displaystyle\int}

\author[Matthieu Brachet]{Matthieu Brachet}
\title[PDEs on the Sphere]{Numerical approximation of propagation problems on the sphere using a compact scheme}
\institute[IECL]{Institut Elie Cartan de Lorraine,\newline UMR CNRS 7502, Dép. de Mathématiques,\newline Metz, France} 
\date[Paris, April 7th, 2017]{April 7th, 2017} 

\begin{document}

\begin{frame}
\titlepage
\includegraphics[scale=0.22]{iecl.jpg}
\includegraphics[scale=0.41]{ul.jpg}
\end{frame}

\begin{frame}{}
\begin{itemize}
\item Works at Institut Elie Cartan de Lorraine, UMR CNRS 7502 (Metz, France),
\item PhD supervised by Jean-Pierre Croisille.
\end{itemize}
\end{frame}

%% ***************************************************************************************************************************************

\begin{frame}{Outline}
\begin{enumerate}
\item The Cubed-Sphere
\item Compact differential operators on the Cubed-Sphere
\item Numerical results
\end{enumerate}
\end{frame}


\begin{frame}{The Cubed-Sphere}
\begin{columns}
\column{0.45\textwidth}
\begin{figure}
   \def\svgwidth{1 \textwidth}
\input{drawing12.pdf_tex}
\end{figure}

\column{0.5\textwidth}
\begin{itemize}
\item 2 sets of great circles $C_i^{(1)}$ and $C_j^{(2)}$, $-N/2 \leq i,j \leq N/2$,

\item $C_0^{(1)}$ and $C_0^{(2)}$ intersect with an angle of $90$ degrees

\item $\mathbf{x}_{i,j}=C_i^{(1)} \cap C_j^{(2)}$ are grid points on a panel of the Cubed-Sphere.
\end{itemize}
\end{columns}
\end{frame}


\begin{frame}{The Cubed Sphere grid: 6 panels}
\begin{figure}
\begin{center}
\hspace{-1.cm}
\includegraphics[scale=0.28]{plot_CS.png}
\caption{\footnotesize
The Cubed-Sphere grid with $N=16$ ($16^2$ cells by panel).
}
\end{center}
\label{fig:1.4}
\end{figure}
\end{frame}




\begin{frame}{Coordinate angles on a panel}
\begin{block}{Angles $\xi$ and $\eta$}
\begin{itemize}
\item \textbf{First equatorial angle} $\xi$ with $-\pi/4 \leq \xi \leq \pi/4$ ("zonal"),
\item \textbf{Second equatorial angle} $\eta$ with $-\pi/4 \leq \eta \leq \pi/4$ ("meridional"). 
\end{itemize}
\end{block}

%\begin{columns}
%\column{0.25\textwidth}
\begin{figure}
\includegraphics[scale=0.225]{xieta.png}
\end{figure}

%\column{0.7\textwidth}
%\begin{block}{Basis and metric}
%\begin{itemize}
%\item Local basis : $(\mathbf{g}_{\xi}, \mathbf{g}_{\eta} ) = ( \partial_{\xi} \mathbf{x}, \partial_{\eta} \mathbf{x} ) \in \mathbb{T}\mathbb{S}_R^2$,
%
%\item Metric tensor :
%$$G = \left[ \begin{array}{rcl}
%\mathbf{g}_{\xi} \cdot \mathbf{g}_{\xi} & \mathbf{g}_{\xi} \cdot \mathbf{g}_{\eta} \\ 
%\mathbf{g}_{\eta} \cdot \mathbf{g}_{\xi} & \mathbf{g}_{\eta} \cdot \mathbf{g}_{\eta}
%\end{array} \right] = \left[ \begin{array}{rcl}
%G_{1,1}&G_{1,2}\\
%G_{2,1}&G_{2,2}
%\end{array} \right]$$
%
%\item Dual basis $(\mathbf{g}^{\xi},\mathbf{g}^{\eta})$ :
%$$
%\left\{
%\begin{array}{ccc}
%G_{1,1}\mathbf{g}^{\xi} + G_{1,2}\mathbf{g}^{\eta} & = & \mathbf{g}_{\xi} \\ 
%G_{2,1}\mathbf{g}^{\xi} + G_{2,2}\mathbf{g}^{\eta} & = & \mathbf{g}_{\eta} \\ 
%\end{array} 
%\right.
%$$
%\end{itemize}
%\end{block}
%\end{columns}
\end{frame}

%% ****************************************************************************************************************************************


\begin{frame}{Gradient, Divergence and Curl}

\begin{block}{Differential operators}
Let $\mathbf{u} : \mathbb{S}_a^2 \mapsto \mathbb{T}\mathbb{S}_a^2$ and $h : \mathbb{S}_a^2 \mapsto \mathbb{R}$ be regular functions. Then :
\begin{itemize}
\item Gradient of $h$:
\begin{equation}
\nabla_T h = \dfrac{\partial h}{\partial \xi} \mathbf{g}^{\xi} + \dfrac{\partial h}{\partial \eta} \mathbf{g}^{\eta}
\end{equation}
\item Divergence of $\mathbf{u}$:
\begin{equation}
\nabla_T \cdot \mathbf{u} = \dfrac{\partial \mathbf{u}}{\partial \xi}\cdot \mathbf{g}^{\xi} + \dfrac{\partial \mathbf{u}}{\partial \eta} \cdot \mathbf{g}^{\eta}
\end{equation}\\
\item Curl of $\mathbf{u}$:
\begin{equation}
\nabla_T \times \mathbf{u} = \dfrac{\partial \mathbf{u}}{\partial \xi}\times \mathbf{g}^{\xi} + \dfrac{\partial \mathbf{u}}{\partial \eta} \times \mathbf{g}^{\eta}
\end{equation}

\end{itemize}

\end{block}
\end{frame}

%% ***************************************************************************************************************************************

%\begin{frame}{Three-point Hermitian Derivative Operator}
%\begin{block}{Local formula : }
%Standard local finite difference scheme of a sequence $(u_j)_{j \in \mathbb{Z}}$ :
%\begin{equation}
%\dfrac{u_{j+1}-u_{j-1}}{2\Delta x} = \delta_x u_j \;\;\; j \in \mathbb{Z}
%\end{equation}
%\end{block}
%
%\begin{block}{Second order accuracy}
%Approximation of $u'(x_j)$ :
%\begin{equation}
%\delta_x u_j = u'(x_j) + \dfrac{1}{12} \partial_x^{(3)} u(x_j) \Delta x^2 + \mathcal{O}(\Delta x^4)
%\end{equation}
%\end{block}
%\end{frame}

%% ***************************************************************************************************************************************

\begin{frame}{Fourth order Hermitian derivative}
\begin{block}{Finite difference derivative}
\begin{itemize}
\item Periodic Data : $(u_j)_{j \in \mathbb{Z}}$ with $u_{j+J}=u_j$, $J$ the period.

\item Hermitian derivative : $(u_{x,j})_{j \in \mathbb{Z}}$ solution of 
$$
\dfrac{1}{6}u_{x,j+1}+\dfrac{4}{6}u_{x,j}+\dfrac{1}{6}u_{x,j-1}=\dfrac{u_{j+1}-u_{j-1}}{2 \Delta x} \text{, } \Delta x > 0
$$ 

\item Fourth order accuracy.
\end{itemize}
\end{block}
\end{frame}

%% ***************************************************************************************************************************************

\begin{frame}{Hermitian derivatives along great circles}
\begin{block}{Principle}
\begin{itemize}
\item Calculate Hermitian derivatives along a series of great circles.  
\end{itemize}
\end{block}

\begin{block}{Great circle = coordinate lines}
\begin{itemize}
\item Coordinate lines on each panel on the CS are great circles sections.
\item Each of these sections are extended into a full great circle.
\end{itemize}
\end{block}
\end{frame}

%% ***************************************************************************************************************************************

\begin{frame}{Interpolating data along a coordinate great circle}
\begin{figure}
\includegraphics[scale=0.18]{fig35.jpg}
\end{figure}
\end{frame}

%% ***************************************************************************************************************************************

\begin{frame}{Discrete differential operators}
$\Delta = \Delta \xi = \Delta \eta = \pi / (2N)$ represents the discrete parameter,
\begin{block}{Discrete gradient, divergence and curl}
$$
\bullet \hspace{0.2cm} \nabla_{T}h=\frac{\partial h}{\partial\xi}_{|\eta}\mathbf{g}^{\xi}
+\frac{\partial h}{\partial\eta}_{|\xi}\mathbf{g}^{\eta} \approx \nabla_{T,\Delta} h=h_{\xi}\mathbf{g}^{\xi}
+h_{\eta} \mathbf{g}^{\eta}
$$

$$
\bullet \hspace{0.2cm} \nabla_{T} \cdot \mathbf{u}=\frac{\partial \mathbf{u}}{\partial\xi}_{|\eta} \cdot \mathbf{g}^{\xi}
+\frac{\partial \mathbf{u}}{\partial\eta}_{|\xi} \cdot \mathbf{g}^{\eta} \approx \nabla_{T,\Delta} \cdot \mathbf{u}=\mathbf{u}_{\xi} \cdot \mathbf{g}^{\xi}
+\mathbf{u}_{\eta} \cdot \mathbf{g}^{\eta}
$$

$$
\bullet \hspace{0.2cm} \nabla_{T} \times \mathbf{u}=\frac{\partial \mathbf{u}}{\partial\xi}_{|\eta} \times \mathbf{g}^{\xi}
+\frac{\partial \mathbf{u}}{\partial\eta}_{|\xi} \times \mathbf{g}^{\eta} \approx \nabla_{T,\Delta} \times \mathbf{u}=\mathbf{u}_{\xi} \times \mathbf{g}^{\xi}
+\mathbf{u}_{\eta} \times \mathbf{g}^{\eta}
$$

where $(\mathbf{g}^{\xi},\mathbf{g}^{\eta})$ is the dual basis at $(\xi_i, \eta_j)$ and $h_{\xi}$, $h_{\eta}$, $\mathbf{u}_{\xi}$ and $\mathbf{u}_{\eta}$ are the Hermitian derivatives at points $(\xi_i, \eta_j)$ on each panel $(k)$.
\end{block}
\end{frame}

%% ***************************************************************************************************************************************

\begin{frame}{Discrete differential operators : error estimate}
\begin{block}{Proposition :}
Let $h:\mathbb{S}^2 \mapsto \mathbb{R}$ and $\mathbf{u}: \mathbb{S}^2 \mapsto \mathbb{T}\mathbb{S}^2$ be regular functions. Then:
\begin{itemize}
\item Gradient : \begin{equation}
\| (\nabla_{T}h)^* - \nabla_{T,\Delta}h^* \|_{\infty} \leq \mathcal{O} \left( \Delta^3 \right)
\end{equation}
\item Divergence : \begin{equation}
\| (\nabla_{T}\cdot \mathbf{u})^* - \nabla_{T,\Delta}\cdot \mathbf{u}^* \|_{\infty} \leq \mathcal{O} \left( \Delta^3 \right)
\end{equation}
\item Curl : \begin{equation}
\| (\nabla_{T}\times \mathbf{u})^* - \nabla_{T,\Delta}\times \mathbf{u}^* \|_{\infty}  \leq  \mathcal{O} \left( \Delta^3 \right)
\end{equation}
\end{itemize}
The operator "$*$" stands for : "restricted to the Cubed-Sphere".
\end{block}
\begin{block}{Effective order of accuracy}
Fourth order accuracy is observed in the three cases!
\end{block}
\end{frame}

%% ***************************************************************************************************************************************

%\begin{frame}{Numerical order of accuracy of approximated curl}
%\begin{block}{Approximate curl of a tangential vector field $\mathbf{u} = u(\theta) \mathbf{e}_{\lambda}$ (zonal dependance)}
%$u(\theta)=\left\lbrace
%\begin{array}{ll}
%\frac{80}{e_n} \exp\left[ \frac{1}{(\theta-\theta_0)(\theta-\theta_1)} \right] & \text{ if } \theta_0 \leq \theta \leq \theta_1 \\
%0 & \text{ else}
%\end{array}\right.$ 
%
%with $e_n$ a normalization constant.
%\begin{figure}
%\includegraphics[scale=0.35]{rate_vort.png}
%\caption{\footnotesize Convergence rate of the Hermitian curl of the zonal 
%velocity $\mathbf{u} = u(\theta) \mathbf{e}_{\lambda}$.}
%\end{figure}
%\end{block}
%\end{frame}

%% ***************************************************************************************************************************************

%\begin{frame}{Solid Body Rotation [Williamson and al., 1992]}
%$\bullet$ $(\lambda, \theta)$ : longitude and latitude for $(NS)$ axis,
%
%$\bullet$ $(\lambda', \theta')$ : longitude, latitude (axis $(NS)$ rotated by an angle $\alpha$).
%
%\begin{columns}
%\column{0.45\textwidth}
%\begin{figure}
%\def\svgwidth{0.6 \textwidth}
%\vspace{0.5cm}
%\input{drawing34.pdf_tex}
%\end{figure}
%\column{0.45\textwidth}
%\begin{block}{}
%\begin{equation*}
%\left \{
%\begin{array}{l}
%\dfrac{\partial h}{\partial t} + \mathbf{c} \left( \mathbf{x} \right) \cdot \nabla_T h = 0\\[6pt]
%h(0,\mathbf{x}) = h_0(\mathbf{x})\\
%\end{array}
%\right.
%\end{equation*}
%with $\mathbf{c} \left( \lambda', \theta' \right) = \omega_0  cos \theta' \mathbf{e}_{\lambda'}$.
%\end{block}
%\end{columns}
%\begin{block}{}
%Solution :
%$$h \left( \mathbf{x} , t \right) = h_0 \left( R_{\omega_0 t}^{-1} \mathbf{x} \right)$$
%with $R_{\theta}$ rotation matrix with angle $\theta$.
%\end{block}
%\end{frame}

%% ***************************************************************************************************************************************

\begin{frame}{Semi-discrete scheme}
\begin{block}{Method of lines}
\begin{itemize}
\item \textbf{First step :} discretization in space,
\item \textbf{Second step :} discretization in time.
\end{itemize}
\end{block}

\begin{block}{Example}
\begin{itemize}
\item Advection equation :
$$\partial_t h + \mathbf{c}(\mathbf{x},t) \cdot \nabla_T h = 0$$
is discretized by a centered scheme
$$\dfrac{d}{dt} h_{i,j}^k + \mathbf{c}(\mathbf{x}_{i,j}^k,t) \cdot \nabla_{T,\Delta} h_{i,j}^k = 0$$
\end{itemize}
\end{block}
\end{frame}


%% ***************************************************************************************************************************************

\begin{frame}{Time discretization}
\textbf{Semi-discrete equation }(discrete in space and continuous in time):
\begin{equation}
\left\lbrace
\begin{array}{rcl}
\dfrac{\partial q}{\partial t} & = & F(q,t), \\
q(t=0,\mathbf{x}) & = & q_0(\mathbf{x}) \\
\end{array}
\right.
\end{equation}

\begin{block}{Filtering}
$\mathcal{F}$ is a spatial filter function.
\end{block}

\begin{block}{Runge-Kutta order 4 + Spatial filtering}

\begin{enumerate}
\item $K_1 = F(q^n, t^n)$,
\item $K_2 = F(q^n + \frac{\Delta t}{2} K_1, t^n + \frac{\Delta t}{2})$,
\item $K_3 = F(q^n + \frac{\Delta t}{2} K_2, t^n + \frac{\Delta t}{2})$,
\item $K_4 = F(q^n + \Delta t K_3, t^n + \Delta t)$
\item $\hat{q}^{n+1} = q^n + \frac{\Delta t}{6} \left( K_1 + 2 K_2 + 2 K_3 + K_4 \right)$
\item $q^{n+1} = \mathcal{F}(\hat{q}^{n+1})$
\end{enumerate}
\end{block}
\end{frame}

%% ***************************************************************************************************************************************

\begin{frame}{Filtering function $\mathcal{F}$ }
\begin{block}{High frequencies filtering \textit{(Alpert, Tam, Visbal, Bailly, ...)}}
\begin{itemize}
\item $\mathcal{F}(u)_j$ is the one dimensional filtering operator acting on the data $u_j$ by :
$$
\mathcal{F}(u)_j = \sum_{p=-M}^M f_{p} u_{j+p}\approx u_j + \mathcal{O}\left( \Delta x^{\alpha} \right)
$$
\item $\alpha$ : order of the filtering
\end{itemize}
\end{block}
\begin{columns}
\column{0.45\textwidth}
\begin{block}{10-th order filtering}
$\begin{pmatrix}
f_0\\ f_1 = f_{-1}\\ f_2 = f_{-2} \\ f_3=f_{-3} \\ f_4=f_{-4} \\ f_5 = f_{-5}
\end{pmatrix}=\begin{pmatrix}
772/1024\\ 210/1024\\ -120/1024\\ 45/1024\\ -10/1024\\ 1/1024
\end{pmatrix}$
\end{block}

\column{0.52\textwidth}
\begin{itemize}
\item On the Cubed-Sphere, an efficient symmetric filter is:
\begin{block}{}
\begin{equation*}
\mathcal{F} = \dfrac{1}{2} \left( \mathcal{F}_{\xi} \circ \mathcal{F}_{\eta} + \mathcal{F}_{\eta} \circ \mathcal{F}_{\xi} \right)
\end{equation*}
\end{block}
\item A symmetric filter gives better results than a non-symmetric one.
\end{itemize}
\end{columns}
\end{frame}

%% ***************************************************************************************************************************************

\begin{frame}{Moving vortex test case [R. Nair, C. Jablonowski, 2008]}
\begin{block}{Two antipodal vortices moving around the sphere}
\begin{figure}
\href{run:ref_7363145849_test_2.avi}{\includegraphics[scale=0.5]{rate_NJ.png}} 
\caption{Convergence analysis for the R. Nair and C. Jablonowski test case; $cfl = 0.7$ ;$\alpha = \pi/4,$ for $N = 40$, $50$, $60$, $80$, $100$ and $150$.}
\end{figure}
\end{block}
\end{frame}

%% ***************************************************************************************************************************************

%\begin{frame}{Symmetric filter function}
%\begin{block}{Numericals Observations}
%\begin{itemize}
%\item A symmetric filter gives significant better results than a non-symmetric one.
%\item A good symmetric filter is given by
%\begin{equation}
%\mathcal{F} = \dfrac{1}{2} \left( \mathcal{F}_{\xi} \circ \mathcal{F}_{\eta} + \mathcal{F}_{\eta} \circ \mathcal{F}_{\xi} \right)
%\end{equation}
%\end{itemize}
%where $\mathcal{F}_{\xi}$ (resp. $\mathcal{F}_{\eta}$) is the 10-th order filter in the $\xi$ direction (resp. $\eta$).
%\end{block}
%\end{frame}

%% ****************************************************************************************************************************************

\begin{frame}{Shallow Water equations (SW)}
\begin{block}{SW equations in non conservative form (vector form of the momentum equation)}
\begin{equation}
\left\lbrace
\begin{array}{rcl}
\dfrac{\partial h^{\star}}{\partial t} + \nabla_T \cdot \left( h^{\star} \mathbf{v} \right) & = & 0 \\
\dfrac{\partial \mathbf{v}}{\partial t} + \nabla_T \left( \dfrac{1}{2}|\mathbf{v}|^2 + gh \right) + \left( f + \zeta \right) \mathbf{n} \times \mathbf{v} & = & 0
\end{array}
\right.
\end{equation}
where 
\begin{itemize}
\item $h$ is the fluid thickness and $\mathbf{v}$ the tangential velocity,
\item $h^{\star}=h-h_s$ with $h_s$ the relief, in general $h_s=0$,
\item $\mathbf{n}$ is the normal exterior vector, 
\item $\zeta = \left( \nabla_T \times \mathbf{v} \right) \cdot \mathbf{n}$ is the vorticity,
\item $f$ is the Coriolis parameter.
\end{itemize}
\end{block}
\end{frame}

%% ***************************************************************************************************************************************

\begin{frame}{Conservation properties}
\begin{block}{}
If $(h, \mathbf{v})$ is solution of the SW equations, the following quantities are conserved :

\begin{itemize}
\item \textbf{mass :} 
$\gint_{\mathbb{S}^2_a} h(t, \mathbf{x}) d \sigma(\mathbf{x})$
\item \textbf{energy :}
$ \gint_{\mathbb{S}^2_a} \left( \dfrac{1}{2}g(h^2 - h_s^2) + \dfrac{1}{2} h | \mathbf{v} |^2 \right) d \sigma(\mathbf{x})$
\item \textbf{potential enstrophy :}
$\gint_{\mathbb{S}^2_a} \dfrac{\left( f + \zeta \right)^2}{2gh} d \sigma(\mathbf{x})$
\end{itemize}
\end{block}

Need of a careful evaluation of the numerical conservation!

\textbf{Discrete quadrature formula :}

$$I(f) = a \Delta \xi \Delta \eta   \sum_{k=(I)}^{(VI)} \sum_{i,j=-N/2}^{N/2}  \sqrt{G_{i,j}^k} f_{i,j}^{k} $$

\end{frame}

%% ***************************************************************************************************************************************

\begin{frame}{SW time independent test case [Williamson and al., 1992]}

\begin{itemize}
\item $h = h_0 - \dfrac{1}{g} \left( a \Omega u_0 + \dfrac{u_0^2}{2} \right)\left( - \cos \lambda \cos \theta \sin \alpha + \sin \theta \cos \alpha \right)^2$
\item $\mathbf{v} = u \mathbf{e}_{\lambda}+ v \mathbf{e}_{\theta}$ with :
\begin{equation*}
\left\lbrace \begin{array}{rcl}
 u & = & u_0 ( \cos \theta \cos \alpha + \cos \lambda \sin \theta \sin \alpha)\\
 v & = & -u_0 \sin \lambda \sin \alpha
 \end{array} \right.
\end{equation*}
\end{itemize}
is a time independent solution of the SW equations with $\alpha$ parameter angle.

\end{frame}

%% ***************************************************************************************************************************************

\begin{frame}{SW time independent test case [Williamson and al., 1992]}
\begin{figure}
\includegraphics[scale=0.24]{ref_7367706276_snapshot_err.png}
\includegraphics[scale=0.24]{ref_7367706276_solution.png}
\end{figure}
\begin{itemize}
\item Steady state geostrophic flow with $\alpha=\pi/4$ on a $6 \times 32 \times 32$ grid after $5$ days.
\item Relative error on the height $h$ at final time (left), $\sim 10^{-6}$.
\item $h$ at final time (right),
\item $\Delta t \approx 10$ minutes, $713$ times iterations, $6146$ grid points, $\frac{\sqrt{g h_0} \Delta t}{a \Delta \xi } \approx 0.3$
\end{itemize}
\end{frame}

%% ***************************************************************************************************************************************

\begin{frame}{Error Analysis}
\begin{figure}
\includegraphics[scale=0.55]{rate_SW2_alpha2.png}
\end{figure}
\begin{itemize}
\item Relative error on $h$ after 5 days with $\alpha = \pi/4$.
\item Spatial order of accuracy is 4.
\end{itemize}
\end{frame}

%% ***************************************************************************************************************************************

\begin{frame}{Conservation}
\begin{figure}
\includegraphics[scale=0.4]{ref_7367706276_conservationA.png}
\end{figure}
\begin{itemize}
\item Relative error on conservation with $6 \times 32 \times 32$ grid and $\alpha=\pi/4$.
\item The worst relative conservation is $4 \times 10^{-9}$ on the energy (713 it.)!
\end{itemize}
\end{frame}

%% ***************************************************************************************************************************************

%\begin{frame}{Convergence Numerical Analysis}
%\begin{figure}
%\includegraphics[scale=0.4]{rate_SW2.png}
%\includegraphics[scale=0.4]{rate_SW2_alpha2.png}
%\end{figure}
%\begin{itemize}
%\item Relative error afer 5 days of steady state geostrophic flow. The times step are $\Delta t \approx 2.5$ ($N=128$), $5$($N=64$) and $10$($N=32$) minutes.
%\item $\alpha = 0$ (left).
%\item $\alpha = \pi/4$ (right).
%\item Spatial order of accuracy is 4!
%\end{itemize}
%\end{frame}


%% ***************************************************************************************************************************************

\begin{frame}{Isolated mountain [Williamson and al., 1992]}

Similar initial functions with $\alpha = 0$ than in the previous:

\begin{itemize}
\item $h = h_0 - \dfrac{1}{g} \left( a \Omega u_0 + \dfrac{u_0^2}{2} \right)(\sin \theta)^2$
\item $\mathbf{v} = u_0 \cos \theta \mathbf{e}_{\lambda}$.
\end{itemize}

perturbated by adding an isolated mountain:

$$h_s = h_{s_0} \left( 1 - \dfrac{r}{R} \right)$$

with $h_{s_0}=2000m$ and $r^2=min(R^2, (\lambda + \pi/2)^2+(\theta - \pi/6)^2)$, $R=\pi/9$.

\end{frame}

%% ***************************************************************************************************************************************


\begin{frame}{Isolated mountain (cont.)}
\begin{figure}
\includegraphics[scale=0.4]{ref_7367706559_snapshot_intermediaire499.png}
\end{figure}
\begin{itemize}
\item $h$ after 5 days on a $6 \times 32 \times 32$ grid.
\item $\Delta t \approx 10min$, $2140$ time iterations, $6146$ grid points.
%\item Results similar than high order conservative methods such as FV or DG.
\end{itemize}
\end{frame}

%% ***************************************************************************************************************************************


\begin{frame}{Isolated mountain (cont.)}

\begin{figure}
\includegraphics[scale=0.4]{ref_7367706559_snapshot_intermediaire999.png}
\end{figure}
\begin{itemize}
\item $h$ after 10 days on a $6 \times 32 \times 32$ grid.
\item $\Delta t \approx 10min$, $2140$ time iterations, $6146$ grid points.
%\item Results similar than high order conservative methods such as FV or DG.
\end{itemize}

\end{frame}

%% ***************************************************************************************************************************************


\begin{frame}{Isolated mountain (cont.)}
\begin{figure}
\includegraphics[scale=0.4]{ref_7367706559_snapshot_intermediaire1499.png}
\end{figure}
\begin{itemize}
\item $h$ after 15 days on a $6 \times 32 \times 32$ grid.
\item $\Delta t \approx 10min$, $2140$ time iterations, $6146$ grid points.
\item Results similar than high order conservative methods such as FV or DG.
\end{itemize}
\end{frame}

%% ***************************************************************************************************************************************

\begin{frame}{Conservation : mass, energy and potential enstrophy}
\begin{figure}
\includegraphics[scale=0.4]{ref_7367915037_conservationA.png}
\end{figure}
\begin{itemize}
\item Conservation of mass and energy, grid: $6 \times 32 \times 32$.
\item Mass and energy are conserved up to $10^{-5}$ (relative error).
\item Potential enstrophy is conserved up to $10^{-4}$ (relative error).
\end{itemize}
\end{frame}

%% ***************************************************************************************************************

\begin{frame}{Barotropic instability [J. Galewsky and al., 2004]}

\begin{block}{Zonal steady state of the SW equations}
\begin{equation}
\begin{array}{rcl}
\bar{h}(\theta) & = & h_0 + \dfrac{1}{g}\gint^{\theta}_{-\pi/2} a u(\tau) \left[ f + \dfrac{\tan(\tau)}{a} u(\tau) \right] d \tau \\
\mathbf{v}(\lambda,\theta) & = & u(\theta) \mathbf{e}_{\lambda}
\end{array}
\end{equation}
\end{block}

with :

\begin{itemize}
\item Coriolis parameter : $f = 2 \Omega \sin \theta$,
\item $u(\theta)=\left\lbrace
\begin{array}{ll}
\dfrac{u_{max}}{e_n} \exp\left( \dfrac{1}{(\theta-\theta_0)(\theta-\theta_1)} \right) & \text{ if } \theta_0 \leq \theta \leq \theta_1 \\
0 & \text{ else}
\end{array}\right.$\\
 with $e_n=C^{ste}$, $u_{max} = 80 ms^{-1}$, $\theta_0 = \pi/7$ and $\theta_1 = \pi/2 - \theta_0$.
\end{itemize}
\end{frame}


%% ***************************************************************************************************************************************

\begin{frame}{Barotropic instability (cont.)}
\begin{block}{Perturbation}
The initial data is given by :
\begin{itemize}
\item zonal velocity:
$$\mathbf{v}(\lambda,\theta) = u(\theta) \mathbf{e}_{\lambda}$$
\item height = perturbation of $\bar{h}$:
$$h(\lambda,\theta) = \bar{h}(\lambda,\theta) + \hat{h} \cos \theta \exp \left[ - \left( \dfrac{\lambda}{\alpha} \right)^2 - \left( \dfrac{\theta_2 - \theta}{\beta} \right)^2 \right] \text{, } \hat{h}/\bar{h} \approx 1 \%$$
with $\theta_2 = \pi/4$, $\alpha = 1/3$ and $\beta = 1/15$.
\end{itemize}
\end{block}

\begin{block}{}
This test is particularly challenging for the Cubed-Sphere:

\begin{itemize}
\item $h$ has large variations along the boundary of panel (V)
\item the initial perturbation is located at the boundary between panel (I) and panel (V).
\end{itemize}
\end{block}
\end{frame}

%% ***************************************************************************************************************************************

\begin{frame}{J. Galewsky and al. test case}
\begin{figure}
\href{run:ref_7367787500.avi}{\includegraphics[scale=0.4]{ref_7367767680_snapshot.png}} 
\end{figure}
\begin{itemize}
\item Vorticity (6 days), grid: $6 \times 128 \times 128 \Rightarrow $ correct number of vortices!
\item Results similar to high order conservative methods such as FV or DG.
\end{itemize}
\end{frame}

%% ***************************************************************************************************************************************

\begin{frame}{Conservation : mass, energy and potential enstrophy}
\begin{figure}
\includegraphics[scale=0.3]{ref_7367767680_massenergy.png}
\includegraphics[scale=0.3]{ref_7367767680_enstrophy.png}
\end{figure}
\begin{itemize}
\item Conservation of mass and energy, grid: $6 \times 128 \times 128$.
\item Mass and energy are conserved up to $10^{-7}$ (relative error).
\item Potential enstrophy is conserved up to $10^{-3}$ (relative error).
\end{itemize}
\end{frame}

%% ***************************************************************************************************************************************

\begin{frame}{Computational complexity}
\begin{itemize}
\item All the computations are performed with a Matlab sequential code on a work station (Intel(R) Xeon(R) CPU E5-2620 v2 @ 2.10GHz).
\item Computational cost of $\nabla_T h$ $\approx 96N^2$ for $12N^2$ unknowns (due to the tridiagonal matrices).
\item Typical CPU time : $1.5$ hours for $6$ days with $N=80$ ($2140$ it.).
\end{itemize}
\end{frame}

%% ***************************************************************************************************************************************

\begin{frame}{Conclusion and future works}
\begin{itemize}
\item Fully centered scheme in space.
\item Explicit RK4 time scheme with filtering.
\item Results similar to high order conservative schemes in terms of accuracy.
\item Current work : mathematical convergence analysis for gradient, divergence, curl and the time discretization.
\item Discretization of the Laplacian and Biharmonic to design more advanced artificial viscosity operators.
\item Implicit time stepping required for slow wave problems (Rossby).
\end{itemize}
\end{frame}

%% ***************************************************************************************************************************************

\begin{frame}
\begin{center}
Thank you for your attention.
\end{center}
\end{frame}





\end{document}