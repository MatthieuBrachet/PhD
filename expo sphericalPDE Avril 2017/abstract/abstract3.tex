
\documentclass[10pt]{article}
\usepackage{latexsym}
\usepackage{bbm}
\usepackage{graphicx}
\usepackage{epsfig}
\usepackage{amsmath}
\usepackage{amsfonts}

\setlength{\paperheight}{297mm}\setlength{\paperwidth}{210mm}
\setlength{\oddsidemargin}{10mm}\setlength{\evensidemargin}{10mm}
\setlength{\topmargin}{0mm}\setlength{\headheight}{10mm}\setlength{\headsep}{8mm}
\setlength{\textheight}{240mm}\setlength{\textwidth}{160mm}
\setlength{\marginparsep}{0mm}\setlength{\marginparwidth}{0mm}
\setlength{\footskip}{10mm}
\voffset -13mm\hoffset -10mm\parindent=0cm
\def\titre#1{\begin{center}{\Large{\bf #1}}\end{center}}
\def\orateur#1#2{\begin{center}{\underline{\large{\bf #1}}}, {#2}\end{center}}
\def\auteur#1#2{\begin{center}{\large{\bf #1}}, {#2}\end{center}}
\def\auteurenbasdepage#1#2#3{\small{\bf #1}, \small{#2}\\ \small{\tt #3}\\ }
\def\motscles#1{%
	\ifx#1\IsUndefined\relax\else\noindent{\normalsize{\bf Mots-cl\'es :}} #1\\ \fi}
\renewcommand{\refname}{\normalsize R\'ef\'erences}

\begin{document}
\thispagestyle{empty}

\def\Titre{Numerical approximation of propagation problems on the sphere using a compact scheme}
\def\NomOrateur{Matthieu BRACHET}
\def\AdresseCourteOrateur{IECL, Universit\'e Lorraine, Metz}
\def\AdresseLongueOrateur{Institut Elie Cartan de Lorraine, UMR 7502, Univ. Lorraine, Metz}
\def\EmailOrateur{email}

\def\NomAuteurI{Jean-Pierre CROISILLE}
\def\AdresseCourteAuteurI{IECL, Universit\'e Lorraine, Metz}
\def\AdresseLongueAuteurI{Institut Elie Cartan de Lorraine, UMR 7502, Univ. Lorraine, Metz}
\def\EmailAuteurI{jean-pierre.croisille@univ-lorraine.fr}

\titre{\Titre}% 

\orateur{\NomOrateur}{\AdresseCourteOrateur}
\auteur{\NomAuteurI}{\AdresseCourteAuteurI}

\motscles{\listmotcles}

In this presentation, we continue to investigate a compact scheme
introduced in \cite{Croisille2013, Croisille2015} with emphasis on PDEs on the sphere involving
rotational and/or shear flows.
Our scheme is based on the Cubed Sphere. Each point of the grid carries
data for the principal unknown $q$ and the tangential gradient $\nabla_T
q$. The approximation in space is fully centered with accuracy close to
$4$. The numerical diffusion is kept as minimal as posible in order to
preserve the accuracy of the scheme, even after a large number of time
iterations. This diffusion mainly consists in a high frequency filtering
of the 1/-1 mode associated with the grid.
Recent progress has focused on the importance of the symmetric form of
the filtering. Numerical results will be shown, in particular vortex propagation \cite{Nair2008} and shear flows problems
\cite{Galewsky2004} for large physical time simulations. These results
indicate the interest of the present approach for applications in
mathematical climatology.
More details can be found in \cite{Brachetyear}.

\begin{figure}[ht]
\begin{center}
\includegraphics[height=4.5cm]{vort_galewsky.png}
\includegraphics[height=4.5cm]{ref_7366156130_normerreur_test_2.png}
\end{center}
\caption{Vorticity of the J. Galewsky et al. test case \cite{Galewsky2004} after 6 days with $6 \times 63^2 + 2$ points on the CS (left), relative error on the advective equation \cite{Nair2008} with $6 \times 40^2 + 2$ points(right)}
\end{figure}

\bibliographystyle{plain}
\begin{thebibliography}{99}

\bibitem{Croisille2013} {\sc J.-P. Croisille}, {\sl Hermitian compact interpolation on the cubed-sphere grid}, Jour. Sci. Comp. , 57, 2013, pp. 193-212.

\bibitem{Croisille2015} {\sc J.-P. Croisille}, {\sl Hermitian approximation of the spherical divergence on the Cubed-Sphere}, J. Comp. App. Maths., 280, 2015, pp. 188-201.

\bibitem{Nair2008} {\sc R. D. Nair, C. Jablonowski}, {\sl Moving Vortices on the Sphere : a test case for horizontal advection problem}, Mon. Wea. Rev. , 136, 2008, pp. 689--711.

\bibitem{Galewsky2004} {\sc J. Galewsky, R. Scott, K. Richard and L. M. Polvani}, {\sl An initial-value problem for testing numerical models of the global shallow-water equations}, Tellus A , 56, 2004, pp. 429--440.

\bibitem{Brachetyear} {\sc M. Brachet, J.-P. Croisille}, {\sl Numerical simulations of propagation problemes
on the sphere using a compact scheme}, preprint.

\end{thebibliography}

\vfill
\auteurenbasdepage{BRACHET Matthieu}{Institut Elie Cartan de Lorraine, Universit\'e de Lorraine,
Site de Metz, B\^at.  A, Ile du Saulcy, F-57045 Metz Cedex 1}{matthieu.brachet@univ-lorraine.fr}
\auteurenbasdepage{CROISILLE Jean-Pierre}{Institut Elie Cartan de Lorraine, Universit\'e de Lorraine,
Site de Metz, B\^at.  A, Ile du Saulcy, F-57045 Metz Cedex 1}{jean-pierre.croisille@univ-lorraine.fr}

\end{document}
