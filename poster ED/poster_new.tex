\documentclass[portrait,final,a0paper,fontscale=0.34]{baposter}%0.277
\usepackage[utf8]{inputenc}
\usepackage{calc}
\usepackage{graphicx}
\usepackage{amsmath}
\usepackage{amssymb}
\usepackage{relsize}
\usepackage{multirow}
\usepackage{rotating}
\usepackage{bm}
\usepackage{url}
\usepackage{graphicx}
\usepackage{multicol}
\usepackage{palatino}

%\graphicspath{{images/}{../images/}}
\usetikzlibrary{calc}

\newcommand{\captionfont}{\footnotesize}
\newcommand{\SET}[1]  {\ensuremath{\mathcal{#1}}}
\newcommand{\MAT}[1]  {\ensuremath{\boldsymbol{#1}}}
\newcommand{\VEC}[1]  {\ensuremath{\boldsymbol{#1}}}
\newcommand{\Video}{\SET{V}}
\newcommand{\video}{\VEC{f}}
\newcommand{\track}{x}
\newcommand{\Track}{\SET T}
\newcommand{\LMs}{\SET L}
\newcommand{\lm}{l}
\newcommand{\PosE}{\SET P}
\newcommand{\posE}{\VEC p}
\newcommand{\negE}{\VEC n}
\newcommand{\NegE}{\SET N}
\newcommand{\Occluded}{\SET O}
\newcommand{\occluded}{o}

\def\gint{\displaystyle\int}

%%%%%%%%%%%%%%%%%%%%%%%%%%%%%%%%%%%%%%%%%%%%%%%%%%%%%%%%%%%%%%%%%%%%%%%%%%%%%%%%
%%%% Some math symbols used in the text
%%%%%%%%%%%%%%%%%%%%%%%%%%%%%%%%%%%%%%%%%%%%%%%%%%%%%%%%%%%%%%%%%%%%%%%%%%%%%%%%

%%%%%%%%%%%%%%%%%%%%%%%%%%%%%%%%%%%%%%%%%%%%%%%%%%%%%%%%%%%%%%%%%%%%%%%%%%%%%%%%
% Multicol Settings
%%%%%%%%%%%%%%%%%%%%%%%%%%%%%%%%%%%%%%%%%%%%%%%%%%%%%%%%%%%%%%%%%%%%%%%%%%%%%%%%
\setlength{\columnsep}{1.5em}
\setlength{\columnseprule}{0mm}

%%%%%%%%%%%%%%%%%%%%%%%%%%%%%%%%%%%%%%%%%%%%%%%%%%%%%%%%%%%%%%%%%%%%%%%%%%%%%%%%
% Save space in lists. Use this after the opening of the list
%%%%%%%%%%%%%%%%%%%%%%%%%%%%%%%%%%%%%%%%%%%%%%%%%%%%%%%%%%%%%%%%%%%%%%%%%%%%%%%%
\newcommand{\compresslist}{%
\setlength{\itemsep}{1pt}%
\setlength{\parskip}{0pt}%
\setlength{\parsep}{0pt}%
}

%%%%%%%%%%%%%%%%%%%%%%%%%%%%%%%%%%%%%%%%%%%%%%%%%%%%%%%%%%%%%%%%%%%%%%%%%%%%%%
%%% Begin of Document
%%%%%%%%%%%%%%%%%%%%%%%%%%%%%%%%%%%%%%%%%%%%%%%%%%%%%%%%%%%%%%%%%%%%%%%%%%%%%%

\begin{document}

%%%%%%%%%%%%%%%%%%%%%%%%%%%%%%%%%%%%%%%%%%%%%%%%%%%%%%%%%%%%%%%%%%%%%%%%%%%%%%
%%% Here starts the poster
%%%---------------------------------------------------------------------------
%%% Format it to your taste with the options
%%%%%%%%%%%%%%%%%%%%%%%%%%%%%%%%%%%%%%%%%%%%%%%%%%%%%%%%%%%%%%%%%%%%%%%%%%%%%%
% Define some colors

\definecolor{lightergreen}{rgb}{0.7843,0.9568,1}
\definecolor{lightgreen}{rgb}{0.3921,0.3921,0.7843}

\hyphenation{resolution occlusions} 

\begin{poster}%
  % Poster Options
  {
  % Show grid to help with alignment
  grid=false,
  % Column spacing
  colspacing=1em,
  % Color style
  bgColorOne=white,
  bgColorTwo=white,
  borderColor=lightgreen,
  headerColorOne=black,
  headerColorTwo=lightgreen,
  headerFontColor=white,
  boxColorOne=white,
  boxColorTwo=lightgreen,
  % Format of textbox
  textborder=roundedleft,
  % Format of text header
  eyecatcher=false,
  headerborder=closed,
  headerheight=0.1\textheight,
%  textfont=\sc, An example of changing the text font
  headershape=roundedright,
  headershade=shadelr,
  headerfont=\Large\bf\textsc, %Sans Serif
  textfont={\setlength{\parindent}{1.5em}},
  boxshade=plain,
%  background=shade-tb,
  background=plain,
  linewidth=2pt
  }
  % Eye Catcher
  {\includegraphics[height=5em]{images/graph_occluded.pdf}} 
  % Title
  {\bf\textsc{Schémas Compacts Hermitiens sur la Sphère : Applications en Climatologie et Océanographie numérique}\vspace{0.4em}}
  % Authors
  {\textsc{ M. BRACHET, J.-P. Croisille (directeur de thèse)}}
  % University logo
  {% The makebox allows the title to flow into the logo, this is a hack because of the L shaped logo.
    \includegraphics[height=10.0em]{images/logo_iecl_ul.png}
  }

%%%%%%%%%%%%%%%%%%%%%%%%%%%%%%%%%%%%%%%%%%%%%%%%%%%%%%%%%%%%%%%%%%%%%%%%%%%%%%
%%% Now define the boxes that make up the poster
%%%---------------------------------------------------------------------------
%%% Each box has a name and can be placed absolutely or relatively.
%%% The only inconvenience is that you can only specify a relative position 
%%% towards an already declared box. So if you have a box attached to the 
%%% bottom, one to the top and a third one which should be in between, you 
%%% have to specify the top and bottom boxes before you specify the middle 
%%% box.
%%%%%%%%%%%%%%%%%%%%%%%%%%%%%%%%%%%%%%%%%%%%%%%%%%%%%%%%%%%%%%%%%%%%%%%%%%%%%%
    %
    % A coloured circle useful as a bullet with an adjustably strong filling
    \newcommand{\colouredcircle}{%
      \tikz{\useasboundingbox (-0.2em,-0.32em) rectangle(0.2em,0.32em); \draw[draw=black,fill=lightblue,line width=0.03em] (0,0) circle(0.18em);}}

%%%%%%%%%%%%%%%%%%%%%%%%%%%%%%%%%%%%%%%%%%%%%%%%%%%%%%%%%%%%%%%%%%%%%%%%%%%%%%
  \headerbox{Problème mathématique :}{name=problem,column=0, span=3,row=0}{
%%%%%%%%%%%%%%%%%%%%%%%%%%%%%%%%%%%%%%%%%%%%%%%%%%%%%%%%%%%%%%%%%%%%%%%%%%%%%%%
   Proposer et étudier une nouvelle méthode numérique permettant de prévoir les mouvements de l'atmosphère en se basant sur un modèle mathématique \eqref{SWEC} issu des équations de Navier-Stokes. 
   
  L'équation Shallow Water \eqref{SWEC} permet de modéliser l'épaisseur de l'atmosphère $h$ et sa vitesse $\mathbf{u}$ en tenant compte de la gravité et de la force de Coriolis :

  \begin{equation}
  \label{SWEC}
  \left\lbrace
  \begin{array}{rcl}
  \dfrac{\partial \mathbf{u}}{\partial t} + \mathbf{u} \cdot \nabla \mathbf{u} + g \nabla h + f \mathbf{k} \wedge \mathbf{u} & = & \mathbf{0} \\
  \dfrac{\partial h}{\partial t} + \nabla \cdot \left( h \mathbf{u} \right) & = & 0
  \end{array}
  \right.\text{ pour } \mathbf{x} \in \mathbb{S}_R^2 \text{ et } t>0.
  \end{equation}
 }

%%%%%%%%%%%%%%%%%%%%%%%%%%%%%%%%%%%%%%%%%%%%%%%%%%%%%%%%%%%%%%%%%%%%%%%%%%%%%%
  \headerbox{Difficultés :}{name=difficult,column=0,below=problem}{
%%%%%%%%%%%%%%%%%%%%%%%%%%%%%%%%%%%%%%%%%%%%%%%%%%%%%%%%%%%%%%%%%%%%%%%%%%%%%%
   Difficultés liées au problème physique :
   \begin{itemize}
   \item On souhaite éviter la singularité liée aux pôles Nord et Sud,
   \item Tendre vers le problème complet prennant en compte les paramètres physiques.
   \end{itemize}
   
	Difficultés numériques :
	\begin{itemize}
	\item importance de la construction d'une méthode de calcul rapide,
	\item précision élevée à la fois en espace et en temps.
	\end{itemize}	   
  }
  
%%%%%%%%%%%%%%%%%%%%%%%%%%%%%%%%%%%%%%%%%%%%%%%%%%%%%%%%%%%%%%%%%%%%%%%%%%%%%%
  \headerbox{Méthode numérique :}{name=method,column=0,span=1, below=difficult }{
%%%%%%%%%%%%%%%%%%%%%%%%%%%%%%%%%%%%%%%%%%%%%%%%%%%%%%%%%%%%%%%%%%%%%%%%%%%%%%
\begin{itemize}
\item \textbf{Maillage : } de type "Cubed-Sphere"

\begin{center}
\includegraphics[scale=0.4]{./images/CS_lauritzen.png}


\begin{flushright}
\tiny{\textit{Illustration P. Lauritzen}}
\end{flushright}
\end{center}



\item \textbf{Discrétisation en temps : } méthode de Runge-Kutta d'ordre 4 (RK4),
\item \textbf{Ecriture des opérateurs : } écriture des opérateurs en coordonnées locales $( \xi, \eta)$ et discrétisation (par exemple pour le gradient ) :
\begin{equation}
\nabla h = \mathbf{g}^{\xi} \dfrac{\partial h}{\partial \xi}+\mathbf{g}^{\eta} \dfrac{\partial h}{\partial \eta}
\end{equation}

avec $(\mathbf{g}^{\xi}, \mathbf{g}^{\eta})$ la base duale de $(\partial_{\xi} \mathbf{x}, \partial_{\eta} \mathbf{x})$ par rapport à la métrique $\mathbf{G}$.

\item \textbf{Discrétisation spatiale :} Utilisation de schémas compacts d'ordre 4 :
$$\dfrac{1}{6}\bar{\delta}_x h_j+\dfrac{2}{3}\bar{\delta}_x h_j + \dfrac{1}{6}\bar{\delta}_x h_{j-1} = \dfrac{h_{j+1}- h_{j-1}}{2\Delta x}
$$

\end{itemize}
  }

%%%%%%%%%%%%%%%%%%%%%%%%%%%%%%%%%%%%%%%%%%%%%%%%%%%%%%%%%%%%%%%%%%%%%%%%%%%%%%
  \headerbox{Résultats numériques : équation d'advection}{name=advection,column=1,span=2, below=problem}{
%%%%%%%%%%%%%%%%%%%%%%%%%%%%%%%%%%%%%%%%%%%%%%%%%%%%%%%%%%%%%%%%%%%%%%%%%%%%%%
Résolution de l'équation d'advection :
\begin{equation}
  \label{advection}
  \left\lbrace
  \begin{array}{rcl}
  \dfrac{\partial h}{\partial t} + \mathbf{c}( \mathbf{x}, t ) \cdot \nabla h & = & 0 \\
  h(\mathbf{x},t) & = & h_0( \mathbf{x} )
  \end{array}
  \right. \text{ pour } \mathbf{x} \in \mathbb{S}_R^2 \text{ et } t>0
  \end{equation}
 
  
  \textbf{Test de Nair et Jablonowski \cite{Nair2008} :} rotation d'un vortex autour de la sphère.
  
\begin{center}
  \includegraphics[scale=0.3]{courbe_AE.png}
  \vspace{1cm}
  \includegraphics[scale=0.35]{erreur_AE.png}
  
  \textbf{Figure 1:} Solution à $t=12$ jours (gauche), erreur relative (droite). 
\end{center}

\textbf{Bilan :}  L'équation d'advection \eqref{advection} est un modèle très simplifié mais important. On observe une bonne précision sur ce test complexe où apparaissent deux tourbillons pouvant être difficiles à retrouver numériquement (problème de sous-résolution ou de dissipation numérique par exemple). 

}

  %%%%%%%%%%%%%%%%%%%%%%%%%%%%%%%%%%%%%%%%%%%%%%%%%%%%%%%%%%%%%%%%%%%%%%%%%%%%%%
  \headerbox{Résultats numériques : eq. Shallow Water linéarisée }{name=SWEC,column=1,span=2, below=advection}{
%%%%%%%%%%%%%%%%%%%%%%%%%%%%%%%%%%%%%%%%%%%%%%%%%%%%%%%%%%%%%%%%%%%%%%%%%%%%%%
L'équation LSWE \eqref{LSWEC} représente une perturbation de \eqref{SWEC} autour d'une position d'équilibre. Nous avons démontré que cette équation possède des propriétés intéréssantes de conservation d'énergie et de masse.
\begin{equation}
  \label{LSWEC}
  \left\lbrace
  \begin{array}{rcl}
  \dfrac{\partial \mathbf{u}}{\partial t} + g \nabla h + f \mathbf{k} \wedge \mathbf{u} & = & 0\\
  \dfrac{\partial h}{\partial t} + H \nabla \cdot  \mathbf{u} & = & 0 \\
  h(\mathbf{x},t)  =  h_0( \mathbf{x} ) & \text{ et } & \mathbf{u}(\mathbf{x},t)  =  \mathbf{u}_0( \mathbf{x} )
  \end{array}
  \right. \text{ pour } \mathbf{x} \in \mathbb{S}_R^2 \text{ et } t>0
  \end{equation}
 
  
  \textbf{Test de conservation :} 
  \begin{itemize}
  \item Solution stationnaire : $\mathbf{u}(\mathbf{x},t) = u(\theta) \mathbf{e}_{\lambda}$ et $h(\mathbf{x},t) = h_0 - \dfrac{R}{g}\gint_0^{\theta} f(\theta)u(\theta) d \theta$
  \item Masse et énergie :
  $$\gint_{ \mathbf{S}_R^2} h(\mathbf{x})  d \mathbf{x} \text{ et }\gint_{ \mathbf{S}_R^2} gh^2 + H |\mathbf{u}|^2  d \mathbf{x}$$
  \end{itemize}
\begin{center}
  \includegraphics[height=4cm]{conservation_LSWE.png}
  \vspace{1cm}
  \includegraphics[scale=0.33]{erreur_LSWE.png}
  
  \textbf{Figure 2:} Masse et énergie relative (gauche), erreur relative (droite). 
\end{center}

\textbf{Bilan :} La méthode numérique utilisée préserve de manière satisfesante les relations de conservation. De plus, sur des tests dont la solution exacte est connue, on observe des résultats de précision très encourageants (de l'ordre de $5 \times 10^{-2} \% $ pour la conservation de solution stationnaire).
  }
  
  %%%%%%%%%%%%%%%%%%%%%%%%%%%%%%%%%%%%%%%%%%%%%%%%%%%%%%%%%%%%%%%%%%%%%%%%%%%%%%
  \headerbox{References :}{name=references,column=0,below=method}{
%%%%%%%%%%%%%%%%%%%%%%%%%%%%%%%%%%%%%%%%%%%%%%%%%%%%%%%%%%%%%%%%%%%%%%%%%%%%%%
    %\smaller
    \bibliographystyle{ieee}
    \renewcommand{\section}[2]{\vskip 0.05em}
      \begin{thebibliography}{1}\itemsep=-0.01em
      \setlength{\baselineskip}{0.4em}
        
       \bibitem{Croisille2013}
        J.-P. Croisille.
        \newblock {H}ermitian compact interpolation on the Cubed-Sphere grid
        \newblock In {Jour. of Sci. Comp.}
        
       \bibitem{Galewsky2004}
        J. Galewsky, R. K. Scott, L. M. Polvani.
        \newblock {A}n initial value test case for the shallow water equations
        \newblock In {Tellus}
        
       \bibitem{Nair2008}
        R. D. Nair, C. Jablonowski.
        \newblock {M}oving Vortices on the Sphere: A Test Case for Horizontal Advection Problems
        \newblock In {American Meteorology Society}
        
       \bibitem{Williamson1994}
        D. L. Williamson, J. B. Drake, J. J. Hack.
        \newblock {A} Standard Test Set for Numerical Approximations to the Shallow Water Equation in Spherical Geometry
        \newblock In {1991}
        
      \end{thebibliography}
   \vspace{0.3em}
  }
  
  %%%%%%%%%%%%%%%%%%%%%%%%%%%%%%%%%%%%%%%%%%%%%%%%%%%%%%%%%%%%%%%%%%%%%%%%%%%%%%
  \headerbox{Conclusion et Perspectives :}{name=conclusion,column=0,span=3,below=references}{
%%%%%%%%%%%%%%%%%%%%%%%%%%%%%%%%%%%%%%%%%%%%%%%%%%%%%%%%%%%%%%%%%%%%%%%%%%%%%%
\textbf{Conclusion : } La méthode numérique proposée ne présente pas de singularités majeurs (tels que les pôles Nord et Sud). De plus, la précision est satisfaisante sur des tests complexes et les relations de conservations sont respectées.

\textbf{Perspectives : } Nous disposons des outils pour travailler sur l'équation Shallow Water  complète \eqref{SWEC} et ainsi proposer une méthode complète pour la résolution de ce problème. Cela permettra de proposer une méthode précise possèdant de bonnes propriétés de conservations pour la résolution d'équations de Climatologie visant à prédire les mouvements de l'atmosphère. Le travail en cours porte sur la résolution du modèle complet en considérant des reliefs sur la sphère.
    
  }

\end{poster}
\end{document}


