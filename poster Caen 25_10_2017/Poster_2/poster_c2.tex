\documentclass{sciposter}
%\documentclass[a0,portrait]{sciposter}
%\documentclass[draft]{a0poster}
%\documentclass[portrait]{sciposter}
%\documentclass[a0paper,portrait]{sciposter}
%\documentclass[11pt, a4paper]{article}
\usepackage[utf8]{inputenc}  %%PC ou Unix

\usepackage{a0size}
\usepackage{amsmath}
\usepackage{amssymb}
\usepackage{multicol}
%\usepackage[francais,english]{babel} 
\usepackage[english]{babel}
\usepackage{setspace}
\usepackage[pdftex,final]{graphicx}
 \graphicspath{Figures/}
\usepackage{epsfig}
\usepackage{dsfont}
\usepackage{bm}
\usepackage{pstricks,pst-grad}
\usepackage{color}
  
\usepackage{relsize}
\usepackage{url}			% For \url
%\usepackage{epstopdf}	% Included EPS files automatically converted to PDF to include with pdflatex
%%%%%%

\definecolor{BoxCol}{rgb}{0.2921,0.3921,0.6843}


\definecolor{SectionCol}{rgb}{1,1,1}
% uncomment for dark blue \section text 
\definecolor{TextColG}{rgb}{0,0.5,0}
\definecolor{TextColR}{rgb}{0.5,0,0}
\newcommand{\Rcolor}{\textcolor{TextColR}}
\newcommand{\Gcolor}{\textcolor{TextColG}}

%%%%%%%%%%%%%%%%%%%%%%%%%%%%%%%%
\renewcommand{\thesection}{\arabic{section}}
\newcommand{\sect}[1]{\setcounter{equation}{0}\section{#1}}
\renewcommand{\theequation}{\arabic{equation}} 
 \newcommand{\seqna}{\begin{eqnarray}}
\newcommand{\eeqna}{\end{eqnarray}}
  
\newcounter{Theo}[section]  %\setcounter{noEx}{1}  
\def\theTheo{\arabic{Theo} de la section \thesection}
\def\numeroTheo{\arabic{Theo}}
\newenvironment{Theo}[1]{\vspace{-\lastskip}\vspace{1em}
\refstepcounter{Theo}\noindent{\bf Th\'eor\`eme \numeroTheo. #1}
\endgraf \leftskip1em{\parskip0pt\noindent}\ignorespaces
}{\par\vspace{1pt}}

\newcounter{Lemme}[section]  %\setcounter{noEx}{1}  
\def\theLemme{\arabic{Lemme} de la section \thesection}
\def\numeroLemme{\arabic{Lemme}}
\newenvironment{Lemme}[1]{\vspace{-\lastskip}\vspace{1em}
\refstepcounter{Lemme}\noindent{\bf Lemme \numeroLemme. #1}
\endgraf \leftskip1em{\parskip0pt\noindent}\ignorespaces
}{\par\vspace{1pt}}

\newcounter{Corollaire}[section]  %\setcounter{noEx}{1}  
\def\theCorollaire{\arabic{Corollaire} de la section \thesection}
\def\numeroCorollaire{\arabic{Corollaire}}
\newenvironment{Corollaire}[1]{\vspace{-\lastskip}\vspace{1em}
\refstepcounter{Corollaire}\noindent{\bf Corollaire \numeroCorollaire. #1}
\endgraf \leftskip1em{\parskip0pt\noindent}\ignorespaces
}{\par\vspace{1pt}}

%%%%%%%%%%%%%%%%%%%%%%%%%%
\let\bb\mathbb
\def \R {\mathbb{R}}
\def \N {\mathbb{N}}
\def \T {\mathbb{T}}
\def \L {\mathbb{L}}
\def \Z{\mathbb{Z}}
 
\def\gint{\displaystyle\int}
\def\gsum{\displaystyle\sum\limits}

%-------------------------------------------------------------------------------------------------------
\title{Numerical approximation of propagation problems on the sphere using a compact scheme}
%\title{Behaviour of BDF in High-frequency modes}

\bigskip
%%\author{Christian Dogbe \hspace{2 in}
%%Michael Pucher \hspace{4 in}
%%Junichi Yamagishi \hspace{3 in}
%}

\smallskip
\institute{{\LARGE\bfseries M. Brachet, J.-P. Croisille}\hspace{9.5 in}  \\\hspace{0.5 in}
Institut Elie Cartan de Lorraine,  Université de Lorraine\hspace{1 in}\\\hspace{0 in}
B.P. 70239, F-54506 Vandoeuvre-lès-Nancy Cedex, France\hspace{6.5 in}\\}

\email{\hspace{1 in}  matthieu.brachet@univ-lorraine.fr\hspace{0.1 in}  }

% The following commands can be used to alter the default logo settings
\leftlogo[1.1]{Figures/LogoFNM}  % defines logo to left of title (with scale factor)
\rightlogo[1.4]{Figures/unicaen3}   % same but on right

%-------------------------------------------------------------------------------------------------------
\begin{document}

%define conference poster is presented at (appears as footer)
\conference{VI$^{\mbox{e}}$ Colloque EDP-Normandie, Caen 2017}

\maketitle
%\renewcommand{\abstractname}{Abstract} 
%\renewcommand{\abstractname}{R\'esum\'e} 


\vspace{0.5 cm}

%\vskip-1.5cm
\begin{abstract}
\begin{center}
\parbox{\hsize}{
 \textcolor{blue}{We present numerical results for the shallow water equations on  a rotating sphere  using the compact scheme introduced in \cite{C2015}. The scheme uses the Cubed-Sphere grid and is based on a centered finite difference scheme. The discretization is performed by a Runge-Kutta method combined with a high-frequencies filtering procedure. Numerical results are displayed  for several classical test cases  \cite{BCyear, G2004}. Numerical results, for an associated  quadrature method on the Cubed Sphere, permitting to assess the conservation properties of the scheme will also be displayed.}}
\end{center}
\end{abstract}

\vskip1cm

\begin{multicols}{2} %%%DETERMINE LE NOMBRE DE COLONNES ****************************************************************

%-------------------------------------------------------------------------------------------------------
\section{Problem}
   Propose a method based on hermitian schemes to solve equations from fluid mechanics. A particular interest is on the Shallow Water equations \eqref{SWEC} with the unknowns $(\mathbf{u},h)$

\begin{equation}
  \label{SWEC}
  \left\lbrace
  \begin{array}{rcl}
  \dfrac{\partial \mathbf{u}}{\partial t} + \mathbf{u} \cdot \nabla \mathbf{u} + g \nabla h + f \mathbf{k} \wedge \mathbf{u} & = & \mathbf{0} \\
  \dfrac{\partial h}{\partial t} + \nabla \cdot \left( h \mathbf{u} \right) & = & 0
  \end{array}
  \right.\text{ for all } \mathbf{x} \in \mathbb{S}_R^2 \text{ and } t>0.
\end{equation}


%-------------------------------------------------------------------------------------------------------
\section{Geometry}
\textbf{Mesh grid : } the "Cubed-Sphere"

\begin{center}
\includegraphics[scale=1]{CS2.png}

\textbf{Figure 1 :} The Cubed-Sphere grid.
\end{center}

The \textbf{Cubed-Sphere grid} is composed of six panels on the sphere. Each panel is obtained as intersection points of great circles.

Each operator can be expressed in a coordinate system $(\xi,\eta)$ associated with a panel.

\begin{itemize}
\item The \textbf{gradient} operator :
\begin{equation}
\nabla h = \mathbf{g}^{\xi} \dfrac{\partial h}{\partial \xi} + \mathbf{g}^{\eta} \dfrac{\partial h}{\partial \eta}
\end{equation}
\item The \textbf{divergence} operator :
\begin{equation}
\nabla \cdot \mathbf{u} = \mathbf{g}^{\xi} \cdot \dfrac{\partial \mathbf{u}}{\partial \xi} + \mathbf{g}^{\eta} \cdot \dfrac{\partial \mathbf{u}}{\partial \eta}
\end{equation}
\item The \textbf{curl} operator :
\begin{equation}
\nabla \wedge \mathbf{u} = \mathbf{g}^{\xi} \wedge \dfrac{\partial \mathbf{u}}{\partial \xi} + \mathbf{g}^{\eta} \wedge \dfrac{\partial \mathbf{u}}{\partial \eta}
\end{equation}
\end{itemize}

where $(\mathbf{g}^{\xi}, \mathbf{g}^{\eta})$ is the dual basis of $(\partial_{\xi} \mathbf{x},\partial_{\eta} \mathbf{x})$.

\vspace{.8cm}

%-------------------------------------------------------------------------------------------------------
\section{Discretization}
The discretization use the following procedure \cite{C2015} :
\begin{itemize}
\item \textbf{Space discretization :} partial derivative $\dfrac{\partial h}{\partial \xi}$ is approximated by $h_{\xi}$ on each grid point. $h_{\xi}$ is calculated by solving a system in the form 
\begin{equation}
\dfrac{1}{6} h_{\xi,i+1} + \dfrac{4}{6} h_{\xi,i} + \dfrac{1}{6} h_{\xi,i-1} = \dfrac{h_{i+1} - h_{i-1}}{2 \Delta \xi}
\end{equation}
This \textbf{Hermitian scheme} is 4-th order accuracy. It permit to assemble each operator.

\item \textbf{Time discretization} The semi-discretized equation
\begin{equation}
\left\lbrace
\begin{array}{rcl}
q'(t) &=& F(q(t),t)\\
q(0)  &=& q_0 
\end{array}
\right.
\end{equation}
is solved with a \textbf{4-th order Runge Kutta method} :

\begin{enumerate}
\item $K^{(1)} = F \left( q^n, t^n \right)$,
\item $K^{(2)} = F \left( q^n + \dfrac{\Delta t}{2} K^{(1)}, t^n + \dfrac{\Delta t}{t}\right)$,
\item $K^{(3)} = F \left( q^n + \dfrac{\Delta t}{2} K^{(2)}, t^n + \dfrac{\Delta t}{t}\right)$,
\item $K^{(4)} = F \left( V^n + \Delta t K^{(3)}, t^n + \Delta t\right)$,  
\item $\tilde{q}^{n+1} = q^n  + \frac{\Delta t}{6} \left( K^{(1)} + 2 K^{(2)} + 2 K^{(3)} + K^{(4)} \right)$,
\item $q^{n+1} = \mathcal{F} \left(  \tilde{q}^{n+1} \right)$.
\end{enumerate}

$\mathcal{F}$ is a 10-th order filtering procedure.
The filter permit to remove parasitic waves.
\end{itemize}

\vspace{0.3cm}



%-------------------------------------------------------------------------------------------------------
\section{Advection equation in dimension 1}
Condider the 1D advection equation :
\begin{equation}
\left\lbrace
\begin{array}{rcl}
\dfrac{\partial h}{\partial t} + c \dfrac{\partial h}{\partial x} & = & 0 \\
h(t=0,x) & = & h_0(x)
\end{array}
\right. \text{ with } x \in [0,1] \text{ and } t>0,
\label{eq:advection1D}
\end{equation}
The context is periodic : $h(t,x+1) = h(t,x)$ for all $t>0$ and $x \in [0,1]$.

For all $1 \leq i \leq N$ and $n=0,1,...$, the quantity $h_i^n$ is an approximation of $h(t^n, x_i)$ obtained with the previous scheme on the equation \eqref{eq:advection1D}.

\begin{Theo}

Without the filtering procedure, the scheme is numerically fourth order of accuracy and stable if 
\begin{equation}
\lambda = \dfrac{c \Delta t}{\Delta x} \leq \lambda_{\infty} = 2 \sqrt{\dfrac{2}{3}} \approx 1.6329
\end{equation}
\end{Theo}

With a filtering procedure of order $10$, the scheme is numerically stable if $\lambda \leq \lambda_{10} \approx 1.6883$ . $\lambda_{10}$ estimated numerically.

\begin{Theo}

With or without filtering procedure $\mathcal{F}$, the scheme is conservative. For all $n = 0,1, ...$
\begin{equation}
\Delta x \gsum_{i} h_i^{n+1} =  \Delta x \gsum_{i} h_i^{n}.
\end{equation}
\end{Theo}



















%-------------------------------------------------------------------------------------------------------
\section{Spherical Shallow Water}
We use the previous scheme on the equation \eqref{SWEC} with a 10-th order filter. The Galewsky test case \cite{G2004} uses the initial condition :
\begin{equation}
\left\lbrace
\begin{array}{rcl}
\bar{h}_0(\lambda, \theta) & = & h_0 - \dfrac{1}{g} \gint_{-\pi/2}^{\theta} a u_{\lambda}(\tau) 
\left( f + \dfrac{\tan(\tau)}{a}u_{\lambda}(\tau) \right) d\tau\\
\mathbf{v} & = & u_{\lambda}(\theta) \mathbf{e}_{\lambda}
\end{array}
\right.
\end{equation}
$(\lambda, \theta)$ is the longitud-latitud coordinate system, $u_{\lambda}(\tau) = \dfrac{1}{e_n} \exp \left[ \dfrac{1}{(\tau - \theta_0)(\tau - \theta_1)} \right] $ if $\tau \in [ \theta_0, \theta_1]$ and $0$ outside of this domain. This initial data is stationnary, we perturbate it using $h_0$ instead of $\bar{h}_0$ :
\begin{equation}
h_0(\lambda, \theta) = \bar{h}_0(\lambda, \theta ) + \hat{h} \cos \theta \exp \left[ - (\lambda/\alpha)^2 - ((\theta-\theta_0)/\beta)^2 \right].
\end{equation}

\begin{center}
\includegraphics[scale=1.1]{ref_7369437806_snapshot_intermediaire599.png}\\
\includegraphics[scale=1]{ref_7369437806_massenergy.png}
\includegraphics[scale=1]{ref_7369437806_enstrophy.png}

\textbf{Figure 3  : } Vorticity at day 6, relative error on conservation of mass, energy and potential enstrophy on a $6 \times 128 \times 128$ grid.
\end{center}


%%%%%%%%%%%%%%%%%%%%%%%%%%%%%
\bibliographystyle{plain}
\begin{thebibliography}{99}
\bibitem{BCyear} {\sc M. Brachet, J.-P. Croisille}, \textit{Numerical simulations of propagation problemes on the sphere
using a compact scheme}, preprint.

\bibitem{C2015} {\sc J.-P. Croisille}, \textit{Hermitian approximation of the spherical divergence on the Cubed-Sphere}, J.
Comp. App. Maths., 280, 2015, pp. 188-201.

\bibitem{G2004} {\sc J. Galewsky, R. Scott, K. Richard and L. M. Polvani}, \textit{An initial-value problem for testing
numerical models of the global shallow-water equations}, Tellus A , 56, 2004, pp. 429?440.

\end{thebibliography}

\end{multicols}
\end{document}
