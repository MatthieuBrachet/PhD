\documentclass[portrait,final,a0paper,fontscale=0.34]{baposter}%0.277
\usepackage[utf8]{inputenc}
\usepackage{calc}
\usepackage{graphicx}
\usepackage{amsmath}
\usepackage{amssymb}
\usepackage{relsize}
\usepackage{multirow}
\usepackage{rotating}
\usepackage{bm}
\usepackage{url}
\usepackage{graphicx}
\usepackage{multicol}
\usepackage{palatino}

%\graphicspath{{images/}{../images/}}
\usetikzlibrary{calc}

\newcommand{\captionfont}{\footnotesize}
\newcommand{\SET}[1]  {\ensuremath{\mathcal{#1}}}
\newcommand{\MAT}[1]  {\ensuremath{\boldsymbol{#1}}}
\newcommand{\VEC}[1]  {\ensuremath{\boldsymbol{#1}}}
\newcommand{\Video}{\SET{V}}
\newcommand{\video}{\VEC{f}}
\newcommand{\track}{x}
\newcommand{\Track}{\SET T}
\newcommand{\LMs}{\SET L}
\newcommand{\lm}{l}
\newcommand{\PosE}{\SET P}
\newcommand{\posE}{\VEC p}
\newcommand{\negE}{\VEC n}
\newcommand{\NegE}{\SET N}
\newcommand{\Occluded}{\SET O}
\newcommand{\occluded}{o}

\def\gint{\displaystyle\int}
\def\gsum{\displaystyle\sum\limits}

%%%%%%%%%%%%%%%%%%%%%%%%%%%%%%%%%%%%%%%%%%%%%%%%%%%%%%%%%%%%%%%%%%%%%%%%%%%%%%%%
%%%% Some math symbols used in the text
%%%%%%%%%%%%%%%%%%%%%%%%%%%%%%%%%%%%%%%%%%%%%%%%%%%%%%%%%%%%%%%%%%%%%%%%%%%%%%%%

%%%%%%%%%%%%%%%%%%%%%%%%%%%%%%%%%%%%%%%%%%%%%%%%%%%%%%%%%%%%%%%%%%%%%%%%%%%%%%%%
% Multicol Settings
%%%%%%%%%%%%%%%%%%%%%%%%%%%%%%%%%%%%%%%%%%%%%%%%%%%%%%%%%%%%%%%%%%%%%%%%%%%%%%%%
\setlength{\columnsep}{1.5em}
\setlength{\columnseprule}{0mm}

%%%%%%%%%%%%%%%%%%%%%%%%%%%%%%%%%%%%%%%%%%%%%%%%%%%%%%%%%%%%%%%%%%%%%%%%%%%%%%%%
% Save space in lists. Use this after the opening of the list
%%%%%%%%%%%%%%%%%%%%%%%%%%%%%%%%%%%%%%%%%%%%%%%%%%%%%%%%%%%%%%%%%%%%%%%%%%%%%%%%
\newcommand{\compresslist}{%
\setlength{\itemsep}{1pt}%
\setlength{\parskip}{0pt}%
\setlength{\parsep}{0pt}%
}

%%%%%%%%%%%%%%%%%%%%%%%%%%%%%%%%%%%%%%%%%%%%%%%%%%%%%%%%%%%%%%%%%%%%%%%%%%%%%%
%%% Begin of Document
%%%%%%%%%%%%%%%%%%%%%%%%%%%%%%%%%%%%%%%%%%%%%%%%%%%%%%%%%%%%%%%%%%%%%%%%%%%%%%

\begin{document}

%%%%%%%%%%%%%%%%%%%%%%%%%%%%%%%%%%%%%%%%%%%%%%%%%%%%%%%%%%%%%%%%%%%%%%%%%%%%%%
%%% Here starts the poster
%%%---------------------------------------------------------------------------
%%% Format it to your taste with the options
%%%%%%%%%%%%%%%%%%%%%%%%%%%%%%%%%%%%%%%%%%%%%%%%%%%%%%%%%%%%%%%%%%%%%%%%%%%%%%
% Define some colors

\definecolor{lightergreen}{rgb}{0.7843,0.9568,1}
\definecolor{lightgreen}{rgb}{0.2921,0.3921,0.6843}
\definecolor{white}{rgb}{1,1,1}

\hyphenation{resolution occlusions} 

\begin{poster}%
  % Poster Options
  {
  % Show grid to help with alignment
  grid=false,
  % Column spacing
  colspacing=1em,
  % Color style
  bgColorOne=white,
  bgColorTwo=white,
  borderColor=lightgreen,
  headerColorOne=black,
  headerColorTwo=lightgreen,
  headerFontColor=white,
  boxColorOne=white,
  boxColorTwo=lightgreen,
  % Format of textbox
  textborder=roundedleft,
  % Format of text header
  eyecatcher=false,
  headerborder=closed,
  headerheight=0.1\textheight,
%  textfont=\sc, An example of changing the text font
  headershape=roundedright,
  headershade=shadelr,
  headerfont=\Large\bf\textsc, %Sans Serif
  textfont={\setlength{\parindent}{1.5em}},
  boxshade=plain,
%  background=shade-tb,
  background=plain,
  linewidth=2pt
  }
  % Eye Catcher
  {\includegraphics[height=5em]{images/graph_occluded.pdf}} 
  % Title
  {\bf\textsc{Numerical approximation of propagation problems on the sphere using a compact scheme}\vspace{0.4em}}
  % Authors
  {\textsc{ M. BRACHET, J.-P. Croisille (Supervisor)}}
  % University logo
  {% The makebox allows the title to flow into the logo, this is a hack because of the L shaped logo.
    \includegraphics[height=8.0em]{images/logo_iecl_ul.png}
  }


%%%%%%%%%%%%%%%%%%%%%%%%%%%%%%%%%%%%%%%%%%%%%%%%%%%%%%%%%%%%%%%%%%%%%%%%%%%%%%
%%% Now define the boxes that make up the poster
%%%---------------------------------------------------------------------------
%%% Each box has a name and can be placed absolutely or relatively.
%%% The only inconvenience is that you can only specify a relative position 
%%% towards an already declared box. So if you have a box attached to the 
%%% bottom, one to the top and a third one which should be in between, you 
%%% have to specify the top and bottom boxes before you specify the middle 
%%% box.
%%%%%%%%%%%%%%%%%%%%%%%%%%%%%%%%%%%%%%%%%%%%%%%%%%%%%%%%%%%%%%%%%%%%%%%%%%%%%%
    %
    % A coloured circle useful as a bullet with an adjustably strong filling
    \newcommand{\colouredcircle}{%
      \tikz{\useasboundingbox (-0.2em,-0.32em) rectangle(0.2em,0.32em); \draw[draw=black,fill=lightblue,line width=0.03em] (0,0) circle(0.18em);}}

%%%%%%%%%%%%%%%%%%%%%%%%%%%%%%%%%%%%%%%%%%%%%%%%%%%%%%%%%%%%%%%%%%%%%%%%%%%%%%
  \headerbox{Problem :}{name=problem,column=0, span=3,row=0}{
%%%%%%%%%%%%%%%%%%%%%%%%%%%%%%%%%%%%%%%%%%%%%%%%%%%%%%%%%%%%%%%%%%%%%%%%%%%%%%%
   Propose a method based on hermitian scheme to solve equations from fluid mechanics. A particular interest is on the Shallow Water equations \eqref{SWEC} with the unknowns $(\mathbf{u},h)$

\begin{equation}
  \label{SWEC}
  \left\lbrace
  \begin{array}{rcl}
  \dfrac{\partial \mathbf{u}}{\partial t} + \mathbf{u} \cdot \nabla \mathbf{u} + g \nabla h + f \mathbf{k} \wedge \mathbf{u} & = & \mathbf{0} \\
  \dfrac{\partial h}{\partial t} + \nabla \cdot \left( h \mathbf{u} \right) & = & 0
  \end{array}
  \right.\text{ pour } \mathbf{x} \in \mathbb{S}_R^2 \text{ et } t>0.
\end{equation}
 }
 
  
%%%%%%%%%%%%%%%%%%%%%%%%%%%%%%%%%%%%%%%%%%%%%%%%%%%%%%%%%%%%%%%%%%%%%%%%%%%%%%
  \headerbox{Geometry :}{name=geometry,column=0,span=1,below=problem}{
%%%%%%%%%%%%%%%%%%%%%%%%%%%%%%%%%%%%%%%%%%%%%%%%%%%%%%%%%%%%%%%%%%%%%%%%%%%%%%
\textbf{Mesh grid : } the "Cubed-Sphere"

\begin{center}
\includegraphics[scale=0.25]{CS2.png}
\end{center}

The \textbf{Cubed-Sphere grid} is composed of six panels on the sphere. Each panel is obtain like intersection points of greats circles.

Each operator can be expressed in a coordinate system $(\xi,\eta)$ associated with a panel.

\begin{itemize}
\item The \textbf{gradient} operator is given by 
\begin{equation}
\nabla h = \mathbf{g}^{\xi} \dfrac{\partial h}{\partial \xi} + \mathbf{g}^{\eta} \dfrac{\partial h}{\partial \eta}
\end{equation}
\item The \textbf{divergence} operator is given by 
\begin{equation}
\nabla \cdot \mathbf{u} = \mathbf{g}^{\xi} \cdot \dfrac{\partial \mathbf{u}}{\partial \xi} + \mathbf{g}^{\eta} \cdot \dfrac{\partial \mathbf{u}}{\partial \eta}
\end{equation}
\item The \textbf{curl} operator is given by 
\begin{equation}
\nabla \wedge \mathbf{u} = \mathbf{g}^{\xi} \wedge \dfrac{\partial \mathbf{u}}{\partial \xi} + \mathbf{g}^{\eta} \wedge \dfrac{\partial \mathbf{u}}{\partial \eta}
\end{equation}
\end{itemize}

where $(\mathbf{g}^{\xi}, \mathbf{g}^{\eta})$ is the dual basis of $(\partial_{\xi} \mathbf{x},\partial_{\eta} \mathbf{x})$.
  }
  
  



%%%%%%%%%%%%%%%%%%%%%%%%%%%%%%%%%%%%%%%%%%%%%%%%%%%%%%%%%%%%%%%%%%%%%%%%%%%%%%
  \headerbox{Discretization :}{name=discretization,column=0,span=1,below=geometry}{
%%%%%%%%%%%%%%%%%%%%%%%%%%%%%%%%%%%%%%%%%%%%%%%%%%%%%%%%%%%%%%%%%%%%%%%%%%%%%%
The discretization use the following procedure [Croisille, 2015] :
\begin{itemize}
\item \textbf{Space discretization :} partial derivative $\dfrac{\partial h}{\partial \xi}$ is approximated by $h_{\xi}$ on each grid point. $h_{\xi}$ is calculated by solving a system in the form 
\begin{equation}
\dfrac{1}{6} h_{\xi,i+1} + \dfrac{4}{6} h_{\xi,i} + \dfrac{1}{6} h_{\xi,i-1} = \dfrac{h_{i+1} - h_{i-1}}{2 \Delta \xi}
\end{equation}
This \textbf{Hermitian scheme} is 4-th order accuracy. It permit to assemble each operator.

\item \textbf{Time discretization} The semi-discretized equation
\begin{equation}
\left\lbrace
\begin{array}{rcl}
q'(t) &=& F(q(t),t)\\
q(0)  &=& q_0 
\end{array}
\right.
\end{equation}
is solved with a \textbf{4-th order Runge Kutta method} :
\begin{center}
\begin{enumerate}
\item $K^{(1)} = F \left( q^n, t^n \right)$,
\item $K^{(2)} = F \left( q^n + \dfrac{\Delta t}{2} K^{(1)}, t^n + \dfrac{\Delta t}{t}\right)$,
\item $K^{(3)} = F \left( q^n + \dfrac{\Delta t}{2} K^{(2)}, t^n + \dfrac{\Delta t}{t}\right)$,
\item $K^{(4)} = F \left( V^n + \Delta t K^{(3)}, t^n + \Delta t\right)$,  
\item $S = K^{(1)} + 2 K^{(2)} + 2 K^{(3)} + K^{(4)}$,
\item $q^{n+1} = \mathcal{F} \left( q^n  + \frac{\Delta t}{6} S \right)$.
\end{enumerate}
\end{center}

where $\mathcal{F}$ is a high-frequencies filtering procedure.
In practice, we choice a 10-th order filter $\mathcal{F}$ to suppress the stray waves but other filters are available. 
\end{itemize}
  }
  
  

  
  
  
  
  %%%%%%%%%%%%%%%%%%%%%%%%%%%%%%%%%%%%%%%%%%%%%%%%%%%%%%%%%%%%%%%%%%%%%%%%%%%%%%
  \headerbox{1D Advection equation:}{name=toymodel,column=1,span=2,below=problem}{
%%%%%%%%%%%%%%%%%%%%%%%%%%%%%%%%%%%%%%%%%%%%%%%%%%%%%%%%%%%%%%%%%%%%%%%%%%%%%%
Condider the 1D advection equation :
\begin{equation}
\left\lbrace
\begin{array}{rcl}
\dfrac{\partial h}{\partial t} + c \dfrac{\partial h}{\partial x} & = & 0 \\
h(t=0,x) & = & h_0(x)
\end{array}
\right. \text{ with } x \in [0,1] \text{ and } t>0,
\label{eq:advection1D}
\end{equation}
The context is periodic : $h(t,x+1) = h(t,x)$ for all $t>0$ and $x \in [0,1]$.

For all $1 \leq i \leq N$ and $n=0,1,...$, the quantity $h_i^n$ is an approximation of $h(t^n, x_i)$ obtained with the previous scheme on the equation \eqref{eq:advection1D}.

\textbf{Proposition 1 :}
Without filtering procedure, the scheme is numerically fouth order and stable if 
\begin{equation}
\lambda = \dfrac{c \Delta t}{\Delta x} \leq \lambda_{\infty} = 2 \sqrt{\dfrac{2}{3}} \approx 1.6329
\end{equation}
With a filtering procedure order 10, the scheme is numerically stable if $\lambda \leq \lambda_{10} \approx 1.6883$ with
$\lambda_{10}$ estimated numerically.

\begin{center}
\begin{tabular}{|c|c|}
\hline
\textbf{Filter's order $p$} & $\lambda_p$ \\
\hline \hline
$10$ & $1.6883$ \\
$8$ & $1.7114$ \\
$6$ & $1.7485$ \\
$4$ & $1.8156$ \\
$2$ & $1.9749$ \\
\hline
\end{tabular}
\textbf{Table 1: } Stability condition function of order of $\mathcal{F}$.
\end{center}

\textbf{Proposition 2 :}
With or without filtering procedure $\mathcal{F}$, the scheme is conservative. For all $n = 0,1, ...$
\begin{equation}
\Delta x \gsum_{i} h_i^{n+1} =  \Delta x \gsum_{i} h_i^{n}.
\end{equation}

  }
  
  
  
    %%%%%%%%%%%%%%%%%%%%%%%%%%%%%%%%%%%%%%%%%%%%%%%%%%%%%%%%%%%%%%%%%%%%%%%%%%%%%%
  \headerbox{Spherical Shallow Water:}{name=swe,column=1,span=2,below=toymodel}{
%%%%%%%%%%%%%%%%%%%%%%%%%%%%%%%%%%%%%%%%%%%%%%%%%%%%%%%%%%%%%%%%%%%%%%%%%%%%%%
We use the previous scheme on the equation \eqref{SWEC} with a 10-th order filter. The Galewsky test case [Galewsky, Scott, \textit{et al.} 2004] use the initial condition :
\begin{equation}
\left\lbrace
\begin{array}{rcl}
\bar{h}_0(\lambda, \theta) & = & h_0 - \dfrac{1}{g} \gint_{-\pi/2}^{\theta} a u_{\lambda}(\tau) 
\left( f + \dfrac{\tan(\tau)}{a}u_{\lambda}(\tau) \right) d\tau\\
\mathbf{v} & = & u_{\lambda}(\theta) \mathbf{e}_{\lambda}
\end{array}
\right.
\end{equation}
$(\lambda, \theta)$ is the longitud-latitud coordinate system, $u_{\lambda}(\tau) = \dfrac{1}{e_n} \exp \left[ \dfrac{1}{(\tau - \theta_0)(\tau - \theta_1)} \right] $ if $\tau \in [ \theta_0, \theta_1]$ avec $0$ outside of this domain. This initial data is stationnary, we perturbate it using $h_0$ instead of $\bar{h}_0$ :
\begin{equation}
h_0(\lambda, \theta) = \bar{h}_0(\lambda, \theta ) + \hat{h} \cos \theta \exp \left[ - (\lambda/\alpha)^2 - ((\theta-\theta_0)/\beta)^2 \right].
\end{equation}

\begin{center}
\includegraphics[scale=.38]{ref_7369437806_snapshot_intermediaire599.png}\\
\includegraphics[scale=.39]{ref_7369437806_massenergy.png}
\includegraphics[scale=.36]{ref_7369437806_enstrophy.png}

\textbf{Figure 1 : } Vorticity at $6$ days, conservation error for mass, energy and potential enstrophy on a $6 \times 128 \times 128$ grid.
\end{center}

  }
  
  
  
   %%%%%%%%%%%%%%%%%%%%%%%%%%%%%%%%%%%%%%%%%%%%%%%%%%%%%%%%%%%%%%%%%%%%%%%%%%%%%%
  \headerbox{Conclusion:}{name=conclusion,column=1,span=2,below=swe}{
%%%%%%%%%%%%%%%%%%%%%%%%%%%%%%%%%%%%%%%%%%%%%%%%%%%%%%%%%%%%%%%%%%%%%%%%%%%%%%
For a periodic $1D$ problem, the scheme is stable under a condition and accurate order 4. The scheme is conservative even if we add a filtering procedure. This filter permit to remove parasitic waves.

On the sphere, the method give good results on several tests cases [Brachet, Croisille, preprint]. The Cubed-Sphere grid has not singularities problems like on a Longitud-Latitud grid. The geometry is practically cartesian. This property permit to use this efficiant and accurate method.


  }


\end{poster}
\end{document}


