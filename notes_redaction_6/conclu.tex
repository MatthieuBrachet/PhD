%% Conclusion générale

Dans cette thèse, nous présentons un nouveau schéma aux différences finies pour la résolution d'équations aux dérivées partielles d'évolution sur la Sphère.

Le contexte de discrétisation est celui des opérateurs périodiques. C'est pourquoi nous étudions la précision de ces opérateurs ainsi que leurs propriétés spectrales. Les opérateurs aux différences finis utilisés permettent d'approcher la dérivée première avec une grande précision. De plus, nous avons étudié les opérateurs de filtrage en dimension 1, leur existence, leur précision ainsi que leurs propriétés spectrales.
Après semi-discrétisation en espace, un schéma explicite en temps de Runge-Kutta d'ordre 4 couplé à un opérateur de filtrage donne d'excellents résultats sur des équations classiques telles que l'équation de transport, l'équation des ondes et l'équation de Burgers. Malgré le caractère hyperbolique de ces équations, l'ajout de l'opérateur de filtrage nous permet de conserver des schémas centrés en espace sans que les ondes parasites deviennent néfastes pour le bon déroulement des algorithmes. Les opérateurs de filtrage permettent en effet d'améliorer la stabilité en ajoutant de la dissipation numérique au schéma. Ces derniers, n'affectent pas du tout la conservation des quantités primaires telles que la masse.

Le maillage utilisé est celui de la Cubed-Sphere. Il est construit à partir d'ensembles de grands cercles. Le maillage obtenu possède de nombreuses symétries. Ces symétries, ainsi que la structure quasi-cartésienne, sont utilisées pour construire une formule de quadrature précise et permettent d'introduire un produit scalaire vérifiant l'orthogonalité d'un grand nombre d'harmoniques sphériques. Les formules de quadratures sont issues des formules des trapèzes et de Simpson. En perturbant la formule des trapèzes, on obtient une nouvelle formule plus précise. Cette dernière donne d'excellents résultats sur les tests numériques avec un taux de convergence proche de 4 sur un ensemble de fonctions intégrables. 

Les portions de grands cercles qui constituent le maillage permettent d'utiliser un schéma hermitien centré périodique. Cependant, sur un grand cercle donné, les points de discrétisations ne sont pas toujours des points du maillage, les données sont complétées à l'aide de Splines Cubique. On obtient ainsi des opérateurs gradient, divergence et vorticité approchés. Les opérateurs obtenus sont consistants au moins à l'ordre 3. Lors des tests numériques effectués, un ordre 4 ou supérieur est observé. Un opérateur de filtrage symétrisé sera aussi utilisé. Il est construit sur le même principe que les opérateurs différentiels. Il permet d'atténuer les ondes parasites en affectant un minimum le calcul. Nous avons démontré la consistance de ces opérateurs.

Nous utilisons ces opérateurs différentiels et de filtrage, ainsi qu'un algorithme de type RK4 pour résoudre des équations aux dérivées partielles d'évolution sur la Sphère. Les problèmes considérés sont de deux ordres. D'une part, nous considérons des équations d'advections linéaires et non linéaires. Les tests numériques effectués permettent d'illustrer la précision du schéma. Les ordres de convergence ainsi que les niveaux d'erreurs sont très bons. De plus, malgré le caractère hyperbolique des équations, l'opérateur de filtrage est suffisant pour utiliser des schémas centrés sur toutes les simulations, y compris des simulations présentant des chocs.
D'autres part, nous avons considéré des systèmes d'équations du type Shallow Water. Les tests effectués sur l'équation Shallow Water linéarisé et l'équation Shallow Water donnent des résultats comparables à ceux obtenus par des méthodes de Galerkin ou de volumes finis. Les niveaux d'erreurs sont très faibles. Bien que le schéma ne soit pas conservatif, l'erreur de conservation est très faible. Pour la masse le comportement est satisfaisant. Pour les autres quantités conservées, les erreurs sont similaires à celles obtenues par d'autres méthodes y compris sur des tests difficiles.

Le schéma aux différences finis considéré est centré et l'opérateur de filtrage utilisé affecte peu les calculs. Bien que les équations soient hyperboliques, ce schéma est suffisant sur les tests effectués. Les perspectives de ce travail concernent des simulations en temps long. En effet, nous n'avons utilisé ici que des schémas explicites. Ces derniers possèdent des contraintes sur le pas de temps importantes. Il serait intéressant de développer un schéma implicite de manière a pouvoir considérer des pas de temps plus grands et ainsi des simulations sur des temps plus longs. Une autre perspective est de considérer des problèmes en dimension 3 pour les simulations de fluides sur la sphère de type Navier-Stokes 3D. Enfin il serait d'étudier des méthodes de zoom de type "Local Defect Correction" ainsi que la parallélisation du code existant.

