%% *** SECTION ************************************************************************************************************************

\section{Numerical results}

In this section, we present numerical results obtained for a familly of partial differential equation \eqref{eq:adv}, \eqref{eq:lswe} and \eqref{eq:swe} on the sphere. Firstly, with the advection equation, a known velocity is used such that an explicit solution is also known \cite{Nair-Machenhauer, Nair-Jablonowski}. Then, we present two tests for the linearized shallow water equation \eqref{eq:lswe} and numerical results for shallow water equation \eqref{eq:swe}. Most of this tests are introduced by \cite{Williamson-Drake-Hack-Jakob-Swarztrauber, Galewsky-Scott-Polvani}.
When the analytic solution is known, we quantify the relative error (excepted for the damped case of (LSWE)) with the following formula :

\begin{equation}
\label{eq:error1}
I_p = \dfrac{\|\psi^n - \psi_t \|_p}{\| \psi_t \|_p}
\end{equation}

$p \in \lbrace 1, 2, \infty \rbrace$, $\psi_t$ is the analytic solution at time $t$, $\psi^n$ is the calculated solution at time $t^n$. 
$ \| . \| _p$ denote sur $\ell^p$ norm on the sphere. The numerical integral is given by \cite{Croisille-12}.

\subsection{Advection equation}

Numerical results are obtained on \eqref{eq:adv}. This tests case are originates from \cite{Nair-Machenhauer, Nair-Jablonowski}. The velocity fields $(t,\mathbf{x}) \mapsto \mathbf{c} ( t,\mathbf{x})$ is prescribed as follow. The simplest problem of this kind is the Cosinus-bell advection problem \cite{Williamson-Drake-Hack-Jakob-Swarztrauber, Galewsky-Scott-Polvani}. It correspond to a solid body velocity. Numerical results using the scheme (\ref{eq:300.41-1}) have been reported in  \cite{Croisille-10}. 
Here we consider two more challenging problems of the form (\ref{eq:978.23.1}) involving transport and roll-up effects of vortices. 
The mathematical basis leading to the analytical solution is given in \cite{Nair-Cote-Stanisforth}.

\subsubsection{Nair-Machenhauer test case}

\label{sec:4.1}
This test case \cite{Nair-Machenhauer}
considers
as initial condition $h(0,\mathbf{x})$
two vortices located
at diammetrally opposite points on the sphere.
These points are called $P$ (for North) and $P^\prime$ (for South).
The point $P$ has coordinates $(\lambda_P,\theta_P)$ in the reference
longitude-latitude system.
The longitude-latitude coordinate system 
$(\lambda^\prime,\theta^\prime)$ is associated to 
the axis $P  P^\prime$. The two vortices evolve in a roll-up
letting appear a finer and finer rollup structure when time increases.
We call $\mathbb{S}^2_R$ the sphere with radius $R$, 
\cite{Nair-Cote-Stanisforth,Nair-Machenhauer}.
Let us define the velocity $\mathbf{x} \in \mathbb{R}^2 \mapsto V(\mathbf{x})$ 
by
\begin{equation}
\left\{
\begin{array}{l}
\rho(\mathbf{x})=\rho_0 \cos(\theta^\prime)\\
V(\mathbf{x}) = v_0 \dfrac{3 \sqrt{3} }{2} \sech^2 ( \rho ) \tanh ( \rho )
\end{array}
\right.
\end{equation} 
The angular velocity $\theta^\prime \mapsto \omega_r(\theta^\prime)$ 
corresponding to $V$ 
is defined by:
\begin{equation}
   \omega_r ( \theta' ) = \left\{ 
   \begin{array}{ll}
      V/( R \rho ) & \text{ if } \rho \neq 0 \\
      0 & \text{ if} \rho =0
   \end{array}
   \right.
\label{vitesse_angulaire}
\end{equation}
Define now the tangential velocity on the sphere $\mathbb{S}^2_R$ 
appearing in (\ref{eq:978.23.1}) by
by:
\begin{equation}
\mathbf{c}(\mathbf{x},t)=c_{\lambda^\prime} \mathbf{e}_{\lambda^\prime}+
c_{\theta^\prime} \mathbf{e}_{\theta^\prime}
\end{equation}
with
\begin{equation}
\left\{
\begin{array}{l}
c_{\lambda^\prime}=\cos(\theta^\prime) \omega_r(\theta^\prime)\\
c_{\theta^\prime}=0
\end{array}
\right.
\end{equation}

Switching back to the $(\mathbf{e}_\lambda,\mathbf{e}_\theta)$ basis, one obtains

\begin{equation}
\mathbf{c}(\mathbf{x},t)=\mathbf{c}(\mathbf{x})=c_{\lambda} \mathbf{e}_{\lambda}+
c_{\theta} \mathbf{e}_{\theta}
\end{equation}

with
\begin{equation}
\left\{
\begin{array}{l}
c_{\lambda} = R \omega_r ( \theta' ) \left[ \sin \theta_p \cos \theta - \cos \theta_p \cos ( \lambda - \lambda_p ) \sin \theta \right]\\
c_{\theta} = R \omega_r ( \theta' ) 
\left[ \cos \theta_p \sin ( \lambda - \lambda_p ) \right]
\label{eq:78.10}
\end{array}
\right.
\end{equation}

Now, the solution $(\mathbf{x}, t) \in \mathbb{S}_R^2 \times \mathbb{R}_{-}+ \mapsto
\phi(\mathbf{x},t)$ is given in coordinates $(\lambda^\prime,\theta^\prime,t)$ by

\begin{equation}
\phi ( \lambda^\prime , \theta', t ) 
= 
1 - \tanh \left[ \dfrac{\rho_0\cos(\theta^\prime)}{\gamma} \sin ( \lambda' 
- \omega_r(\theta^\prime) t ) \right],\;\;\; \rho(\theta^\prime)= 
\rho_0 \cos(\theta^\prime)
\label{NM_exacte}
\end{equation}

The constant $\gamma$ determines the strenght of $\nabla_T\phi$ and $\rho_0>0$ is a reference distance to the center of the vortex.
Let $T>0$ be the physical time of evolution and $v_0 = 2 \pi R / T$ ($R$=radius) be a reference velocity. 
In this article, for a smooth flow, we choose the parameters $\rho_0 = 3$ and $ \gamma = 5$.

In Figures \ref{erreur_cfl=0.05} and \ref{erreur_cfl=0.5}, the error history is reported for CFL numbers $\CFL=0.05$ and $\CFL=0.5$.
The Cubed Sphere grid has a parameter $N=35$, which corresponds to an equatorial resolution $\Delta \lambda = 2.6 \deg$. 
This is a spatial resolution similar to the one in \cite{Nair-Jablonowski} where a Discontinuous Galerkin (DG) scheme is used.
As can be observed, the error growth is regular.
The error behaviour is similar for the angles $\alpha=0$ and $\alpha=45\deg$, which shows that there is no apparent influence of the corners of the Cubed Sphere.

We have performed two different runs. The first corresponds to a $\CFL=0.05$. In this case, the error is dominated by the space approximation.
The error levels that are reached are of the same order of magnitude than the ones obtained with the DG scheme in \cite{Nair-Jablonowski}.
In the second case, we use $\CFL=0.5$.  The scheme remains stable.
The error consists of the combination of the space and time accuracy.
The error is slightly smaller than with $\CFL=0.05$. This is a standard behaviour for convection problems.

Table \ref{table:2.4} reports the convergence rate 
in the three norms $1$, $2$ and $\sup$. It can be observed
that the error is of order $4$. 

\begin{figure}[ht!]
\includegraphics[scale=0.3]{ref_7367656360_normerreur_test_1.png}
\includegraphics[scale=0.3]{ref_7367665245_normerreur_test_1.png}
\label{erreur_cfl=0.05}
\caption{Error plots with $N=35$; $\CFL=0.05$. Left panel: 
The point $P$ defining the axis has spherical coordinates  $(\lambda_P,  \theta_P) = (\pi / 4, \pi / 4)$. and $(\lambda_P, \theta_P) = (0,0)$ (right) for the Nair and Machenhauer test case.}
\end{figure}

\begin{figure}[ht!]
\includegraphics[scale=0.3]{ref_7367656531_normerreur_test_1.png}
\includegraphics[scale=0.3]{ref_7367656543_normerreur_test_1.png}
\label{erreur_cfl=0.5}
\caption{Error curves $N=35$; $cfl=0.5$; $(\lambda_P,  \theta_P) = (\pi / 4, \pi / 4)$ (left) and $(\lambda_P, \theta_P) = (0,0)$ (right) for the Nair and Machenhauer test case.}
\end{figure}

\begin{table}[ht!]
\label{table:2.4}
\begin{tabular}{|c||cc|cc|cc|}
\hline
$N$ & $max_n |e_1^n|$ & order  & $max_n |e_2^n|$ & order  & $max_n |e_{\infty}^n|$ & order \\
\hline
\hline
$40\;(2.25\deg)$ & $1.989 (-3)$ & -  & $7.255 (-3)$ & - & $4.039(-2)$  & - \\
\hline 
$50\;(1.80\deg)$ & $7.638 (-4)$ & $4.2891$ & $3.161(-3)$ & $3.7232$ & $1.918 (-2)$ & $3.7108$ \\
\hline
$60\;(1.50\deg)$ & $3.023(-4)$ & $5.2256$ & $1.313 (-3)$ & $4.8188$ & $7.556 (-3)$ & $5.1092$ \\
\hline
$80\;(1.125\deg)$ & $5.2979 (-5)$ & $6.0537$ & $2.391(-4)$ & $5.9204$ & $1.561(-3)$ & $5.4818$ \\
\hline
$100\;(0.90\deg)$ & $1.5036(-5)$ & $5.6441$ & $6.4568(-5)$ & $5.8669$ & $4.329(-4)$ & $5.7478$\\
\hline
$150\;(0.60\deg)$ & $1.9244(-6)$ & $5.0703$ & $9.2082(-6)$ & $4.8034$ & $7.6848(-5)$ & $4.2634$\\
\hline
\end{tabular}
\caption{Convergence analysis for the Nair and Machenhauer test case \cite{Nair-Machenhauer}. 
$N=31$; $\CFL = 0.7$; $(\lambda_p, \theta_p) = (0,0)$.}
\end{table}

\subsubsection{Nair-Jablonowski test case}

\label{sec:4.2}
In \cite{Nair-Jablonowski} a modification of the stationary vortex problem of Section \ref{sec:4.1} was suggested. It combines 
the deformational roll-up effect of the 
preceding case
with a solid body rotation.
The analytical solution is given in \cite{Nair-Jablonowski}.
The advection velocity $\mathbf{c}(\mathbf{x},t)$ in (\ref{eq:978.23.1}) is obtained as the sum

\begin{equation}
\mathbf{c}=\mathbf{c}_s +\mathbf{c}_r
\end{equation}

where $\mathbf{c}_s$ is a solid rotation velocity and 
$\mathbf{c}_r$ is a "static" velocity centered at the center 
of the vortex. The velocity $\mathbf{c}_r$ is actually time dependant 
since in \eqref{eq:78.10}

$(\lambda_P, \theta_P)$ must be remplaced by the solid body advected position given  by 
\begin{equation}
(\lambda_s', \theta_s') = (\lambda_0' + w_s t, \theta_0')
\end{equation}
where $(\lambda_0', \theta_0')$ is the initial position of the vortex.
On the other hand, the solid-body velocity is given by

\begin{equation}
c_{\lambda, r} = R \omega_s \left( \sin \theta_p \cos \theta - \cos \theta_p \cos ( \lambda - \lambda_p ) \sin \theta \right)
\label{vitesse_lambda_bump}
\end{equation}
\begin{equation}
c_{\theta, r} = - R \omega_s \cos \theta_p \sin ( \lambda - \lambda_p )
\label{vitesse_theta_bump}
\end{equation}

where $\omega_s = v_0 / R = 2 \pi / T $ and $( \lambda_p, \theta_p$ ) is the coordinates of the point $P$.

The error growth is reported on Fig. \ref{erreur_cfl=0.05a}
and \ref{erreur_cfl=0.5a}. The magnitude of the error
is very close from the stationary case. However it appears 
slightly less regular. Table \ref{table:2} reports
as before fourth-order accuracy. Note also that the level
of error are very close to the ones reported in
\cite{Nair-Jablonowski} with a DG scheme. Finally
Fig. \ref{coupe-NJ-1} displays a slice of the vortex after 12 days
withgrid sizes $N=30$ and $N=60$. 
The matching with the finest grid
is excellent. No dispersion or dispersion is obervable.



\begin{figure}[ht!]
\includegraphics[scale=0.3]{ref_7367657139_snapshot_test_2_nday_0.png}
\includegraphics[scale=0.3]{ref_7367657143_snapshot_test_2_nday_3.png}

\includegraphics[scale=0.3]{ref_7367657147_snapshot_test_2_nday_6.png}
\includegraphics[scale=0.3]{ref_7367657152_snapshot_test_2_nday_9.png}

\includegraphics[scale=0.3]{ref_7367657157_snapshot_test_2_nday_12.png}
\caption{Nair and Jablonowski test-case. Approximate solution of the vortex after 
0, 3, 6, 9 and 12 days. The resolution is $N=31$. Numerical parameters are 
$N=31$, $\CFL = 0.7$ and $\alpha = 3 \pi / 4$.}
\label{SNAPSHOT_NJ}
\end{figure}

\begin{figure}[ht!]
\includegraphics[scale=0.3]{ref_7367657290_normerreur_test_2.png}
\includegraphics[scale=0.3]{ref_7367657345_normerreur_test_2.png}
\label{erreur_cfl=0.05a}
\caption{Error curves $N=35$; $\CFL=0.05$; $\alpha = \pi / 4$ (left) et $\alpha = 0$ (right) for the Nair and Jablonowski test case \cite{Nair-Jablonowski}.}
\end{figure}

\begin{figure}[ht!]
\includegraphics[scale=0.3]{ref_7367657356_normerreur_test_2.png}
\includegraphics[scale=0.3]{ref_7367657366_normerreur_test_2.png}
\label{erreur_cfl=0.5a}
\caption{Error curves $N=35$; $\CFL=0.5$; $\alpha = \pi / 4$ (left) et $\alpha = 0$ (right) for the Nair and Jablonowski test case \cite{Nair-Jablonowski}.}
\end{figure}

\begin{table}[ht!]
\begin{tabular}{|c||cc|cc|cc|}
\hline
$N$ & $max_n |e_1^n|$ & ordre  & $max_n |e_2^n|$ & ordre  & $max_n |e_{\infty}^n|$ & ordre \\
\hline
\hline
$40$ & $0.3037 (-2)$ & -  & $0.1061(-2)$ & - & $.6220 (-1)$  & - \\
\hline 
$50$ & $0.1471 (-2)$ & $3.2487$ & $0.5656(-2)$ & $2.8192$ & $0.3664 (-1)$ & $2.3715$ \\
\hline
$60$ & $0.7373(-3)$ & $3.7884$ & $0.3033(-2)$ & $3.4179$ & $0.2178(-1)$ & $2.8529$ \\
\hline
$80$ & $0.2311(-3) $ & $4.0327$ & $0.9921(-3)$ & $3.8841$ & $0.7691(-2)$ & $3.6184$ \\
\hline
$100$ & $9.4841(-5)$ & $3.9914$ & $0.4108(-3)$ & $3.9513$ & $0.3205(-2)$ & $3.9228$\\
\hline
$150$ & $1.8779 (-5)$ & $3.9941$ & $8.3160 (-5)$ & $3.9395$ & $0.7584(-3)$ & $3.5543$\\
\hline
\end{tabular}
\label{table:2}
\caption{Convergence analysis for the Nair and Jablonowski test case \cite{Nair-Jablonowski} ; $cfl = 0.7$ ; $\alpha = \pi /4$.}
\end{table}

\begin{figure}[ht!]
\includegraphics[scale=0.5]{ref_7363158648_coupefaceI_equateur_test_2.jpg}
\label{coupe-NJ-1}
\caption{Nair and Jablonowski test case \cite{Nair-Jablonowski}. Slice 
of the vortex after $12$ days. Solid line: exact solution, circles:
approximate solution with $N=30$. Crosses: approximate solution with $N=60$}
\end{figure}

\subsection{Linearized shallow water equation}

The equation (LSWE) \eqref{eq:lswe} is obtained as a perturbation of $(\mathbf{0},H)$ an equilibrium of \eqref{eq:swe}. This equation is an intermediate between advection equation and shallow water equation.
Using our numerical scheme, it permit to highlight the importance of using a symetric filter instead :
\begin{equation}
\mathcal{F} = \mathcal{F}_{\xi} \circ \mathcal{F}_{\eta}
\end{equation}
because dissymetry between panels can create instabillities.

We present two test cases. The first is a damped case of LSWE, the other is a time independant zonal solution.

\subsubsection{A damped case of LSWE}

This test serves to assess the accuracy of the 
gradient and divergence approximation () and () when
used in the LSWE system (\ref{eq:lswe}). 
Consider the two exponentially in time damped functions
\begin{equation}
\tilde{\mathbf{v}} (t,\mathbf{x})=\mathbf{u}_0 \varphi(\theta) e^{-\sigma t}\mathbf{e}_\lambda(\mathbf{x})\\
\tilde{\eta}(t,\mathbf{x})= \eta_0 \varphi(\theta) \sin(\lambda)e^{-\sigma t} 
\end{equation}

The system to solve is
\begin{equation}
\label{eq:lswe_damped}
(LSWE) \left\{
\begin{array}{l}
\dfrac{\partial \mathbf{v}}{\partial t} (t,\mathbf{x})+ \mathbf{g} \nabla_T \eta(t,\mathbf{x}) + f(\mathbf{x}) \mathbf{k}(\mathbf{x}) \times
\mathbf{v}(t,\mathbf{x})=S_{\eta}(t,\mathbf{x})\\
\dfrac{\partial \eta}{\partial t}(t,\mathbf{x})+ H \nabla_T . \mathbf{v}(t,\mathbf{x})=S_{\mathbf{v}}(t,\mathbf{x})
\end{array}
\right.
\end{equation}

In \eqref{eq:lswe_damped}, $\mathbf{g}$ is the gravity vector and $\mathbf{k}(\mathbf{x})$ is the exterior
normal vector.
The source terms $S_{\eta}$ and $S_{\mathbf{v}}$ are defined 
such as the functions $(\tilde{\mathbf{v}}(t,\mathbf{x}), \tilde \eta(t,\mathbf{x}))$ be solution
of (\ref{eq:lswe_damped}).
\textbf{Remarks:\\
- Gradient and divergence are calculated according to ().\\
- Filtrage ou non ?
- Valeurs numériques de $H,g, u_0, \eta_0$ ?
- Autres essais  avec temps plus long ?}

A numerical grid convergence analysis is reported in 
Table \ref{table:4}.

\begin{figure}[ht!]
\begin{tabular}{|c||c|c|c|c|c|}
\hline
& N=40 & rate & N=60  & rate & N=80 \\
\hline 
\hline 
$\vert \mathbf{v}^1_{ex}(T)-\mathbf{v}^1_{cal}(T)\vert_{h,I}$ & 2.52(-5)  &  4.15 & 4.69(-6) &  2.98 & 1.40(-6)   \\
\hline 
$\vert \eta_{ex}(T)-\eta_{cal}(T)\vert_{h,I}$ & 1.06(-5)  &  3.85 & 2.70(-6) &  3.32 & 1.04(-6) \\
\hline 
\end{tabular}
\caption{Hermitian scheme applied to an exponential decaying solution of the LSWE. Final time = 1h30}
\label{table:4}
\end{figure}

\textbf{Tableau a revoir}

A convergence rate between $3$ and $4$ can be observed.

\subsection{A time independent zonal solution 
of LSWE}
In this test case, we consider a time independent 
solution of LSWE depending on the latitude only.
Consider a tangential velocity field of the form. 
If the parameter function $\theta \mapsto \varphi(\theta)=\exp(\frac{1}{(\theta-\theta_0)(\theta-\theta_1)})$, then
we define the 
spherical velocity $\mathbf{v}(\mathbf{x})$ on the sphere by:
\begin{equation}
\mathbf{v}(\mathbf{x})=u_0 \varphi(\theta) \mathbf{e}_\lambda(\mathbf{x})
\end{equation}

The momentum equation () is equivalent to the 
following relation:

\begin{equation}
\eta(\mathbf{x})=\eta_{eq}-\frac{a}{g}\int_0^\theta f(s) \varphi(s) ds
\end{equation}

Our numerical test consists now in testing if the scheme preserve during time stepping 
the time independant solution defined by .....
This is a zonal divergence free solution
of the LSWE depending on the latitude $\theta$ only.
This test case is meaningful, 
first to assess the accuracy of the spatial approximation. In particular, 
spurious modes can pollute the numerical solution.
Second this test allows to test the accuracy of the numerical divergence 
preserving on 
large intervals of time.
The numerical results are reported in Table \ref{table:5}.

\begin{figure}[ht!]
\begin{tabular}{|c|c|c|c|c|c|}
\hline 
&N=40 & rate & N=60  & rate & N=80 \\
\hline
\hline
$\vert \mathbf{v}^1_{ex}(T)-\mathbf{v}^1_{cal}(T)\vert_{h,I}$ & 2.73(-5)  &  4.09 & 5.18(-6) &  4.24 & 1.53(-6) \\
\hline 
$\vert \eta_{ex}(T)-\eta_{cal}(T)\vert_{h,I}$ & 1.21(-5)  &  3.95 & 2.43(-6) &  3.14 & 7.38(-7) \\
\hline 
\end{tabular}
\caption
{Hermitian scheme applied to a time independent solution of the LSWE. Final time = 1h30}
\label{table:5}
\end{figure}

\subsection{Shallow water equation}

The shallow water equation \eqref{eq:swe} is solved with the previsous scheme. Numerical results are obtained on the test cases extracted from \cite{Williamson-Drake-Hack-Jakob-Swarztrauber, Galewsky-Scott-Polvani}.

The following parameters are constant :
\begin{equation}
\begin{array}{rcll}
a & = & 6.37122 \times 10^6m & \text{ is the Earth radius,}\\
\Omega & = & 7.292 \times 10^{-5} s^{-1}& \text{ is the rotational velocity,}\\
g & = & 9.80616 m \cdot s^{-2}& \text{ is the gravity constant}\\
\end{array}
\end{equation}

Unless specification, $h^{\star} = h$.

The following invariant should be present:
\begin{itemize}
\item mass : $I_1=\gint_{\mathbb{S}_a^2} h^{\star}d\mathbf{s}$ \\

\item energy : $I_2 = \gint_{\mathbb{S}_a^2} \frac{1}{2} h^{\star} \mathbf{v}^2 + \frac{1}{2}g \left( h^2 - h_s^2 \right) d\mathbf{s}$ \\

\item potential enstrophy : $I_{3}=\gint_{\mathbb{S}_a^2} \dfrac{\left( \zeta + f \right)^2}{2 h^{\star}} d\mathbf{s}$ with $\zeta$ the vorticity\\
\end{itemize}
We measure the error on conservation for this quantities $q$ with the normalized integral :
\begin{equation}
\dfrac{\gint_{\mathbb{S}_a^2} \left( q(t,\mathbf{s})-q(0,\mathbf{s}) \right) d \mathbf{s}}{\gint_{\mathbb{S}_a^2} q(0,\mathbf{s})d \mathbf{s}}
\end{equation}
Integral of divergence and vorticity on the sphere are theorically nul, so we can't use the normalized integral to measure error. Then, the error on conservation for this quantities $p$ are measured with:
\begin{equation}
\gint_{\mathbb{S}_a^2} p(t,\mathbf{s})d \mathbf{s}
\end{equation}
with $p = \nabla_T \cdot \mathbf{v}$ or $p = \left( \nabla_T \wedge \mathbf{v}  \right) \cdot \mathbf{k}$.

\subsubsection{Test case 2 of Williamson and al. : Global steady state}

The test 2 of \cite{Williamson-Drake-Hack-Jakob-Swarztrauber} is a steady state solution to the non-linear shallow water equation \eqref{eq:swe}. It's a zonal solution around an axes rotated with an angle $\alpha$ (like for the Nair-Jablonowski test case).
Then the Coriolis parameter is given by :
\begin{equation}
f=2 \Omega \left( - cos \lambda cos \theta sin \alpha + sin \theta cos \alpha \right)
\end{equation}

The analytical solution is :
\begin{equation}
\mathbf{v} = u \mathbf{e}_{\lambda} + v \mathbf{e}_{\theta}
\end{equation}
with :
\begin{equation}
\left\lbrace
\begin{array}{rcl}
u & = & u_0 \left( cos \theta cos \alpha + cos \lambda sin \theta sin \alpha \right)\\
v & = & -u_0 sin \lambda sin \alpha
\end{array}
\label{eq:W2 velocity}
\right.
\end{equation}

and the height $h$ :
\begin{equation}
h=h_0- \dfrac{1}{g} \left( a \Omega u_0 + \dfrac{u_0^2}{2} \right) \left( -cos \lambda cos \theta sin \alpha + sin \theta cos \alpha \right)^2 
\label{eq:W2 height}
\end{equation}

constants are :
\begin{equation}
\begin{array}{rcl}
gh_0 & = & 2.94 \times 10^4 m^2/s^2\\
u_0 & = & 2 \pi R / (12 \text{days}) \\
\end{array}
\end{equation}

We test differents values of parameter $\alpha$. Results with $\alpha=\pi/4$ are given in figure \ref{fig:W2 alpha=pi/4}. In table \ref{tab:W2 error order} we can observe the order of convergence close to 4 in accordance with numerical results on each operator.

\begin{figure}[ht!]
\includegraphics[scale=0.3]{ref_7367706276_snapshot_solution.png}
\includegraphics[scale=0.3]{ref_7367706276_erreur.png}
\caption{Numerical results of steady state geostrophic flow in direction of $\alpha=\pi/4$ on  $32 \times 32 \times 6$ grid after $5$ model days of simulation. Contour lines are plotted between 1150 m to 2950m with step of 200m (left). Relative error (right) }
\label{fig:W2 alpha=pi/4}
\end{figure}

The conservation relation are given by the figure \ref{fig:W2 conservation alpha=pi/4}. The conservation is fairly satisfactory but this test is independant of time. We need to obtain others results for unsteady state test case. 

\begin{figure}[ht!]
\includegraphics[scale=0.3]{ref_7367706276_mass.png}
\includegraphics[scale=0.3]{ref_7367706276_energy.png}\\
\includegraphics[scale=0.3]{ref_7367706276_enstrophy.png}
\includegraphics[scale=0.3]{ref_7367706276_conservationBdiv.png}\\
\includegraphics[scale=0.3]{ref_7367706276_conservationBvort.png}
\caption{Numerical conservation of steady state geostrophic flow in direction of $\alpha=\pi/4$ on  $32 \times 32 \times 6$ grid after $5$ model days of simulation.}
\label{fig:W2 conservation alpha=pi/4}
\end{figure}

\begin{table}[ht!]
\begin{tabular}{|c|c||cc|cc|cc|}
\hline
$\alpha$ & grid & $max_n |e_1^n |$ & order & $max_n |e_2^n |$ & order &  $max_n |e_{\infty}^n |$ & order \\
\hline
\hline
              & $6 \times 32 \times 32$   & $1.1422 (-6)$ & - & $1.3885 (-6)$ & - & $2.4469 (-6)$ & - \\
$\alpha = 0$  & $6 \times 64 \times 64$   & $7.1216 (-8)$ & $3.9131$ & $8.6513 (-8)$ & $3.9141$ & $1.5229 (-7)$ & $3.9157$ \\
              & $6 \times 128 \times 128$ & $XXX$ & - & $XXX$ & - & $XXX$ & - \\
\hline
\hline
                & $6 \times 32 \times 32$   & $7.5712 (-7)$ & - & $1.0446 (-6)$ & - & $2.7809 (-6)$ & - \\
$\alpha = \pi/4$& $6 \times 64 \times 64$   & $4.7213 (-8)$ & $3.9129$ & $6.5124 (-8)$ & $3.9133$ & $1.7387 (-7)$ & $3.9092$ \\
                & $6 \times 128 \times 128$ & $XXX$ & - & $XXX$ & - & $XXX$ & - \\
\hline
\end{tabular}
\caption{Relative error afer 5 days model simulation of steady state geostrophic flow. The CFL condition is $0.9$.}
\label{tab:W2 error order}
\end{table}

\subsubsection{Test case 5 of Williamson and al. : Isolated mountain}

The test case 5 of \cite{Williamson-Drake-Hack-Jakob-Swarztrauber} is similar to the previous but with a topography. It permit to check the performance of the scheme with a topography and a time dependant solution.
No analyical solution is not available.

The initial solution is the same than the previsous \eqref{eq:W2 velocity} and \eqref{eq:W2 height} but with the following constant :
\begin{equation}
\begin{array}{rcl}
h_0 & = & 5960m \\
u_0 & = & 20 m/s \\
\alpha & = & 0 \\
\end{array}
\end{equation}

The bottom topography is an isolated conic mountain wih $h_{s_0}=2000m$ of height. The topography is :
\begin{equation}
h_s = h_{s_0} \left( 1 - \dfrac{r}{r_0} \right)
\end{equation}
with $r=min \left( r_0, \sqrt{\left( \lambda - \lambda_c \right)^2 + \left( \theta - \theta_c \right)^2} \right)$. $r_0=\pi/9$ and $(\lambda_c, \theta_c)$ is the location of the mountain, then $(\lambda_c, \theta_c) = (3 \pi /2, \pi /6)$.

Numerical results of the height $h$ at times $5$, $10$ and $15$ days are given in figure \ref{fig:W5 snapshot} with a grid $6 \times 32 \times 32$. Conservations results are in figure \ref{fig:W5 conservation}.

\begin{figure}[ht!]
\includegraphics[scale=0.3]{ref_7367706559_snapshot_intermediaire499.png}\\
\includegraphics[scale=0.3]{ref_7367706559_snapshot_intermediaire999.png}\\
\includegraphics[scale=0.3]{ref_7367706559_snapshot_intermediaire1499.png}
\caption{Numerical results of isolated mountain test case with grid  $32 \times 32 \times 6$ at time 5, 10 and 15 days. Contour line are plotted from 5050 m to 5950 m with interval of 50 m.}
\label{fig:W5 snapshot}
\end{figure}

\begin{figure}[ht!]
\includegraphics[scale=0.3]{ref_7367706559_conservationA.png}
\includegraphics[scale=0.3]{ref_7367706559_conservationB.png}
\caption{Conservation of isolated mountain test case with grid  $32 \times 32 \times 6$ at time 5, 10 and 15 days.}
\label{fig:W5 conservation}
\end{figure}

\subsubsection{Test case 6 of Williamson and al. : Rossby-Haurtwitz waves}

resuts ref : 7367707271

\subsubsection{Test case of Galewsky}


