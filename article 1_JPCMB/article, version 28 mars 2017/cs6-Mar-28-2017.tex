%% LyX 1.6.9 created this file.  For more info, see http://www.lyx.org/.
%% Do not edit unless you really know what you are doing.
\documentclass[oneside,english]{amsart}
%\documentclass[oneside,english]{article}
\usepackage[T1]{fontenc}
\usepackage[utf8]{inputenc}
\usepackage{amsmath,amssymb,graphicx}
\usepackage{amsthm}
\usepackage{graphicx}
%\usepackage{subfigure}
%\usepackage{epsfig}
%\usepackage{color}
%%%%%%%%%%%%%%%%%%%%%%%%%%%%%%%%%%%%%%%%%%%%%%%%
\graphicspath{{./Images/}}
%usepackage[dvips]{color}
%%%%%%%%%%%%%%%%%%%%%%%%%%%%%%%%%%%%%%%%%%%%%%%%%%%%%
\usepackage[left=2.5cm,right=2.5cm,top=3cm,bottom=3cm]{geometry}
%%%%%%%%%%%%%%%%%%%%%%%%%%%%%%%%%%%%%%%%%%%%%%%%%%%
%\newcommand{\cal}{\mathcal}
\newcommand{\beq}{\begin{equation}}
\newcommand{\eeq}{\end{equation}}
%%%%%%%%%%%%%%%%%%%%%%%%%%%%%%%%%%%%%%%%%%%%%%%%%%%%%
\def\sinc{\mathop{\rm sinc}\nolimits}  
\def\cal{\mathop{\rm cal}\nolimits}  
\def\CFL{\mathop{\rm CFL}\nolimits}  
\def\ex{\mathop{\rm ex}\nolimits}  
\def\atan{\mathop{\rm atan}\nolimits}  
\def\sech{\mathop{\rm sech}\nolimits}  
\def\tanh{\mathop{\rm tanh}\nolimits}  
\def\div{\mathop{\rm div}\nolimits}  
\def\days{\mathop{\rm days}\nolimits}  
\def\Re{\mathop{\rm Re}\nolimits}  
\newcommand{\bgrad}{\mbox{\boldmath$\nabla$}}
\newcommand{\bnabla}{\mbox{\boldmath$\nabla$}}
\newcommand{\bF}{\mbox{\boldmath$F$}}
\newcommand{\bM}{\mbox{\boldmath$M$}}
\newcommand{\bg}{\mbox{\boldmath$g$}}
\newcommand{\bq}{\mbox{\boldmath$q$}}
\newcommand{\bx}{\mbox{\boldmath$x$}}
\newcommand{\bc}{\mbox{\boldmath$c$}}
\newcommand{\be}{\mbox{\boldmath$e$}}
\newcommand{\bn}{\mbox{\boldmath$n$}}
\newcommand{\bu}{\mbox{\boldmath$u$}}
\newcommand{\bv}{\mbox{\boldmath$v$}}
\newcommand{\bs}{\mbox{\boldmath$s$}}
\newcommand{\bi}{\mbox{\boldmath$i$}}
\newcommand{\bj}{\mbox{\boldmath$j$}}
\newcommand{\bk}{\mbox{\boldmath$k$}}
\newcommand{\bvarphi}{\mbox{\boldmath$\varphi$}}
\newcommand{\bomega}{\mbox{\boldmath$\omega$}}
\newcommand{\bpsi}{\mbox{\boldmath$\psi$}}
%%%%%%%%%%%%%%%%%%%%%%%%%%%%%%%%%%%%%%%%%%%
% ** PERSONAL NEWTHEOREMS
\newtheorem{thm}{Theorem}[section]
\theoremstyle{definition}
\newtheorem{defi}[thm]{Definition}
\newtheorem{prop}[thm]{Proposition}
\theoremstyle{remark}
\newtheorem{remark}[thm]{Remark}

\def\gint{\displaystyle\int}


\makeatother

\usepackage{babel}

\begin{document}

\title{Numerical simulation of propagation problems on the sphere using a compact scheme}


\begin{abstract}
In this paper we continue the study of the compact scheme 
on the sphere introduced in 
\cite{Croisille-10, Croisille-12}. This scheme uses 
discrete differential operators on the Cubed Sphere. 
Considering propagation problems, the scheme is fully centered
and used a symmetric filer to eliminate high frequency
oscillating modes.
Numerical results on a broad series of numerical test cases
in climatology are presented, including
linar convection problems, the linearized shallow water 
and the non linear shallow water equations. 
These results demonstrate the
potential of the present approach 
in a variety of situations of interest accuracy and the 
stability of the present approach.
\end{abstract}

\author{M. Brachet and J.-P. Croisille\dag\ddag}
\address{\dag Universit\'e de Lorraine, D\'epartement de Math\'ematiques, F-57045 Metz, France\\
\ddag C.N.R.S., Institut Elie Cartan de Lorraine, UMR 7502, F-57045 Metz, France}
\email{matthieu.brachet@univ-lorraine.fr, jean-pierre.croisille@univ-lorraine.fr}

%% \date\today
\date{March 28, 2017}
\maketitle

{\sl Keywords: Cubed-Sphere grid - Compact finite difference scheme - 
Hermitian derivative - Vortex propagation}
%%%%%%%%%%%%%%%%%%%%%%%%%%%%%%%%%%%%%%%%%%%%%%%%%%%%%%%%%%%%%%%%%%%%
%%%%%%%%%%%%%%%%%%%%%%%%%%%%%%%%%%%%%%% *** INTRODUCTION ***************************************************************************

\section{Introduction}
\label{sec:1}
In this paper, we continue the development
of the compact scheme approach introduced in \cite{} and \cite{}
for hyperbolic problems on the sphere.
In recent years a lot of effort
has been devoted to import
ideas of numerical gas dynamics to numerical climatology.
Two particular examples are finite volumes upwind schemes 
with numerical fluxes. This approach involves
reconstruction with piecewise cubic recontsruction.
A variant is the Discontinuous Galerkin approach
in which several unknowns in a single computational cells are used.
In each of these two cases, the machinery of upwind schemes is adapted
to the specificity of the spherical context
and of the test cases. This involves in particular 
slope limiting. 

An important challenge for numerical schemes in numerical climatology 
is to calculate as accurately as possible 
the solution of linear equations up to a large physical time.
Here we show that our compact scheme performs well
for the three following problems:
\begin{itemize}
\item 
The linear scalar equation
advection 
\begin{equation}
\label{eq:adv}
\dfrac{\partial h}{\partial t}  (t,\mathbf{x})+ \mathbf{c}(t,\mathbf{x}) \cdot \nabla_T h(t,\mathbf{x})=0
\end{equation}

The velocity $\mathbf{c}(t,\mathbf{x}) \in (0, +\infty) \times \mathbb{S}^2$ is prescribed
so that the scalar value $h(t,\mathbf{x})$ exhibits a moving rollup vortex.
This test case was introduced
in \cite{Nair-Machenhauer,Nair-Jablonowski}. We refer to Sections
\ref{sec:4.1} and \ref{sec:4.2} for setails.
\item
The linearized shallow water equations (LSWE). This equation
is expressed as:
\begin{equation}
\label{eq:lswe}
(LSWE) \left\{
\begin{array}{l}
\dfrac{\partial \mathbf{v} }{\partial t} (t,\mathbf{x})+ g \nabla_T \eta(t,\mathbf{x}) + f(\mathbf{x}) \mathbf{k}(\mathbf{x}) \wedge
\mathbf{v}(t,\mathbf{x})=0\\
\dfrac{\partial \eta}{\partial t} (t,\mathbf{x})+ H \nabla_T \cdot \mathbf{v}(t,\mathbf{x})=0
\end{array}
\right.
\end{equation}


This system serves as a base for spherical waves on the sphere,
\cite{Paldor}. The state at rest is 
$(\mathbf{v}_0,\eta_0)=(0,H)$ and the 
The 3-components perturbation unknowns is
the vector $(t,\mathbf{x}) \mapsto (\mathbf{v}(t,\mathbf{x}), \eta(t,\mathbf{x}))$.
\item The shallow water equation (SWE). We use the vectorial form as following :
\begin{equation}
\label{eq:swe}
(SWE) \left\lbrace
\begin{array}{rcl}
\dfrac{\partial h^{\star}}{\partial t} (t,\mathbf{x}) + \nabla_T \cdot \left( h^{\star}(t,\mathbf{x}) \mathbf{v}(t,\mathbf{x}) \right) & = & 0 \\
\dfrac{\partial \mathbf{v}}{\partial t}  (t,\mathbf{x}) + \nabla_T \left( \dfrac{1}{2} \mathbf{v}(t,\mathbf{x})^2 + g h(t,\mathbf{x}) \right) + \left( f(\mathbf{x}) + \zeta(t,\mathbf{x}) \right) \mathbf{k}(\mathbf{x}) \wedge \mathbf{v}(t,\mathbf{x}) & = & \mathbf{0} 
\end{array}
\right.
\end{equation}

where $\zeta = \left( \nabla_T \wedge \mathbf{v} \right) \cdot \mathbf{k}$ is the relative vorticity. $h^{\star} = h - h_s$ with $h_s$ the reliefs map.
\end{itemize}

The present approach is related to compact scheme
approach. In this sense, it belongs 
to a classical approach, which can
be traced back to early works
in approximation and interpolation theory, \cite{Collatz}.

Two specific applications in CFD where compact schemes
are  is Aeroacoustics, \cite{Visbal-Gaitonde, Tam-Webb} and Turbulence, \cite{Lele, Kim-Moin}.

Finite difference schemes 
for simulating problems in climatology have
actually attracted interest since 
more than 40 years, \cite{Arakawa}.

In both cases, we show 
that our purely Cartesian approach supports
the comparison with modern conservative schemes on unstructured grids, such as 
Discontinuous Galerkin or Finite-Volume schemes.
Note finally that Finite Difference methods were recently used
in numerical climatology in \cite{Ghader-Nordstrom}.

The outline of the paper is as follows.
In Section \ref{sec:2}, we recall the numerical calculation
of the gradient introduced in \cite{Croisille-10}. Then in Section 
\label{sec:3}, we present the numerical scheme with emphasis 
on the role of the filtering. Dissipation and dispersion analysis 
is given. Finally in Section \ref{sec:4}, we present 
numerical results on two vortex advection problems mentionned above.
These results show the accuracy of our scheme.
The accuracy is comparable to conservative schemes such as DG schemes \cite{Nair-Jablonowski}.
\section{Introduction}
\label{sec:1}
In this paper, we continue the development
of the compact scheme approach introduced in \cite{Croisille-10,Croisille-12}
for partial differential equations problems on the sphere
arising in climatology. A significant part of these problems
consist of propagation equations related in a way or another
to the spherical shallow water equations (SWE)
where the unknown consist of the height 
of the atmosphere $h$ and the wind velocity $\bv$.
These equations represent the central hyperbolic problem
to be solved in a spherical geometry. 
The linearized version of SWE is called LSWE. It
represents the simplest physicall based wave problem.

In recent years efforts
has been devoted by researchers to import and adapt
conservative methods  
steming from numerical gas dynamics to the spherical context.
The particular case of the Cubed-Sphere
is widely used,
either in the 
finite volume upwind context 
\cite{Chen-Xiao, Ullrich-Jablonowski-vanLeer}
of in the one of 
the Discontinuous Galerkin method,
\cite{Bao-Nair-Tufo}.

Conservative schemes 
are applied to the conservative form of the equations.
i.e. for the SW system, the vector height/momentum 
$q=[h,h\bv]$.Such schemes 
guarantee by construction ensure the conservation
of $q$. However many more quantities
must be theoretically conserved: the energy, the potntial enstrophy
and the angular momentum. How the scheme conserves
these quantities for large time mainly depends of the
numerical dissipation of the scheme. This dissipation
must be kept as low as possible for realistic simulations.

Finite difference schemes 
for simulating problems in climatology have
attracted interest since 
more than 40 years, \cite{Arakawa}.
Recently, Finite Difference methods related
to the same approach were recently used
in numerical climatology in \cite{Ghader-Nordstrom}.

The scheme we are considering here is a fully centered
scheme applied to the conconservative form of SWE or LSWE.
In the realm of existing methods 
for fluid flows, our approach is strongly connected to 
the compact schemes for aeroacoustics \cite{Lele, Visbal-Gaitonde, 
Tam-Webb, Bogey-Bailly} and turbulence, \cite{Kim-Moin}.
Here, we want to show that this approach 
isa also relevant for a wide class
of propagation problems in numericl climatology.
In this sense, it belongs 
to a classical approach, which can
be traced back to early works
in approximation and interpolation theory, \cite{Collatz}.

The convective problems over the sphere considered here 
fall into three categories
\begin{itemize}
\item 
The linear scalar equation
advection 
\begin{equation}
\label{eq:adv}
\dfrac{\partial h}{\partial t}  (t,\mathbf{x})+ \mathbf{c}(t,\mathbf{x}) 
\cdot \nabla_T h(t,\mathbf{x})=0
\end{equation}
In (\ref{eq:adv}), the velocity 
$\mathbf{c}(t,\mathbf{x}) \in (0, +\infty) \times \mathbb{S}^2$ 
is prescribed so that the scalar value $h(t,\mathbf{x})$ 
exhibits two moving antipodal vortices.
This test case was introduced
in \cite{Nair-Machenhauer,Nair-Jablonowski}. 
This is a demanding test case, which requires
good accuracy and stability properties.
% We refer to Sections
% \ref{sec:4.1} and \ref{sec:4.2} for setails.
\item
The LSWE system at the rest state $q_0=[H_0,\bv_0=\b0]$
with $H_0 >0$ a constant height, 
is expressed as:
\begin{equation}
\label{eq:lswe}
(LSWE) \left\{
\begin{array}{l}
\dfrac{\partial \eta}{\partial t} (t,\mathbf{x})+ H \nabla_T 
\cdot \mathbf{v}(t,\mathbf{x})=0\\
\dfrac{\partial \mathbf{v} }{\partial t} (t,\mathbf{x})+ 
g \nabla_T \eta(t,\mathbf{x}) + f(\mathbf{x}) \mathbf{k}(\mathbf{x}) \wedge
\mathbf{v}(t,\mathbf{x})=0\\
\end{array}
\right.
\end{equation}
Despite its simplicity, this system still 
is the source of many open mathematicl questions
going back to Laplace. We refer to the recent monograph by
\cite{Paldor} and the references therein. The state at rest is 
$(\mathbf{v}_0,\eta_0)=(0,H)$ and the 
The 3-components perturbation unknowns is
the vector $(t,\mathbf{x}) \mapsto (\mathbf{v}(t,\mathbf{x}), \eta(t,\mathbf{x}))$.
\item The SWE system is nonlinear . In vectorial form it is expressed
as:
\begin{equation}
\label{eq:swe}
(SWE) \left\lbrace
\begin{array}{l}
\dfrac{\partial h^{\star}}{\partial t} (t,\mathbf{x}) + \nabla_T \cdot \left( h^{\star}(t,\mathbf{x}) \mathbf{v}(t,\mathbf{x}) \right) =  0 \\
\dfrac{\partial \mathbf{v}}{\partial t}  (t,\mathbf{x}) + \nabla_T \left( \dfrac{1}{2} \mathbf{v}(t,\mathbf{x})^2 
+ g h(t,\mathbf{x}) \right) + \left( f(\mathbf{x}) + \zeta(t,\mathbf{x}) \right) \mathbf{n}(\mathbf{x}) \wedge \mathbf{v}(t,\mathbf{x}) =  \mathbf{0} 
\end{array}
\right.
\end{equation}

where $\zeta = \left( \nabla_T \wedge \mathbf{v} \right) \cdot \mathbf{k}$ 
is the {\it relative vorticity} and 
$h^{\star} = h - h_s$ with $h_s$ the bottom topography function.
\end{itemize}

The outline of the paper is as follows.
In Section \ref{sec:2}, we recall the numerical calculation
of the gradient introduced in \cite{Croisille-10}. In Section 
\label{sec:3}, our numerical scheme is presented with emphasis 
on the role of the filtering. Dissipation and dispersion analysis 
is given. Finally in Section \ref{sec:4}, we present 
numerical results on a series of test cases
for the three systems of equations above. 
These results show the accuracy and the stability of our approach.
%%%%%%%%%%%%%%%%%%%%%%%%%%%%%%%%%%%%%%%%%%%% *** SECTION ***************************************************************************************************************

\section{Fourth order accurate spherical gradient}
\label{sec:2}
\subsection{Hermitian approach to partial derivatives on the sphere}
\label{sec:2.1}
In this section we present the principle
behind the calculation of the high-order accurate
approximate spherical gradient, as introduced in 
\cite{Croisille-10, Croisille-12}.
Consider a given function $ \mathbf{s} \in \mathbb{S}^2 \mapsto u(\mathbf{s})$ on the unit sphere
and a point $\bar{\mathbf{s}}$ where an approximate value of the tangential gradient
$\nabla_T u(\bar{\mathbf{s}})$ is required.
Consider a point $\mathbf{s}_0$ close to $\bar{\mathbf{s}}$ and two orthogonal great circles
$C_1$ and $C_2$ intersecting at $\mathbf{s}_0$. 
Denote by $\xi$ and $\eta$ the angles along $C_1$ and $C_2$ respectively.
The angles $(\xi,\eta)$ form a local coordinate system 
whose $\mathbf{s}_0(0,0)$ is the center. 
The tangential gradient $\nabla_T u(\bar{\mathbf{s}})$  is given by:
\beq
\label{eq:87.1}
\nabla_T u(\bar{\mathbf{s}})=
\Big(
\partial_\xi u(\bar{\mathbf{s}})
\Big)
g^\xi(\bar{\mathbf{s}})
+
\Big(
\partial_\eta u(\bar{\mathbf{s}})
\Big)
g^\eta(\bar{\mathbf{s}}).
\eeq
In (\ref{eq:87.1}), $(g^\xi(\mathbf{s}),g^\eta(\mathbf{s}))$
denotes the dual basis of the basis 
$(g_\xi(\mathbf{s}),g_\eta(\mathbf{s}))$ where
\beq
\label{eq:32.19}
g_\xi(\mathbf{s})=\frac{\partial \mathbf{s}}{\partial \xi},\;\;\;
g_\eta(\mathbf{s})=\frac{\partial \mathbf{s}}{\partial \eta}.
\eeq
Consider next the iso-$\eta$ curve passing through $\bar{\mathbf{s}}$.
This curve is locally defined as a function of the $\xi$ parameter as:
\beq
 \xi \in (\bar{\xi}-\varepsilon, \bar{\xi}+\varepsilon)
 \mapsto \mathbf{s}(\xi,\bar{\eta}), \;\;\; \varepsilon >0.
\eeq
Due to the orthogonality of the circles $C_1$ and $C_2$, this curve
is also an arc of great circle. We call $\bar{C}$ this great circle.
We define a discrete grid of $\bar{C}$ 
using the points with coordinates $\mathbf{s}_i(i\Delta \xi,\bar{\eta})$. The value
$\Delta \xi >0 $ is the constant step defined by
\beq
\Delta \xi = \frac{\pi}{2N}
\eeq 
and $N>0$ is an integer.
Assume for the moment that the point $\bar{\mathbf{s}}$ belongs to this grid and that 
$\bar{\mathbf{s}}=\mathbf{s}_i$ for some integer $i$.  A simple way 
to approximate $\nabla_T u(\bar{\mathbf{s}})$
is obtained by approximating in (\ref{eq:87.1}) 
the partial derivatives
$\partial_\xi u(\mathbf{s}_i)$ and $\partial_\eta u(\mathbf{s}_i)$.
Consider first $\partial_\xi u(\bar{\mathbf{s}})$.
Since $\bar{\mathbf{s}}=\mathbf{s}_i$, using 
the standard centered divided difference we have;
\beq
\label{eq:75.10}
\partial_\xi u(\mathbf{s}_i) \simeq 
\frac{u(\mathbf{s}_{i+1})- 
u(\mathbf{s}_{i-1})}{2 \Delta \xi}.
\eeq
The divided difference in the right-hand-side is denoted by $\delta_\xi u_i$.
Proceeding in the same manner in the $\eta$ direction, we set up
a grid on the circle $\bar{C}^\prime$:
the iso-$\xi$ great circle through by $\bar{\mathbf{s}}$ receives 
a grid with step size $\Delta \eta = \Delta_\xi$. The points
along this grid are numbered with index $j$ and we
call ${\mathbf{s}}=\mathbf{s}^\prime_j$. The value
$\partial_\eta u (\bar{s})$ is approximated by:
\beq
\label{eq:75.11}
\partial_\eta u(\bar{\mathbf{s}}) \simeq 
\delta_\eta u^\prime_j = \frac{u(\mathbf{s}^\prime_{j+1})- 
u(\mathbf{s}^\prime_{j-1})}{2 \Delta \eta}.
\eeq
A first candidate for the approximate value to $\nabla_T u(\bar{\mathbf{s}})$ is therefore:
\beq
\label{eq:87.2}
\nabla_{T,h} u(\bar{\mathbf{s}})=
\left(\delta_\xi u_i\right)
g^\xi(\bar{\mathbf{s}})
+
\left(\delta_\eta u^\prime_j \right)
g^\eta(\bar{\mathbf{s}}).
\eeq
The vector $\nabla_{T,h} u(\bar{\mathbf{s}})$ is obviously a second order 
approximation to $\nabla_T u(\bar{\mathbf{s}})$.
A simple way to go beyond second order
is to modifiy $\delta_\xi u_i$ using instead
the Hermitian derivative $\delta^H_\xi u_i$.
It is defined in terms of $\delta_\xi u_i$ by the relation
\beq
\label{eq:73.10}
\frac{1}{6} \delta_\xi^H u_{i-1}
+\frac{2}{3} \delta_\xi^H u_i
+\frac{1}{6} \delta_\xi^H u_{i+1}
= 
\delta_\xi u_i.
\eeq
The relation (\ref{eq:73.10}) defines implictely
$ \delta_\xi^H u_i$ as a perturbation of 
$\delta_\xi u_i$ since it can be expressed as:
\beq
\label{eq:73.14}
\delta_\xi^H u_i
= 
\delta_\xi u_i -\frac{\Delta^2\xi}{6} \delta^2_\xi \delta^H_\xi u_i
\eeq
The difference with (\ref{eq:75.10}) 
is that (\ref{eq:73.10}) involves values of $u$ along 
the full on $\bar{C}$ and not only the two neighboor values 
at $i\pm 1$.
Solving (\ref{eq:73.10}) provides $\delta^H u_i$.
This values satisfies the fourth-order consistency relation
\beq
\label{eq:34.18}
\delta^H_\xi u_i=
\partial_\xi u(\bar{\mathbf{s}})+ O(\Delta \xi^4)
\eeq
Proceeding in the same way along the $\eta-$ direction gives 
the approximation
\beq
\label{eq:34.19}
\delta^H_\eta u^\prime_j=
\partial_\eta u(\bar{\mathbf{s}})+ O(\Delta \xi^4)
\eeq
The Hermitian approximate gradient at the point $\bar{\bs}$ 
is therefore:
\beq
\label{eq:87.2.1}
\nabla^H_{T,h} u(\bar{\mathbf{s}})=
\left(\delta^H_\xi u_i\right)
g^\xi(\bar{\mathbf{s}})
+
\left(\delta^H_\eta u^\prime_j\right)
g^\eta(\bar{\mathbf{s}})
\eeq
It results from (\ref{eq:34.18}-\ref{eq:34.19}) that
$\nabla^H_{T,h} u(\bar{\mathbf{s}})$ satisfies the fourth order
consistency relation
\beq
\label{eq:87.4}
\nabla^H_{T,h} u(\bar{\mathbf{s}})=\nabla_{T} u(\bar{\mathbf{s}})+O(\Delta \xi^4)+O(\Delta \eta^4).
\eeq
The Hermitian approximation (\ref{eq:87.4}) is the basis of our approximate gradient on the Cubed-Sphere.
%%%%%%%%%%%%%%%%%%%%%%%%%%%%%%%%%%%%%%%%%%%%%%%%%%%%%%%%%%%%%%%%%%%%%%%%%%%
\subsection{Approximate gradient on the Cubed-Sphere}
%%%%%%%%%%%%%%%%%%%%%%%%%%%%%%%%%%%%%%%%%%%%%%%%%%%%%%%%%%%%%%%%%%%%%%%%%%%%
The Cubed Sphere is a grid of the sphere. This grid and variants were introduced by
various authors. A systematic
presentation was given in
\cite{Ronchi-Iacono-Paolucci}.
This grid has been widely used for numerical climatology.
The Cubed-Sphere is composed of six panels with
label $(k)=(I), (II), (II), (IV), (V)$ and $(VI)$. Each panel matches
the face of the cube, in which the sphere
is embedded. Each panel
supports a Cartesian grid of size $N\times N$.
It is equipped with a 
coordinate system $(\xi,\eta)$.
As in the preceding section,
$\xi$ and $\eta$ are angles along a couple of orthogonal great circles
intersecting at the center of the panel.
A typical panel and the associated grid is represented
on Fig. \ref{fig:1}.
The  grid points are called $\mathbf{s}_{i,j}^{(k)}$ with $ (I) \leq (k) \leq (VI)$ and
$ -N/2 \leq i,j \leq N/2$. 
In panel $(k)$ and for all fixed $j$, the points  $ i \mapsto \mathbf{s}_{i,j}$ are 
located along a great circle. This essential property of the Cubed-Sphere
permits to follow the idea presented in Section \ref{sec:2.1}: 
the calculation of an approximate gradient at $\mathbf{s}^{(k)}_{i,j}$
by mean of Hermitian derivatives.
The approximate gradient
$\nabla_{T,h} u^{(k)}_{i,j}$ is given by
\beq
\label{eq:85.13}
\nabla_{T,h} u^{(k)}_{i,j}=
u_{\xi,i,j}^{(k)} \mathbf{g}^{\xi, (k)}_{i,j} 
+
u_{\eta,i,j}^{(k)} \mathbf{g}^{\eta, (k)}_{i,j} 
\eeq
As in the preceding section, the values $ u_{\xi,i,j}^{(k)}$ and $u_{\eta,i,j}^{(k)}$
are Hermitian approximation of the partial derivatives $\partial_\xi u(\mathbf{s}^{(k)}_{i,j})$ and
$\partial_\eta u(\mathbf{s}^{(k)}_{i,j})$. The calculation of 
these Hermitian derivatives are
based on a set of data located along two great circles called $\bar{C}_{i,j}$ and $\bar{C}^\prime_{i,j}$.
The data along $\bar{C}_{i,j}$ and $\bar{C}^\prime_{i,j}$
are of course based on specific points of the Cubed-Sphere.
The choice of these points is detailed in \cite{Croisille-10, Croisille-12}
and we refer to these two references for more details.
Even if no mathematical proof is available yet, 
numerical evidence show a consistency close to $4$.
%%%%%%%%%%%%%%%%%%%%%%%%%%%%%%%%%%%%%%%%%%%%%%%%%%%%%%%%%%%%%%%%%%%%%%%%%%%%%%%%%
\begin{figure}
   \def\svgwidth{0.4 \textwidth}
\input{drawing13.pdf_tex}
\caption{The points of a typical panel of the Cubed-Sphere are classified in three categories:
(i) Circles correspond to {\sl internal} points; (ii) Squares correspond to {\sl edge} points ;
(iii) Pentagons correspond to {\sl corner} points}
\label{fig:1}
\end{figure}
\section{Compact discrete differncing on the Cubed Sphere}
\label{sec:2}
In this section, we decribe specific properties
of our scheme.
After overviewing the Cubed Sphere in Section
\ref{sec:CS}, the fundamental of
compact differencing along great circles
on the Cubed Sphere is summarized in Section \ref{sec:2.1}.
In Section \ref{sec:2.2} the discrete gradient, divergence and curl
are given.

%%%%%%%%%%%%%%%%%%%%%%%%%%%%%%%%%%%%%%%%%%%%%%%%%%%%%%%%%%%%%%%%%%%%%%%%%%%
\subsection{The Cubed-Sphere}
%%%%%%%%%%%%%%%%%%%%%%%%%%%%%%%%%%%%%%%%%%%%%%%%%%%%%%%%%%%%%%%%%%%%%%%%%%%%
\label{sec:CS}
The Cubed Sphere is a grid of the sphere. This grid and variants were introduced by
various authors. A systematic
presentation was given in
\cite{Ronchi-Iacono-Paolucci}.
This grid has been widely used for numerical climatology.
The Cubed-Sphere is composed of six panels with
label $(k)=(I), (II), (II), (IV), (V)$ and $(VI)$. Each panel matches
the face of the cube, in which the sphere
is embedded. Each panel
supports a Cartesian grid of size $N\times N$.
It is equipped with a 
coordinate system $(\xi,\eta)$.
As in the preceding section,
$\xi$ and $\eta$ are angles along a couple of orthogonal great circles
intersecting at the center of the panel.
The Cubed Sphere is represented on Fig. \ref{fig:1}
and a typical panel is shown
on Fig. \ref{fig:2}.
The  grid points are called $\mathbf{s}_{i,j}^{(k)}$ with $ -N/2 \leq i,j \leq N/2$ and 
$ (I) \leq (k) \leq (VI)$ denotes the index of the panel.
%%%%%%%%%%%%%%%%%%%%%%%%%%%%%%%%%%%%%%%%%%%%%%%%%%%%%%%%%%%%%%%%%%%%%%%%%%%%%%%%%%%%%%%%%%%%%%%%%%%%
\subsection{Periodic compact diffentiation along great circles}
\label{sec:2.1}
Let us recall the basics of compact differentiation.
The centerded finite difference on a regular grid 
with step $\Delta x$ is
\beq
\label{eq:87.13}
\delta_x u_j =\frac{u_{j+1}-u_{j-1}}{2 \Delta x}
\eeq
Relation (\ref{eq:87.13}) is a second order approximation of the derivative
$u^\prime(x_j)$ in the sense that
\beq
\label{eq:87.14}
\delta_x u^\ast_j =u^\prime(x_j)+\frac{\Delta x^2}{12}
u^{(4)}(\xi_j)
\eeq
for a certain value of $\xi_j \in [x_{j-1}, x_{j+1}]$.
Consider now the {\it implicit relation} with unknown $\delta^H_x u_{j}$
\beq
\label{eq:73.10.27}
\frac{1}{6} \delta_\xi^H u_{j-1}
+\frac{2}{3} \delta_\xi^H u_j
+\frac{1}{6} \delta_\xi^H u_{j+1}
= 
\delta_\xi u_j.
\eeq
The relation (\ref{eq:73.10.27}) is a three point compact relation, 
meaning that the stencil of dependence 
consists of the three points is $x_{j-1},x_j,x_{j+1}$. 
It is easily proved, that in the periodic setting $\delta_x^H u_j$ satisfies the consistency relation
\beq
\label{eq:87.14.1}
\delta^H_x u^\ast_j =u^\prime(x_j)
-\frac{\Delta x^4}{180}
u^{(6)}(\xi_j).
\eeq
% When (\ref{eq:87.14.1}) is considered
% on the interval $[0, 2 \pi]$ with periodic
% boundary conditions, then the values
% $\delta^H_{x,j}$ are approximations
% of order $4$ at all points $x_j$.
We refer to \cite{Lele} and to {BenArtzi-Croisille-Fishelov-4}.

The fundamental of our numerical scheme
is to perform the calculation 
of  Hermitian derivatives (\ref{eq:73.10.27}) 
along a series of great circles on the sphere.
These great circles are associated to
coordinates lines on each panel.
In a second time, discrete differential operators are
deduced from thos preliminary computation 
Details can be found in \cite{Croisille-10, Croisille-12}.
%%%%%%%%%%%%%%%%%%%%%%%%%%%%%%%%%%%%%%%%%%%%%%%%%%%%%%%%%%
\subsection{Hermitian gradient, divergence and curl operators}
%%%%%%%%%%%%%%%%%%%%%%%%%%%%%%%%%%%%%%%%%%%%%%%%%%%%%
\label{sec:2.2}
On the panel $(k)$, $(I) \leq (k) \leq (VI)$, the local basis is 
\beq
\label{eq:32.19}
g_\xi(\mathbf{s})=\frac{\partial \mathbf{s}}{\partial \xi},\;\;\;
g_\eta(\mathbf{s})=\frac{\partial \mathbf{s}}{\partial \eta}.
\eeq
and the dual basis is called $(g^\xi, g^\eta)$.
Let $\bx \in \mathbb{S}^2 \mapsto f(\bx)$ be a regular function.
The gradient is defined by
%, divergence and curl are expressed 
%in local coordinates $(\xi,\eta)$ by
\beq
\nabla_T f= 
\partial_\xi f \bg^\xi +\partial_\eta f \bg^\eta
\eeq
Let $f^k_{i,j}$ 
be a given scalar gridfunction.
\beq
\label{eq:95.13}
\nabla_{T,\Delta} f= 
 f_{\xi,i,j}^k (\bg^\xi)_{i,j}^k 
+
f_{\eta,i,j}^k (\bg^\eta)_{i,j}^k 
\eeq

Similarly if 
$\bx \in \mathbb{S}^2 \mapsto \mathbb{R}^2$
is a regular tangential vector field,
the divergence and curl are:
\beq
\left\{
\begin{array}{l}
\nabla_T \cdot \bv= 
\partial_\xi \bv \cdot  \bg^\xi +\partial_\eta \bv \cdot \bg^\eta\\
(\nabla_T \times \bv = 
\partial_\xi \bv \times \bg ^xi +\partial_\eta \bv \times \bg ^eta
\end{array}
\right.
\eeq
The discrete divergence and curl of 
a discrete tangential vector grid function
$\bv^k_{i,j}$ are called 
$\nabla_{T,Delta} \dot \bv_{i,j}^k$ 
and $\nabla_{T,Delta} \times\bv_{i,j}^k$ respectively.
They are defined by
\beq
\left\{
\begin{array}{l}
\nabla_{T,\Delta} \cdot (\bv)_{i,j}^k= 
(\bv)_{\xi,i,j}^k \cdot  (\bg^\xi)_{i,j}^k 
+
\nabla_{T,Delta}
(\bv)^k_{i,j} \cdot (\bg^\eta)^k_{i,j}\\
\nabla_{T,\Delta} \times (\bv)_{i,j}^k = 
(\bv)_{i,j}^k \times (\bg ^xi)^k_{i,j} 
+
(\bv)_{i,j}^k \times (\bg ^eta)^k_{i,j}
\end{array}
\right.
\eeq
Due to (\ref{eq:87.14.1}), we expect these discrete operators
to be fourth order accurate.

\begin{prop}
If $f(\bx)$ and $\bv(\bx)$ are any given scalar (resp. tangential vector)
function on the sphere. If $f^\ast$ and $\bv^\ast$
are their restriction to the Cubed Sphere, then the
consistency relations hold:
\beq
\label{eq:11.91}
\begin{aligned}
\nabla_{T,\Delta} f^k_{i,j}-(\nabla_T f)^{\ast,k}_{i,j}&=O (\Delta^3)\\
\nabla_{T,\Delta} \cdot \bv^k_{i,j}
-(\nabla_T \cdot  f)^{\ast,k}_{i,j}&=O (\Delta^3)\\
\nabla_{T,\Delta} \times \bv^k_{i,j}
-(\nabla_T \times  \bv)^{\ast,k}_{i,j}&=O (\Delta^3)
\end{aligned}
\eeq
\end{prop}
\begin{proof}
Suppose that in relation (\ref{eq:87.14.1}) the data $f_j$ are perturbed
by a term of order $O(\Delta^4)$. Then the resulting
perturbation of $f_\xi$ is at most of order $O(\Delta^3)$.
The consistency error drops therefore from order $4$ to order $3$.
\end{proof}
%%%%%%%%%%%%%%%%%%%%%%%%%%%%%%%%%%%%%%%%%%%%%%%%%%%%%%%%%%%%%%%%%%%%%%%%%%%%%%%%%%%%%%%
\begin{figure}
\includegraphics[scale=0.3]{Images/fig6.jpg}
\caption{
The Cubed Sphere with grid parameter $N=16$.
The total number of gridpoints is $6 \times N^2+2=1538$ in this case.
The panels $(I)$, $(II)$, $(III)$ and $(IV)$ are located
around the equatorial plane $z=0$. The index of the north panel is $(V)$ 
and the one of the south panel is $(VI)$. 
}
\label{fig:1}
\end{figure}
%%%%%%%%%%%%%%%%%%%%%%%%%%%%%%%%%%%%%%%%%%%%%%%%%%%%%%%%%%%%%%%%%%%%%%%%
\begin{figure}
   \def\svgwidth{0.3 \textwidth}
\input{drawing13.pdf_tex}
\caption{
The points of a typical panel of the Cubed-Sphere are classified in three categories:
(i) Circles correspond to {\sl internal} points; (ii) Squares correspond to {\sl edge} points ;
(iii) Pentagons correspond to {\sl corner} points}
\label{fig:2}
\end{figure}
%%%%%%%%%%%%%%%%%%%%%%%%%%%%%%%%%%%%%%%%%%%%%%%%%%%%%%%%%%%%%%%%%%%%%
%%%%%%%%%%%%%%%%%%%%%%%%%%%%%%%%%%%%%%%%%%%%%%%%%%%%%%%%%%%%%%%%%%%%%%%
%%%% *** SECTION ***********************************************************************************************************************

\section{Centered compact scheme with filtering}
%%%%%%%%%%%%%%%%%%%%%%%%%%%%%%%%%%%%%%%%%%%%%%%%%%%%%%%%%%%%%%%%%%%%%%%%%%%%%%%%%%%%%%%%%%%
Our approximate scheme
uses a centered finite-difference approximation on the Cubed-Sphere.
In particular we do not use any kind of upwinding.
A stabilization mechanism is a linear filtering
at each time step. This kind of numerical 
algorithm is stronlgy related to so-called 
hyperviscous numerical diffusion.
%%%%%%%%%%%%%%%%%%%%%%%%%%%%%%%%%%%%%%%%%%%%%%%%%%%%%%%%%%%%%%%%%%%%%%%%%55
\subsection{Method of lines}
%%%%%%%%%%%%%%%%%%%%%%%%%%%%%%%%%%%%%%%%%%%%%%%%%%%%%%%%%%%%%%%%%%%%%%%%%%%%%%%%%%%%%%%%%%%
We consider the convection equation on the sphere
with tangential velocity 
$(\mathbf{x} ,t) \in \mathbb{S}^2 \times \mathbb{R}_+ \mapsto \mathbf{c}(\mathbf{x},t)$.
\beq
\label{eq:978.23.1}
\left\{
\begin{array}{l}
\partial_t h(\mathbf{x},t)+\mathbf{c}(\mathbf{x},t) \cdot \nabla_s h(\mathbf{x},t)=0,\\
h(\mathbf{x},0)=h_0(\mathbf{x}).
\end{array}
\right.
\eeq
Existence and uniqueness for (\ref{eq:978.23.1}) is
obtained by the method of characteristics.
The task is to calculate
an approximation of $h(\mathbf{x},t)$.

Let
$t \mapsto h^{(k)}_{i,j}(t)$ be a semidiscrete approximation
in space of (\ref{eq:978.23.1}). This semi-discrete 
approximation is 
the solution of the differential system:
\beq
\label{eq:978.23.1a}
\left\{
\begin{array}{l}
{d h_{i,j}^k(t) \over dt} +\mathbf{c}^{(k)}_{i,j}(t)\cdot  \mathbf{\nabla} h h_{i,j}^k(t)=0,
\quad -M\leq i,j\leq M,\;\;\; I\leq k\leq VI,\\
h^{(k)}_{i,j}(0)=h_0(\mathbf{s}^{(k)}_{i,j})
\end{array}
\right.
\eeq
where $\mathbf{c}^{(k)}_{i,j}(t) \triangleq \mathbf{c}(\mathbf{s}^{(k)}_{i,j})(t)$.
Denote by $H(t) \triangleq  h^{(k)}_{i,j}(t)$ the gridfuntion
with components $h^{(k)}_{i,j}$. The equation (\ref{eq:978.23.1}) 
is expressed in vector form as:
\beq
\label{eq:71.10.3}
\frac{d}{dt}H(t)=J(t) H(t)
\eeq
where $J(t)$ is the matrix corresponding to
\beq
[J(t) H(t)]_{i,j}^{(k)} \triangleq 
-\bc^{(k)}_{i,j}(t)\cdot  \bgrad_{T,h} h_{i,j}^k(t).
\eeq
Since fourth order accuracy in space is expected, we proceed with 
the explicit in time RK 4 approxaimtion in time for
(\ref{eq:71.10.3}).
Let $\Delta t>0$ be a time-step. The RK4 time-scheme
is applied to (\ref{eq:71.10.3}).

\begin{equation}
\label{eq:300.41-1}
\left\{ 
\begin{array}{l}
K^{(0)} = J(t^n)H^{n}\\
K^{(1)} = J(t^{n+1/2})(H^{n}+\frac{1}{2}\Delta t K^{(0)})\\
K^{(2)} = J(t^{n+1/2})(H^{n}+\frac{1}{2}\Delta t K^{(1)})\\
K^{(3)} = J(t^{n+1})(H^{n}+\Delta t K^{(2)})\\
H^{n+1} = H^{n}
+\Delta t\Bigg(\frac{1}{6}K^{(0)}+\frac{1}{3}K^{(1)}
+\frac{1}{3}K^{(2)}+\frac{1}{6}K^{(3)}\Bigg).
\end{array}\right.
\end{equation}
%%%%%%%%%%%%%%%%%%%%%%%%%%%%%%%%%%%%%%%%%%%%%%%%%%%%%%%%%%%%%%%%%%%%%%%%%%%%%
\subsection{Dissipation and dispersion analysis}
Finite difference schemes are usually described
by their dissipation and dispersion properties. 
A customary analysis of this kind is the so called {\sl modified equation}
analysis, \cite{Shokin}.  This analysis is usually 
performed on the linear advection equation
\beq
\partial_t u + c \partial_x u=0
\eeq
We recall next the main features of this analysis.
Consider a numerical scheme
\beq
\label{eq:76.10}
\frac{u^{n+1}_j-u^n_j}{\Delta t}
+ L_h u|^n_j =0
\eeq
In (\ref{eq:76.10}), the operator $L_h u^n_j$ approximates $-c \partial_x u$.
In operator form, the scheme (\ref{eq:76.10}) is rewritten as
\beq
\label{eq:56.10}
\frac{e^{\Delta t \partial_t}-1}{\Delta t} u^n_j = -L_h u|^n_j
\eeq
The Logarithm series provides the formal expansion 
\beq
\label{eq:87.12}
\partial_t = \frac{e^{\Delta t \partial_t}-1}{\Delta t}
-\frac{\Delta t}{2}\left(\frac{e^{\Delta t \partial_t}-1}{\Delta t}\right)^2
+\frac{\Delta t}{3}\left(\frac{e^{\Delta t \partial_t}-1}{\Delta t}\right)^3
-\frac{\Delta t}{4}\left(\frac{e^{\Delta t \partial_t}-1}{\Delta t}\right)^4
+....
\eeq
Using (\ref{eq:56.10}) in the right hand side of (\ref{eq:87.12}),
we obtain an identity of the form
\beq
\label{eq:34.13}
\partial_t u+ c \partial_x u=c \Big[
  E_1 h \partial_x^{(2)}u 
+ E_2 h^2 \partial_x^{(3)}u 
+ E_3 h^3 \partial_x^{(4)}u 
+ E_4 h^4 \partial_x^{(5)}u 
+...
\Big]
\eeq
This identity is called the modified equation of the scheme. It represents a transport equation
with a perturbation in the form of an asymptotic expansion
in powers of $h$. The coefficients $E_\alpha$ depends
only the Courant number $\lambda= c \Delta t /h$. 
In the case of the scheme (), the modifed equation 
is expressed as:
\beq
\partial_t u+ c \partial_x u=c\Big[ \frac{h^4}{360}(3 \lambda^4+2) \partial_x^{(5)}u
+ \sum_\alpha h^{\alpha}E_\alpha \partial_x^{(\alpha+1)}u
\eeq
The first coefficients are 
\beq
\label{eq:72.10}
\left\{
\begin{array}{l}
E_4= \frac{1}{360}(3 \lambda^4+2)\\
E_5= \frac{1}{144}\lambda^5\\
E_6= \frac{1}{3024}(-2+9\lambda^6)\\
E_7= \frac{1}{1152}\lambda^7\\
E_8= \frac{1}{25920}(\lambda^4-1)(5 \lambda^4-1)\\
E_9= -\frac{1}{4320}\lambda^5
\end{array}
\right.
\eeq
As a result the scheme is $4-$order. In addition it is found to
be dissipative for the term in $\partial_x^{(6)}$ but antidissipative
with the terms in $\partial_x^{(8)}$ and  $\partial_x^{(10)}$.
%%%%%%%%%%%%%%%%%%%%%%%%%%%%%%%%%%%%%%%%%%%%%%%%%%%%%%%%%%%%%%%%%%%%%%%%%%%%%%%%%
\subsection{Tenth-order hyperviscosity}
%%%%%%%%%%%%%%%%%%%%%%%%%%%%%%%%%%%%%%%%%%%%%%%%%%%%%%%%%%%%%%%%%%%%%%%%%%%%%%
\label{sec:4}
As described in the previous section, stabilisation mechanism could be useful
to obtain a better stability profile. 
In such a situation a high-order filtering is added at each time step. 
Numerical practice showed
that a tenth-order filter from \cite{Visbal-Gaitonde} gives good results.
At each time step the value $H^{(n)}$ in (\ref{eq:300.41-1}) is replaced 
by $\mathcal{F} H^{(n)}$ where $\mathcal{F}$ is the filtering operator
acting on the gridfunctions defined by 
the composition of two one-dimensional filters along the $\xi$ and the $\eta$ directions on each panel:
\beq
\mathcal{F}=\mathcal{F}_\xi \circ \mathcal{F}_{\eta}
\eeq
For a one-dimensional grid function $u_j$,
The filter $\mathcal{F}$ belongs to the class 
of the filters 
\beq
\label{eq:75.10.3}
\alpha_f u_{F,i-1}+
u_{F,i}+
\alpha_f u_{F,i+1}=
\sum_{^j=0}^J \frac{a_j}{2}(u_{i+j}+u_{i-j})
\eeq
This kind of filtering was originally introduced
in the Atmospheric Sciences community \cite{Alpert}.
The values of the coefficients in (\ref{eq:75.10}) are given
in \cite{Visbal-Gaitonde}. 
Our numerical results were performed 
with the explicit filter, corresponding to
$\alpha_f=0$ and to the coefficients:
\beq
\label{eq:978.25.3}
\left(
\begin{array}{c}
a_0\\
a_1\\
a_2\\
a_3\\
a_4\\
a_5
\end{array}
\right)
=
\left(
\begin{array}{c}
193/256\\
105/256\\
-15/64\\
45/512\\
-5/256\\
1/512
\end{array}
\right).
\eeq
The fact that () acts as a filter is reflected 
by the values of 
coefficients in (\ref{eq:72.10}) in the modified equation modified equation.
The term in $\partial_x^{(10)} u$ is now dissipative insted of being
antidissitaive without filter. It is now 
\beq
E_9= -\frac{1}{138240}\frac{32\lambda^6-135}{\lambda}
\eeq
All these results were obtained with MAPLE.

%%%%%%%%%%%%%%%%%%%%%%%%%%%%%%%%%%%%%%%%%%%%%%%%%%%%%%%%%%%%%%%%%%%%%%%%%%%%%%%%%%%%
\section{A centered scheme on the Cubed Sphere}
%%%%%%%%%%%%%%%%%%%%%%%%%%%%%%%%%%%%%%%%%%%%%%%%%%%%%%%%%%%%%%%%%%%%%%%%%%%%%%%%%%%%%%%%%%%
%%%%%%%%%%%%%%%%%%%%%%%%%%%%%%%%%%%%%%%%%%%%%%%%%%%%%%%%%%%%%%%%%%%%%%%%%55
\subsection{Method of lines}
%%%%%%%%%%%%%%%%%%%%%%%%%%%%%%%%%%%%%%%%%%%%%%%%%%%%%%%%%%%%%%%%%%%%%%%%%%%%%%%%%%%%%%%%%%%
As recalled in the introduction,
our objectives is to discretize the propagation problems
(\ref{eq:adv}), (\ref{eq:lswe}) or (\ref{eq:swe}). 
Our basic approximation is space is purely centered.
In each case we adopt the discrete differential operators 
deined in (\ref{eq:95.13}, \ref{eq:11.91}) keeping the time continuous (method of lines).
Consider first (\ref{eq:adv}). Let
$t \mapsto h^{k}_{i,j}(t)$ be a semidiscrete approximation
in space of (\ref{eq:adv}), 
solution of the differential system:
\beq
\label{eq:978.23.1a}
\left\{
\begin{array}{l}
{d h_{i,j}^{k}(t) \over dt} +\mathbf{c}^{k}_{i,j}(t)\cdot  \mathbf{\nabla}_T h_{i,j}^{k}(t)=0,
\quad -M\leq i,j\leq M,\;\;\; (I)\leq k\leq (VI),\\
h^{k}_{i,j}(0)=h_0(\mathbf{s}^{k}_{i,j})
\end{array}
\right.
\eeq
where $\mathbf{c}^{k}_{i,j}(t) \triangleq \mathbf{c}(\mathbf{s}^{k}_{i,j})(t)$.
Denote by $H(t) \triangleq  h^{(k)}_{i,j}(t)$ the gridfuntion
with components $h^{(k)}_{i,j}$. 
We proceed in a simlar way (\ref{eq:lswe}) and (\ref{eq:swe}). In each case the 
unknown vector is $ t \mapsto \bq(t)=[h(t), \bv(t)]^T$ is represented on the Cubed Sphere
by the semidiscrete function
\beq
t \mapsto \bq^k_{i,j}(t)=[h^k_{i,j}(t), \bv^k_{i,j}(t)]^T, \;\;\; 
-N/2\leq i,j\leq N/2,\;\;\; (I) \leq k\leq (VI)
\eeq
The system (\ref{eq:lswe}) is approximated 
by the semi-discrete scheme
\beq
\label{eq:978.23.2a}
\left\{
\begin{array}{l}
{d h_{i,j}^k(t) \over dt} +H \nabla_{T,\Delta} \cdot \bv_{T,\Delta}=0\\
{d \bv_{i,j}^k(t) \over dt} +g \nabla_{T,\Delta} \eta^k_{i,j}(t) 
+ f(\bs^k_{i,j}) \bn^k_{i,j} \times \bv^k_{i,j}(t)=0
\end{array}
\right.
\eeq
and the system (\ref{eq:swe}) is approximated by:
\beq
\label{eq:978.23.2b}
\left\{
\begin{array}{l}
{d h_{i,j}^k(t) \over dt} +\nabla_{T,\Delta} \cdot h_{i,j}^k(t)\bv_{T,\Delta}=0\\
{d \bv_{i,j}^k(t) \over dt} +\nabla_{T,\Delta}
\left( \frac{1}{2}\vert \bv^k_{i,j}(t) \vert ^2+ g h^k_{i,j}(t)\right)
+ \left(f(\bs^k_{i,j}) + \zeta^k_{i,j}(t)\right) \bn^k_{i,j} \times \bv^k_{i,j}(t)=0
\end{array}
\right.
\eeq
The centered approximation (\ref{eq:978.23.1a}), (\ref{eq:978.23.2a}) or (\ref{eq:978.23.2b}) of course requires
some numerical diffusion added. 
All the numerical experiments reported in Section \ref{sec:4}
have been performed mith a minimal numerical damping.
In fact, these cases are not dominated 
by shock propagation, and a minimal damping 
is highly desirable in order to preserve fourth order accuracy.
Our method is directly inspired from \cite{Lele, Visbal-Gaitonde, Bogey-Bailly}.
In these works, finite difference schemes are used
for aeroacoustics problems. The schemes are centered 
and the numerical damping conisists only of a high frequency filtering
hich eliminates the +1/-1 mode attached to the grid.
%%%%%%%%%%%%%%%%%%%%%%%%%%%%%%%%%%%%%%%%%%%%%%%%%%%%%%%%%%%%%%%%%%%%%%%%%%%%%%%%%%%5
\subsection{Time stepping using the RK4 time scheme and the associated modified equation}
%%%%%%%%%%%%%%%%%%%%%%%%%%%%%%%%%%%%%%%%%%%%%%%%%%%%%%%%%%%%%%%%%%%%%%%%%%%%%%%%%%%%
Consider for example the equation (\ref{eq:978.23.1a}).
It  is expressed in vector form as:
\beq
\label{eq:71.10.3}
\frac{d}{dt}H(t)=J(t) H(t)
\eeq
where $J(t)$ is the matrix corresponding to
\beq
[J(t) H(t)]_{i,j}^{(k)} \triangleq 
-\bc^{(k)}_{i,j}(t)\cdot  \bgrad_{T,h} h_{i,j}^k(t).
\eeq
The spatial approximation is perfomed using 
the standard RK 4 time-scheme.
Let $\Delta t>0$ be a time-step. The RK4 time-scheme
is applied to (\ref{eq:71.10.3}).

\begin{equation}
\label{eq:300.41-1}
\left\{ 
\begin{array}{l}
K^{(0)} = J(t^n)H^{n}\\
K^{(1)} = J(t^{n+1/2})(H^{n}+\frac{1}{2}\Delta t K^{(0)})\\
K^{(2)} = J(t^{n+1/2})(H^{n}+\frac{1}{2}\Delta t K^{(1)})\\
K^{(3)} = J(t^{n+1})(H^{n}+\Delta t K^{(2)})\\
H^{n+1} = H^{n}
+\Delta t\Bigg(\frac{1}{6}K^{(0)}+\frac{1}{3}K^{(1)}
+\frac{1}{3}K^{(2)}+\frac{1}{6}K^{(3)}\Bigg).
\end{array}\right.
\end{equation}
The RK4 times scheme carries by itslef
some damping. A classical analysis
of this damping is the so called {\sl modified equation}
analysis, \cite{Shokin}.  This analysis usually 
relies on the linear equation
\beq
\partial_t u + c \partial_x u=0, \;\;\; c>0
\eeq
Consider a numerical scheme
\beq
\label{eq:76.10}
\frac{u^{n+1}_j-u^n_j}{\Delta t}
+ L_{\Delta x} u|^n_j =0
\eeq
In (\ref{eq:76.10}), the operator $L_{\Delta x} u^n_j$ approximates $-c \partial_x u$.
In operator form, the scheme (\ref{eq:76.10}) is rewritten as
\beq
\label{eq:56.10}
\frac{e^{\Delta t \partial_t}-1}{\Delta t} u^n_j = -L_{\Delta x} u|^n_j
\eeq
The Logarithm series provides the formal expansion 
\beq
\label{eq:87.12}
\partial_t = \frac{e^{\Delta t \partial_t}-1}{\Delta t}
-\frac{\Delta t}{2}\left(\frac{e^{\Delta t \partial_t}-1}{\Delta t}\right)^2
+\frac{\Delta t}{3}\left(\frac{e^{\Delta t \partial_t}-1}{\Delta t}\right)^3
-\frac{\Delta t}{4}\left(\frac{e^{\Delta t \partial_t}-1}{\Delta t}\right)^4
+....
\eeq
Using (\ref{eq:56.10}) in the right hand side of (\ref{eq:87.12}),
we obtain an identity of the form
\beq
\label{eq:34.13}
\partial_t u+ c \partial_x u=c \Big[
  E_1 \Delta x \partial_x^{(2)}u 
+ E_2 (\Delta x)^2 \partial_x^{(3)}u 
+ E_3 (\Delta x)^3 \partial_x^{(4)}u 
+ E_4 (\Delta x)^4 \partial_x^{(5)}u 
+...
\Big]
\eeq
This identity is called the modified equation of the scheme. It represents a transport equation
with a perturbation in the form of an asymptotic expansion
in powers of $h$. The coefficients $E_\alpha$ depends
only the Courant number $\lambda= c \Delta t /\Delta x$. 
In the case where $L_{\Delta x} u^n_j= -c \delta_x^H u^n_j$ (see (\ref{eq:73.10.27})), the modified equation 
is expressed as:
\beq
\label{eq:11.53}
\partial_t u+ c \partial_x u=c\Big[ \frac{(\Delta x)^4}{360}(3 \lambda^4+2) \partial_x^{(5)}u
+ \sum_\alpha (\Delta x)^{\alpha}E_\alpha \partial_x^{(\alpha+1)}u
\eeq
The first coefficients are 
\beq
\label{eq:72.10}
\left\{
\begin{array}{l}
E_4= \frac{1}{360}(3 \lambda^4+2),\\
E_5= \frac{1}{144}\lambda^5,\\
E_6= \frac{1}{3024}(-2+9\lambda^6),\\
E_7= \frac{1}{1152}\lambda^7,\\
E_8= \frac{1}{25920}(\lambda^4-1)(5 \lambda^4-1),\\
E_9= -\frac{1}{4320}\lambda^5.
\end{array}
\right.
\eeq
As a result the scheme is $4-$order. In addition it is found to
be dissipative for the term in $\partial_x^{(6)}$ but antidissipative
with the terms in $\partial_x^{(8)}$ and  $\partial_x^{(10)}$.
All the calculations were performed with MAPLE.
%%%%%%%%%%%%%%%%%%%%%%%%%%%%%%%%%%%%%%%%%%%%%%%%%%%%%%%%%%%%%%%%%%%%%%%%%%%%%%%%%
\subsection{Tenth-order filter operator}
%%%%%%%%%%%%%%%%%%%%%%%%%%%%%%%%%%%%%%%%%%%%%%%%%%%%%%%%%%%%%%%%%%%%%%%%%%%%%%
\label{sec:4.5}
As previously mentionned a numerical dissipation in space is required 
to obtain a better stability profile.
Otherwise stated, the built-in numerical damping
of the RK4 time scheme
is not enough to ensure realistic 
In such a situation a high-order filtering is added at each time step. 
Numerical practice showed
that a tenth-order filter from \cite{Visbal-Gaitonde} gives good results.

For a one-dimensional grid function $u_j$,
The filter $\mathcal{F}$ belongs to the class 
of the filters
\beq
\label{eq:75.10.3}
\alpha_f u_{F,i-1}+
u_{F,i}+
\alpha_f u_{F,i+1}=
\sum_{^j=0}^J \frac{a_j}{2}(u_{i+j}+u_{i-j})
\eeq
This kind of filtering was in fact introduced
in the Atmospheric Sciences community \cite{Alpert}.
The values of the coefficients in (\ref{eq:75.10.3}) are given
in \cite{Visbal-Gaitonde}. 
Our numerical results in Section \ref{sec:4} were performed 
with the explicit filter, corresponding to
$\alpha_f=0$ and to the coefficients:
\beq
\label{eq:978.25.3}
\left(
\begin{array}{c}
a_0\\
a_1\\
a_2\\
a_3\\
a_4\\
a_5
\end{array}
\right)
=
\left(
\begin{array}{c}
193/256\\
105/256\\
-15/64\\
45/512\\
-5/256\\
1/512
\end{array}
\right).
\eeq
The fact that (\ref{eq:978.25.3}) acts as a numerical dissipation is reflected 
by the values of 
coefficients in (\ref{eq:72.10}) in the modified equation (\ref{eq:11.53}):
when $H^{n+1}$ is replaced by the filtered value
$\mathcal{F}(H^{n+1})$, then the term in $\partial_x^{(10)} u$ becomes dissipative instead of being
antidissitaive without filter. Its value is found as:
\beq
E_9= -\frac{1}{138240}\frac{32\lambda^6-135}{\lambda}
\eeq

At each time step the value $H^{(n+1)}$ in (\ref{eq:300.41-1}) is replaced 
by $\mathcal{F} H^{(n)}$ where $\mathcal{F}$ is the filtering operator
acting on the gridfunctions.

On the Cubed Sphere, we use a symmetric filter operator of the form
\begin{equation}
\mathcal{F}=\dfrac{1}{2} \left( \mathcal{F}_\xi \circ \mathcal{F}_{\eta} +  \mathcal{F}_{\eta} \circ \mathcal{F}_{\xi} \right)
\end{equation}
This has been found as essential to preserve 
the symmetry of the computations for large physical times.
%%%%%%%%%%%%%%%%%%%%%%%%%%%%%%%%%%%%%%%%%%%%%%%%%%%%%%%%%%%%%%%%%%%%%%%%%%%%%%%%
\section{Numerical results}
%%%%%%%%%%%%%%%%%%%%%%%%%%%%%%%%%%%%%%%%%%%%%%%%%%%%%%%%%%%%%%%%%%%%%%%%%%%%%%%%
\label{sec:4}
In this section, we present numerical results obtained for a familly of partial differential equation \eqref{eq:adv}, \eqref{eq:lswe} and \eqref{eq:swe} on the sphere. Firstly, with the advection equation, a known velocity is used such that an explicit solution is also known \cite{Nair-Machenhauer, Nair-Jablonowski}. Then, we present two tests for the linearized shallow water equation \eqref{eq:lswe} and numerical results for shallow water equation \eqref{eq:swe}. Most of this tests are introduced by \cite{Williamson-Drake-Hack-Jakob-Swarztrauber, Galewski-Scott-Polvani}.
When the analytic solution is known, we quantify the relative error (excepted for the damped case of (LSWE)) with the following formula :

\begin{equation}
\label{eq:error1}
I_p = \dfrac{\|\psi^n - \psi_{t^n} \|_p}{\| \psi_{t^n} \|_p}
\end{equation}

$p \in \lbrace 1, 2, \infty \rbrace$, $\psi_t^n$ is the analytic solution at time $t^n$, $\psi^n$ is the calculated solution at time $t^n$. 
$ \| . \| _p$ denote sur $\ell^p$ norm on the sphere. The numerical integration is in \cite{Croisille-12}.

\subsection{Advection equation}

Numerical results are obtained on \eqref{eq:adv}. This tests case are originates from \cite{Nair-Machenhauer, Nair-Jablonowski}. The velocity fields $(t,\mathbf{x}) \mapsto \mathbf{c} ( t,\mathbf{x})$ is prescribed as follow. The simplest problem of this kind is the Cosinus-bell advection problem \cite{Williamson-Drake-Hack-Jakob-Swarztrauber, Galewski-Scott-Polvani}. It correspond to a solid body velocity. Numerical results using the scheme (\ref{eq:300.41-1}) have been reported in  \cite{Croisille-10}. 
Here we consider two more challenging problems of the form (\ref{eq:978.23.1a}) involving transport and roll-up effects of vortices. 
The mathematical basis leading to the analytical solution is given in \cite{Nair-Machenhauer, Nair-Jablonowski}.

\subsubsection{Nair-Machenhauer test case}

\label{sec:4.1}
This test case \cite{Nair-Machenhauer}
considers
as initial condition $h(0,\mathbf{x})$
two vortices located
at diammetrally opposite points on the sphere.
These points are called $P$ (for North) and $P^\prime$ (for South).
The point $P$ has coordinates $(\lambda_P,\theta_P)$ in the reference
longitude-latitude system.
The longitude-latitude coordinate system 
$(\lambda^\prime,\theta^\prime)$ is associated to 
the axis $P  P^\prime$. The two vortices evolve in a roll-up
letting appear a finer and finer rollup structure when time increases.
We call $\mathbb{S}^2_R$ the sphere with radius $R$, 
\cite{Nair-Cote-Stanisforth,Nair-Machenhauer}.
Let us define the velocity $\mathbf{x} \in \mathbb{R}^2 \mapsto V(\mathbf{x})$ 
by
\begin{equation}
\left\{
\begin{array}{l}
\rho(\mathbf{x})=\rho_0 \cos(\theta^\prime)\\
V(\mathbf{x}) = v_0 \dfrac{3 \sqrt{3} }{2} \sech^2 ( \rho ) \tanh ( \rho )
\end{array}
\right.
\end{equation} 
The angular velocity $\theta^\prime \mapsto \omega_r(\theta^\prime)$ 
corresponding to $V$ 
is defined by:
\begin{equation}
   \omega_r ( \theta' ) = \left\{ 
   \begin{array}{ll}
      V/( R \rho ) & \text{ if } \rho \neq 0 \\
      0 & \text{ if} \rho =0
   \end{array}
   \right.
\label{vitesse_angulaire}
\end{equation}
Define now the tangential velocity on the sphere $\mathbb{S}^2_R$ 
appearing in (\ref{eq:978.23.1a}) by:
\begin{equation}
\mathbf{c}(\mathbf{x},t)=c_{\lambda^\prime} \mathbf{e}_{\lambda^\prime}+
c_{\theta^\prime} \mathbf{e}_{\theta^\prime}
\end{equation}
with
\begin{equation}
\left\{
\begin{array}{l}
c_{\lambda^\prime}=\cos(\theta^\prime) \omega_r(\theta^\prime)\\
c_{\theta^\prime}=0
\end{array}
\right.
\end{equation}

Switching back to the $(\mathbf{e}_\lambda,\mathbf{e}_\theta)$ basis, one obtains

\begin{equation}
\mathbf{c}(\mathbf{x},t)=\mathbf{c}(\mathbf{x})=c_{\lambda} \mathbf{e}_{\lambda}+
c_{\theta} \mathbf{e}_{\theta}
\end{equation}

with
\begin{equation}
\left\{
\begin{array}{l}
c_{\lambda} = R \omega_r ( \theta' ) \left[ \sin \theta_p \cos \theta - \cos \theta_p \cos ( \lambda - \lambda_p ) \sin \theta \right]\\
c_{\theta} = R \omega_r ( \theta' ) 
\left[ \cos \theta_p \sin ( \lambda - \lambda_p ) \right]
\label{eq:78.10}
\end{array}
\right.
\end{equation}

Now, the solution $(\mathbf{x}, t) \in \mathbb{S}_R^2 \times \mathbb{R}_{-}+ \mapsto
\phi(\mathbf{x},t)$ is given in coordinates $(\lambda^\prime,\theta^\prime,t)$ by

\begin{equation}
\phi ( \lambda^\prime , \theta', t ) 
= 
1 - \tanh \left[ \dfrac{\rho_0\cos(\theta^\prime)}{\gamma} \sin ( \lambda' 
- \omega_r(\theta^\prime) t ) \right],\;\;\; \rho(\theta^\prime)= 
\rho_0 \cos(\theta^\prime)
\label{NM_exacte}
\end{equation}

The constant $\gamma$ determines the strenght of $\nabla_T\phi$ and $\rho_0>0$ is a reference distance to the center of the vortex.
Let $T>0$ be the physical time of evolution and $v_0 = 2 \pi R / T$ ($R$=radius) be a reference velocity. 
In this article, for a smooth flow, we choose the parameters $\rho_0 = 3$ and $ \gamma = 5$.

In Figures \ref{erreur_cfl=0.05} and \ref{erreur_cfl=0.5}, the error history is reported for CFL numbers $\CFL=0.05$ and $\CFL=0.5$.
The Cubed Sphere grid has a parameter $N=35$, which corresponds to an equatorial resolution $\Delta \lambda = 2.6 \deg$. 
This is a spatial resolution similar to the one in \cite{Nair-Jablonowski} where a Discontinuous Galerkin (DG) scheme is used.
As can be observed, the error growth is regular.
The error behaviour is similar for the angles $\alpha=0$ and $\alpha=45\deg$, which shows that there is no apparent influence of the corners of the Cubed Sphere.

We have performed two different runs. The first corresponds to a $\CFL=0.05$. In this case, the error is dominated by the space approximation.
The error levels that are reached are of the same order of magnitude than the ones obtained with the DG scheme in \cite{Nair-Jablonowski}.
In the second case, we use $\CFL=0.5$.  The scheme remains stable.
The error consists of the combination of the space and time accuracy.
The error is slightly smaller than with $\CFL=0.05$. This is a standard behaviour for convection problems.

Table \ref{table:2.4} reports the convergence rate 
in the three norms $1$, $2$ and $\sup$. It can be observed
that the error is of order greater or equal than $4$. 

\begin{figure}[ht!]
\includegraphics[scale=0.3]{ref_7367656360_normerreur_test_1.png}
\includegraphics[scale=0.3]{ref_7367665245_normerreur_test_1.png}
\caption{Error plots with $N=35$; $\CFL=0.05$. Left panel: 
The point $P$ defining the axis has spherical coordinates  $(\lambda_P,  \theta_P) = (\pi / 4, \pi / 4)$. and $(\lambda_P, \theta_P) = (0,0)$ (right) for the Nair and Machenhauer test case.}
\label{erreur_cfl=0.05}
\end{figure}

\begin{figure}[ht!]
\includegraphics[scale=0.3]{ref_7367656531_normerreur_test_1.png}
\includegraphics[scale=0.3]{ref_7367656543_normerreur_test_1.png}
\caption{Error curves $N=35$; $cfl=0.5$; $(\lambda_P,  \theta_P) = (\pi / 4, \pi / 4)$ (left) and $(\lambda_P, \theta_P) = (0,0)$ (right) for the Nair and Machenhauer test case.}
\label{erreur_cfl=0.5}
\end{figure}

\begin{table}[ht!]
\begin{tabular}{|c||cc|cc|cc|}
\hline
$N$ & $max_n |e_1^n|$ & order  & $max_n |e_2^n|$ & order  & $max_n |e_{\infty}^n|$ & order \\
\hline
\hline
$40\;(2.25\deg)$ & $1.989 (-3)$ & -  & $7.255 (-3)$ & - & $4.039(-2)$  & - \\
\hline 
$50\;(1.80\deg)$ & $7.638 (-4)$ & $4.3852$ & $3.161(-3)$ & $3.8066$ & $1.918 (-2)$ & $3.4122$ \\
\hline
$60\;(1.50\deg)$ & $3.023(-4)$ & $5.1767$ & $1.313 (-3)$ & $4.9069$ & $7.556 (-3)$ & $5.2026$ \\
\hline
$80\;(1.125\deg)$ & $5.2979 (-5)$ & $6.1413$ & $2.391(-4)$ & $6.0061$ & $1.561(-3)$ & $5.5612$ \\
\hline
$100\;(0.90\deg)$ & $1.5036(-5)$ & $5.7073$ & $6.4568(-5)$ & $5.9326$ & $4.329(-4)$ & $5.8121$\\
\hline
$150\;(0.60\deg)$ & $1.9244(-6)$ & $5.1120$ & $9.2082(-6)$ & $4.8429$ & $7.6848(-5)$ & $4.2985$\\
\hline
\end{tabular}
\caption{Convergence analysis for the Nair and Machenhauer test case \cite{Nair-Machenhauer}. 
$N=31$; $\CFL = 0.7$; $(\lambda_p, \theta_p) = (0,0)$.}
\label{table:2.4}
\end{table}

\subsubsection{Nair-Jablonowski test case}

\label{sec:4.2}
In \cite{Nair-Jablonowski} a modification of the stationary vortex problem of Section \ref{sec:4.1} was suggested. It combines 
the deformational roll-up effect of the 
preceding case
with a solid body rotation.
The analytical solution is given in \cite{Nair-Jablonowski}.
The advection velocity $\mathbf{c}(\mathbf{x},t)$ in (\ref{eq:978.23.1a}) is obtained as the sum

\begin{equation}
\mathbf{c}=\mathbf{c}_s +\mathbf{c}_r
\end{equation}

where $\mathbf{c}_s$ is a solid rotation velocity and 
$\mathbf{c}_r$ is a "static" velocity centered at the center 
of the vortex. The velocity $\mathbf{c}_r$ is actually time dependant 
since in \eqref{eq:78.10}

$(\lambda_P, \theta_P)$ must be remplaced by the solid body advected position given  by 
\begin{equation}
(\lambda_s', \theta_s') = (\lambda_0' + w_s t, \theta_0')
\end{equation}
where $(\lambda_0', \theta_0')$ is the initial position of the vortex.
On the other hand, the solid-body velocity is given by

\begin{equation}
c_{\lambda, r} = R \omega_s \left( \sin \theta_p \cos \theta - \cos \theta_p \cos ( \lambda - \lambda_p ) \sin \theta \right)
\label{vitesse_lambda_bump}
\end{equation}
\begin{equation}
c_{\theta, r} = - R \omega_s \cos \theta_p \sin ( \lambda - \lambda_p )
\label{vitesse_theta_bump}
\end{equation}

where $\omega_s = v_0 / R = 2 \pi / T $ and $( \lambda_p, \theta_p$ ) is the coordinates of the point $P$.

The error growth is reported on Fig. \ref{erreur_cfl=0.05a}
and \ref{erreur_cfl=0.5a}. The magnitude of the error
is very close from the stationary case. However it appears 
slightly less regular. Table \ref{table:2} reports
as before fourth-order accuracy. Note also that the level
of error are very close to the ones reported in
\cite{Nair-Jablonowski} with a DG scheme. Finally
Fig. \ref{coupe-NJ-1} displays a slice of the vortex after 12 days
withgrid sizes $N=30$ and $N=60$. 
The matching with the finest grid
is excellent. No dispersion or dispersion is obervable.



\begin{figure}[ht!]
\includegraphics[scale=0.3]{ref_7367657139_snapshot_test_2_nday_0.png}
\includegraphics[scale=0.3]{ref_7367657143_snapshot_test_2_nday_3.png}

\includegraphics[scale=0.3]{ref_7367657147_snapshot_test_2_nday_6.png}
\includegraphics[scale=0.3]{ref_7367657152_snapshot_test_2_nday_9.png}

\includegraphics[scale=0.3]{ref_7367657157_snapshot_test_2_nday_12.png}
\caption{Nair and Jablonowski test-case. Approximate solution of the vortex after 
0, 3, 6, 9 and 12 days. The resolution is $N=31$. Numerical parameters are 
$N=31$, $\CFL = 0.7$ and $\alpha = 3 \pi / 4$.}
\label{SNAPSHOT_NJ}
\end{figure}

\begin{figure}[ht!]
\includegraphics[scale=0.3]{ref_7367657290_normerreur_test_2.png}
\includegraphics[scale=0.3]{ref_7367657345_normerreur_test_2.png}
\caption{Error curves $N=35$; $\CFL=0.05$; $\alpha = \pi / 4$ (left) et $\alpha = 0$ (right) for the Nair and Jablonowski test case \cite{Nair-Jablonowski}.}
\label{erreur_cfl=0.05a}
\end{figure}

\begin{figure}[ht!]
\includegraphics[scale=0.3]{ref_7367657356_normerreur_test_2.png}
\includegraphics[scale=0.3]{ref_7367657366_normerreur_test_2.png}
\caption{Error curves $N=35$; $\CFL=0.5$; $\alpha = \pi / 4$ (left) et $\alpha = 0$ (right) for the Nair and Jablonowski test case \cite{Nair-Jablonowski}.}
\label{erreur_cfl=0.5a}
\end{figure}

\begin{table}[ht!]
\begin{tabular}{|c||cc|cc|cc|}
\hline
$N$ & $max_n |e_1^n|$ & ordre  & $max_n |e_2^n|$ & ordre  & $max_n |e_{\infty}^n|$ & ordre \\
\hline
\hline
$40$ & $2.2199 (-3)$ & -  & $8.1592 (-3)$ & - & $4.9298 (-2)$  & - \\
\hline 
$50$ & $9.9676 (-4)$ & $3.6687$ & $3.9476(-3)$ & $3.3266$ & $2.8948(-2)$ & $2.4393$ \\
\hline
$60$ & $4.4566(-4)$ & $4.4957$ & $1.9545(-3)$ & $3.9262$ & $1.6474(-2)$ & $3.1484$ \\
\hline
$80$ & $1.3189(-4) $ & $4.2937$ & $5.8682(-4)$ & $4.2429$ & $5.6752(-3)$ & $3.7580$ \\
\hline
$100$ & $5.3380(-5)$ & $4.0990$ & $2.3789 (-4)$ & $4.0916$ & $2.3177(-3)$ & $4.0582$\\
\hline
$150$ & $1.0401 (-5)$ & $4.0669$ & $4.7173 (-5)$ & $4.0232$ & $4.9192(-4)$ & $3.8542$\\
\hline
\end{tabular}
\caption{Convergence analysis for the Nair and Jablonowski test case \cite{Nair-Jablonowski} ; $cfl = 0.7$ ; $\alpha = \pi /4$.}
\label{table:2}
\end{table}

\begin{figure}[ht!]
\includegraphics[scale=0.5]{ref_7363158648_coupefaceI_equateur_test_2.jpg}
\caption{Nair and Jablonowski test case \cite{Nair-Jablonowski}. Slice 
of the vortex after $12$ days. Solid line: exact solution, circles:
approximate solution with $N=30$. Crosses: approximate solution with $N=60$}
\label{coupe-NJ-1}
\end{figure}

\subsection{Linearized shallow water equation}

The equation (LSWE) \eqref{eq:lswe} is obtained as a perturbation of  \eqref{eq:swe} around of $(0,H)$ an equilibrium. This equation is an intermediate between advection equation and shallow water equation.
Using our numerical scheme, it permit to highlight the importance of using a symetric filter instead of:
\begin{equation}
\mathcal{F} = \mathcal{F}_{\xi} \circ \mathcal{F}_{\eta}
\end{equation}
because dissymetry between panels can create instabillities.

We present two test cases. The first is a damped case of LSWE, the other is a time independant zonal solution.

\subsubsection{A damped case of LSWE}

This test serves to assess the accuracy of the 
gradient and divergence approximation () and () when
used in the LSWE system (\ref{eq:lswe}). 
Consider the two exponentially in time damped functions
\begin{equation}
\left\lbrace
\begin{array}{rcl}
\tilde{\mathbf{v}} (t,\mathbf{x}) & = & \mathbf{u}_0 \varphi(\theta) e^{-\sigma t}\mathbf{e}_\lambda(\mathbf{x})\\
\tilde{\eta}(t,\mathbf{x}) & = & \eta_0 \varphi(\theta) \sin(\lambda)e^{-\sigma t}\\
\end{array}
\right.
\end{equation}

The system to solve is
\begin{equation}
\label{eq:lswe_damped}
(LSWE) \left\{
\begin{array}{l}
\dfrac{\partial \mathbf{v}}{\partial t} (t,\mathbf{x})+ \mathbf{g} \nabla_T \eta(t,\mathbf{x}) + f(\mathbf{x}) \mathbf{k}(\mathbf{x}) \times
\mathbf{v}(t,\mathbf{x})=S_{\eta}(t,\mathbf{x})\\
\dfrac{\partial \eta}{\partial t}(t,\mathbf{x})+ H \nabla_T . \mathbf{v}(t,\mathbf{x})=S_{\mathbf{v}}(t,\mathbf{x})
\end{array}
\right.
\end{equation}

In \eqref{eq:lswe_damped}, $\mathbf{g}$ is the gravity vector and $\mathbf{k}(\mathbf{x})$ is the exterior
normal vector.
The source terms $S_{\eta}$ and $S_{\mathbf{v}}$ are defined 
such as the functions $(\tilde{\mathbf{v}}(t,\mathbf{x}), \tilde \eta(t,\mathbf{x}))$ be solution
of (\ref{eq:lswe_damped}).
\textbf{Remarks:\\
- Gradient and divergence are calculated according to ().\\
- Filtrage ou non ?
- Valeurs numériques de $H,g, u_0, \eta_0$ ?
- Autres essais  avec temps plus long ?}

A numerical grid convergence analysis is reported in 
Table \ref{table:4}.

\begin{figure}[ht!]
\begin{tabular}{|c||c|c|c|c|c|}
\hline
& N=40 & rate & N=60  & rate & N=80 \\
\hline 
\hline 
$\vert \mathbf{v}^1_{ex}(T)-\mathbf{v}^1_{cal}(T)\vert_{h,I}$ & 2.52(-5)  &  4.15 & 4.69(-6) &  2.98 & 1.40(-6)   \\
\hline 
$\vert \eta_{ex}(T)-\eta_{cal}(T)\vert_{h,I}$ & 1.06(-5)  &  3.85 & 2.70(-6) &  3.32 & 1.04(-6) \\
\hline 
\end{tabular}
\caption{Hermitian scheme applied to an exponential decaying solution of the LSWE. Final time = 1h30}
\label{table:4}
\end{figure}

\textbf{Tableau a revoir}

A convergence rate between $3$ and $4$ can be observed.

\subsubsection{A time independent zonal solution of LSWE}

In this test case, we consider a time independent 
solution of LSWE depending on the latitude only.
Consider a tangential velocity field of the form. 
If the parameter function $\theta \mapsto \varphi(\theta)=\exp(\frac{1}{(\theta-\theta_0)(\theta-\theta_1)})$, then
we define the 
spherical velocity $\mathbf{v}(\mathbf{x})$ on the sphere by:
\begin{equation}
\mathbf{v}(\mathbf{x})=u_0 \varphi(\theta) \mathbf{e}_\lambda(\mathbf{x})
\end{equation}

The momentum equation () is equivalent to the 
following relation:

\begin{equation}
\eta(\mathbf{x})=\eta_{eq}-\frac{a}{g}\int_0^\theta f(s) \varphi(s) ds
\end{equation}

Our numerical test consists now in testing if the scheme preserve during time stepping 
the time independant solution defined by .....
This is a zonal divergence free solution
of the LSWE depending on the latitude $\theta$ only.
This test case is meaningful, 
first to assess the accuracy of the spatial approximation. In particular, 
spurious modes can pollute the numerical solution.
Second this test allows to test the accuracy of the numerical divergence 
preserving on 
large intervals of time.
The numerical results are reported in Table \ref{table:5}.

\begin{figure}[ht!]
\begin{tabular}{|c|c|c|c|c|c|}
\hline 
&N=40 & rate & N=60  & rate & N=80 \\
\hline
\hline
$\vert \mathbf{v}^1_{ex}(T)-\mathbf{v}^1_{cal}(T)\vert_{h,I}$ & 2.73(-5)  &  4.09 & 5.18(-6) &  4.24 & 1.53(-6) \\
\hline 
$\vert \eta_{ex}(T)-\eta_{cal}(T)\vert_{h,I}$ & 1.21(-5)  &  3.95 & 2.43(-6) &  3.14 & 7.38(-7) \\
\hline 
\end{tabular}
\caption
{Hermitian scheme applied to a time independent solution of the LSWE. Final time = 1h30}
\label{table:5}
\end{figure}

\subsection{Shallow water equation}

The shallow water equation \eqref{eq:swe} is solved with the previsous scheme. Numerical results are obtained on the test cases extracted from \cite{Williamson-Drake-Hack-Jakob-Swarztrauber, Galewski-Scott-Polvani}.

The following parameters are:
\begin{equation}
\begin{array}{rcll}
a & = & 6.37122 \times 10^6m & \text{ is the Earth radius,}\\
\Omega & = & 7.292 \times 10^{-5} s^{-1}& \text{ is the rotational velocity,}\\
g & = & 9.80616 m \cdot s^{-2}& \text{ is the gravity constant}\\
f & = & 2 \Omega sin \theta & \text{ exept for test 2 of Williamson and al. : global steady state.}
\end{array}
\end{equation}

Unless specification, $h^{\star} = h$.

The following invariant should be present:
\begin{itemize}
\item mass : $I_1=\gint_{\mathbb{S}_a^2} h^{\star}d\mathbf{s}$ \\

\item energy : $I_2 = \gint_{\mathbb{S}_a^2} \frac{1}{2} h^{\star} \mathbf{v}^2 + \frac{1}{2}g \left( h^2 - h_s^2 \right) d\mathbf{s}$ \\

\item potential enstrophy : $I_{3}=\gint_{\mathbb{S}_a^2} \dfrac{\left( \zeta + f \right)^2}{2 h^{\star}} d\mathbf{s}$ with $\zeta$ the vorticity\\
\end{itemize}
We measure the error on conservation for this quantities $q$ with the normalized integral :
\begin{equation}
\dfrac{\gint_{\mathbb{S}_a^2} \left( q(t,\mathbf{s})-q(0,\mathbf{s}) \right) d \mathbf{s}}{\gint_{\mathbb{S}_a^2} q(0,\mathbf{s})d \mathbf{s}}
\end{equation}
Integral of divergence and vorticity on the sphere are theorically nul, so we can't use the normalized integral to measure error. Then, the error on conservation for this quantities $p$ are measured with:
\begin{equation}
\gint_{\mathbb{S}_a^2} p(t,\mathbf{s})d \mathbf{s}
\end{equation}
with $p = \nabla_T \cdot \mathbf{v}$ or $p = \left( \nabla_T \wedge \mathbf{v}  \right) \cdot \mathbf{k}$.

\subsubsection{Test case 2 of Williamson and al. : Global steady state}

The test 2 of \cite{Williamson-Drake-Hack-Jakob-Swarztrauber} is a steady state solution to the non-linear shallow water equation \eqref{eq:swe}. It's a zonal solution around an axes rotated with an angle $\alpha$ (like for the Nair-Jablonowski test case). We test differents values $\alpha = 0$ or $\alpha = \pi/4$ to test the influence of corners on the cubed-sphere grid.
The Coriolis parameter, with influence of the rotation angle $\alpha$, is given by:
\begin{equation}
f=2 \Omega \left( - cos \lambda cos \theta sin \alpha + sin \theta cos \alpha \right)
\end{equation}

The analytical solution is :
\begin{equation}
\mathbf{v} = u \mathbf{e}_{\lambda} + v \mathbf{e}_{\theta}
\end{equation}
with :
\begin{equation}
\left\lbrace
\begin{array}{rcl}
u & = & u_0 \left( cos \theta cos \alpha + cos \lambda sin \theta sin \alpha \right)\\
v & = & -u_0 sin \lambda sin \alpha
\end{array}
\label{eq:W2 velocity}
\right.
\end{equation}

and the height $h$ :
\begin{equation}
h=h_0- \dfrac{1}{g} \left( a \Omega u_0 + \dfrac{u_0^2}{2} \right) \left( -cos \lambda cos \theta sin \alpha + sin \theta cos \alpha \right)^2 
\label{eq:W2 height}
\end{equation}

constants are :
\begin{equation}
\begin{array}{rcl}
gh_0 & = & 2.94 \times 10^4 m^2/s^2\\
u_0 & = & 2 \pi R / (12 \text{days}) \\
\end{array}
\end{equation}

Results with $\alpha=\pi/4$ are given in figure \ref{fig:W2 alpha=pi/4}. In table \ref{tab:W2 error order} we can observe the order of convergence close to 4 in accordance with numerical results on each operator.

\begin{figure}[ht!]
\includegraphics[scale=0.3]{ref_7367706276_snapshot_solution.png}
\includegraphics[scale=0.3]{ref_7367706276_erreur.png}
\caption{Numerical results of steady state geostrophic flow in direction of $\alpha=\pi/4$ on  $32 \times 32 \times 6$ grid after $5$ model days of simulation. Contour lines are plotted between 1150 m to 2950m with step of 200m (left). Relative error (right) }
\label{fig:W2 alpha=pi/4}
\end{figure}

The conservation relation are given by the figure \ref{fig:W2 conservation alpha=pi/4}. The conservation is fairly satisfactory but this test is independant of time. We need to obtain others results for unsteady state test case. 

\begin{figure}[ht!]
\includegraphics[scale=0.3]{ref_7367706276_mass.png}
\includegraphics[scale=0.3]{ref_7367706276_energy.png}\\
\includegraphics[scale=0.3]{ref_7367706276_enstrophy.png}
\includegraphics[scale=0.3]{ref_7367706276_conservationBdiv.png}\\
\includegraphics[scale=0.3]{ref_7367706276_conservationBvort.png}
\caption{Numerical conservation of steady state geostrophic flow in direction of $\alpha=\pi/4$ on  $32 \times 32 \times 6$ grid after $5$ model days of simulation.}
\label{fig:W2 conservation alpha=pi/4}
\end{figure}

\begin{table}[ht!]
\begin{tabular}{|c|c||cc|cc|cc|}
\hline
$\alpha$ & grid & $max_n |e_1^n |$ & order & $max_n |e_2^n |$ & order &  $max_n |e_{\infty}^n |$ & order \\
\hline
\hline
              & $6 \times 32 \times 32$   & $1.1422 (-6)$ & - & $1.3885 (-6)$ & - & $2.4469 (-6)$ & - \\
$\alpha = 0$  & $6 \times 64 \times 64$   & $7.1216 (-8)$ & $4.0035$ & $8.6513 (-8)$ & $4.0045$ & $1.5229 (-7)$ & $4.0061$ \\
              & $6 \times 128 \times 128$ & $4.4469 (-9)$ & $3.9942$ & $5.4018 (-9)$ & $4.0014$ & $9.5186 (-9)$ & $3.9999$ \\
\hline
\hline
                & $6 \times 32 \times 32$   & $7.5712 (-7)$ & - & $1.0446 (-6)$ & - & $2.7809 (-6)$ & - \\
$\alpha = \pi/4$& $6 \times 64 \times 64$   & $4.7213 (-8)$ & $4.0032$ & $6.5124 (-8)$ & $4.0036$ & $1.7387 (-7)$ & $3.9995$ \\
                & $6 \times 128 \times 128$ & $2.9487 (-9)$ & $4.0010$ & $4.0672 (-9)$ & $4.0011$ & $1.0858 (-8)$ & $4.0012$ \\
\hline
\end{tabular}
\caption{Relative error afer 5 days model simulation of steady state geostrophic flow. The CFL condition is $0.9$.}
\label{tab:W2 error order}
\end{table}

\subsubsection{Test case 5 of Williamson and al. : Isolated mountain}

The test case 5 of \cite{Williamson-Drake-Hack-Jakob-Swarztrauber} is similar to the previous but with a topography. It permit to check the performance of the scheme with a topography and a time dependant solution.
Analyical solution is not available.

The initial solution is the same than the previsous \eqref{eq:W2 velocity} and \eqref{eq:W2 height} but with the following constant :
\begin{equation}
\begin{array}{rcl}
h_0 & = & 5960m \\
u_0 & = & 20 m/s \\
\alpha & = & 0 \\
\end{array}
\end{equation}

The bottom topography is an isolated conic mountain wih $h_{s_0}=2000m$ of height, then:
\begin{equation}
h_s = h_{s_0} \left( 1 - \dfrac{r}{r_0} \right)
\end{equation}
with $r=min \left( r_0, \sqrt{\left( \lambda - \lambda_c \right)^2 + \left( \theta - \theta_c \right)^2} \right)$. $r_0=\pi/9$ and $(\lambda_c, \theta_c)=(3 \pi /2, \pi /6)$ is the location of the mountain center. As for the previous, stationnary test case, the error on potential enstrophy conservation is greater than the energy and mass conservation error. But all are well satisfactory with relative error close to $10^{-5}$. 

Numerical results of the height $h$ at times $5$, $10$ and $15$ days are given in figure \ref{fig:W5 snapshot} with a grid $6 \times 32 \times 32$. Conservations results are in figure \ref{fig:W5 conservation}.

\begin{figure}[ht!]
\includegraphics[scale=0.3]{ref_7367706559_snapshot_intermediaire499.png}\\
\includegraphics[scale=0.3]{ref_7367706559_snapshot_intermediaire999.png}\\
\includegraphics[scale=0.3]{ref_7367706559_snapshot_intermediaire1499.png}
\caption{Numerical results of isolated mountain test case with grid  $32 \times 32 \times 6$ at time 5, 10 and 15 days. Contour line are plotted from 5050 m to 5950 m with interval of 50 m.}
\label{fig:W5 snapshot}
\end{figure}

\begin{figure}[ht!]
\includegraphics[scale=0.3]{ref_7367706559_conservationA.png}
\includegraphics[scale=0.3]{ref_7367706559_conservationB.png}
\caption{Conservation of isolated mountain test case with grid  $32 \times 32 \times 6$.}
\label{fig:W5 conservation}
\end{figure}

\subsubsection{Test case 6 of Williamson and al. : Rossby-Haurtwitz waves}

resuts ref : 7367707271

The Rossby-Haurwitz waves is the 6th test case of \cite{Williamson-Drake-Hack-Jakob-Swarztrauber}. Like in the previous test case, no analytic solution is available.
The initial velocity is $\mathbf{v}=u \cdot \mathbf{e}_{\lambda}+v \cdot \mathbf{e}_{\theta}$ with:
\begin{equation}
\left\lbrace
\begin{array}{rcl}
u & = & a \omega cos \theta + a K cos^{R-1} \theta \left( R sin^2 \theta - cos^2 \theta \right) cos R\lambda\\
v & = & -a K R cos^{R-1} \theta sin \theta sin R \lambda
\end{array}
\label{eq:W6 velocity}
\right.
\end{equation} 

the height $h$ is :
\begin{equation}
gh = gh_0 + a^2 A(\theta) + a^2 B(\theta) cos R \lambda + a^2 C(\theta) cos 2 R \lambda 
\label{eq:W6 height}
\end{equation}

Maps $A$, $B$ and $C$ are function of the latitude $\theta$:

\begin{equation}
\begin{array}{rcl}
A(\theta) & = & \dfrac{\omega}{2} \left( 2 \Omega + \omega \right) cos^2 \theta + \dfrac{1}{4} K^2 cos^{2R} \theta \left[ (R+1) cos^2 \theta+ (2R^2 -R -2) - 2R^2 cos^{-2} \theta \right]\\
B(\theta) & = & \dfrac{2 (\Omega +\omega) K }{(R+1)(R+2)} cos^R \theta \left[ (R^2 + 2R +2) - (R+1)^2 cos^2 \theta  \right] \\
C(\theta) & = & \dfrac{1}{4} K^2 cos^{2R} \theta \left[ (R+1) cos^2 \theta - (R+2) \right]
\end{array}
\end{equation}

and constants are:

\begin{equation}
\begin{array}{rcl}
\omega & = & 7.848 \times 10^{-6} s^{-1}\\
K & = & 7.848 \times 10^{-6} s^{-1}\\
h_0 & = & 8 \times 10^3 m \\
R & = & 4
\end{array}
\end{equation} 

We plot numerical results after 7 and 14 days model simulation in figure \ref{fig:W6 snapshot}. Conservation results are in figure \ref{fig:W6 conservation}. Mass and Energy are close to be conservative but enstrophy is greater. Potential enstrophy is difficult to be conservative. Divergence is close to $0$ and vorticity is little and close to be a constant. 

\begin{figure}[ht!]
\includegraphics[scale=0.3]{ref_7367723128_snapshot_intermediaire699.png}
\includegraphics[scale=0.3]{ref_7367723128_snapshot_intermediaire1399.png}\\
\caption{Numerical results of Rossby-Haurwitz wave test case with grid  $128 \times 128 \times 6$ at time 7 and 14 days. Contour line are plotted from 8100 m to 10500 m with interval of 100 m.}
\label{fig:W6 snapshot}
\end{figure}

\begin{figure}[ht!]
\includegraphics[scale=0.3]{ref_7367723128_conservationB.png}
\includegraphics[scale=0.3]{ref_7367723128_mass.png}\\
\includegraphics[scale=0.3]{ref_7367723128_energy.png}
\includegraphics[scale=0.3]{ref_7367723128_enstrophy.png}
\caption{Conservation of Rossby-Haurwitz wave test case with grid  $128 \times 128 \times 6$.}
\label{fig:W6 conservation}
\end{figure}

\subsubsection{Barotropic instability}

The barotropic instability test, introduced in \cite{Galewski-Scott-Polvani}, is a perturbation of a steady state zonal solution.

The zonal steady state is given by :

\begin{equation}
h(\lambda, \theta) = h_0 - \dfrac{1}{g} \gint_{-\pi/2}^{\theta} a u_{\lambda}(\tau) \left( f + \dfrac{tan(\tau)}{a}u_{\lambda}(\tau) \right) d\tau
\label{eq:G1 steady}
\end{equation}

where $h_0$ is such that the global mean layer is equal to $10 000$ meters and $u_{\lambda}$ is given by the formula:

\begin{equation}
u_{\lambda}(\tau) = \left\lbrace
\begin{array}{ll}
0 & \text{ if } \tau \leq \theta_0 \\
\dfrac{u_{max}}{e_n} exp \left[ \dfrac{1}{(\tau - \theta_0)(\tau - \theta_1)} \right] & \text{ if } \theta_0 \leq \tau \leq \theta_1 \\
0 & \text{ if } \theta_1 \leq \tau \\
\end{array}
\right.
\end{equation}

which constants:
\begin{equation}
\begin{array}{rcl}
u_{max} & = & 80 m/s \\
\theta_0 & = & \pi /7 \\
\theta_1 & = & \pi/2 - \theta_0 \\
e_{n} & = & exp \left[ \dfrac{-4}{(\theta_1 - \theta_0)^2} \right]\\
\end{array}
\end{equation}

the velocity is :
\begin{equation}
\mathbf{v} = u_{\lambda}(\theta) \mathbf{e}_{\lambda}
\label{eq:G1 velocity}
\end{equation}

This steady state is not stable, we perturbate it by adding $h'$ to $h$ :

\begin{equation}
h'(\lambda, \theta) = \hat{h} \cos (\theta) \exp \left[ - \left( \dfrac{\lambda}{\alpha} \right)^2 - \left( \dfrac{\theta_2 - \theta}{\beta} \right)^2 \right] 
\end{equation}

with $\hat{h} = 120$ meters, $\alpha = 1/3$, $\beta = 1/15$ and $\theta_2 = \pi/4$.

As we can see in figure \ref{fig:G1 snaphot}, it is necessary to use a sufficiant mesh to plot the vorticity efficiantly. 

\begin{figure}
\includegraphics[scale=0.5]{ref_7367708164_snapshot_intermediaire598.png}\\
\includegraphics[scale=0.5]{ref_7367708686_snapshot_intermediaire599.png}\\
\includegraphics[scale=0.5]{ref_7367709339_snapshot_intermediaire599.png}
\caption{Numerical results of the barotropic instability at day 6 with differents mesh. $32 \times 32 \times 6$ (top), $64 \times 64 \times 6$ (2nd panel) and $128 \times 128 \times 6$ (bottom). ADD RESULTS WITH N=95}
\label{fig:G1 snaphot}
\end{figure}

This test is very challenging for the Cubed-Sphere because the perturbation is  located at the border between to panels and $h$ change with a larger gradient at the border of the face V.

Conservation of quantities are good (see figure \ref{fig:G1 conservation}), we observe perturbation of this quantities with apparition of the instability close to day 4.

\begin{figure}
\includegraphics[scale=0.3]{ref_7367709339_conservationB.png}
\includegraphics[scale=0.3]{ref_7367709339_mass.png}\\
\includegraphics[scale=0.3]{ref_7367709339_energy.png}
\includegraphics[scale=0.3]{ref_7367709339_enstrophy.png}
\caption{Conservation of Barotropic instability test case with grid  $128 \times 128 \times 6$.}
\label{fig:G1 conservation}
\end{figure}
%%%%%%%%%%%%%%%%%%%%%%%%%%%%%%%%%%%%%%%%%%%%%%%%%%%%%%%%%%
%%%% %% Conclusion générale

Dans cette thèse, nous présentons un nouveau schéma aux différences finies pour la résolution d'équations aux dérivées partielles d'évolution sur la sphère.

Après semi-discrétisation en espace, un schéma explicite en temps de Runge-Kutta d'ordre 4 couplé à un opérateur de filtrage donne d'excellents résultats sur des équations classiques telles que l'équation de transport, l'équation des ondes et l'équation de Burgers. Malgré le caractère hyperbolique de ces équations, l'ajout de l'opérateur de filtrage nous permet de conserver des schémas centrés en espace sans que les ondes parasites deviennent néfastes pour le bon déroulement des algorithmes. Les opérateurs de filtrage permettent en effet d'améliorer la stabilité en ajoutant de la dissipation numérique au schéma. Ces derniers, n'affectent pas la conservation de la masse.

Le maillage utilisé est celui de la Cubed-Sphere. Il est construit à partir de sections de grands cercles. Cela permet de définir un schéma hermitien centré de nature périodique. On obtient ainsi une version discrète naturelle des opérateurs gradient, divergence et vorticité. Les opérateurs obtenus sont consistants au moins à l'ordre 3. Lors des tests numériques effectués, un ordre 4 ou supérieur est observé.

D'autres part, nous avons considéré des systèmes d'équations du type Shallow Water sphériques. Les tests effectués sur l'équation Shallow Water linéarisée et l'équation Shallow Water donnent des résultats comparables à ceux obtenus par des méthodes de Galerkin ou de volumes finis d'ordres élevés. Les niveaux d'erreurs sont très faibles. Bien que le schéma ne soit pas a priori conservatif, l'erreur de conservation est très faible. Pour la masse le comportement est satisfaisant. Pour l'énergie et l'enstrophie potentielle, les erreurs sont similaires à celles obtenues par d'autres méthodes y compris sur des tests difficiles tels que le test de la montagne isolée ou le cas tests barotropique avec instabilité. Le schéma aux différences finis considéré est centré et l'opérateur de filtrage utilisé affecte peu les calculs. Bien que les équations soient hyperboliques, ce schéma est suffisant sur les tests effectués.

Les perspectives de ce travail concernent des simulations en temps long. En effet, nous nous sommes restreints ici à des schémas explicites. Ces derniers imposent des restrictions importantes sur le pas de temps. Il serait intéressant de développer un schéma implicite de manière à pouvoir considérer des pas de temps plus grands et ainsi des simulations sur des temps plus longs. Une autre perspective est de considérer des méthodes de zoom de type "Local Defect Correction" de manière à obtenir une meilleure représentation de phénomènes locaux tels que les tourbillons. Au niveau équations, un objectif futur est de travailler avec les équations de Navier-Stokes en dimension 3 plus proche de la réalité physique.



\section{Conclusion}
%%%%%%%%%%%%%%%%%%%%%%%%%%%%%%%%%%%%%%%%%%%%%%%%%%%%%%%%%%%%%%%%%%%%%%%%%%
In this paper, we show that compact approximations in the
on the Cubed Sphere lead to accurate numerical simulations
for hyperbolic problems on the sphere.
The present approach makes use 
of the standard finite difference design.
Since the scheme is centered, some numerical viscosity
needs to be added for time dependent solutions.
In our case, a high order filter is added.
Fourth order accuracy was numerically observed.
The quality of the numerical results can be compared
to the best conservative schemes on irregular grids, such as Discontinuous Galerkin schemes.

Current research involves mathematical analysis
of the scheme, in particular conservation properties.
The same logic can be applied to any form
of the SW system on the sphere.
%%%%%%%%%%%%%%%%%%%%%%%%%%%%%%%%%%%%%%%%%%%%%%%%%%%%%
%%%%%%%%%%%%%%%%%%%%%%%%%%%%%%%%%%%%%%%%%%%%%%%%%%%
\bibliographystyle{plain}
\bibliography{../BIB/bibox_4apr11_no_issue}
\end{document}
