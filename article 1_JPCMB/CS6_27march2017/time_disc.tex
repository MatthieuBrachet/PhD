%% *** SECTION ***********************************************************************************************************************

\section{Centered compact scheme with filtering}
%%%%%%%%%%%%%%%%%%%%%%%%%%%%%%%%%%%%%%%%%%%%%%%%%%%%%%%%%%%%%%%%%%%%%%%%%%%%%%%%%%%%%%%%%%%
Our approximate scheme
uses a centered finite-difference approximation on the Cubed-Sphere.
In particular we do not use any kind of upwinding.
A stabilization mechanism is a linear filtering
at each time step. This kind of numerical 
algorithm is stronlgy related to so-called 
hyperviscous numerical diffusion.
%%%%%%%%%%%%%%%%%%%%%%%%%%%%%%%%%%%%%%%%%%%%%%%%%%%%%%%%%%%%%%%%%%%%%%%%%55
\subsection{Method of lines}
%%%%%%%%%%%%%%%%%%%%%%%%%%%%%%%%%%%%%%%%%%%%%%%%%%%%%%%%%%%%%%%%%%%%%%%%%%%%%%%%%%%%%%%%%%%
We consider the convection equation on the sphere
with tangential velocity 
$(\mathbf{x} ,t) \in \mathbb{S}^2 \times \mathbb{R}_+ \mapsto \mathbf{c}(\mathbf{x},t)$.
\beq
\label{eq:978.23.1}
\left\{
\begin{array}{l}
\partial_t h(\mathbf{x},t)+\mathbf{c}(\mathbf{x},t) \cdot \nabla_s h(\mathbf{x},t)=0,\\
h(\mathbf{x},0)=h_0(\mathbf{x}).
\end{array}
\right.
\eeq
Existence and uniqueness for (\ref{eq:978.23.1}) is
obtained by the method of characteristics.
The task is to calculate
an approximation of $h(\mathbf{x},t)$.

Let
$t \mapsto h^{(k)}_{i,j}(t)$ be a semidiscrete approximation
in space of (\ref{eq:978.23.1}). This semi-discrete 
approximation is 
the solution of the differential system:
\beq
\label{eq:978.23.1a}
\left\{
\begin{array}{l}
{d h_{i,j}^k(t) \over dt} +\mathbf{c}^{(k)}_{i,j}(t)\cdot  \mathbf{\nabla} h h_{i,j}^k(t)=0,
\quad -M\leq i,j\leq M,\;\;\; I\leq k\leq VI,\\
h^{(k)}_{i,j}(0)=h_0(\mathbf{s}^{(k)}_{i,j})
\end{array}
\right.
\eeq
where $\mathbf{c}^{(k)}_{i,j}(t) \triangleq \mathbf{c}(\mathbf{s}^{(k)}_{i,j})(t)$.
Denote by $H(t) \triangleq  h^{(k)}_{i,j}(t)$ the gridfuntion
with components $h^{(k)}_{i,j}$. The equation (\ref{eq:978.23.1}) 
is expressed in vector form as:
\beq
\label{eq:71.10.3}
\frac{d}{dt}H(t)=J(t) H(t)
\eeq
where $J(t)$ is the matrix corresponding to
\beq
[J(t) H(t)]_{i,j}^{(k)} \triangleq 
-\bc^{(k)}_{i,j}(t)\cdot  \bgrad_{T,h} h_{i,j}^k(t).
\eeq
Since fourth order accuracy in space is expected, we proceed with 
the explicit in time RK 4 approxaimtion in time for
(\ref{eq:71.10.3}).
Let $\Delta t>0$ be a time-step. The RK4 time-scheme
is applied to (\ref{eq:71.10.3}).

\begin{equation}
\label{eq:300.41-1}
\left\{ 
\begin{array}{l}
K^{(0)} = J(t^n)H^{n}\\
K^{(1)} = J(t^{n+1/2})(H^{n}+\frac{1}{2}\Delta t K^{(0)})\\
K^{(2)} = J(t^{n+1/2})(H^{n}+\frac{1}{2}\Delta t K^{(1)})\\
K^{(3)} = J(t^{n+1})(H^{n}+\Delta t K^{(2)})\\
H^{n+1} = H^{n}
+\Delta t\Bigg(\frac{1}{6}K^{(0)}+\frac{1}{3}K^{(1)}
+\frac{1}{3}K^{(2)}+\frac{1}{6}K^{(3)}\Bigg).
\end{array}\right.
\end{equation}
%%%%%%%%%%%%%%%%%%%%%%%%%%%%%%%%%%%%%%%%%%%%%%%%%%%%%%%%%%%%%%%%%%%%%%%%%%%%%
\subsection{Dissipation and dispersion analysis}
Finite difference schemes are usually described
by their dissipation and dispersion properties. 
A customary analysis of this kind is the so called {\sl modified equation}
analysis, \cite{Shokin}.  This analysis is usually 
performed on the linear advection equation
\beq
\partial_t u + c \partial_x u=0
\eeq
We recall next the main features of this analysis.
Consider a numerical scheme
\beq
\label{eq:76.10}
\frac{u^{n+1}_j-u^n_j}{\Delta t}
+ L_h u|^n_j =0
\eeq
In (\ref{eq:76.10}), the operator $L_h u^n_j$ approximates $-c \partial_x u$.
In operator form, the scheme (\ref{eq:76.10}) is rewritten as
\beq
\label{eq:56.10}
\frac{e^{\Delta t \partial_t}-1}{\Delta t} u^n_j = -L_h u|^n_j
\eeq
The Logarithm series provides the formal expansion 
\beq
\label{eq:87.12}
\partial_t = \frac{e^{\Delta t \partial_t}-1}{\Delta t}
-\frac{\Delta t}{2}\left(\frac{e^{\Delta t \partial_t}-1}{\Delta t}\right)^2
+\frac{\Delta t}{3}\left(\frac{e^{\Delta t \partial_t}-1}{\Delta t}\right)^3
-\frac{\Delta t}{4}\left(\frac{e^{\Delta t \partial_t}-1}{\Delta t}\right)^4
+....
\eeq
Using (\ref{eq:56.10}) in the right hand side of (\ref{eq:87.12}),
we obtain an identity of the form
\beq
\label{eq:34.13}
\partial_t u+ c \partial_x u=c \Big[
  E_1 h \partial_x^{(2)}u 
+ E_2 h^2 \partial_x^{(3)}u 
+ E_3 h^3 \partial_x^{(4)}u 
+ E_4 h^4 \partial_x^{(5)}u 
+...
\Big]
\eeq
This identity is called the modified equation of the scheme. It represents a transport equation
with a perturbation in the form of an asymptotic expansion
in powers of $h$. The coefficients $E_\alpha$ depends
only the Courant number $\lambda= c \Delta t /h$. 
In the case of the scheme (), the modifed equation 
is expressed as:
\beq
\partial_t u+ c \partial_x u=c\Big[ \frac{h^4}{360}(3 \lambda^4+2) \partial_x^{(5)}u
+ \sum_\alpha h^{\alpha}E_\alpha \partial_x^{(\alpha+1)}u
\eeq
The first coefficients are 
\beq
\label{eq:72.10}
\left\{
\begin{array}{l}
E_4= \frac{1}{360}(3 \lambda^4+2)\\
E_5= \frac{1}{144}\lambda^5\\
E_6= \frac{1}{3024}(-2+9\lambda^6)\\
E_7= \frac{1}{1152}\lambda^7\\
E_8= \frac{1}{25920}(\lambda^4-1)(5 \lambda^4-1)\\
E_9= -\frac{1}{4320}\lambda^5
\end{array}
\right.
\eeq
As a result the scheme is $4-$order. In addition it is found to
be dissipative for the term in $\partial_x^{(6)}$ but antidissipative
with the terms in $\partial_x^{(8)}$ and  $\partial_x^{(10)}$.
%%%%%%%%%%%%%%%%%%%%%%%%%%%%%%%%%%%%%%%%%%%%%%%%%%%%%%%%%%%%%%%%%%%%%%%%%%%%%%%%%
\subsection{Tenth-order hyperviscosity}
%%%%%%%%%%%%%%%%%%%%%%%%%%%%%%%%%%%%%%%%%%%%%%%%%%%%%%%%%%%%%%%%%%%%%%%%%%%%%%
\label{sec:4}
As described in the previous section, stabilisation mechanism could be useful
to obtain a better stability profile. 
In such a situation a high-order filtering is added at each time step. 
Numerical practice showed
that a tenth-order filter from \cite{Visbal-Gaitonde} gives good results.
At each time step the value $H^{(n)}$ in (\ref{eq:300.41-1}) is replaced 
by $\mathcal{F} H^{(n)}$ where $\mathcal{F}$ is the filtering operator
acting on the gridfunctions defined by 
the composition of two one-dimensional filters along the $\xi$ and the $\eta$ directions on each panel:
\begin{equation}
\mathcal{F}=\dfrac{1}{2} \left( \mathcal{F}_\xi \circ \mathcal{F}_{\eta} +  \mathcal{F}_{\eta} \circ \mathcal{F}_{\xi} \right)
\end{equation}

For a one-dimensional grid function $u_j$,
The filter $\mathcal{F}$ belongs to the class 
of the filters 
\beq
\label{eq:75.10.3}
\alpha_f u_{F,i-1}+
u_{F,i}+
\alpha_f u_{F,i+1}=
\sum_{^j=0}^J \frac{a_j}{2}(u_{i+j}+u_{i-j})
\eeq
This kind of filtering was originally introduced
in the Atmospheric Sciences community \cite{Alpert}.
The values of the coefficients in (\ref{eq:75.10}) are given
in \cite{Visbal-Gaitonde}. 
Our numerical results were performed 
with the explicit filter, corresponding to
$\alpha_f=0$ and to the coefficients:
\beq
\label{eq:978.25.3}
\left(
\begin{array}{c}
a_0\\
a_1\\
a_2\\
a_3\\
a_4\\
a_5
\end{array}
\right)
=
\left(
\begin{array}{c}
193/256\\
105/256\\
-15/64\\
45/512\\
-5/256\\
1/512
\end{array}
\right).
\eeq
The fact that () acts as a filter is reflected 
by the values of 
coefficients in (\ref{eq:72.10}) in the modified equation modified equation.
The term in $\partial_x^{(10)} u$ is now dissipative insted of being
antidissitaive without filter. It is now 
\beq
E_9= -\frac{1}{138240}\frac{32\lambda^6-135}{\lambda}
\eeq
All these results were obtained with MAPLE.
