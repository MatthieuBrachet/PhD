%% *** INTRODUCTION ***************************************************************************

\section{Introduction}
\label{sec:1}
In this paper, we continue the development
of the compact scheme approach introduced in \cite{} and \cite{}
for hyperbolic problems on the sphere.
In recent years a lot of effort
has been devoted to import
ideas of numerical gas dynamics to numerical climatology.
Two particular examples are finite volumes upwind schemes 
with numerical fluxes. This approach involves
reconstruction with piecewise cubic recontsruction.
A variant is the Discontinuous Galerkin approach
in which several unknowns in a single computational cells are used.
In each of these two cases, the machinery of upwind schemes is adapted
to the specificity of the spherical context
and of the test cases. This involves in particular 
slope limiting. 

An important challenge for numerical schemes in numerical climatology 
is to calculate as accurately as possible 
the solution of linear equations up to a large physical time.
Here we show that our compact scheme performs well
for the three following problems:
\begin{itemize}
\item 
The linear scalar equation
advection 
\begin{equation}
\label{eq:adv}
\dfrac{\partial h}{\partial t}  (t,\mathbf{x})+ \mathbf{c}(t,\mathbf{x}) \cdot \nabla_T h(t,\mathbf{x})=0
\end{equation}

The velocity $\mathbf{c}(t,\mathbf{x}) \in (0, +\infty) \times \mathbb{S}^2$ is prescribed
so that the scalar value $h(t,\mathbf{x})$ exhibits a moving rollup vortex.
This test case was introduced
in \cite{Nair-Machenhauer,Nair-Jablonowski}. We refer to Sections
\ref{sec:4.1} and \ref{sec:4.2} for setails.
\item
The linearized shallow water equations (LSWE). This equation
is expressed as:
\begin{equation}
\label{eq:lswe}
(LSWE) \left\{
\begin{array}{l}
\dfrac{\partial \mathbf{v} }{\partial t} (t,\mathbf{x})+ g \nabla_T \eta(t,\mathbf{x}) + f(\mathbf{x}) \mathbf{k}(\mathbf{x}) \wedge
\mathbf{v}(t,\mathbf{x})=0\\
\dfrac{\partial \eta}{\partial t} (t,\mathbf{x})+ H \nabla_T \cdot \mathbf{v}(t,\mathbf{x})=0
\end{array}
\right.
\end{equation}


This system serves as a base for spherical waves on the sphere,
\cite{Paldor}. The state at rest is 
$(\mathbf{v}_0,\eta_0)=(0,H)$ and the 
The 3-components perturbation unknowns is
the vector $(t,\mathbf{x}) \mapsto (\mathbf{v}(t,\mathbf{x}), \eta(t,\mathbf{x}))$.
\item The shallow water equation (SWE). We use the vectorial form as following :
\begin{equation}
\label{eq:swe}
(SWE) \left\lbrace
\begin{array}{rcl}
\dfrac{\partial h^{\star}}{\partial t} (t,\mathbf{x}) + \nabla_T \cdot \left( h^{\star}(t,\mathbf{x}) \mathbf{v}(t,\mathbf{x}) \right) & = & 0 \\
\dfrac{\partial \mathbf{v}}{\partial t}  (t,\mathbf{x}) + \nabla_T \left( \dfrac{1}{2} \mathbf{v}(t,\mathbf{x})^2 + g h(t,\mathbf{x}) \right) + \left( f(\mathbf{x}) + \zeta(t,\mathbf{x}) \right) \mathbf{k}(\mathbf{x}) \wedge \mathbf{v}(t,\mathbf{x}) & = & \mathbf{0} 
\end{array}
\right.
\end{equation}

where $\zeta = \left( \nabla_T \wedge \mathbf{v} \right) \cdot \mathbf{k}$ is the relative vorticity. $h^{\star} = h - h_s$ with $h_s$ the reliefs map.
\end{itemize}

The present approach is related to compact scheme
approach. In this sense, it belongs 
to a classical approach, which can
be traced back to early works
in approximation and interpolation theory, \cite{Collatz}.

Two specific applications in CFD where compact schemes
are  is Aeroacoustics, \cite{Visbal-Gaitonde, Tam-Webb} and Turbulence, \cite{Lele, Kim-Moin}.

Finite difference schemes 
for simulating problems in climatology have
actually attracted interest since 
more than 40 years, \cite{Arakawa}.

In both cases, we show 
that our purely Cartesian approach supports
the comparison with modern conservative schemes on unstructured grids, such as 
Discontinuous Galerkin or Finite-Volume schemes.
Note finally that Finite Difference methods were recently used
in numerical climatology in \cite{Ghader-Nordstrom}.

The outline of the paper is as follows.
In Section \ref{sec:2}, we recall the numerical calculation
of the gradient introduced in \cite{Croisille-10}. Then in Section 
\label{sec:3}, we present the numerical scheme with emphasis 
on the role of the filtering. Dissipation and dispersion analysis 
is given. Finally in Section \ref{sec:4}, we present 
numerical results on two vortex advection problems mentionned above.
These results show the accuracy of our scheme.
The accuracy is comparable to conservative schemes such as DG schemes \cite{Nair-Jablonowski}.