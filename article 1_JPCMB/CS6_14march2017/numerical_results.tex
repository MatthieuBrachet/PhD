%% *** SECTION ************************************************************************************************************************

\section{Numerical results}
%\section{Two vortex propagation test-cases with 
%deformational velocity}
In this section we present numerical results obtained
for a family of linear convection equations on the sphere 
with deformational velocity. This family
originates from \cite{Nair-Cote-Stanisforth, Nair-Machenhauer, Nair-Jablonowski}.
The equation is of the form (\ref{eq:978.23.1}).
The velocity $(\mathbf{x},t) \mapsto \mathbf{c}(\mathbf{c},t)$ is prescribed as follows.
The simplest problem of this kind is the Cosine-bell 
advection problem \cite{Swarztrauber-Williamson-Drake}. It corresponds
to a solid body velocity.
Numerical results for this problem
using the scheme (\ref{eq:300.41-1})
have been reported in 
\cite{Croisille-10}. 
Here we consider two more challenging problems of the form
(\ref{eq:978.23.1}) involving transport and roll-up effects
of vortices. 
The mathematical basis leading 
to the analytical solution is given in
\cite{Nair-Cote-Stanisforth}.
%%%%%%%%%%%%%%%%%%%%%%%%%%%%%%%%%%%%%%%%%%%%%%%%%%
%\subsection{Deformational stationary spherical vortex}
%%%%%%%%%%%%%%%%%%%%%%%%%%%%%%%%%%%%%%%%%%%%%%%%%%
%\label{sec:4.1}
\subsection{Nair-Machenhauer test case}
%%%%%%%%%%%%%%%%%%%%%%%%%%%%%%%%%%%%%%%%%%%%%%%%%%%%%%%%%5
\label{sec:4.1}
This test case \cite{Nair-Machenhauer}
considers
as initial condition $h(0,\mathbf{x})$
two vortices located
at diammetrally opposite points on the sphere.
These points are called $P$ (for North) and $P^\prime$ (for South).
The point $P$ has coordinates $(\lambda_P,\theta_P)$ in the reference
longitude-latitude system.
The longitude-latitude coordinate system 
$(\lambda^\prime,\theta^\prime)$ is associated to 
the axis $P  P^\prime$. The two vortices evolve in a roll-up
letting appear a finer and finer rollup structure when time increases.
We call $\Bbb S^2_R$ the sphere with radius $R$, 
\cite{Nair-Cote-Stanisforth,Nair-Machenhauer}.
Let us define the velocity $\mathbf{x} \in \mathbb{R}^2 \mapsto V(\mathbf{x})$ 
by
\begin{equation}
\left\{
\begin{array}{l}
\rho(\mathbf{x})=\rho_0 \cos(\theta^\prime)\\
V(\mathbf{x}) = v_0 \dfrac{3 \sqrt{3} }{2} \sech^2 ( \rho ) \tanh ( \rho )
\end{array}
\right.
\end{equation} 
The angular velocity $\theta^\prime \mapsto \omega_r(\theta^\prime)$ 
corresponding to $V$ 
is defined by:
\begin{equation}
   \omega_r ( \theta' ) = \left\{ 
   \begin{array}{ll}
      V/( R \rho ) & \text{ if } \rho \neq 0 \\
      0 & \text{ if} \rho =0
   \end{array}
   \right.
\label{vitesse_angulaire}
\end{equation}
Define now the tangential velocity on the sphere $\mathbb{S}^2_R$ 
appearing in (\ref{eq:978.23.1}) by
by:
\beq
\mathbf{c}(\mathbf{x},t)=c_{\lambda^\prime} \mathbf{e}_{\lambda^\prime}+
c_{\theta^\prime} \mathbf{e}_{\theta^\prime}
\eeq
with
\beq
\left\{
\begin{array}{l}
c_{\lambda^\prime}=\cos(\theta^\prime) \omega_r(\theta^\prime)\\
c_{\theta^\prime}=0
\end{array}
\right.
\eeq
Switching back to the $(\mathbf{e}_\lambda,\mathbf{e}_\theta)$ basis, one obtains
\beq
\mathbf{c}(\mathbf{x},t)=\mathbf{c}(\mathbf{x})=c_{\lambda} \mathbf{e}_{\lambda}+
c_{\theta} \mathbf{e}_{\theta}
\eeq
with
\begin{equation}
\left\{
\begin{array}{l}
c_{\lambda} = R \omega_r ( \theta' ) \left[ \sin \theta_p \cos \theta - \cos \theta_p \cos ( \lambda - \lambda_p ) \sin \theta \right]\\
c_{\theta} = R \omega_r ( \theta' ) 
\left[ \cos \theta_p \sin ( \lambda - \lambda_p ) \right]
\label{eq:78.10}
\end{array}
\right.
\end{equation}
Now, the solution $(\mathbf{x}, t) \in \mathbb{S}_R^2 \times \mathbb{R}_{-}+ \mapsto
\phi(\mathbf{x},t)$ is given in coordinates $(\lambda^\prime,\theta^\prime,t)$ by
\begin{equation}
\phi ( \lambda^\prime , \theta', t ) 
= 
1 - \tanh \left[ \dfrac{\rho_0\cos(\theta^\prime)}{\gamma} \sin ( \lambda' 
- \omega_r(\theta^\prime) t ) \right],\;\;\; \rho(\theta^\prime)= 
\rho_0 \cos(\theta^\prime)
\label{NM_exacte}
\end{equation}
The constant $\gamma$
determines the strenght of $\nabla_T\phi$ and $\rho_0>0$ 
is a reference distance to the center of the vortex.
Let $T>0$ be the physical time of evolution and $v_0 = 2 \pi R / T$ ($R$=radius)
be a reference velocity. 
In this article, for a smooth flow, we choose the parameters $\rho_0 = 3$ and $ \gamma = 5$.
%%%%%%%%%%%%%%%%%%%%%%%%%%%%%%%%%%%%%%%%%%%%%%5
%% \subsubsection{Numerical results}
%%%%%%%%%%%%%%%%%%%%%%%%%%%%%%%%%%%%%%%%%%%%%%%%
In Figures \ref{erreur_cfl=0.05} and \ref{erreur_cfl=0.5}, the error history
is reported for CFL numbers $\CFL=0.05$ and $\CFL=0.5$.
The Cubed Sphere grid has a parameter $N=35$, which corresponds 
to an equatorial resolution $\Delta \lambda = 2.6 \deg$. 
This is a spatial resolution similar to the one 
in \cite{Nair-Jablonowski}
where a Discontinuous Galerkin (DG) scheme is used.
As can be observed, the error growth is regular.
The error behaviour is similar for the angles 
$\alpha=0$ and $\alpha=45\deg$, which shows 
that there is no apparent influence
of the corners of the Cubed Sphere.

We have performed two different runs. The first corresponds
to a $\CFL=0.05$. In this case, 
the error is dominated by the 
space approximation.
The error levels that are reached are of the same order of magnitude 
than the ones obtained with the DG scheme in \cite{Nair-Jablonowski}.
In the second case, we use $\CFL=0.5$.  The scheme remains stable.
The error consists of the combination
of the space and time accuracy.
The error is slightly smaller than with $\CFL=0.05$. 
This is a standard behaviour for convection problems.
%corresponds to
%that numerical scheme oftenly exhiblits some kind of enhanced convergence 
%when the $\CFL$ number is closer from $1$, in the case of convection problems.
Table \ref{table:2.4} reports the convergence rate 
in the three norms $1$, $2$ and $\sup$. It can be observed
that the error is of order $4$ for the norms $1$ and $2$. It
is a slightly less accurate for the $\infty$ norm with 
an observed rate of $3.77$. 
%%%%%%%%%%%%%%%%%%%%%%%%%%%%%%%%%%%%%%%%%%%%%%%%%%%%%%%%%%%%%%%%%%%%%%%%%%%%%%%%
\begin{figure}[!ht]
\includegraphics[scale=0.3]{ref_7367656360_normerreur_test_1.png}
\includegraphics[scale=0.3]{ref_7367665245_normerreur_test_1.png}
\label{erreur_cfl=0.05}
\caption{Error plots with $N=35$; $\CFL=0.05$. Left panel: 
The point $P$ defining the axis has spherical coordinates  $(\lambda_P,  \theta_P) = (\pi / 4, \pi / 4)$. and $(\lambda_P, \theta_P) = (0,0)$ (right) for the Nair and Machenhauer test case.}
\end{figure}
%%%%%%%%%%%%%%%%%%%%%%%%%%%%%%%%%%%%%%%%%%%%%%%%%%%%%%%%%%%%%%%%%%%%%%%%%%%%%%%
\begin{figure}[!ht]
\includegraphics[scale=0.3]{ref_7367656531_normerreur_test_1.png}
\includegraphics[scale=0.3]{ref_7367656543_normerreur_test_1.png}
\label{erreur_cfl=0.5}
\caption{Error curves $N=35$; $cfl=0.5$; $(\lambda_P,  \theta_P) = (\pi / 4, \pi / 4)$ (left) and $(\lambda_P, \theta_P) = (0,0)$ (right) for the Nair and Machenhauer test case.}
\end{figure}
%%%%%%%%%%%%%%%%%%%%%%%%%%%%%%%%%%%%%%%%%%%%%%%%%%%%%%%%%%%%%%%%%%%%%%%%%%%%%%
\begin{figure}
\label{table:2.4}
\begin{tabular}{c||cc|cc|cc}
$N$ & $max_n |e_1^n|$ & order  & $max_n |e_2^n|$ & order  & $max_n |e_{\infty}^n|$ & order \\
\hline
\hline
$40\;(2.25\deg)$ & $0.1989 (-2)$ & -  & $0.7255 (-2)$ & - & $0.4039(-1)$  & - \\
\hline 
$50\;(1.80\deg)$ & $0.7638 (-3)$ & $4.2891$ & $0.3161(-3)$ & $3.7232$ & $0.1918 (-1)$ & $3.7108$ \\
\hline
$60\;(1.50\deg)$ & $0.3023(-3)$ & $5.2256$ & $0.1313 (-2)$ & $4.8188$ & $0.7556 (-2)$ & $5.1092$ \\
\hline
$80\;(1.125\deg)$ & $5.2979 (-5)$ & $6.0537$ & $0.2391(-3)$ & $5.9204$ & $0.1561(-2)$ & $5.4818$ \\
\hline
$100\;(0.90\deg)$ & $1.5036(-5)$ & $5.6441$ & $6.4568(-5)$ & $5.8669$ & $0.4329(-3)$ & $5.7478$\\
\hline
$150\;(0.60\deg)$ & $1.9244(-6)$ & $5.0703$ & $9.2082(-6)$ & $4.8034$ & $7.6848(-5)$ & $4.2634$
\end{tabular}
\caption{Convergence analysis for the Nair and Machenhauer test case \cite{Nair-Machenhauer}. 
$N=31$; $\CFL = 0.7$; $(\lambda_p, \theta_p) = (0,0)$.}
\end{figure}
%%%%%%%%%%%%%%%%%%%%%%%%%%%%%%%%%%%%%%%%%%%%%%%%%%%%%%%%%%%%%%%%%%%%%%%%%%%%%%%%%%%%%%%%%%%%%%
%\subsection{Deformational moving spherical vortex}
%%%%%%%%%%%%%%%%%%%%%%%%%%%%%%%%%%%%%%%%%%%%%%%%%%%%%%%%%%%%%%%%%%%%%%%%%%%%%%%%%%%
\subsection{Nair-Jablonowski test case}
%%%%%%%%%%%%%%%%%%%%%%%%%%%%%%%%%%%%%%%%%%%%%%%%%%%%%%%%%%%%%%%%%%%%%%%%%%%%%%%%%%%
\label{sec:4.2}
In \cite{Nair-Jablonowski} a modification of the stationary vortex problem of
Section \ref{sec:4.1} was suggested. It combines 
the deformational roll-up effect of the 
preceding case
with a solid body rotation.
The analytical solution is given in \cite{Nair-Jablonowski}.
The advection velocity $\mathbf{c}(\mathbf{x},t)$ in (\ref{eq:978.23.1}) is
obtained as the sum
\beq
\mathbf{c}=\mathbf{c}_s +\mathbf{c}_r
\eeq
where $\mathbf{c}_s$ is a solid rotation velocity and 
$\mathbf{c}_r$ is a "static" velocity centered at the center 
of the vortex. The velocity $\mathbf{c}_r$ is actually time dependant 
since in \eqref{eq:78.10}
%\eqref{vitesse_lambda_mach}-\eqref{vitesse_theta_mach====}, 
$(\lambda_P, \theta_P)$ must be remplaced by the solid body advected 
position given  by 
\begin{equation}
(\lambda_s', \theta_s') = (\lambda_0' + w_s t, \theta_0')
\end{equation}
where $(\lambda_0', \theta_0')$ is the initial position of the vortex.
On the other hand, the solid-body velocity is given by
\begin{equation}
c_{\lambda, r} = R \omega_s \left( \sin \theta_p \cos \theta - \cos \theta_p \cos ( \lambda - \lambda_p ) \sin \theta \right)
\label{vitesse_lambda_bump}
\end{equation}
\begin{equation}
c_{\theta, r} = - R \omega_s \cos \theta_p \sin ( \lambda - \lambda_p )
\label{vitesse_theta_bump}
\end{equation}
where

$\omega_s = v_0 / R = 2 \pi / T $ and $( \lambda_p, \theta_p$ )  
is the coordinates of the point $P$.
%%%%%%%%%%%%%%%%%%%%%%%%%%%%%%%%%%%%%%%%%%%%%%%%%%%%
%%\subsubsection{Numerical results}
The error growth is reported on Fig. \ref{erreur_cfl=0.05a}
and \ref{erreur_cfl=0.5a}. The magnitude of the error
is very close from the stationary case. However it appears 
slightly less regular. Table \ref{table:2} reports
as before fourth-order accuracy. Note also that the level
of error are very close to the ones reported in
\cite{Nair-Jablonowski} with a DG scheme. Finally
Fig. \ref{coupe-NJ-1} displays a slice of the vortex after 12 days
withgrid sizes $N=30$ and $N=60$. 
The matching with the finest grid
is excellent. No dispersion or dispersion is obervable.


%%%%%%%%%%%%%%%%%%%%%%%%%%%%%%%%%%%%%%%%%%%%%%%%%%%%%%%%%%%%%%%%%%%%%%%%%%%%%%%
\begin{figure}[!ht]
\includegraphics[scale=0.3]{ref_7367657139_snapshot_test_2_nday_0.png}
\includegraphics[scale=0.3]{ref_7367657143_snapshot_test_2_nday_3.png}

\includegraphics[scale=0.3]{ref_7367657147_snapshot_test_2_nday_6.png}
\includegraphics[scale=0.3]{ref_7367657152_snapshot_test_2_nday_9.png}

\includegraphics[scale=0.3]{ref_7367657157_snapshot_test_2_nday_12.png}
\caption{Nair and Jablonowski test-case. Approximate solution of the vortex after 
0, 3, 6, 9 and 12 days. The resolution is $N=31$. Numerical parameters are 
$N=31$, $\CFL = 0.7$ and $\alpha = 3 \pi / 4$.}
\label{SNAPSHOT}
\end{figure}
%%%%%%%%%%%%%%%%%%%%%%%%%%%%%%%%%%%%%%%%%%%%%%%%%%%%%%%%%%%%%%%%%%%%%%%%%%%%%%%%
\begin{figure}[!ht]
\includegraphics[scale=0.3]{ref_7367657290_normerreur_test_2.png}
\includegraphics[scale=0.3]{ref_7367657345_normerreur_test_2.png}
\label{erreur_cfl=0.05a}
\caption{Error curves $N=35$; $\CFL=0.05$; $\alpha = \pi / 4$ (left) et $\alpha = 0$ (right) for the Nair and Jablonowski test case \cite{Nair-Jablonowski}.}
\end{figure}
%%%%%%%%%%%%%%%%%%%%%%%%%%%%%%%%%%%%%%%%%%%%%%%%%%%%%%%%%%%%%%%%%%%%%%%%%%%%%%%
\begin{figure}[!ht]
\includegraphics[scale=0.3]{ref_7367657356_normerreur_test_2.png}
\includegraphics[scale=0.3]{ref_7367657366_normerreur_test_2.png}
\label{erreur_cfl=0.5a}
\caption{Error curves $N=35$; $\CFL=0.5$; $\alpha = \pi / 4$ (left) et $\alpha = 0$ (right) for the Nair and Jablonowski test case \cite{Nair-Jablonowski}.}
\end{figure}
%%%%%%%%%%%%%%%%%%%%%%%%%%%%%%%%%%%%%%%%%%%%%%%%%%%%%%%%%%%%%%%%%%%%%%%%%%%%%%%%%
\begin{figure}
\begin{tabular}{c||cc|cc|cc}
$N$ & $max_n |e_1^n|$ & ordre  & $max_n |e_2^n|$ & ordre  & $max_n |e_{\infty}^n|$ & ordre \\
\hline
\hline
$40$ & $0.3037 (-2)$ & -  & $0.1061(-2)$ & - & $.6220 (-1)$  & - \\
\hline 
$50$ & $0.1471 (-2)$ & $3.2487$ & $0.5656(-2)$ & $2.8192$ & $0.3664 (-1)$ & $2.3715$ \\
\hline
$60$ & $0.7373(-3)$ & $3.7884$ & $0.3033(-2)$ & $3.4179$ & $0.2178(-1)$ & $2.8529$ \\
\hline
$80$ & $0.2311(-3) $ & $4.0327$ & $0.9921(-3)$ & $3.8841$ & $0.7691(-2)$ & $3.6184$ \\
\hline
$100$ & $9.4841(-5)$ & $3.9914$ & $0.4108(-3)$ & $3.9513$ & $0.3205(-2)$ & $3.9228$\\
\hline
$150$ & $1.8779 (-5)$ & $3.9941$ & $8.3160 (-5)$ & $3.9395$ & $0.7584(-3)$ & $3.5543$
\end{tabular}
\label{table:2}
\caption{Convergence analysis for the Nair and Jablonowski test case \cite{Nair-Jablonowski} ; $cfl = 0.7$ ; $\alpha = \pi /4$.}
\end{figure}
%%%%%%%%%%%%%%%%%%%%%%%%%%%%%%%%%%%%%%%%%%%%%%%%%%%%%%%%%%%%%%%%%%%%%%%%%%%%%%%%%%%%%%%%%
\begin{figure}[!ht]
\includegraphics[scale=0.5]{ref_7363158648_coupefaceI_equateur_test_2.jpg}
\label{coupe-NJ-1}
\caption{Nair and Jablonowski test case \cite{Nair-Jablonowski}. Slice 
of the vortex after $12$ days. Solid line: exact solution, circles:
approximate solution with $N=30$. Crosses: approximate solution with $N=60$}
\end{figure}
%%%%%%%%%%%%%%%%%%%%%%%%%%%%%%%%%%%%%%%%%%%%%%%%%%%%%%%%%%%%%%%%%%%%%%%%%%%%%%
\subsection{A damped case of LSWE}
%%%%%%%%%%%%%%%%%%%%%%%%%%%%%%%%%%%%%%%%%%%%%%%%%%%%%%%%%%%%%%%%%%%%%%%
This test serves to assess the accuracy of the 
gradient and divergence approximation () and () when
used in the LSWE system (\ref{eq:76.10}). 
Consider the two exponentially in time damped functions
\beq
\tilde{\mathbf{v}} (t,\mathbf{x})=\mathbf{u}_0 \varphi(\theta) e^{-\sigma t}\mathbf{e}_\lambda(\mathbf{x})\\
\tilde{\eta}(t,\mathbf{x})= \eta_0 \varphi(\theta) \sin(\lambda)e^{-\sigma t} 
\eeq
The system to solve is
\beq
\label{eq:76.11}
(LSWE) \left\{
\begin{array}{l}
\partial_t \mathbf{v}(t,\mathbf{x})+ \mathbf{g} \nabla_T \eta(t,\mathbf{x}) + f(\mathbf{x}) \mathbf{k}(\mathbf{x}) \times
\mathbf{v}(t,\mathbf{x})=S_{\eta}(t,\mathbf{x})\\
\partial_t \eta(t,\mathbf{x})+ H \nabla_T . \mathbf{v}(t,\mathbf{x})=S_{\mathbf{v}}(t,\mathbf{x})
\end{array}
\right.
\eeq
In (\ref{eq:76.11}), $\mathbf{g}$ is the gravity vector and $\mathbf{k}(\mathbf{x})$ is the exterior
normal vector.
The source terms $S_{\eta}$ and $S_{\mathbf{v}}$ are defined 
such as the functions $(\tilde{\mathbf{v}}(t,\mathbf{x}), \tilde \eta(t,\mathbf{x}))$ be solution
of (\ref{eq:76.11}).
Remarks:\\
- Gradient and divergence are calculated according to ().\\
- Filtrage ou non ?
- Valeurs numériques de $H,g, u_0, \eta_0$ ?
- Autres essais  avec temps plus long ?
A numerical grid convergence analysis is reported in 
Table \ref{table:4}.
%%%%%%%%%%%%%%%%%%%%%%%%%%%%%%%%%%%%%%%%%%%
\begin{figure}
\begin{tabular}{||c|c|c|c|c|c||}
\hline
& N=40 & rate & N=60  & rate & N=80 \\
\hline 
$\vert \mathbf{v}^1_{ex}(T)-\mathbf{v}^1_{cal}(T)\vert_{h,I}$ & 2.52(-5)  &  4.15 & 4.69(-6) &  2.98 & 1.40(-6)   \\
\hline 
$\vert \eta_{ex}(T)-\eta_{cal}(T)\vert_{h,I}$ & 1.06(-5)  &  3.85 & 2.70(-6) &  3.32 & 1.04(-6) \\
\hline 
\end{tabular}
\caption{Hermitian scheme applied to an exponential decaying solution of the LSWE. Final time = 1h30}
\label{table:4}
\end{figure}

A convergence rate between $3$ and $4$ can be observed.

%%%%%%%%%%%%%%%%%%%%%%%%%%%%%%%%%%%%%%%%%%%%%%%%%%%%%%%%%%%%%%%%%%%%%%%%%%
\subsection{A time independent solution zonal solution 
of LSWE}
In this test case, we consider a time independent 
solution of LSWE depending on the latitude only.
Consider a tangential velocity field of the form. 
If the parameter function $\theta \mapsto \varphi(\theta)=\exp(\frac{1}{(\theta-\theta_0)(\theta-\theta_1)})$, then
we define the 
spherical velocity $\mathbf{v}(\mathbf{x})$ on the sphere by:
\beq
\mathbf{v}(\mathbf{x})=u_0 \varphi(\theta) \mathbf{e}_\lambda(\mathbf{x})
\eeq
The momentum equation () is equivalent to the 
following relation:
%% The matching atmosphere thickness is $\eta(\bx)$:
\beq
\eta(\mathbf{x})=\eta_{eq}-\frac{a}{g}\int_0^\theta f(s) \varphi(s) ds
\eeq
Our numerical test consists now in testing if the scheme preserve during time stepping 
the time independant solution defined by .....
This is a zonal divergence free solution
of the LSWE depending on the latitude $\theta$ only.
This test case is meaningful, 
first to assess the accuracy of the spatial approximation. In particular, 
spurious modes can pollute the numerical solution.
Second this test allows to test the accuracy of the numerical divergence 
preserving on 
large intervals of time.
The numerical results are reported in Table \ref{table:5}.

\begin{figure}
\begin{tabular}{||c|c|c|c|c|c||}
\hline 
&N=40 & rate & N=60  & rate & N=80 \tabularnewline
\hline
$\vert \mathbf{v}^1_{ex}(T)-\mathbf{v}^1_{cal}(T)\vert_{h,I}$ & 2.73(-5)  &  4.09 & 5.18(-6) &  4.24 & 1.53(-6) \\
\hline 
$\vert \eta_{ex}(T)-\eta_{cal}(T)\vert_{h,I}$ & 1.21(-5)  &  3.95 & 2.43(-6) &  3.14 & 7.38(-7) \\
\hline 
\end{tabular}
\caption
{Hermitian scheme applied to a time independent solution of the LSWE. Final time = 1h30}
\label{table:5}
\end{figure}