%% *** SECTION ***************************************************************************************************************

\section{Fourth order accurate spherical gradient}
\label{sec:2}
\subsection{Hermitian approach to partial derivatives on the sphere}
\label{sec:2.1}
In this section we present the principle
behind the calculation of the high-order accurate
approximate spherical gradient, as introduced in 
\cite{Croisille-10, Croisille-12}.
Consider a given function $ \mathbf{s} \in \mathbb{S}^2 \mapsto u(\mathbf{s})$ on the unit sphere
and a point $\bar{\mathbf{s}}$ where an approximate value of the tangential gradient
$\nabla_T u(\bar{\mathbf{s}})$ is required.
Consider a point $\mathbf{s}_0$ close to $\bar{\mathbf{s}}$ and two orthogonal great circles
$C_1$ and $C_2$ intersecting at $\mathbf{s}_0$. 
Denote by $\xi$ and $\eta$ the angles along $C_1$ and $C_2$ respectively.
The angles $(\xi,\eta)$ form a local coordinate system 
whose $\mathbf{s}_0(0,0)$ is the center. 
The tangential gradient $\nabla_T u(\bar{\mathbf{s}})$  is given by:
\beq
\label{eq:87.1}
\nabla_T u(\bar{\mathbf{s}})=
\Big(
\partial_\xi u(\bar{\mathbf{s}})
\Big)
g^\xi(\bar{\mathbf{s}})
+
\Big(
\partial_\eta u(\bar{\mathbf{s}})
\Big)
g^\eta(\bar{\mathbf{s}}).
\eeq
In (\ref{eq:87.1}), $(g^\xi(\mathbf{s}),g^\eta(\mathbf{s}))$
denotes the dual basis of the basis 
$(g_\xi(\mathbf{s}),g_\eta(\mathbf{s}))$ where
\beq
\label{eq:32.19}
g_\xi(\mathbf{s})=\frac{\partial \mathbf{s}}{\partial \xi},\;\;\;
g_\eta(\mathbf{s})=\frac{\partial \mathbf{s}}{\partial \eta}.
\eeq
Consider next the iso-$\eta$ curve passing through $\bar{\mathbf{s}}$.
This curve is locally defined as a function of the $\xi$ parameter as:
\beq
 \xi \in (\bar{\xi}-\varepsilon, \bar{\xi}+\varepsilon)
 \mapsto \mathbf{s}(\xi,\bar{\eta}), \;\;\; \varepsilon >0.
\eeq
Due to the orthogonality of the circles $C_1$ and $C_2$, this curve
is also an arc of great circle. We call $\bar{C}$ this great circle.
We define a discrete grid of $\bar{C}$ 
using the points with coordinates $\mathbf{s}_i(i\Delta \xi,\bar{\eta})$. The value
$\Delta \xi >0 $ is the constant step defined by
\beq
\Delta \xi = \frac{\pi}{2N}
\eeq 
and $N>0$ is an integer.
Assume for the moment that the point $\bar{\mathbf{s}}$ belongs to this grid and that 
$\bar{\mathbf{s}}=\mathbf{s}_i$ for some integer $i$.  A simple way 
to approximate $\nabla_T u(\bar{\mathbf{s}})$
is obtained by approximating in (\ref{eq:87.1}) 
the partial derivatives
$\partial_\xi u(\mathbf{s}_i)$ and $\partial_\eta u(\mathbf{s}_i)$.
Consider first $\partial_\xi u(\bar{\mathbf{s}})$.
Since $\bar{\mathbf{s}}=\mathbf{s}_i$, using 
the standard centered divided difference we have;
\beq
\label{eq:75.10}
\partial_\xi u(\mathbf{s}_i) \simeq 
\frac{u(\mathbf{s}_{i+1})- 
u(\mathbf{s}_{i-1})}{2 \Delta \xi}.
\eeq
The divided difference in the right-hand-side is denoted by $\delta_\xi u_i$.
Proceeding in the same manner in the $\eta$ direction, we set up
a grid on the circle $\bar{C}^\prime$:
the iso-$\xi$ great circle through by $\bar{\mathbf{s}}$ receives 
a grid with step size $\Delta \eta = \Delta_\xi$. The points
along this grid are numbered with index $j$ and we
call ${\mathbf{s}}=\mathbf{s}^\prime_j$. The value
$\partial_\eta u (\bar{s})$ is approximated by:
\beq
\label{eq:75.11}
\partial_\eta u(\bar{\mathbf{s}}) \simeq 
\delta_\eta u^\prime_j = \frac{u(\mathbf{s}^\prime_{j+1})- 
u(\mathbf{s}^\prime_{j-1})}{2 \Delta \eta}.
\eeq
A first candidate for the approximate value to $\nabla_T u(\bar{\mathbf{s}})$ is therefore:
\beq
\label{eq:87.2}
\nabla_{T,h} u(\bar{\mathbf{s}})=
\left(\delta_\xi u_i\right)
g^\xi(\bar{\mathbf{s}})
+
\left(\delta_\eta u^\prime_j \right)
g^\eta(\bar{\mathbf{s}}).
\eeq
The vector $\nabla_{T,h} u(\bar{\mathbf{s}})$ is obviously a second order 
approximation to $\nabla_T u(\bar{\mathbf{s}})$.
A simple way to go beyond second order
is to modifiy $\delta_\xi u_i$ using instead
the Hermitian derivative $\delta^H_\xi u_i$.
It is defined in terms of $\delta_\xi u_i$ by the relation
\beq
\label{eq:73.10}
\frac{1}{6} \delta_\xi^H u_{i-1}
+\frac{2}{3} \delta_\xi^H u_i
+\frac{1}{6} \delta_\xi^H u_{i+1}
= 
\delta_\xi u_i.
\eeq
The relation (\ref{eq:73.10}) defines implictely
$ \delta_\xi^H u_i$ as a perturbation of 
$\delta_\xi u_i$ since it can be expressed as:
\beq
\label{eq:73.14}
\delta_\xi^H u_i
= 
\delta_\xi u_i -\frac{\Delta^2\xi}{6} \delta^2_\xi \delta^H_\xi u_i
\eeq
The difference with (\ref{eq:75.10}) 
is that (\ref{eq:73.10}) involves values of $u$ along 
the full on $\bar{C}$ and not only the two neighboor values 
at $i\pm 1$.
Solving (\ref{eq:73.10}) provides $\delta^H u_i$.
This values satisfies the fourth-order consistency relation
\beq
\label{eq:34.18}
\delta^H_\xi u_i=
\partial_\xi u(\bar{\mathbf{s}})+ O(\Delta \xi^4)
\eeq
Proceeding in the same way along the $\eta-$ direction gives 
the approximation
\beq
\label{eq:34.19}
\delta^H_\eta u^\prime_j=
\partial_\eta u(\bar{\mathbf{s}})+ O(\Delta \xi^4)
\eeq
The Hermitian approximate gradient at the point $\bar{\bs}$ 
is therefore:
\beq
\label{eq:87.2.1}
\nabla^H_{T,h} u(\bar{\mathbf{s}})=
\left(\delta^H_\xi u_i\right)
g^\xi(\bar{\mathbf{s}})
+
\left(\delta^H_\eta u^\prime_j\right)
g^\eta(\bar{\mathbf{s}})
\eeq
It results from (\ref{eq:34.18}-\ref{eq:34.19}) that
$\nabla^H_{T,h} u(\bar{\mathbf{s}})$ satisfies the fourth order
consistency relation
\beq
\label{eq:87.4}
\nabla^H_{T,h} u(\bar{\mathbf{s}})=\nabla_{T} u(\bar{\mathbf{s}})+O(\Delta \xi^4)+O(\Delta \eta^4).
\eeq
The Hermitian approximation (\ref{eq:87.4}) is the basis of our approximate gradient on the Cubed-Sphere.
%%%%%%%%%%%%%%%%%%%%%%%%%%%%%%%%%%%%%%%%%%%%%%%%%%%%%%%%%%%%%%%%%%%%%%%%%%%
\subsection{Approximate gradient on the Cubed-Sphere}
%%%%%%%%%%%%%%%%%%%%%%%%%%%%%%%%%%%%%%%%%%%%%%%%%%%%%%%%%%%%%%%%%%%%%%%%%%%%
The Cubed Sphere is a grid of the sphere. This grid and variants were introduced by
various authors. A systematic
presentation was given in
\cite{Ronchi-Iacono-Paolucci}.
This grid has been widely used for numerical climatology.
The Cubed-Sphere is composed of six panels with
label $(k)=(I), (II), (II), (IV), (V)$ and $(VI)$. Each panel matches
the face of the cube, in which the sphere
is embedded. Each panel
supports a Cartesian grid of size $N\times N$.
It is equipped with a 
coordinate system $(\xi,\eta)$.
As in the preceding section,
$\xi$ and $\eta$ are angles along a couple of orthogonal great circles
intersecting at the center of the panel.
A typical panel and the associated grid is represented
on Fig. \ref{fig:1}.
The  grid points are called $\mathbf{s}_{i,j}^{(k)}$ with $ (I) \leq (k) \leq (VI)$ and
$ -N/2 \leq i,j \leq N/2$. 
In panel $(k)$ and for all fixed $j$, the points  $ i \mapsto \mathbf{s}_{i,j}$ are 
located along a great circle. This essential property of the Cubed-Sphere
permits to follow the idea presented in Section \ref{sec:2.1}: 
the calculation of an approximate gradient at $\mathbf{s}^{(k)}_{i,j}$
by mean of Hermitian derivatives.
The approximate gradient
$\nabla_{T,h} u^{(k)}_{i,j}$ is given by
\beq
\label{eq:85.13}
\nabla_{T,h} u^{(k)}_{i,j}=
u_{\xi,i,j}^{(k)} \mathbf{g}^{\xi, (k)}_{i,j} 
+
u_{\eta,i,j}^{(k)} \mathbf{g}^{\eta, (k)}_{i,j} 
\eeq
As in the preceding section, the values $ u_{\xi,i,j}^{(k)}$ and $u_{\eta,i,j}^{(k)}$
are Hermitian approximation of the partial derivatives $\partial_\xi u(\mathbf{s}^{(k)}_{i,j})$ and
$\partial_\eta u(\mathbf{s}^{(k)}_{i,j})$. The calculation of 
these Hermitian derivatives are
based on a set of data located along two great circles called $\bar{C}_{i,j}$ and $\bar{C}^\prime_{i,j}$.
The data along $\bar{C}_{i,j}$ and $\bar{C}^\prime_{i,j}$
are of course based on specific points of the Cubed-Sphere.
The choice of these points is detailed in \cite{Croisille-10, Croisille-12}
and we refer to these two references for more details.
Even if no mathematical proof is available yet, 
numerical evidence show a consistency close to $4$.
%%%%%%%%%%%%%%%%%%%%%%%%%%%%%%%%%%%%%%%%%%%%%%%%%%%%%%%%%%%%%%%%%%%%%%%%%%%%%%%%%
\begin{figure}
   \def\svgwidth{0.4 \textwidth}
\input{drawing13.pdf_tex}
\caption{The points of a typical panel of the Cubed-Sphere are classified in three categories:
(i) Circles correspond to {\sl internal} points; (ii) Squares correspond to {\sl edge} points ;
(iii) Pentagons correspond to {\sl corner} points}
\label{fig:1}
\end{figure}