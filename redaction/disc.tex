% disc.tex

\chapter{Méthodes de discrétisation}

\section{Discrétisation spatiale}

La résolution numérique d'une équations aux dérivées partielles (EDP) nécessite une discrétisation du problème
initials. Différentes méthodes ont été développées dans cet objectif,
en voici les exemples les plus courrants :

\begin{itemize}
\item Les méthodes d'éléments finis reposent sur la formulation
  variationnelle du problème initial,

\item Les méthodes de volumes finis utilisent la forme intégrale de
  l'EDP,

\item Les différences finies emploient la formulation forte de
  l'équation à résoudre.
\end{itemize}

Il est tout de même à noter que certaines méthodes possèdent des propriétés mixtes. On pensera, par exemple, au schéma boîte \cite{Abbas2011} qui possèdent des propriétés communent aux différences finies et aux volumes finis ou aux méthodes de Galerkin discontinues qui reposent sur les éléments finis mais possèdent des propriétés de conservations de matière typiques des volumes finis.

Dans cette partie, nous présentons les méthodes de différences finies
qui permettent une discrétisation et une résolution
numérique. L'accent sera mis sur les schémas d'ordre élevé \cite{Desquesnes2007} et en
particulier sur les schémas compacts étudiées par S. K. Lele dans
\cite{Lele1991}. On effectuera aussi une étude sur les fréquences qui
jouent un r\^ole important dans la résolution des équations hyperboliques. 

\subsection{Méthodes de différences finies}

Dans la suite nous considérons une fonction :

$$u : [0, 1] \rightarrow \mathbb{R}$$

 supposée suffisament régulière. Nous supposons connues les valeurs de $u(x_i)$
(nous noterons $u_i$ cette valeur) et $u$ $1-$périodique\footnote{Le choix de périodicité est fait par simplicité d'écriture mais des méthodes existent pour effectuer ce calcul avec des conditions de Dirichlet, Neumann, ... Voir par exemple \cite{Abbas2011} et \cite{Brachet2015}} (donc $u(0) = u(1)$).

Les points $x_i$ sont les points du maillage. Nous les définissons par
$x_0 = 0$ et $x_i = x_0 + i \times h$ avec $h = \frac{1}{N+1}$, pour
tout $ 0 \leq i \leq N+1 $.

L'objectif est ici de déterminer une valeur approchée de la dérrivée
première $\partial_xu(x_i)$ notée $\delta_x u_i$\footnote{On peut
  appliquer le m\^eme raisonnement pour déterminer une approximationd es
dérivées d'ordre supérieur. Voir \cite{Lele1991} pour plus de détails.} gr\^ace à une relation de la forme :

\begin{equation}
\label{eq:differences finies}
\alpha_0 \delta_x u_i + \sum_{j=1}^M \alpha_j \left( \delta_x u_{i+j} + \delta_x
u_{i-j} \right) = \sum_{k=1}^N a_k \dfrac{u_{i+j} - u_{i-j}}{2kh}
\end{equation}

La difficulté étant de déterminer les coefficients les coefficients
$(\alpha_j)_{0 \leq j \leq M}$ et $(a_k)_{1 \leq k \leq N}$ pour que la
méthode soit aussi précise que souhaité.

\begin{definition}
On dit que la formule \eqref{eq:differences finies} est précise à
l'ordre $q$ si elle définit $\delta_x u_i$ tels que :
\begin{equation}
 \delta_x u_i - \partial_x u(x_i) = \mathcal{O} \left( h^q \right)
\end{equation}

lorsque $u$ est suffisament régulière.
\end{definition}

Dans la suite nous parlerons de schémas classiques et de schémas
compacts.

\subsection{Schémas Classiques }

Les schémas classiques sont construits de manière à pouvoir calculer
immédiatement $\delta_x u_i$ comme une combinaison linéaire de ses
voisins.

\begin{definition}
Un schéma d'approximation de la forme \eqref{eq:differences finies}
est dit classique lorsque pour tout $0 \leq i \leq N+1$, on a
$\alpha_i = \delta_{0,i}$\footnote{$\delta_{p,q}$ est le symbole de
  Kronecker. Il vaut $1$ lorsque $p=q$ et $0$ sinon.}
\end{definition}

Un schéma classique d'approximation de $\partial_x u$ est de la forme
:

\begin{equation}
  \label{eq:schemas classiques}
 \delta_x u_i = \sum_{k=1}^N a_k \dfrac{u_{i+j} - u_{i-j}}{2kh}
\end{equation}

Les développements de Taylor nous donnent immédiatement la relation
suivante :

\begin{equation}
\label{eq:taylor}
 \sum_{k=1}^N a_k \dfrac{u(x+jh) - u(x-jh)}{2kh} = \sum_{p=0}^{P}
 \sum_{k=1}^N a_k \dfrac{(kh)^p + (-kh)^p}{p! 2 k h} \partial_x^{(p)} u(x)
 + \mathcal{O} \left( h^{P+1} \right)
\end{equation}

ainsi pour avoir une méthode d'ordre aussi élevé que possible, on souhaite avoir pour $p \in \mathbb{N}$ :

\begin{equation}
  \label{eq:prop schemas explicites}
\sum_{k=1}^N a_k \dfrac{k^{p-1} + (-k)^{p-1}}{k} = \delta_{p,1}
\end{equation}

On remarque que la propriété \eqref{eq:prop schemas explicites} est automatiquement vérifiée pour $p$ impair. On peut en déduire qu'un schéma classique centré de la forme \eqref{eq:schemas classiques} est automatiquement d'ordre pair. De plus, un schéma à $N$ points sera d'ordre $2N$ au plus. 

\begin{exemple}
  Voici quelques exemples de schémas classiques centrés :

  \begin{itemize}
  \item Schéma classique d'ordre 2 :
    \begin{equation}
      \label{eq:classique ordre 2}
      \dfrac{u(x+h)-u(x-h)}{2h} = \partial_x u(x) + \mathcal{O} \left( h^2 \right)
    \end{equation}

  \item Schéma classique d'ordre 4 :
    \begin{equation}
      \label{eq:classique ordre 4}
      \dfrac{4}{3} \dfrac{u(x+h)-u(x-h)}{2h} - \dfrac{1}{3} \dfrac{u(x+2h)-u(x-2h)}{4h}  = \partial_x u(x) + \mathcal{O} \left( h^4 \right)
    \end{equation}

    \item Schéma classique d'ordre 6 :
    \begin{equation}
      \label{eq:classique ordre 6}
      \dfrac{3}{2} \dfrac{u(x+h)-u(x-h)}{2h} - \dfrac{3}{5} \dfrac{u(x+2h)-u(x-2h)}{4h} + \dfrac{1}{10} \dfrac{u(x+3h)-u(x-3h)}{6h} = \partial_x u(x) + \mathcal{O} \left( h^6 \right)
    \end{equation}
    
  \end{itemize}
\end{exemple}

Il est clair que plus l'ordre d'une méthode est élevé plus cette méthode est précise. Le choix de la méthode de discrétisation est crucial dans la résolution numérique d'une équation.

\subsection{Schémas Compacts }

Les schémas compacts, ou Hermitiens, reposent sur une idée semblable à celle des schémas classiques mais prennant en compte l'intégralité des points disponibles. L'idée est d'utiliser l'équation \eqref{eq:differences finies} mais contrairement aux schémas classiques, nous ne supposons pas $\alpha_i = \delta_{i,0}$

Ainsi, si les $(a_i)_{1 \leq i \leq N}$ sont déterminés par un schéma explicite comme précédemment, il ne reste plus qu'à déterminer les coefficients $(\alpha_j)_{0 \leq j \leq M}$. Par développement limité et identification des facteurs, on déduit que les équations suivantes doivent être vérifiées :

\begin{equation}
  \label{eq:compact-eq1}
  \alpha_0 + 2 \sum_{j=1}^M \alpha_j = 2 \sum_{i=1}^N a_i
\end{equation}

ainsi que pour tout $k=1, ...$:

\begin{equation}
  \label{eq:compact-eq2}
  \left( k+1 \right) \sum_{j=1}^M \alpha_j \left( j^k + (-j)^k \right) = \sum_{i=1}^N a_i \left( i^k + (-i)^k \right)
\end{equation}

NOTE : VERIFIER CES EQUATIONS!!!

On peut commencer par noter que la propriété \eqref{eq:compact-eq2} est automatiquement vérifiée pour $k$ impair. De plus, si les $(a_i)_{1 \leq i \leq N}$ sont préalablement définis, il ne reste plus qu'à déterminer les $M+1$ coefficients $(\alpha_j)_{0 \leq j \leq M}$. La méthode est alors une discrétisation par différences finies\footnote{Cependant, il est important de noter que le calcul des $\delta_x u_i$ est implicite et demande la résolution d'un système linéaire.} à l'ordre $2(M+N)$.








Dois je mettre une méthode pour traiter les CL? (voir article avec J.P. Chehab)

\subsection{Analyse spectrale}


\section{Discrétisation temporelle}

\subsection{Méthodes de Runge-Kutta : généralités}

\subsection{Méthodes de Runge-Kutta explicite}

\subsection{Quelques méthodes de Runge-Kutta implicites : DIRK et SDIRK}

\subsection{Stabilité des méthodes de Runge-Kutta}


\section{Filtrage numérique}

\subsection{\'Etude générale}

\subsection{Filtrage explicite de type S. Redonnet}

\subsection{Filtrage Implicite : cas de Visbal (titre à revoir)}


\section{Interpolation par spline cubique}

L'objectif de cette section est de présenter l'outils d'interpolation utilisé : les méthodes de spline cubique.

On se donne les valeurs $(y_i)_{0 \leq i \leq N+1}$ associées aux points de maillage $(x_i)_{0 \leq i \leq N+1}$. Les points de maillages sont tels que : pour tous $i$ et $j$ entre $0$ et $N+1$ tels que $i < j$ alors $x_i < x_j$.

Le problème est de trouver une fonction $\phi$ telle que :

\begin{equation}
\text{Pour tout } i \hspace{0.5cm}   \phi(x_i) = y_i
\end{equation}

On peut chercher $\phi$ sous la forme suivante :

\begin{equation}
\phi (x) = \sum_{i=0}^{N+1} \phi_i(x) y_i
\label{eq:interpolation fct caract}
\end{equation}

pour tout $x \in \mathbb{R}$. Pour tout $i$, $\phi_i$ est telle que pour tout $j$ on a :

\begin{equation}
\phi_i ( x_j ) = \delta_{i,j}
\label{eq:fct cardinal}
\end{equation}

où $\delta_{i,j}$ est le symbole de Kronecker.

\begin{remarque}
Le problème \eqref{eq:fct cardinal} est toujours un problème d'interpolation mais le choix de $(\phi_i)_{0 \leq i \leq N+1}$ est indépendant de la donnée $(y_i)_{0 \leq i \leq N+1}$. De plus l'écriture \eqref{eq:interpolation fct caract} permet de mettre en évidence la linéarité du problème par rapport à $(y_i)_{0 \leq i \leq N+1}$.
\end{remarque}

Les fonctions $(\phi_i)_{0 \leq i \leq N+1}$ sont appelées fonctions cardinales. Lorsque $\phi_i$ est nulle exceptée sur un voisinage de $x_i$, on dit que l'interpolation est locale.

\subsection{Spline Cubique}

Dans cette partie, nous recherchons les fonctions $(\phi_i)_{0 \leq i \leq N+1}$ sous la forme d'un polynôme d'ordre 3 par morceaux. C'est à dire que pour tout $x \in [x_i, x_{i+1}]$, on a :

\begin{equation}
\phi_i(x) = c_1 + c_2 (x-x_i) + c_3 (x-x_i)^2 + c_4 (x-x_i)^3
\end{equation}

De plus, les fonctions $(\phi_i)_{0 \leq i \leq N+1}$ doivent être suffisamment régulière. Ici on cherche $\phi_i \in \mathcal{C}^2$.

\subsection{Polynôme d'interpolation de degré 3}

Dans cette partie uniquement, on cherche $P\in \mathbb{R}_3 [x]$ tel que 

\begin{itemize}
\item Pour tout $x \in [a,b]$, $P(x) = y_a + y_a' (x-a) + c_3 (x-a)^2 + c_4 (x-a)^3$,
\item $P(a)=y_a$,
\item $P(b)=y_b$,
\item $P'(a)=y_a'$,
\item $P'(b)=y_b'$.
\end{itemize}

en supposant, naturellement, $a \neq b$.

On note d'abord que les conditions 2 et 4 sont triviallement vérifiée par la forme donnée à $P$ dans la condition 1.
Il reste à voir quelle forme doivent prendre $c_3$ et $c_4$ pour que les conditions 3 et 5 soient vérifiées.

On sait que $P(b) = y_b$ donc :

$$(b-a)^2 (c_3 + c_4 (b-a)) = y_b - y_a - y_a' (b-a)$$

d'où :

\begin{equation}
c_ 3 + c_4 (b-a) = \dfrac{y_b - y_a}{(b-a)^2} - \dfrac{y_a'}{b-a}
\label{cond:p(b)=yb}
\end{equation}

De manière similaire, comme $P'(b) = y_b'$, on a :

\begin{equation}
2 c_3 + 3 c_4 (b-a) = \dfrac{y_b' - y_a'}{b-a}
\label{cond:p'(b)=yb'}
\end{equation}

En combinant \eqref{cond:p(b)=yb} et \eqref{cond:p(b)=yb}, on peut montrer que :

\begin{equation}
\left\lbrace
\begin{array}{rcl}
c_3 & = & 3 \dfrac{y_b - y_a}{(b-a)^2} - \dfrac{2 y_a' + y_b'}{b-a} \\
c_ 4 & = & 2 \dfrac{y_a - y_b}{(b-a)^3} + \dfrac{y_b' + y_a'}{(b-a)^2}
\end{array}
\right.
\label{cond:c3 and c4}
\end{equation}

\subsection{Construction de la Spline cubique}

Soit $i$ un entier entre $1$ et $N$.
Alors $\phi$ restreint à $[x_i, x_{i+1}]$ est de la forme :

\begin{equation}
\phi_{|[x_i, x_{i+1}]} (x) = y_i + y_i' (x-x_i) + c_3^i (x-x_i)^2 + c_4^i (x-x_i)^3
\end{equation}

avec $c_3^i = 3 \dfrac{y_{i+1} - y_i}{(x_{i+1}-x_i)^2} - \dfrac{2 y_i' + y_{i+1}'}{x_{i+1}-x_i}$ et 
$c_4^i = 2 \dfrac{y_i - y_{i+1}}{(x_{i+1}-x_i)^3} + \dfrac{y_{i+1}' + y_i'}{(x_{i+1}-x_i)^2}$.

De la même manière, on a :

\begin{equation}
\phi_{|[x_{i-1}, x_{i}]} (x) = y_{i-1} + y_{i-1}' (x-x_{i-1}) + c_3^{i-1} (x-x_{i-1})^2 + c_4^{i-1} (x-x_{i-1})^3
\end{equation}

avec $c_3^{i-1} = 3 \dfrac{y_i - y_{i-1}}{(x_{i}-x_{i-1})^2} - \dfrac{2 y_{i-1}' + y_{i}'}{x_{i}-x_{i-1}}$ et 
$c_4^{i-1} = 2 \dfrac{y_{i-1} - y_i}{(x_{i}-x_{i-1})^3} + \dfrac{y_{i}' + y_{i-1}'}{(x_{i}-x_{i-1})^2}$.

Car $\phi$ est une fonction $\mathcal{C}^2$ donc sa dérivée est continue. De plus, sa dérivée seconde est continue aussi, donc :

\begin{equation}
\phi_{|[x_i, x_{i+1}]}''(x_i) = \phi_{|[x_{i-1}, x_{i}]}''(x_i) 
\end{equation}

d'où :

\begin{equation}
c_3^i = c_3^{i-1} + 3 c_4^{i-1} ( x_i - x_{i-1} )
\end{equation}

Ce que l'on peut écrire (en simplifiant) pour tout $1 \leq i \leq N$ :

\begin{multline}
- \dfrac{3}{(x_i - x_{i-1})^2} y_{i-1} + \left( \dfrac{3}{(x_i - x_{i-1})} - \dfrac{3}{(x_{i+1} - x_{i})} \right) y_i + \dfrac{3}{(x_{i+1} - x_{i})^2} y_{i+1} = ...\\
... \dfrac{1}{x_i - x_{i-1}} y_{i-1}' + \left( \dfrac{2}{x_{i+1}-x_i} + \dfrac{2}{x_{i}-x_{i-1}} \right)y_i' + \dfrac{1}{x_{i-1} - x_{i}} y_{i+1}' 
\label{eq:spline cubique int}
\end{multline}

L'équation \eqref{eq:spline cubique int} donne $N$ relation pour les points intérieurs du maillages alors qu'il y a $N+2$ inconnues $y_i'$. Il manque donc deux conditions pour que le problème soit bien posé.

Si la fonction à déterminer est une fonction cardinale $\phi_j$, on note que $y_i = \delta_{i,j}$. Ainsi le terme de gauche de \eqref{eq:spline cubique int} est nul lorsque $i>j+1$. Cependant, la spline cubique n'est a priori pas une interpolation locale.

JE PENSE QU'ON PEUT MONTRER QUE \eqref{eq:spline cubique int} EST UN SCHEMA COMPACT D'ORDRE 4 ET DONC TRADUIRE CA COMME UN SCHEMA AU DF! VERIFIER L'ORDRE DE CETTE METHODE.

\subsection{Spline cubique naturelle}

\subsection{Spline cubique 'Not a knot'}

Pour la condition 'Not a Knot', on suppose que $\phi$ est $\mathcal{C}^3$ en $x_1$ et en $x_N$.

Ainsi, on doit avoir :

\begin{equation}
\left\lbrace
\begin{array}{rcl}
\phi_{|[x_0, x_1]}^{(3)}(x_1) & = & \phi_{|[x_1, x_2]}^{(3)}(x_1) \\
\phi_{|[x_N, x_{N+1}]}^{(3)}(x_N) & = & \phi_{|[x_{N-1}, x_N]}^{(3)}(x_N)
\end{array}
\right.
\label{NOK_condition}
\end{equation}

Ce qui se traduit par :

\begin{equation}
\left\lbrace
\begin{array}{rcl}
6 c_4^0 & = & 6 c_4^1 \\
6 c_4^{N} & = & 6 c_4^{N+1} \\
\end{array}
\right.
\end{equation}

écrit à l'aide des $(y_i)$ et $(y_i')$, on obtient :

\begin{multline}
\dfrac{1}{(x_1-x_0)^3} y_0 - \left[ \dfrac{1}{(x_1 - x_0)^3} + \dfrac{1}{(x_2-x_1)^3} \right] y_1 + \dfrac{1}{(x_2 - x_1)^3} y_2 = ... \\
... \dfrac{1}{2} \left[ \dfrac{1}{(x_2-x_1)^2} y_2' + \left( \dfrac{1}{(x_2-x_1)^2} - \dfrac{1}{(x_1-x_0)^2} \right) y_1 - \dfrac{1}{(x_1 - x_0)^2} y_0' \right]
\end{multline}

ainsi que, de la même manière :

\begin{multline}
\dfrac{1}{(x_N-x_{N-1})^3} y_{N-1} - \left[ \dfrac{1}{(x_N - x_{N-1})^3} + \dfrac{1}{(x_{N+1}-x_N)^3} \right] y_N + \dfrac{1}{(x_{N+1} - x_N)^3} y_{N+1} = ... \\
... \dfrac{1}{2} \left[ \dfrac{1}{(x_{N+1}-x_N)^2} y_{N+1}' + \left( \dfrac{1}{(x_{N+1}-x_N)^2} - \dfrac{1}{(x_N-x_{N-1})^2} \right) y_N - \dfrac{1}{(x_N - x_{N-1})^2} y_{N-1}' \right]
\end{multline}
















