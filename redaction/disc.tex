% disc.tex

\chapter{Méthodes de discrétisation}

\section{Discrétisation spatiale}

La résolution d'une équations aux dérivées partielles (EDP) par des
voies numériques nécessite une discrétisation du problème
initiales. Différentes méthodes ont été développées dans cet objectif,
en voici les exemples les plus courrants :

\begin{itemize}
\item Les méthodes d'éléments finis reposent sur la formulation
  variationnelle du problème initial,

\item Les méthodes de volumes finis utilisent la forme intégrale de
  l'EDP,

\item Les différences finies emploient la formulation forte de
  l'équation à résoudre.
\end{itemize}

Dans cette partie, nous présentons les méthodes de différences finies
qui permettent une discrétisation et une résolution
numérique. L'accent sera mis sur les schémas d'ordre élevé \cite{Desquesnes2007} et en
particulier sur les schémas compacts étudiées par S. K. Lele dans
\cite{Lele1991}. On effectuera aussi une étude sur les fréquences qui
jouent un r\^ole important dans la résolution des équations hyperboliques. 

\subsection{Méthodes de différences finies}

Dans la suite nous considérons une fonction :

$$u : [0, 1] \rightarrow \mathbb{R}$$

 supposée suffisament régulière. Nous supposons connues les valeurs de $u(x_i)$
(nous noterons $u_i$ cette valeur) et $u$ $1-$périodique (donc $u(0) =
 u(1)$.

Les points $x_i$ sont les points du maillage. Nous les définissons par
$x_0 = 0$ et $x_i = x_0 + i \times h$ avec $h = \frac{1}{N+1}$, pour
tout $ 0 \leq i \leq N+1 $.

L'objectif est ici de déterminer une valeur approchée de la dérrivée
première $\partial_xu(x_i)$ notée $\delta_x u_i$\footnote{On peut
  appliquer le m\^me raisonnement pour déterminer une approximationd es
dérivées d'ordre supérieur. Voir \cite{Lele1991} pour plus de détails.} gr\^ace à une relation de la forme :

\begin{equation}
\label{eq:differences finies}
\alpha_0 \delta_x u_i + \sum_{j=1}^M \alpha_j \left( \delta_x u_{i+j} + \delta_x
u_{i-j} \right) = \sum_{k=1}^N a_k \dfrac{u_{i+j} - u_{i-j}}{2kh}
\end{equation}

La difficulté étant de déterminer les coefficients les coefficients
$(\alpha_j)_{0 \leq j \leq M}$ et $(a_k)_{1 \leq k \leq N}$ pour que la
méthode soit aussi précise que souhaité.

\begin{definition}
On dit que la formule \eqref{eq:differences finies} est précise à
l'ordre $q$ si elle définit $\delta_x u_i$ tels que :
\begin{equation}
 \delta_x u_i - \partial_x u(x_i) = \mathcal{O} \left( h^q \right)
\end{equation}

lorsque $u$ est suffisament régulière.
\end{definition}

Dans la suite nous parlerons de schémas classiques et de schémas
compacts.

\subsection{Schémas Classiques }

Les schémas classiques sont construits de manière à pouvoir calculer
immédiatement $\delta_x u_i$ comme une combinaison linéaire de ses
voisins immédiats.

\begin{definition}
Un schéma d'approximation de la forme \eqref{eq:differences finies}
est dit classique lorsque pour tout $0 \leq i \leq N+1$, on a
$\alpha_i = \delta_{0,i}$\footnote{$\delta_{p,q}$ est le symbole de
  Kronecker. Il vaut $1$ lorsque $p=q$ et $0$ sinon.}
\end{definition}

Un schéma classique d'approximation de $\partial_x u$ est de la forme
:

\begin{equation}
 \delta_x u_i = \sum_{k=1}^N a_k \dfrac{u_{i+j} - u_{i-j}}{2kh}
\end{equation}

Les développements de Taylor nous donnent immédiatement la relation
suivante :

\begin{equation}
\label{eq:taylor}
 \sum_{k=1}^N a_k \dfrac{u(x+jh) - u(x-jh)}{2kh} = \sum_{p=0}^{P}
 \sum_{k=1}^N a_k \dfrac{(kh)^p + (-kh)^p}{p! 2 k h} \partial_x^{(p)} u(x)
 + \mathcal{O} \left( h^{P+1} \right)
\end{equation}

ainsi pour avoir une méthode d'ordre $P+1$, il faut que pour tout $0
\leq p \leq P$ :

\begin{equation}
\sum_{k=1}^N a_k \dfrac{k^p + (-k)^p}{k} = \delta_{p,1}
\end{equation}


\subsection{Schémas Compacts }









Dois je mettre une méthode pour traiter les CL? (voir article avec J.P. Chehab)

\subsection{Analyse spectrale}


\section{Discrétisation temporelle}

\subsection{Méthodes de Runge-Kutta : généralités}

\subsection{Quelques méthodes : Runge-Kutta explicite, DIRK et SDIRK}


\section{Filtrage numérique}

\subsection{\'Etude générale}

\subsection{Filtrage explicite de type S. Redonnet}

\subsection{Filtrage Implicite : cas de Visbal (titre à revoir)}


\section{\'Etude des schémas sur l'équation d'advection}

\subsection{Equation d 'advection périodique 1D}

\subsubsection{\'Etude théorique}

\subsection{Analyse de dispersion et dissipation de quelques schémas}

\section{Conclusion}
