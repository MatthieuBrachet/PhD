% advsph.tex

\chapter{\'Equation d'advection sphérique}

\section{Test de Williamson : Bump instationnaire}

\subsection{Préliminaires}

Soit $\mathbf{x} = (x,y,z)^T \in \mathbb{S}_R^2$ un point de la sphère de centre $O$ et de rayon $R$. Il existe alors $\lambda$ et $\theta$ , respectivement longitude et latitude, dans $] 0, 2 \pi ] \times ] - \pi /2, \pi/2 [ $ tels que :

\begin{equation}
\left\lbrace 
\begin{array}{rcl}
x & = & R \cos \theta \cos \lambda \\
y & = & R \cos \theta \sin \lambda \\
z & = & R \sin \theta
\end{array}
\right.
\end{equation}

Ainsi :

\begin{equation}
\dfrac{d \mathbf{x}}{dt} = u \mathbf{e}_{\lambda} + v \mathbf{e}_{\theta}
\end{equation}

où $\mathbf{e}_{\lambda}$ et $\mathbf{e}_{\theta}$ sont donnés dans la remarque \ref{base_lonlat}. Par identification, on a :

\begin{equation}
\left\lbrace 
\begin{array}{rcl}
u & = & R cos ( \theta ) \dfrac{d \lambda}{dt} \\
v & = & R \dfrac{d \theta}{dt}
\end{array}
\right.
\end{equation}.

Lorsque $\mathbf{x}$ est transporté parallèlement à l'équateur, à vitesse angulaire constante, on a :

\begin{equation}
\left\lbrace 
\begin{array}{rcl}
\dfrac{d \lambda}{dt} & = & \omega \\
\dfrac{d \theta}{dt} & = & 0
\end{array}
\right.
\end{equation}.

\subsection{Résolution exacte}

\subsubsection{Cas 1 : transport parallère à l'équateur}

Le but de cette partie est de résoudre l'équation d'advection sphèrique suivante :

\begin{equation}
\label{eq:advection spherique 1}
\left\lbrace
\begin{array}{r cl}
\dfrac{\partial h}{\partial t} + \mathbf{c} ( \mathbf{x} ) \cdot \nabla_ T h & = & 0 \\
h(\mathbf{x},0) & = & h_0 ( \mathbf{x} )
\end{array}
\right. \text{ pour tout } \mathbf{x} \in \mathbf{S}_R^2 \text{ et } t \geq 0
\end{equation}

Avec $\mathbf{c} ( \mathbf{x} ) = R \omega cos ( \theta )$ \footnote{Dans ce premier cas, le transport est parallèle à l'équateur et à vitesse constante.}.
On cherche la solution par la méthode des caractéristiques, sur la caractéristique $t \rightarrow\mathbf{x}(t) = (\lambda (t), \theta(t) )$ on a :

\begin{equation}
\left\lbrace
\begin{array}{rcl}
\partial_t \lambda & = & u_0 cos ( \theta ) \\
\partial_t \theta & = & 0
\end{array}
\right.
\end{equation}

d'où : $\lambda(t) = \lambda_0 + \Omega t$ et $\theta(t) = \theta_0$ avec $\Omega = u_0 cos ( \theta )$.

Ainsi la solution de \eqref{eq:advection spherique 1} est :

\begin{equation}
h( \mathbf{x}, t ) = h_0 ( \lambda - \Omega t, \theta ) = h_0 ( R_{-t}  \mathbf{x} )
\end{equation}

Avec $R_{-t}$ la rotation autour de l'axe $(Oz)$ d'angle $-\Omega t$.


\subsection{Tests numériques}

\section{Test de Nair et Machenhauer}

\section{Test de Nair et Jablonowski}