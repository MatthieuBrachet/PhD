% opsph.tex
% opérateurs sphériques sur la sphère

\chapter{Définition et discrétisation des opérateurs sphérique sur la sphére}

\section{Discrétisation sur la sphère : la cube-sphère}

La construction d'un maillage sur la sphère présente de nombreuses difficultés, Dans \ref{Ronchi1996}, on présente les critères pour avoir un "bon" maillage de la sphère :

\begin{itemize}
\item absence de singularités,
\item préserver la forme générale des équation hydrostatiques,
\item maillage aussi régulier que possible,
\item ...
\end{itemize}

En 1972, Robert Sadourny propose un maillage basé sur la projection d'un cube pour mailler la sphère \ref{Sadourny1972}.

Soit $C$ un cube centré sur $O=(0,0,0)$ inscrit dans une sphère de rayon R $\mathbb{S}_R$ de centre $O$.
Si $c^{(K)}$ est un point de la $K_{\text{ième}}$ face du cube ($K \in I, II, III, IV, V, VI$) 
Le point $c^{(K)}$ est projeté sur la sphère. Si ce point est un point de maillage du cube, il permet par projection de mailler la sphère. 

\subsection{Construction}

Dans 

\subsection{Coordonnées}

\section{Gradient sphérique}

Soit $u : \mathbb{S}^2 \longrightarrow \mathbb{R}$ une fonction sur la sphère. Le gradient de $u$ est donné par :

\begin{equation}
\label{eq:gradCS}
\nabla_T u = \dfrac{\partial u}{\partial \xi} \mathbf{g}^{\xi} + \dfrac{\partial u}{\partial \eta} \mathbf{g}^{\eta}
\end{equation}

où $( \mathbf{g}^{\xi}, \mathbf{g}^{\eta})$ est la base duale de $( \mathbf{g}_{\xi}, \mathbf{g}_{\eta})$.

\subsection{Définition}

\subsection{Schémas compact pour le calcul du gradient}

\section{Divergence sphérique}

\subsection{Définition}

$(\xi, \eta)$ définit un système de coordonnées sur la sphère sur chaque panel.

Soit un champ de vecteurs sur la sphère $\mathbf{F} :  \mathbb{S}^2 \longrightarrow \mathbb{TS}^2$, alors on peut calculer la divergence sur chaque panel grâce à :

\begin{equation}
\label{eq:divCS}
\nabla_T \cdot \mathbf{F} = \dfrac{1}{\sqrt{\overline{\mathbf{G}}}} \left[  \dfrac{\partial}{\partial \xi} \left( \sqrt{\overline{\mathbf{G}}} \mathbf{F} \cdot \mathbf{g}^{\xi} \right) + \dfrac{\partial}{\partial \eta} \left( \sqrt{\overline{\mathbf{G}}} \mathbf{F} \cdot \mathbf{g}^{\eta} \right)  \right]
\end{equation}

où $( \mathbf{g}^{\xi}, \mathbf{g}^{\eta})$ est la base duale de $( \mathbf{g}_{\xi}, \mathbf{g}_{\eta})$ et $\overline{\mathbf{G}} = |det \left( \overline{\mathbf{G}} \right)|$ le déterminant de la métrique en valeur absolue.

\subsection{Schémas compact pour le calcul de la divergence sphérique}

\section{Rotationnel sphérique}

\subsection{Définition}

\subsection{Calcul numérique du rotationnel sphérique}

\section{Laplacien sphérique}

\subsection{Définition}

\subsection{Calcul numérique du Laplacien sphérique}

\section{Intégration sur la sphère}

\subsection{Définition}

\subsection{Calcul numérique}