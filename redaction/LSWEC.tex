% LSWEC.tex
\chapter{Linearized Shallow Water Equation with Coriolis Force}

Le but de cette partie est de présenter deux test pour l'équation Shallow-Water linéarisée avec force de Coriolis (LSWEC) :

\begin{equation}
\label{LSWEC}
\left\lbrace
\begin{array}{r @{=} l}
\partial_t \mathbf{u} & -f \mathbf{k} \wedge \mathbf{u} - g \nabla \eta + \mathbf{F} \\
\partial_t \eta & -H \nabla \cdot \mathbf{u} + G
\end{array}
\right.
\end{equation}

ici $\mathbf{F}$ et $G$ sont des fonctions de forcage.

l'équation est résolue sur la sphère $\mathbb{S}_a^2$ dont le rayon est $a$. $\mathbf{x}$, un point de $\mathbb{S}_a^2$ est un point de latitude $\lambda$ et de longitude $\theta$ the longitude. Un vecteur $\mathbf{u} : \mathbb{S}_a^2 \rightarrow \mathbb{T}\mathbb{S}_a^2$ est déterminé par :

$$\mathbf{u} = u \mathbf{e}_{\theta} + v \mathbf{e}_{\lambda}.$$ 

Le paramètre de Coriolis est :

\begin{equation}
f=2 \omega sin ( \theta' ) = 2 \omega \left( -cos \lambda cos \theta sin \alpha + sin \theta cos \alpha \right)
\label{coriolis_parameter}
\end{equation}

En ce qui concerne la mesure de l'erreur, nous considérons l'erreur sous la forme suivante :

$$e_{i} = max_n \dfrac{\| \eta^n - \eta(t^n) \|_{i}}{\| \eta(0) \|_{i}}$$

$i \in \left\lbrace 1, 2, \infty \right\rbrace$.

\section{Linéarisation de l'équation Shallow Water}

On considère l'équation Shallow Water sans reliefs ($\eta^{\star} \equiv 0$) :

\begin{equation}
\label{eq:SWE_without relief}
\left\lbrace
\begin{array}{rcl}
\dfrac{\partial \mathbf{u}}{\partial t} + \left( \mathbf{u} \cdot \nabla \right) \mathbf{u} + f \mathbf{k} \wedge \mathbf{u} + g \nabla \eta & = & \mathbf{0} \\
\dfrac{\partial \eta}{\partial t} + \nabla \cdot \left( \eta \mathbf{u} \right) & = & 0
\end{array}
\right.
\end{equation}

Pour linéariser ce système autour de la solution stationnaire $(H, \overline{\mathbf{u}}) = (H,\mathbf{0})$, on considère le couple de solutions :

\begin{equation}
\left\lbrace
\begin{array}{rcl}
\eta & = & H + \tilde{\eta} \\
\mathbf{u} & = & \tilde{\mathbf{u}} \\
\end{array}
\right.
\end{equation}

En incorporant cette solution dans \eqref{eq:SWE_without relief} et en simplifiant les termes d'ordre 2, on obtient le nouveau système d'équations aux dérivées partielles :

\begin{equation}
\left\lbrace
\begin{array}{r @{=} l}
\partial_t \tilde{\mathbf{u}}& -f \mathbf{k} \wedge \tilde{\mathbf{u}} - g \nabla \tilde{\eta}\\
\partial_t \tilde{\eta} & -H \nabla \cdot \tilde{\mathbf{u}}
\end{array}
\right.
\end{equation}

Il s'agit de l'équation de Shallow Water linéarisée.
Pour simplifier les notations, dans ce chapitre, on remplacera $\tilde{\mathbf{u}}$ par $\mathbf{u}$ ainsi que $\tilde{\eta}$ par $\eta$.





\section{Quelques relations de conservation pour l'équation Shallow Water linéarisée}

Dans cette partie, nous considérons l'équation \eqref{LSWEC} sans forcage (\textit{i.e.} $\mathbf{F} \equiv \mathbf{0}$ et $G \equiv 0$).

\begin{remarque}
\label{remark_stokes}
Une remarque préliminaire est la suivante  :

Si $\Omega_1$ et $\Omega_2$ sont les deux hémisphères de $\mathbb{S}^2_a$ alors :
\begin{itemize}
\item $\Omega_1 \cup \Omega_2 = \mathbb{S}^2_a $,
\item $\overbrace{\Omega_1 \cap \Omega_2}^{\circ} = \varnothing$
\end{itemize}

par le théorème de Stokes, on a :

\begin{equation}
\gint_{\Omega_i}  \nabla \cdot \mathbf{u} = \gint_{\partial \Omega_i} \mathbf{u} \cdot \mathbf{n}_i
\end{equation}

pour $i \in \left\lbrace 1, 2 \right\rbrace$ et $\mathbf{n}_i$ est la normale extérieur de $\Omega_i$ sur $\mathbb{S}_a^2$ ($\mathbf{n}_i$ est tangent à la sphère).
Alors, comme $\mathbf{n_1} = -\mathbf{n_2}$ et $\partial \Omega_1 = \partial \Omega_2$ on a :

$$\gint_{\mathbb{S}_R^2}  \nabla \cdot \mathbf{u} = \gint_{\Omega_1}  \nabla \cdot \mathbf{u} + \gint_{\Omega_2}  \nabla \cdot \mathbf{u} = \gint_{\partial \Omega_1} \mathbf{u} \cdot \mathbf{n}_1 + \gint_{\partial \Omega_2} \mathbf{u} \cdot \mathbf{n}_2 = 0$$
\end{remarque}

\begin{proposition}
(Conservation de la masse)
La masse totale $\gint_{\mathbb{S}_a^2} \eta$ est constante au cours du temps.
\end{proposition}

\begin{proof}
Nous intégrons sur $\mathbb{S}_a^2$ la seconde équation de \eqref{LSWEC} et tenons compte de la remarque \ref{remark_stokes}.

$$\dfrac{\partial}{\partial t} \gint_{\mathbb{S}_a^2} \eta = \gint_{\mathbb{S}_a^2} \dfrac{\partial \eta}{\partial t} = \gint_{\mathbb{S}_a^2} \nabla \cdot \mathbf{u} = 0$$
\end{proof}

\begin{proposition}
(Conservation de l'énergie)
Si $\mathbf{F} = \mathbf{0}$, alors l'énergie totale $\gint_{\mathbf{S}_a^2 }g  \eta^2 + H | u |^2$ est constante par rapport au temps $t$.
\end{proposition}

\begin{proof}
\begin{itemize}
\item premièrement, notons que $\mathbf{u}$ est orthogonal à $\mathbf{k} \wedge \mathbf{u}$ alors :

$$\gint_{\mathbb{S}_a^2} \dfrac{\partial \mathbf{u}}{\partial t} \cdot \mathbf{u} = \dfrac{1}{2} \dfrac{\partial}{\partial t} \gint_{\mathbb{S}_a^2} \mathbf{u}^2 = -g \gint_{\mathbb{S}_a^2} \nabla \eta \cdot \mathbf{u}$$

en d'autres termes :

\begin{equation}
\dfrac{1}{2} \dfrac{\partial}{\partial t} \| u \|_{L^2(\mathbb{S}_a^2)}^2 = -g \gint_{\mathbb{S}_a^2} \nabla \eta \cdot \mathbf{u}
\label{energy_eq1}
\end{equation}

\item avec une idée similaire :

$$\dfrac{\partial \eta}{\partial t} \times \eta = -H \eta \nabla \cdot \mathbf{u} $$

et après intégrations sur $\mathbb{S}_a^2$ :

\begin{equation}
\dfrac{1}{2} \dfrac{\partial}{\partial t} \| \eta \|^2_{L^2(\mathbb{S}_a^2)} = -H \gint_{\mathbb{S}_a^2} \eta \nabla \cdot \mathbf{u}
\label{energy_eq2}
\end{equation}

\item Il est connu que :

\begin{equation}
\nabla \cdot \left( \mathbf{A} B \right) = \left( \mathbf{A} \cdot \nabla \right) B + \left( B \nabla \cdot \mathbf{A} \right)
\label{energy_eq3}
\end{equation}

en utilisant \eqref{energy_eq3} et la remarque \ref{remark_stokes}, nous avons le résultat :

\begin{equation}
\dfrac{\partial}{\partial t} \gint_{\mathbb{S}^2_a} \left( g  \eta^2 + H | u |^2 \right) = 0
\end{equation}
\end{itemize}
\end{proof}

\section{Solution exponentielle}

Pour ce test, $\alpha = 0$ et les fonctions de forcage $\mathbf{F}$ et $G$ sont ajustées pour que la solution exacte de \eqref{LSWEC} soit :

\begin{equation}
\left\lbrace
\begin{array}{r @{=} l}
u & \frac{\sqrt{gH}}{10} \psi ( \theta )  e^{-\sigma t} \\
v & 0 \\
\eta & \psi( \theta ) e^{-\sigma t}
\end{array}
\right.
\end{equation}

où :

\begin{equation*}
\psi ( \theta ) = 
\left\lbrace
\begin{array}{l}
0 \text{ if } \theta > \theta_1\\
\frac{1}{K}e^{\frac{1}{(\theta - \theta_0)(\theta-\theta_1)}} \text{ if } \theta_0 \leq \theta \leq \theta_1 \\
0 \text{ if } \theta < \theta_0\\
 
\end{array}
\right.
\label{galewski_fun}
\end{equation*}

$K = e^{-\dfrac{4}{(\theta_0 - \theta_1)^2}}$ est une constante de normalisation.

Pour les calculs numériques, nous choisissons $\sigma = 10^{-4}$, $\theta_0 = -\dfrac{3 \pi}{16}$ and $\theta_1 = \dfrac{3 \pi}{16}$.

ainsi qu'une condition CFL de la forme suivante :

\begin{equation}
CFL = \dfrac{c \Delta t}{a \Delta \xi}
\end{equation}

avec $c = max(c_{grav}, c_{cor})$, $c_{grav} = \sqrt{gH}$ et $c_{cor} = a \omega$, $\Delta \xi = \pi / 2N$.

Les tables \ref{CV_order4_hp10}, \ref{CV_order4_hp100} et \ref{CV_order4_hp1000} sont obtenues avec un schéma d'ordre 4 en temps (RK4) et d'ordre 4 en espace. Pour les Tables \ref{CV_order8_hp10}, \ref{CV_order8_hp100} et \ref{CV_order8_hp1000} le schéma est le même en temps mais nous utilisons un schéma compact à l'ordre 8 pour la divergence et 4 pour le gradient.


\begin{table}[ht]
\begin{center}
\begin{tabular}{c|c|c|c|c|c|c}
$N$ & $e_{\infty}$ & ordre & $e_2$ & ordre & $e_1$ & ordre \\ 
\hline 
\hline
$20$ & $1.2920 (-4)$ & - & $5.6417 (-5)$ & - & $4.7733 (-5)$ & - \\ 
\hline 
$40$ & $1.9620 (-5)$ & $2.9171$ & $5.4901 (-6)$ & $3.4823$ & $3.6578 (-6)$ & $3.8394$ \\ 
\hline 
$60$ & $3.7239 (-6)$ & $4.1827$ & $1.1867 (-6)$ & $3.8554$ & $7.3660 (-7)$ & $4.0336$ \\
\hline 
$80$ & $1.0788 (-6)$ & $4.3689$ & $3.7154 (-7)$ & $4.0951$ & $2.1765 (-7)$ & $4.2992$ \\ 
\hline 
$100$ & $4.1779(-7)$ & $4.2988$ & $1.4789 (-7)$ & $4.1745$ & $8.5641 (-8)$ & $4.2268$  \\ 
\end{tabular} 
\caption{Analyse de convergence $CFL=0.5$ et $H=10$.}
\label{CV_order4_hp10}
\end{center}
\end{table}

\begin{table}[ht]
\begin{center}
\begin{tabular}{c|c|c|c|c|c|c}
$N$ & $e_{\infty}$ & ordre & $e_2$ & ordre & $e_1$ & ordre \\ 
\hline 
\hline
$20$ & $0.0038$ & - & $0.0015$ & - & $0.0012$ & - \\ 
\hline 
$40$ & $3.8502 (-4)$ & $3.4220$ & $1.1233 (-4)$ & $3.8738$ & $7.0664 (-5)$ & $4.2331$ \\ 
\hline 
$60$ & $6.0715 (-5)$ & $4.6491$ & $1.7546 (-5)$ & $4.6731$ & $1.1497 (-5)$ & $4.5705$ \\
\hline 
$80$ & $1.4527 (-5)$ & $5.0434$ & $4.8548 (-6)$ & $4.5309$ & $3.2599 (-6)$ & $4.4446$ \\ 
\hline 
$100$ & $5.8355(-6)$ & $4.1331$ & $1.9027 (-6)$ & $4.2447$ & $1.2637 (-6)$ & $4.2944$  \\ 
\end{tabular} 
\caption{Analyse de convergence $CFL=0.5$ et $H=100$.}
\label{CV_order4_hp100}
\end{center}
\end{table}

\begin{table}[ht]
\begin{center}
\begin{tabular}{c|c|c|c|c|c|c}
$N$ & $e_{\infty}$ & ordre & $e_2$ & ordre & $e_1$ & ordre \\ 
\hline 
\hline
$20$ & $0.0629$ & - & $0.0283$ & - & $0.0277$ & - \\ 
\hline 
$40$ & $0.0044$ & $3.9757$ & $0.0014$ & $4.4935$ & $0.0011$ & $4.8219$ \\ 
\hline 
$60$ & $6.6608 (-4)$ & $4.7519$ & $1.9743 (-4)$ & $4.9304$ & $1.4907 (-4)$ & $5.0306$ \\
\hline 
$80$ & $1.6033 (-4)$ & $5.0222$ & $5.3660 (-5)$ & $4.5217$ & $4.2043 (-5)$ & $4.4634$ \\ 
\hline 
$100$ & $6.9510(-5)$ & $3.7874$ & $2.1324 (-5)$ & $4.1819$ & $1.6013 (-5)$ & $4.3743$  \\ 
\end{tabular} 
\caption{Analyse de convergence avec $CFL=0.5$ et $H=1000$.}
\label{CV_order4_hp1000}
\end{center}
\end{table}

\begin{table}[ht]
\begin{center}
\begin{tabular}{c|c|c|c|c|c|c}
$N$ & $e_{\infty}$ & ordre & $e_2$ & ordre & $e_1$ & ordre \\ 
\hline 
\hline
$20$ & $1.0303 (-4)$ & - & $5.1192 (-5)$ & - & $5.0087 (-5)$ & - \\ 
\hline 
$40$ & $1.1221 (-5)$ & $3.3140$ & $4.7304 (-6)$ & $3.5596$ & $3.3176 (-6)$ & $4.0573$ \\ 
\hline 
$60$ & $2.8566 (-6)$ & $3.4436$ & $1.0696 (-6)$ & $3.7421$ & $6.5310 (-7)$ & $4.0908$ \\
\hline 
$80$ & $1.0837 (-6)$ & $3.4180$ & $3.2702 (-7)$ & $4.1789$ & $1.8005 (-7)$ & $4.5438$ \\ 
\hline 
$100$ & $4.2901(-7)$ & $4.1993$ & $1.2667 (-7)$ & $4.2980$ & $6.7292 (-8)$ & $4.4600$  \\ 
\end{tabular} 
\caption{Analyse de convergence avec $CFL=0.5$ et $H=10$.}
\label{CV_order8_hp10}
\end{center}
\end{table}

\begin{table}[ht]
\begin{center}
\begin{tabular}{c|c|c|c|c|c|c}
$N$ & $e_{\infty}$ & ordre & $e_2$ & ordre & $e_1$ & ordre \\ 
\hline 
\hline
$20$ & $0.0025$ & - & $0.0012$ & - & $0.0011$ & - \\ 
\hline 
$40$ & $1.7414 (-4)$ & $3.9820$ & $6.5746 (-5)$ & $4.3409$ & $4.6314 (-5)$ & $4.7345$ \\ 
\hline 
$60$ & $2.7156 (-5)$ & $4.6772$ & $9.1924 (-6)$ & $4.9520$ & $5.9158 (-6)$ & $5.1795$ \\
\hline 
$80$ & $7.9310 (-6)$ & $4.3404$ & $2.3353 (-6)$ & $4.8320$ & $1.3738 (-6)$ & $5.1487$ \\ 
\hline 
$100$ & $2.7584(-6)$ & $4.7860$ & $8.5491 (-7)$ & $4.5538$ & $4.9340 (-7)$ & $4.6405$  \\ 
\end{tabular} 
\caption{Analyse de convergence avec $CFL=0.5$ et $H=100$.}
\label{CV_order8_hp100}
\end{center}
\end{table}


\begin{table}[ht]
\begin{center}
\begin{tabular}{c|c|c|c|c|c|c}
$N$ & $e_{\infty}$ & ordre & $e_2$ & ordre & $e_1$ & ordre \\ 
\hline 
\hline
$20$ & $0.0376$ & - & $0.0206$ & - & $0.0241$ & - \\ 
\hline 
$40$ & $0.0019$ & $4.4618$ & $7.3624 (-4)$ & $4.9794$ & $7.3567 (-4)$ & $5.2151$ \\ 
\hline 
$60$ & $1.5134 (-4)$ & $6.3682$ & $5.3116 (-5)$ & $6.6173$ & $5.5023 (-5)$ & $6.5266$ \\
\hline 
$80$ & $2.0280 (-5)$ & $7.0877$ & $7.1205 (-6)$ & $7.0863$ & $7.3563 (-6)$ & $7.0958$ \\ 
\hline 
$100$ & $4.9820(-6)$ & $6.3615$ & $1.9414 (-6)$ & $5.8892$ & $1.7583 (-6)$ & $6.4857$  \\ 
\end{tabular} 
\caption{Analyse de convergence avec $CFL=0.5$ et $H=1000$.}
\label{CV_order8_hp1000}
\end{center}
\end{table}


\section{Test sans forcage}


Ce second test est considéré sans forcage ($\mathbf{F} \equiv \mathbf{0}$ et $G \equiv 0$) et $\alpha$ est quelconque. Dans ce contexte, on construit une solution stationnaire telle que le champ de vitesse est zonale dans le système de coordonnées $(\lambda', \theta')$ donné dans \eqref{from classic to prime} avec $(\lambda_P, \theta_P) = (\pi, \pi /2 - \alpha)$ :

$$\mathbf{u}(\theta', \lambda') = u(\theta') \mathbf{e}_{\lambda'}$$

On montre facilement que la vitesse $ \mathbf{u}$ est à divergence nulle avec ce choix.

La perturbation de l'atmosphère $\eta$ est ensuite construite de telle sorte que l'on a :

\begin{equation}
f \mathbf{k} \wedge \mathbf{u} + g \nabla \eta = 0
\end{equation} 

Dans la base $(\mathbf{e}_{\lambda'}, \mathbf{e}_{\theta'})$ :

\begin{equation}
f u \mathbf{e}_{\theta'} + g \left[ \dfrac{1}{a \cdot cos \theta'} \dfrac{\partial \eta}{\partial \lambda'} \mathbf{e}_{\lambda'} + \dfrac{1}{a}\dfrac{\partial \eta'}{\partial \theta'} \mathbf{e}_{\theta'} \right] = 0
\end{equation}

par identification :

\begin{itemize}
\item $\dfrac{1}{a cos \theta'} \dfrac{\partial \eta'}{\partial \lambda'} = 0$, alors $\eta'$ est indépendant de $\lambda'$,

\item $f u + \dfrac{g}{a} \dfrac{\partial \eta'}{\partial \theta'} = 0$, alors (comme on a l'équation \eqref{coriolis_parameter}) :

\begin{equation}
\eta (\theta' ) = \eta_0 - \dfrac{2 \omega a}{g} \gint_0^{\theta'} sin(\tau) u(\tau) d \tau
\end{equation} 
\end{itemize}

ainsi une solution stationnaire est donnée. Nous choisissons $u(\theta') = u_0 \psi( \theta' )$ avec $\psi$ donné par \eqref{galewski_fun}. L'intégrale est calculée numériquement par la méthode des trapèzes composites.

On peut ainsi calculer :

\begin{equation}
\left\lbrace
\begin{array}{rcl}
\mathbf{u}(\mathbf{x}) & = & \mathbf{u}(P_{\alpha} \mathbf{x} ) \\
\eta (\mathbf{x}) & = & \eta(P_{\alpha} \mathbf{x} )
\end{array}
\right.
\end{equation}

où $P_{\alpha}$ est la rotation permettant de passer de l'axe Nord-Sud classique à l'axe Nord-Sud tourné d'un angle $\alpha$.

Numériquement, nous obtenons deux types de résultats :
\begin{itemize}
\item nous représentons l'erreur relative au temps $t^n$, $e_i^n$ avec $i \in \lbrace 1, 2, \infty \rbrace$,
\item Nous représentons aussi les relations de conservations relatives :
\begin{equation}
\dfrac{I^n}{I(0)}
\end{equation}
où  $I^n = \gint_{\mathbb{S}_a^2} \eta (t^n)$ la masse ou l'énergie $I^n = \gint_{\mathbb{S}_a^2} g \eta (t^n)^2 + H |u(t^n) |^2$ et $I(0)$ est la masse ou l'énergie initiale. Cette quantité $I^n/I(0)$ doit etre proche de $1$ si les quantités sont conservées.
\end{itemize} 


