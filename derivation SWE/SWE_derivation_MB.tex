\documentclass[10pt,a4paper]{amsart}

\usepackage[utf8]{inputenc}
\usepackage[francais]{babel}
\usepackage[T1]{fontenc}
\usepackage{amsmath}
\usepackage{tikz}
\usetikzlibrary[patterns]
\usepackage{amsfonts}
\usepackage{amssymb}
\usepackage{graphicx}
\usepackage[left=2cm,right=2cm,top=2cm,bottom=2cm]{geometry}

\newtheorem{exemple}{Exemple}
\newtheorem{theo}{Th\'eoreme}
\newtheorem{proposition}{Proposition}
\newtheorem{definition}{D\'efinition}
\newtheorem{lemme}{Lemme}
\newtheorem{remarque}{Remarque}

\def\gint{\displaystyle\int}

\author{Brachet Matthieu}
\title{Dérivation des équations Shallow Water}
\date\today

\begin{document}

\maketitle

\tableofcontents


\part{Cadre}

\section{\'Equations de Navier-Stokes incompressibles}

On se place dans le cadre d'un fluide placé sur un fond\footnote{Le fond ne dépend pas du temps.} irrégulier définit par :

\begin{equation}\label{Fond}
\psi(x,y,z) = z+b(x,y) = 0
\end{equation}

En chaque point $M(x,y,-b(x,y))$ du fond, la normale extérieure est :

\begin{equation}
\mathbf{n_f}(t,x,y,\xi(t,x,y)) = \dfrac{1}{\sqrt{1+\left( \dfrac{\partial b}{\partial x} \right)^2+\left( \dfrac{\partial b}{\partial y} \right)^2}} \begin{pmatrix}
\dfrac{\partial b}{\partial x} \\ 
\dfrac{\partial b}{\partial y} \\ 
1
\end{pmatrix} 
\end{equation}

De plus, la surface\footnote{La surface varie en fonction du temps} du fluide est donnée par:

\begin{equation}\label{Surface}
\phi(x,y,z) = z-\xi(t,x,y) = 0
\end{equation}

La normale extérieure au point $M(x,y,\xi(t,x,y))$ sur la surface et à l'instant $t$ est donnée par :

\begin{equation}
\mathbf{n_s}(t,x,y,\xi(t,x,y)) = \dfrac{1}{\sqrt{1+\left( \dfrac{\partial \xi}{\partial x} \right)^2+\left( \dfrac{\partial \xi}{\partial y} \right)^2}} \begin{pmatrix}
-\dfrac{\partial \xi}{\partial x} \\ 
-\dfrac{\partial \xi}{\partial y} \\ 
1
\end{pmatrix} 
\end{equation}





\begin{figure}
\vspace{1.2cm}
\begin{tikzpicture}
\draw [dotted] (-5,2) -- (2.2,2);
\draw (2.2,2) node[right]{$z=0$} ;

\draw [samples=100, domain=-5:3, scale=1] plot (\x, {0.8*sin(180/3.14*(\x+1.5))+2});
\draw [samples=20, domain=-5:3, scale=1] plot (\x, {0.2*sin(6*180/3.14*\x)-1});

\draw [>=stealth, <->] (0,2) -- (0,2.8);
\draw (0,2.4) node[right]{$\xi(t, \mathbf{x})$} ;

\draw [>=stealth, <->] (0,2) -- (0,-1);
\draw (0,0.5) node[right]{$b(\mathbf{x})$} ;

\draw [>=stealth, <->] (-.5,-1) -- (-.5,2.8);
\draw (-.5,0.9) node[left]{$H(t, \mathbf{x})$} ;

\fill[color=gray!5, pattern=north east lines] (3,-2) -- (-5,-2) --  plot [samples=20, domain=-5:3](\x, {0.2*sin(6*180/3.14*\x)-1}) ;
\end{tikzpicture}

\caption{Occupation du fluide}
\end{figure}








En chaque point $M(x,y,z)$ avec $-b(x,y) \leq z \leq \xi(t,x,y)$ le fluide est décrit par les équations de Navier-Stokes :

\begin{equation}\label{NSE}
  \left\{
      \begin{array}{rcl}
        \dfrac{\partial \rho}{\partial t} + \nabla \cdot \left( \rho V \right) & = & 0 \\
        \dfrac{\partial ( \rho V )}{\partial t} + \nabla \cdot \left( \rho V \otimes V \right) & = & \nabla \cdot \underline{\underline{\tau}} + \rho f \\
         \dfrac{\partial (\rho e)}{\partial t} + \nabla \cdot \left[ \left( \rho e \right) V \right] & = & \nabla \cdot \left( \underline{\underline{\tau}} \cdot V \right) + \rho f \cdot V - \nabla \cdot \overset{\cdot}{q} + r \\
      \end{array}
    \right.
\end{equation}

Dans ces équations, il est à noté que :

\begin{itemize}
\item $\rho$ désigne la masse volumique du fluide.

\item $V = (u, v, w)^T$ est le vecteur de vitesse d'une particule du fluide.

\item $f$ désigne les forces (massiques) sur le fluide.

\item $e$ est l'énergie totale par unité de masse.

\item $\overset{\cdot}{q}$ est le flux de chaleur perdue.

\item $r$ est la perte de chaleur.

\item $\overline{\overline{\tau}}$ est le tenseur des contraintes.
\end{itemize}

Le tenseur des contraintes prend la forme suivante :
\begin{equation}\label{tenseur_contraintes}
\underline{\underline{\tau}} = -p \underline{\underline{Id}} + \mu \left( \left( \nabla \otimes V \right) + \left( \nabla \otimes V \right)^T \right) + \eta \left( \nabla \cdot V \right) \underline{\underline{Id}}
\end{equation}

$p$ est la pression, $\mu$ la viscosité dynamique et $\eta$ est la viscosité de volume du fluide.

Supposons alors que le fluide est incompressible, cela se traduit par $\rho = \rho_0 = C^{ste}$. L'équation (\ref{NSE}.a) se réécrit alors :

\begin{equation}
\nabla \cdot V = 0
\end{equation}

En incorporant cette nouvelle écriture dans (\ref{NSE}.b) et en remarquant que les équations (\ref{NSE}.a) et (\ref{NSE}.b) forment un système de 2 équations à deux inconnues, on obtient les équations de Navier-Stokes incompressibles :

\begin{equation}\label{NSE_incompressible}
\left\{
\begin{array}{rcl}

\nabla \cdot V & = & 0 \\

\dfrac{\partial V}{\partial t} + \left( V \cdot \nabla \right) V & = & - \nabla p + \mu \nabla \underline{\underline{\tau}} + f

\end{array}
\right.
\end{equation}

Où on a abusivement noté $p$ la pression cinématique $p / \rho$ et $\mu$ la viscosité cinématique $\mu / \rho$.

En coordonnées cartésiennes $(x_1, x_2, x_3) = (x,y,z)$ et avec $U = (u_1, u_2, u_3)^{T} = (u, v, w)^{T}$ et en ne considérant que la force de gravité et la force de coriolis autour de l'axe $\mathbf{k} = (0, 0, 1)^T$ : $-2 f_c \mathbf{k} \wedge V$ ($f_c$ correspond à la vitesse de rotation autour de l'axe) , les équations de Navier-Stokes incompressibles \eqref{NSE_incompressible} s'écrivent :


\begin{equation}\label{NSE_incompressible_cartesien}
\left\{
\begin{array}{rcl}
\dfrac{\partial u}{\partial x} + \dfrac{\partial v}{\partial y}+\dfrac{\partial w}{\partial z}&=&0\\

\dfrac{\partial u}{\partial t} +  \dfrac{\partial u^2}{\partial x} + \dfrac{\partial u v}{\partial y} + \dfrac{\partial u w}{\partial z} + \dfrac{\partial p}{\partial x} & = & \dfrac{\partial}{\partial x}\left( 2 \mu \dfrac{\partial u}{\partial x} \right) + \dfrac{\partial}{\partial y}\left(\mu \dfrac{\partial u}{\partial y} + \mu \dfrac{\partial v}{\partial x}\right) + \dfrac{\partial}{\partial z}\left(\mu \dfrac{\partial u}{\partial z} + \mu \dfrac{\partial w}{\partial x}\right)+2 f_c v\\

\dfrac{\partial v}{\partial t} +  \dfrac{\partial uv}{\partial x} + \dfrac{\partial v^2}{\partial y} + \dfrac{\partial v w}{\partial z} + \dfrac{\partial p}{\partial y} & = &  \dfrac{\partial}{\partial x}\left(\mu \dfrac{\partial u}{\partial y} + \mu \dfrac{\partial v}{\partial x}\right) + \dfrac{\partial}{\partial y}\left( 2 \mu \dfrac{\partial v}{\partial y} \right) + \dfrac{\partial}{\partial z}\left(\mu \dfrac{\partial v}{\partial z} + \mu \dfrac{\partial w}{\partial y}\right)-2 f_c u\\

\dfrac{\partial w}{\partial t} +  \dfrac{\partial wv}{\partial x} + \dfrac{\partial wv}{\partial y} + \dfrac{\partial w^2}{\partial z} + \dfrac{\partial p}{\partial z} & = & -g + \dfrac{\partial}{\partial z}\left(\mu \dfrac{\partial u}{\partial z} + \mu \dfrac{\partial w}{\partial x}\right) + \dfrac{\partial}{\partial z}\left(\mu \dfrac{\partial v}{\partial z} + \mu \dfrac{\partial w}{\partial y}\right) + \dfrac{\partial}{\partial z}\left( 2 \mu \dfrac{\partial w}{\partial z} \right)
\end{array}
\right.
\end{equation}

Pour que le système d'équations aux dérivées partielles soit bien posé, nous avons besoin de conditions au bord.

\section{Conditions en surface}

En l'absence de fluide au dessus de la surface, on constate une condition limite de contrainte nulle au bord :

\begin{equation}\label{tenseur_surface}
\underline{\underline{\tau}} \cdot \mathbf{n_s} = 0 \hspace{2cm} \text{ sur } z=\xi(t,x,y)
\end{equation}

La vitesse normale du fluide est égale à la vitesse normale de la surface :

\begin{itemize}
\item Vitesse normale de la surface :

$$-\dfrac{1}{| \nabla \phi |} \dfrac{\partial \phi}{\partial t}$$

\item Vitesse normale du fluide :

$$V \cdot \mathbf{n_s}$$

\end{itemize}

De là, il découle directement :

\begin{equation} \label{surface_libre}
-\dfrac{\partial \phi}{\partial t} = V \cdot \nabla \phi
\end{equation}

L'équation \eqref{surface_libre} signifie qu'une particule à la surface du fluide reste sur la surface au cours du temps.

\begin{remarque}
L'équation (\ref{surface_libre}) se réécrit :
$$\dfrac{\partial \xi}{\partial t} + u \dfrac{\partial \xi}{\partial x} + v \dfrac{\partial \xi}{\partial y} - w = 0 \hspace{2cm} \text{ sur } z=\xi(t,x,y)$$
On rappelle que $\xi$ représente l'élévation du fluide au dessus de la hauteur moyenne.
\end{remarque}

Au dessus de la surface définie par \eqref{Surface}, il n'y a pas de fluide, la pression est nulle.

\begin{equation}\label{pression_surface}
p=0 \hspace{2cm} \text{ sur } z=\xi(t,x,y)
\end{equation}

\section{Conditions au fond}

Sur le fond, comme au niveau de la surface, il n'y a pas de contraintes normales :

\begin{equation}\label{contrainte_fond}
\overline{\overline{\sigma}} \cdot \mathbf{n_f} = 0 \hspace{2cm} \text{ sur } z=-b(x,y)
\end{equation}

De plus, le fluide ne pénètre pas dans le sol :

\begin{equation}\label{nonpenetration_fond}
U \cdot \nabla \psi = 0 \hspace{2cm} \text{ sur } z=-b(x,y)
\end{equation}

\begin{remarque}
L'équation (\ref{nonpenetration_fond}) se réécrit :
$$u \dfrac{\partial b}{\partial x} + v \dfrac{\partial b}{\partial y} + w = 0 \hspace{2cm} \text{ sur } z=-b(x,y)$$
\end{remarque}


\part{Hypothèse de faible profondeur}

Dans notre cas, on peut supposer que la profondeur est très faible par rapport à la longueur et à la largeur du domaine. en adimensionnant les équations précédentes, on peut voir ce que cela entraine.

\section{Variables sans dimensions}

On pose $H_c$ la profondeur caractéristique, $L$ est la longueur/largeur caractéristique du domaine de travail. Par faible profondeur, on a : $H_c/L = \varepsilon$. Si les vitesses caractéristiques horizontales sont notées $U$ et $V$, on a alors le temps caractéristique : $T=L/U=L/V$. La pression caractéristique est $P=U^2$. 

Dans ce cadre, on pose les constantes sans dimensions :

\begin{itemize}
\item $ \widetilde{x} = x/L$, variable horizontale en abscisse.

\item $ \widetilde{y} = y/L$, variable horizontale en ordonnée.

\item $ \widetilde{z} = z/H_c = z/(\varepsilon L)$, variable de profondeur. On rappelle que cette dernière est considérée comme faible.

\item $ \widetilde{t} = tU/L$, échelle de temps. Le temps caractéristique est donné par $L/U$.

\item $ \widetilde{p} = p/U^2$ est la pression sans dimensions.

\item $ \widetilde{u} = u/U$ composante sans dimension de vitesse selon $x$.

\item $ \widetilde{v} = v/V$ composante sans dimension de vitesse selon $y$.

\item $ \widetilde{w} = w/W=w/(\varepsilon U)$ représente la vitesse sans dimension. La faible profondeur a ici évidemment une influence.
\end{itemize}

On note que compte tenu de la faible profondeur, le paramètre $\varepsilon$ est destiné à tendre vers 0.

En plus de ces variables, on pose $\sigma = \dfrac{\mu}{LU}$, $G=\dfrac{gH_c}{U^2}$ et $F_c = (L f_c)/U$.

\section{\'Equations sans dimensions}

Réécrivons les équations de Navier-Stokes incompressibles  sans dimensions. Pour simplifier les écritures, nous ne notons pas les $\widetilde{ }$ sur les variables. Les équations (\ref{NSE_incompressible_cartesien}.a)-(\ref{NSE_incompressible_cartesien}.d) s'écrivent :

\begin{equation}\label{NSEI_SD_cartesien}
\left\{
\begin{array}{rcl}

\dfrac{\partial u}{\partial x} + \dfrac{\partial v}{\partial y}+\dfrac{\partial w}{\partial z} & = & 0 \\


\dfrac{\partial u}{\partial t} +  \dfrac{\partial u^2}{\partial x} + \dfrac{\partial u v}{\partial y} + \dfrac{\partial u w}{\partial z} + \dfrac{\partial p}{\partial x} & = & \dfrac{\partial}{\partial x}\left( 2 \sigma \dfrac{\partial u}{\partial x} \right) +  \ldots\\

 &  &  \hspace{1cm} \ldots \dfrac{\partial}{\partial y}\left(\sigma \dfrac{\partial u}{\partial y} + \sigma \dfrac{\partial v}{\partial x}\right) + \dfrac{\partial}{\partial z}\left(\dfrac{\sigma}{\varepsilon^2} \dfrac{\partial u}{\partial z} + \sigma \dfrac{\partial w}{\partial x}\right) + 2 F_c v
\\


\dfrac{\partial v}{\partial t} +  \dfrac{\partial uv}{\partial x} + \dfrac{\partial v^2}{\partial y} + \dfrac{\partial v w}{\partial z} + \dfrac{\partial p}{\partial y} & = &  \dfrac{\partial}{\partial x}\left(\sigma \dfrac{\partial u}{\partial y} + \sigma \dfrac{\partial v}{\partial x}\right) + \ldots\\

 &  & \hspace{1cm} \ldots \dfrac{\partial}{\partial y}\left( 2 \sigma \dfrac{\partial v}{\partial y} \right) +  \dfrac{\partial}{\partial z}\left(\dfrac{\sigma}{\varepsilon^2} \dfrac{\partial v}{\partial z} + \sigma \dfrac{\partial w}{\partial y}\right) - 2 F_c u\\


 \varepsilon^2 \left[ \dfrac{\partial w}{\partial t} +  \dfrac{\partial wv}{\partial x} + \dfrac{\partial wv}{\partial y} + \dfrac{\partial w^2}{\partial z} \right] + \dfrac{\partial p}{\partial z} & = & -G + \dfrac{\partial}{\partial x}\left(\sigma \dfrac{\partial u}{\partial z} + \varepsilon^2 \sigma \dfrac{\partial w}{\partial x}\right) + \ldots \\
 
  &  & \hspace{1cm} \ldots \dfrac{\partial}{\partial y}\left(\sigma \dfrac{\partial v}{\partial z} + \varepsilon^2 \sigma \dfrac{\partial w}{\partial y}\right) +  \dfrac{\partial}{\partial z}\left( 2 \sigma \dfrac{\partial w}{\partial z} \right)

\end{array}
\right.
\end{equation}


De la même manière, si $z=\xi(t,x,y)$ l'équation (\ref{tenseur_surface}) devient :


\begin{equation}\label{tenseur_surface_SD}
\left\{
\begin{array}{rcl}
\sigma \dfrac{\partial u}{\partial z} & = & \varepsilon^2 \left( -p\dfrac{\partial \xi}{\partial x} + 2 \sigma \dfrac{\partial u}{\partial x}\dfrac{\partial \xi}{\partial x} + \sigma \left( \dfrac{\partial v}{\partial x} + \dfrac{\partial u}{\partial y} \right) \dfrac{\partial \xi}{\partial y} - \sigma \dfrac{\partial w}{\partial x} \right)\\


\sigma \dfrac{\partial v}{\partial z} & = & \varepsilon^2 \left( \sigma \left( \dfrac{\partial v}{\partial x} + \dfrac{\partial u}{\partial y} \right)\dfrac{\partial \xi}{\partial x} - p\dfrac{\partial \xi}{\partial y} + 2 \sigma \dfrac{\partial v}{\partial y} \dfrac{\partial \xi}{\partial y} - \sigma \dfrac{\partial w}{\partial y} \right)\\


2 \sigma \dfrac{\partial w}{\partial z} - p & = &\sigma \varepsilon^2 \left( \dfrac{\partial w}{\partial x} \dfrac{\partial \xi}{\partial x} +  \dfrac{\partial w}{\partial y} \dfrac{\partial \xi}{\partial y} \right) + \sigma \left( \dfrac{\partial u}{\partial z} \dfrac{\partial \xi}{\partial x} + \dfrac{\partial v}{\partial z} \dfrac{\partial \xi}{\partial y} \right) 

\end{array}
\right.
\end{equation}



En remplacant $\sigma \partial u / \partial z$ et $\sigma \partial v / \partial z$ issus de (\ref{tenseur_surface_SD}.a) et (\ref{tenseur_surface_SD}.b) dans (\ref{tenseur_surface_SD}.c), on obtient la relation :

\begin{equation}\label{noncisaillement_surface_SD}
p - 2 \sigma \dfrac{\partial w}{\partial z} = \varepsilon^2 \left[ \dfrac{\partial \xi}{\partial x} \left( p\dfrac{\partial \xi}{\partial x} - 2 \sigma \dfrac{\partial u}{\partial x} \dfrac{\partial \xi}{\partial x} -  \sigma \left( \dfrac{\partial v}{\partial x} + \dfrac{\partial u}{\partial y} \right)\dfrac{\partial \xi}{\partial y} \right) + \dfrac{\partial \xi}{\partial y} \left( p\dfrac{\partial \xi}{\partial y} - 2 \sigma \dfrac{\partial v}{\partial y} \dfrac{\partial \xi}{\partial y} - \sigma \left( \dfrac{\partial v}{\partial x} + \dfrac{\partial u}{\partial y} \right)\dfrac{\partial \xi}{\partial x} \right) \right]
\end{equation}


Au fond, $z=-b(x,y)$, l'équation (\ref{contrainte_fond}) prend la forme suivante :

\begin{equation}\label{tenseur_fond_SI}
\left\{
\begin{array}{rcl}
\sigma \dfrac{\partial u}{\partial z} & = & \varepsilon^2 \left( p\dfrac{\partial b}{\partial x} - 2 \sigma \dfrac{\partial u}{\partial x}\dfrac{\partial b}{\partial x} - \sigma \left( \dfrac{\partial v}{\partial x} + \dfrac{\partial u}{\partial y} \right) \dfrac{\partial b}{\partial y} - \sigma \dfrac{\partial w}{\partial x} \right)\\

\sigma \dfrac{\partial v}{\partial z} & = & \varepsilon^2 \left( -\sigma \left( \dfrac{\partial v}{\partial x} + \dfrac{\partial u}{\partial y} \right)\dfrac{\partial b}{\partial x} + p\dfrac{\partial b}{\partial y} - 2 \sigma \dfrac{\partial v}{\partial y} \dfrac{\partial b}{\partial y} - \sigma \dfrac{\partial w}{\partial y} \right)\\

-2 \sigma \dfrac{\partial w}{\partial z} + p & = & \sigma \varepsilon^2 \left( \dfrac{\partial w}{\partial x} \dfrac{\partial b}{\partial x} +  \dfrac{\partial w}{\partial y} \dfrac{\partial b}{\partial y} \right) + \sigma \left( \dfrac{\partial u}{\partial z} \dfrac{\partial b}{\partial x} + \dfrac{\partial v}{\partial z} \dfrac{\partial b}{\partial y} \right)

\end{array}
\right.
\end{equation}



\section{Faibles variations}

En supposant, $\sigma = \varepsilon \sigma_0$ (faible viscosité), par la faible profondeur $\varepsilon \longrightarrow 0$, et par les équation (\ref{NSEI_SD_cartesien}.b), (\ref{tenseur_surface_SD}.a) et (\ref{tenseur_fond_SI}.a) on a :

\begin{eqnarray}
\label{faible_var_u}
\dfrac{\partial^2 u}{\partial z^2} = \mathcal{O} \left( \varepsilon \right)\\
\label{faible_var_u_surface}
\dfrac{\partial u}{\partial z} = \mathcal{O} \left( \varepsilon \right) \text{ sur } z=\xi(t,x,y) \\
\label{faible_var_u_fond}
\dfrac{\partial u}{\partial z} = \mathcal{O} \left( \varepsilon \right) \text{ sur } z=-b(x,y)
\end{eqnarray}

De là, il découle le résultat :

\begin{proposition}\label{prop_variation_u}
\begin{equation}
u(t,x,y,z)=u(t,x,y,0)+\mathcal{O} \left( \varepsilon \right)
\end{equation}
\end{proposition}

De la même manière, grâce aux équations (\ref{NSEI_SD_cartesien}.c), (\ref{tenseur_surface_SD}.b) et (\ref{tenseur_fond_SI}.b), on peut montrer que :

\begin{proposition}\label{prop_variation_v}
\begin{equation}
v(t,x,y,z)=v(t,x,y,0)+\mathcal{O} \left( \varepsilon \right)
\end{equation}
\end{proposition}

Grâce à l'équation (\ref{NSEI_SD_cartesien}.d) :

$$ \dfrac{\partial p}{\partial z} = - G + \dfrac{\partial}{\partial x} \left( \sigma \dfrac{\partial u}{\partial z} \right) + \dfrac{\partial}{\partial y} \left( \sigma \dfrac{\partial v}{\partial z} \right) + \dfrac{\partial}{\partial z} \left( 2 \sigma \dfrac{\partial u}{\partial z} \right) + \mathcal{O}\left( \varepsilon^2 \right)$$

Alors, en intérgant entre $z$ et $\xi(t,x,y)$ et en utilisant la relation \eqref{noncisaillement_surface_SD}, on obtient :

\begin{eqnarray*}
p(z) & = & G(\xi(t,x,y) - z ) + \gint_{\xi(t,x,y)}^z \left[ \dfrac{\partial}{\partial x} \left( \sigma \dfrac{\partial u}{\partial z} \right) + \dfrac{\partial}{\partial y} \left( \sigma \dfrac{\partial v}{\partial z} \right) \right] dz + 2 \sigma \dfrac{\partial w}{\partial z} + \mathcal{O} \left( \varepsilon^2 \right)\\
     & = & G(\xi(t,x,y) - z ) - \sigma \left( \dfrac{\partial u}{\partial x} + \dfrac{\partial v}{\partial y} \right) - \sigma \left( \dfrac{\partial u}{\partial x} + \dfrac{\partial v}{\partial y} \right)_{| z=\xi(t,x,y)} + \mathcal{O} \left( \varepsilon^2 \right) \text{en intégrant et par (\ref{NSEI_SD_cartesien}.a)}
\end{eqnarray*}

Ainsi, en négligeant les termes en $\mathcal{O} \left( \varepsilon \right)$ :

\begin{equation}\label{hyp_hydrostatique_ordre1}
p(z) = G(\xi(t,x,y) - z )
\end{equation}

ou ceux en $\mathcal{O} \left( \varepsilon^2 \right)$ :

\begin{equation}\label{hyp_hydrostatique_ordre2}
p(z) = G(\xi(t,x,y) - z ) - \sigma \left( \dfrac{\partial u}{\partial x} + \dfrac{\partial v}{\partial y} \right) - \sigma \left( \dfrac{\partial u}{\partial x} + \dfrac{\partial v}{\partial y} \right)_{| z=\xi(t,x,y)}
\end{equation}


Dans la suite, nous noterons $\textbf{x} = (x,y)^T$.

Soit $H(t,\textbf{x}) = \xi(t, \textbf{x}) + b(\textbf{x}) $ la hauteur du fluide. Si on définit les vitesses moyennes :

\begin{itemize}
\item $\overline{u}(t,\textbf{x}) = \dfrac{1}{H(t,\textbf{x})}\gint_{-b(\textbf{x})}^{\xi(t,\textbf{x})} u(t,\textbf{x},z) dz$ et

\item $\overline{v}(t,\textbf{x}) = \dfrac{1}{H(t,\textbf{x})}\gint_{-b(\textbf{x})}^{\xi(t,\textbf{x})} v(t,\textbf{x},z) dz$. 
\end{itemize}

On peut aussi définir $\widetilde{u}$ et $\widetilde{v}$ par :

$$u = \overline{u} + \widetilde{u}$$

$$v = \overline{v} + \widetilde{v}$$

En un certain sens, $\widetilde{u}$ et $\widetilde{v}$ représentent les variations de $u$ et $v$ sur la hauteur.

Dès lors, par les proposition \ref{prop_variation_u} et \ref{prop_variation_v} :

\begin{equation} \label{variations_faibles}
\left\{
\begin{array}{rcl}
\widetilde{u} & = & \mathcal{O}\left( \varepsilon \right) \\
\widetilde{v} & = & \mathcal{O}\left( \varepsilon \right) 
\end{array}
\right.
\end{equation}

On remarque aisément que

$$\gint_{-b(\textbf{x})}^{\xi(t,\textbf{x})} \widetilde{u}(t,\textbf{x},z) dz = 0$$

et

$$\gint_{-b(\textbf{x})}^{\xi(t,\textbf{x})} \widetilde{v}(t,\textbf{x},z) dz = 0.$$


De là, il découle facilement :

\begin{equation} \label{integrale1}
\gint_{-b(\textbf{x})}^{\xi(t,\textbf{x})} u(t,\textbf{x},z) v(t,\textbf{x},z) dz = H \overline{u}(t,\textbf{x}) \overline{v}(t,\textbf{x}) + \gint_{-b(\textbf{x})}^{\xi(t,\textbf{x})} \widetilde{u}(t,\textbf{x},z) \widetilde{v}(t,\textbf{x},z) dz
\end{equation}

ainsi que 

\begin{equation} \label{integrale2}
\gint_{-b(\textbf{x})}^{\xi(t,\textbf{x})} u(t,\textbf{x},z) u(t,\textbf{x},z) dz = H \overline{u}(t,\textbf{x})^2 + \gint_{-b(\textbf{x})}^{\xi(t,\textbf{x})} \widetilde{u}(t,\textbf{x},z)^2 dz
\end{equation}

et la même équation en $v$.

\part{Equations Shallow-Water} 

\section{\'Equation de continuité}


Comme le fluide est incompressible, on a l'équation de conservation de la masse suivante (\ref{NSEI_SD_cartesien}.a), on intègre cette équation entre $-b(\textbf{x})$ et $\xi(t,\textbf{x})$ en $z$.

$$\begin{array}{rcl}
0 & = & \gint_{-b(\textbf{x})}^{\xi(t,\textbf{x})}\nabla_{(\textbf{x},z)} \cdot V(t,\textbf{x},z) dz\\
  & = & \dfrac{\partial}{\partial x} \gint_{-b(\textbf{x})}^{\xi(t,\textbf{x})} u(t,\textbf{x},z) dz + \dfrac{\partial}{\partial y} \gint_{-b(\textbf{x})}^{\xi(t,\textbf{x})} v(t,\textbf{x},z) dz \\
  &   & - \left( u(t,\textbf{x}, \xi(t,\textbf{x})) \dfrac{\partial \xi}{\partial x}(t,\textbf{x}) + v(t,\textbf{x}, \xi(t,\textbf{x})) \dfrac{\partial \xi}{\partial y}(t,\textbf{x}) - w(t,\textbf{x}, \xi(t,\textbf{x})) \right)\\
    &   & - \left( u(t,\textbf{x}, -b(t,\textbf{x})) \dfrac{\partial b}{\partial x}(t,\textbf{x}) + v(t,\textbf{x}, -b(t,\textbf{x})) \dfrac{\partial b}{\partial y}(t,\textbf{x}) + w(t,\textbf{x}, -b(t,\textbf{x})) \right)
\end{array}$$
  
En utilisant les contraintes physiques (\ref{nonpenetration_fond}) et (\ref{surface_libre}) on a alors :

\begin{equation}\label{eq_conservmasse_SWE}
\dfrac{\partial \xi}{\partial t}(t,\textbf{x}) + \dfrac{\partial}{\partial x} \gint_{-b(\textbf{x})}^{\xi(t,\textbf{x})} u(t,\textbf{x},z) dz + \dfrac{\partial}{\partial y} \gint_{-b(\textbf{x})}^{\xi(t,\textbf{x})} v(t,\textbf{x},z) dz
\end{equation}

En utilisant les notations de vitesse moyenne, l'équation (\ref{eq_conservmasse_SWE}) de conservation de la masse s'écrit (sans noter les dépendances) :

\begin{equation}\label{eq_SWE1}
\dfrac{\partial H}{\partial t} + \dfrac{\partial}{\partial x} \left( H \overline{u} \right)+ \dfrac{\partial}{\partial y} \left( H \overline{v} \right) = 0
\end{equation}

\begin{remarque}
On remarque que nous n'avons fait aucune hypothèse. Seules les conditions physiques de non pénétration dans le sol et de surface libre ont été utilisées.
\end{remarque}


\section{\'Equation de quantité de mouvement}

Le membre de gauche de l'équation de quantité de mouvement (\ref{NSEI_SD_cartesien}.b), en mettant la pression à droite, s'écrit :

$$\dfrac{\partial u(t, \textbf{x}, z)}{\partial t}  + \dfrac{\partial u(t, \textbf{x}, z)^2}{\partial x} + \dfrac{\partial u(t, \textbf{x}, z) v(t, \textbf{x}, z)}{\partial y} + \dfrac{\partial u(t, \textbf{x}, z) w(t, \textbf{x}, z)}{\partial z} $$

on intègre cette équation :

$$\gint_{-b(\textbf{x})}^{\xi(t, \textbf{x})} \dfrac{\partial u(t, \textbf{x}, z)}{\partial t}  + \dfrac{\partial u(t, \textbf{x}, z)^2}{\partial x} + \dfrac{\partial u(t, \textbf{x}, z) v(t, \textbf{x}, z)}{\partial y} + \dfrac{\partial u(t, \textbf{x}, z) w(t, \textbf{x}, z)}{\partial z}  dz$$

et on obtient alors grâce à (\ref{nonpenetration_fond}) et (\ref{surface_libre}) :


\begin{eqnarray*}
\dfrac{\partial H \overline{u}}{\partial t}  & + &  \dfrac{\partial}{\partial x} \gint_{-b(\textbf{x})}^{\xi(t, \textbf{x})} u(t, \textbf{x}, z)^2 dz + \dfrac{\partial}{\partial y} \gint_{-b(\textbf{x})}^{\xi(t, \textbf{x})} u(t, \textbf{x}, z) v(t, \textbf{x}, z) dz \ldots \\
 &  & {} \ldots-u(t, \textbf{x}, \xi(t, \textbf{x})) \left( \dfrac{\partial \xi (t, \textbf{x})}{\partial t} + u(t, \textbf{x}, \xi(t, \textbf{x})) \dfrac{\partial \xi (t, \textbf{x})}{\partial x} + v(t, \textbf{x}, \xi(t, \textbf{x})) \dfrac{\partial \xi (t, \textbf{x})}{\partial y} - w(t, \textbf{x}, \xi(t, \textbf{x})) \right)\\
 &  & {} \ldots-u(t, \textbf{x}, -b(\textbf{x})) \left( u(t, \textbf{x}, -b(\textbf{x})) \dfrac{\partial b (\textbf{x})}{\partial x} + v(t, \textbf{x}, -b(\textbf{x})) \dfrac{\partial b (t, \textbf{x})}{\partial y} + w(t, \textbf{x}, -b\textbf{x})) \right) + \mathcal{O}\left( \varepsilon \right)\\
 & = & {} \dfrac{\partial H \overline{u}}{\partial t} + \dfrac{\partial}{\partial x} \gint_{-b(\textbf{x})}^{\xi(t, \textbf{x})} u(t, \textbf{x}, z)^2 dz + \dfrac{\partial}{\partial y} \gint_{-b(\textbf{x})}^{\xi(t, \textbf{x})} u(t, \textbf{x}, z) v(t, \textbf{x}, z) dz + \mathcal{O}\left( \varepsilon \right) 
\end{eqnarray*}

Alors en utilisant les écritures en vitesse moyenne, \eqref{integrale1} et \eqref{integrale2}, ainsi que la faible profondeur et \eqref{variations_faibles}, on obtient (sans noter les dépendances) lorsque $\varepsilon \rightarrow 0$ (on néglie les termes en $\mathcal{O}\left( \varepsilon \right)$) :

\begin{equation} \label{SWE_2_left}
\dfrac{\partial H \overline{u}}{\partial t} + \dfrac{\partial}{\partial x} \left( H \overline{u}^2 \right) + \dfrac{\partial}{\partial x} \left( H \overline{u} \overline{v} \right)
\end{equation}

Cela revient à négliger les termes en $\mathcal{O}\left( \varepsilon \right)$.

En ce qui concerne le membre de droite (\ref{NSEI_SD_cartesien}.b):

\begin{eqnarray*}
-\dfrac{\partial p(t, \mathbf{x}, z)}{\partial x} & + & \dfrac{\partial}{\partial x} \left( 2 \sigma \dfrac{\partial u(t, \mathbf{x}, z)}{\partial x} \right) + \dfrac{\partial}{\partial y} \left( \sigma \dfrac{\partial u(t, \mathbf{x}, z)}{\partial y} + \sigma \dfrac{\partial v(t, \mathbf{x}, z)}{\partial x}\right) + \ldots\\
&  & \ldots + \dfrac{\partial}{\partial z} \left( \dfrac{\sigma}{\varepsilon^2} \dfrac{\partial u(t, \mathbf{x}, z)}{\partial z} + \sigma \dfrac{\partial w(t, \mathbf{x}, z)}{\partial x}\right) - \Omega v
\end{eqnarray*}

intégrons cette équation terme par terme :

\begin{eqnarray*}
-\gint_{-b(\mathbf{x})}^{\xi(t, \mathbf{x})} \dfrac{\partial p(t, \mathbf{x}, z)}{\partial x}dz & = & - \dfrac{\partial}{\partial x} \gint_{-b(\mathbf{x})}^{\xi(t, \mathbf{x})} p(t, \mathbf{x}, z) dz + p(t, \mathbf{x}, \xi(t, \mathbf{x})) \dfrac{\partial \xi(t, \mathbf{x})}{\partial x} + p(t, \mathbf{x}, -b(\mathbf{x})) \dfrac{\partial b(\mathbf{x})}{\partial x}\\
& = & \dfrac{G}{2} \dfrac{\partial H(t, \mathbf{x})^2}{\partial x}+ p(t, \mathbf{x}, \xi(t, \mathbf{x})) \dfrac{\partial \xi(t, \mathbf{x})}{\partial x} + + p(t, \mathbf{x}, -b(\mathbf{x})) \dfrac{\partial b(\mathbf{x})}{\partial x}\\
\end{eqnarray*}

Ce dernier résultat étant obtenu grâce à l'équation \eqref{hyp_hydrostatique_ordre1}.

De plus, 

$$- \Omega \gint_{-b(\mathbf{x})}^{\xi(t, \mathbf{x})} v dz = - \Omega H \overline{v} $$

Enfin :

\begin{eqnarray*}
\gint_{-b(\mathbf{x})}^{\xi(t, \mathbf{x})} \dfrac{\partial}{\partial x} \left( 2 \sigma \dfrac{\partial u}{\partial x}(t, \mathbf{x}, z) \right) dz & = & \dfrac{\partial}{\partial x} 2 \sigma \gint_{-b(\mathbf{x})}^{\xi(t, \mathbf{x})} \dfrac{\partial u}{\partial x}(t, \mathbf{x}, z) dz - 2 \sigma \dfrac{\partial u}{\partial x}(t, \mathbf{x}, \xi(t, \mathbf{x}) \dfrac{\partial \xi}{\partial x}(t, \mathbf{x})\\
&  & \ldots - 2 \sigma \dfrac{\partial u}{\partial x}(t, \mathbf{x}, -b(\mathbf{x}) \dfrac{\partial b}{\partial x}(\mathbf{x})
\end{eqnarray*}

ainsi que :

\begin{eqnarray*}
\gint_{-b(\mathbf{x})}^{\xi(t, \mathbf{x})} \dfrac{\partial}{\partial y} \left( \sigma \dfrac{\partial u}{\partial y}(t, \mathbf{x}, z) + \sigma \dfrac{\partial v}{\partial x}(t, \mathbf{x}, z) \right) dz & = & \dfrac{\partial}{\partial y} 2 \sigma \gint_{-b(\mathbf{x})}^{\xi(t, \mathbf{x})} \left( \sigma \dfrac{\partial u}{\partial y}(t, \mathbf{x}, z) + \sigma \dfrac{\partial v}{\partial x}(t, \mathbf{x}, z) \right) dz\\
&  & \ldots - \left( \sigma \dfrac{\partial u}{\partial y}(t, \mathbf{x}, \xi(t, \mathbf{x}) + \sigma \dfrac{\partial v}{\partial x}(t, \mathbf{x}, \xi(t, \mathbf{x}) \right)\dfrac{\partial \xi}{\partial x}(t, \mathbf{x})\\
&  & \ldots - \left( \sigma \dfrac{\partial u}{\partial y}(t, \mathbf{x}, -b(\mathbf{x}) + \sigma \dfrac{\partial v}{\partial x}(t, \mathbf{x}, -b( \mathbf{x}) \right)\dfrac{\partial b}{\partial x}(\mathbf{x}))\\
\end{eqnarray*}

Pour le terme en $\partial / \partial z$ on peut obtenir directement :

\begin{eqnarray*}
\gint_{-b(\mathbf{x})}^{\xi(t, \mathbf{x})} \dfrac{\partial}{\partial z} \left( \sigma \dfrac{\partial u}{\partial y}(t, \mathbf{x}, z) + \sigma \dfrac{\partial v}{\partial x}(t, \mathbf{x}, z) \right) dz & = & \sigma \frac{\partial u}{\partial z}(t, \mathbf{x}, \xi(t, \mathbf{x}) + \sigma \dfrac{\partial w}{\partial x}(t, \mathbf{x}, \xi(t, \mathbf{x}) \\
&  & \ldots - \sigma \frac{\partial u}{\partial z}(t, \mathbf{x}, -b(\mathbf{x}) + \sigma \dfrac{\partial w}{\partial x}(t, \mathbf{x}, -b(\mathbf{x}) 
\end{eqnarray*}

Il reste à sommer et à simplifier les termes de bords grâce à \eqref{tenseur_surface_SD} et \eqref{tenseur_fond_SI}. On remarque de plus facilement que :

$$\dfrac{\partial}{\partial x} \gint_{-b(\mathbf{x})}^{\xi(t, \mathbf{x})} 2 \sigma \dfrac{\partial u(t, \mathbf{x}, z)}{\partial x}dz = \mathcal{O}\left( \varepsilon \right)$$

$$\dfrac{\partial}{\partial y} \gint_{-b(\mathbf{x})}^{\xi(t, \mathbf{x})} \sigma \left[ \dfrac{\partial u(t, \mathbf{x},z)}{\partial y} + \dfrac{\partial v(t, \mathbf{x},z)}{\partial x}\right]dz = \mathcal{O}\left( \varepsilon \right)$$

Le résultat est alors donné par (sans noter les dépendances) :

\begin{equation} \label{eq_SWE2}
\dfrac{\partial H \overline{u}}{\partial t} + \dfrac{\partial}{\partial x} \left( H \overline{u}^2 \right) + \dfrac{\partial}{\partial y} \left( H \overline{u} \overline{v} \right) + \dfrac{G}{2}\dfrac{\partial H^2}{\partial x} + 2 \Omega H \overline{v}= 0
\end{equation}

et on obtient un résultat semblable avec l'équation (\ref{NSEI_SD_cartesien}.c).

On obtient les équations Shallow-Water sans viscosité en revenant aux dimensions dans \eqref{eq_SWE1} et \eqref{eq_SWE2} :

\begin{equation} \label{SWE_nonvisqueux}
\left\{
\begin{array}{rcl}
\dfrac{\partial H}{\partial t} + \dfrac{\partial}{\partial x} \left( H \overline{u} \right)+ \dfrac{\partial}{\partial y} \left( H \overline{v} \right) & = & 0\\
\dfrac{\partial H \overline{u}}{\partial t} + \dfrac{\partial}{\partial x} \left( H \overline{u}^2 \right) + \dfrac{\partial}{\partial y} \left( H \overline{u} \overline{v} \right) + \dfrac{g}{2}\dfrac{\partial H^2}{\partial x} + 2 f_c H \overline{v} & = & 0\\
\dfrac{\partial H \overline{v}}{\partial t} + \dfrac{\partial}{\partial x} \left( H \overline{u} \overline{v} \right) + \dfrac{\partial}{\partial y} \left( H \overline{v}^{2} \right) + \dfrac{g}{2}\dfrac{\partial H^2}{\partial y} - 2 f_c H \overline{u} & = & 0\\
\end{array}
\right.
\end{equation}































\begin{thebibliography}{9}
        

\bibitem{Gerbeau Perthame}
         Jean-Frédéric Gerbeau, Benoît Perthame
         \emph{:  Dérivation of viscous Saint-Venant System for Laminar Shallow Water; Numerical Validation}.
         [Research Report],
         RR-4084,
         2000,
         <inria-00072549>.
         
\bibitem{Whitham}
         G. B. Whitham
         \emph{:  Linear and nonlinear waves}.
         Pure \& Applied Mathematics,
         A wiley-interscience series of texts, monographs \& tracts,
         1974.

		
\end{thebibliography}






\end{document}