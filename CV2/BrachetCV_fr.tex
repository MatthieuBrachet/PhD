\documentclass[10pt,a4paper]{report}
\usepackage[utf8]{inputenc}
\usepackage[francais]{babel}
\usepackage[T1]{fontenc}
\usepackage{amsmath}
\usepackage{amsfonts}
\usepackage{amssymb}
\usepackage{graphicx}
\usepackage[left=2.5cm,right=2.5cm,top=3.2cm,bottom=3.2cm]{geometry}

\def\cvpart{\noindent \hrulefill}
\def\sp{\vspace{6mm}}











\begin{document}
\begin{center}
{\fontfamily{pnc}\selectfont
\begin{LARGE}
CURRICULUM VITAE
\end{LARGE}

\begin{large}
Matthieu Brachet
\end{large}

\hrulefill
}
\end{center}

\noindent
84, avenue du Général Leclerc\\
54 000 Nancy, France
\vspace{0.2cm}\\
Téléphone : +33 6 43 50 07 86\\
E-mail : brachetmat@free.fr\\
Page Web : \verb?www.iecl.univ-lorraine.fr/~Matthieu.Brachet/?
\vspace{0.2cm}\\
Nationalité : française\\
Date de naissance : 11 février 1991 (Lille, France)

\cvpart

\noindent
\textbf{Thématiques de recherches : } mécanique des fluides numérique, analyse numérique, schémas aux différences finies, schémas compacts, équations aux dérivées partielles, cubed-sphere.







\sp
{\fontfamily{pnc}\selectfont
\begin{Large}
Expérience professionnelle
\end{Large}
\cvpart
}
\sp

\begin{center}
\begin{tabular}{r p{12cm}}
\textbf{2017 - 2018} & \textbf{ATER}, Université de Lorraine, Institut Elie Cartan de Lorraine (IECL, Nancy). \vspace{2mm}\\

\textbf{2014 - 2017} & \textbf{Doctorant} sous la direction de Prof. J.-P. Croisille à l'Université de Lorraine à l'Institut Elie Cartan de Lorraine (IECL, Metz, France).\newline Titre : "\textit{Schémas compacts hermitiens sur la sphère - Applications en climatologie et océanographie numérique}".\vspace{2mm}\\
	
	& \textbf{Soutenance :} le 3 Juillet 2018.\\
\end{tabular}
\end{center}

















\sp
{\fontfamily{pnc}\selectfont
\begin{Large}
Publications
\end{Large}
\cvpart
}
\sp

[1] {\sc  M. Brachet, J.-P. Croisille}, {\sl Numerical simulations of propagation problemes on the sphere using a compact scheme}, preprint.

\vspace{0.6cm}

[2] {\sc M. Brachet, J.-P. Chehab}, {\sl Stabilized Times Schemes for High Accurate Finite Differences Solutions of Nonlinear Parabolic Equations}, J. of Sci. Comp., 69(3), 946-982, 2016.

\vspace{0.6cm}

[3] {\sc  N. Aissiouene, T. Amtout, M. Brachet, E. Frenod, R. Hild, C. Prud'homme, A. Rousseau, S. Salmon}, {\sl  Hydromorpho: A coupled model for unsteady Stokes/Exner equations and numerical results with FEEL++ library}, ESAIM: Proc and survey, December 2016, Vol. 55, p. 23-40.




\sp
{\fontfamily{pnc}\selectfont
\begin{Large}
Responsabilités collectives
\end{Large}
\hrulefill
}
\sp

\noindent
\begin{itemize}
\item \textbf{Séminaire jeunes chercheurs :} Organisation du séminaire non-permanents (2016 à 2018),
\item \textbf{Rencontre annuelle des doctorants :} organisation de la rencontre annuelle des doctorants de l'IECL (2017).
\end{itemize}








\newpage
\sp
{\fontfamily{pnc}\selectfont
\begin{Large}
Compétences informatiques
\end{Large}
\hrulefill
}
\sp

\noindent
\begin{itemize}
\item \textbf{Language et logiciel :} Matlab (expert), Fortran (pratiques avancées), librairies éléments finis (FreeFem, FEEL++), Maple.
\item \textbf{O. S. :} Linux (pratiques avancées), Windows, Mac.
\end{itemize}











\sp
{\fontfamily{pnc}\selectfont
\begin{Large}
Scolarité
\end{Large}
\cvpart
}
\sp

\begin{center}
\begin{tabular}{r p{12cm}}
\textbf{2012 - 2014} & \textbf{Master en Mathématiques Appliquées et Modélisation} à l'Univ. Picardie Jules Verne (Amiens, France). \\

& \textbf{Mémoire de Master} (mémoire de 2nd année) sous la direction de Prof. J.-P. Chehab. \newline 
Titre : "\textit{Préconditionnement et résolution de problèmes d'évolutions non linéaires}". \\

& \textbf{Projet de recherches} (première année) sous la direction de V. Martin. \newline 
Titre : "\textit{\'Etude de quelques méthodes de Krylov}". \\

 & \\
\end{tabular}
\begin{tabular}{r p{12cm}}
\textbf{2010 - 2012} & \textbf{Licence de Mathématiques} à l'Univ. Picardie Jules Verne (Amiens, France) \\

& \\

\textbf{2008 - 2010} & \textbf{Classes préparatoires} (CPGE : MPSI et MP) au Lycée Pierre d'Ailly (Compiègne, France).\\

& \\

\textbf{2008} & \textbf{Baccalauréat} Scientifique option mathématiques au lycée Cassini (Clermont de l'Oise, France).\\

& \\
\end{tabular}
\end{center}
































\sp
\noindent
{\fontfamily{pnc}\selectfont
\begin{Large}
Présentations et Posters
\end{Large}
\cvpart
}
\sp

\noindent
{\fontfamily{pnc}\selectfont
\textbf{Conférences :}
}

\begin{itemize}
\item \textbf{PDE on the Sphere} (Conférence internationale), Paris, France, avril 2017,
\item \textbf{Journées EDP, IECL}, Metz, France, mars 2017,
\item \textbf{CANUM}, Obernai, France, 11 mai 2016,
\end{itemize}

\vspace{0.3cm}
\noindent
{\fontfamily{pnc}\selectfont
\textbf{Séminaires :}
}

\begin{itemize}
\item \textbf{Journée des Doctorants, IECL}, Nancy, France, décembre 2016,
\item \textbf{Séminaire d'Analyse Appliquée et EDP}, Amiens, France, novembre 2016 (LAMFA),
\item \textbf{Séminaire Doctorants}, Strasbourg, France, mars 2016,
\item \textbf{Séminaire Doctorants}, Rennes, France, novembre 2015,
\item \textbf{Séminaire Doctorants}, Reims, France, octobre 2015.
\item \textbf{Journées doctorantes Lebesgues}, Nantes, France, octobre 2015.
\end{itemize}

\vspace{0.3cm}
\noindent
{\fontfamily{pnc}\selectfont
\textbf{Posters :}
}

\begin{itemize}
\item \textbf{6e Colloque EDP-Normandie}, Caen, France, octobre 2017.
\end{itemize}

\vspace{0.3cm}
\noindent
{\fontfamily{pnc}\selectfont
\textbf{Vulgarisation Scientifique :}
}
\begin{itemize}
\item \textbf{Présentation Alkindi}, Nancy, juin 2018,
\item \textbf{Présentation PréTexte }(Faculté de sciences - Université de Lorraine), Nancy, décembre 2017,
\item \textbf{Séminaire étudiant du CESI }(\'Ecole d'ingénieur), Nancy, décembre 2016,
\item \textbf{Remise de prix des olympiades de mathématiques}, Metz, mai 2016,
\item \textbf{Ma thèse en 180 secondes}, finale régionale, Nancy, avril 2016.
\end{itemize}


















\newpage
\sp
{\fontfamily{pnc}\selectfont
\begin{Large}
Formations scientifiques et écoles d'étés
\end{Large}
\cvpart
}
\sp

\begin{center}
\begin{tabular}{r p{12cm}}
\textbf{été 2015} & \textbf{CEMRACS'15 Coupling Multiphysics} au Centre International de Rencontres Mathématiques (Marseille, France) sous la direction de Pr E. Frenod, Pr. C. Prud'homme, A. Rousseau and Pr. S. Salmon.\newline
Projet : "\textit{Modélisation et résolution numérique d'un système couplant les équations de Stokes instationnaire et l'équation d'Exner (HydroMorpho)}".\\

& \\

\textbf{2014} & \textbf{Fortran 90 parties 1 et 2} à l'Institut du Développement et des Ressources en Informatique Scientifique (Orsay, France) par P. Corde.\\

& \\

\end{tabular}
\end{center}

















\sp
{\fontfamily{pnc}\selectfont
\begin{Large}
Enseignements
\end{Large}
\cvpart
}
\sp

\noindent
\begin{center}
\begin{tabular}{r p{12cm}}
\textbf{2017 - 2018} & \begin{itemize}
\item Cours et travaux dirigés (Méthodes numériques) en L1 Mathématiques (Nancy),
\item Travaux dirigés (Distributions) en première année d'ingénieur à l'ENSEM (Nancy, niveau BAC+3),
\item Travaux dirigés (Analyse Numérique) en seconde année d'ingénieur à l'ENSEM (Nancy, niveau BAC+4),
\item Cours et travaux dirigés (Probabilités) en L1 Biologie (Nancy),
\item Travaux dirigés (Analyse numérique) en L3 Mathématiques (Nancy),
\item Travaux pratiques (Analyse numérique - Matlab) en première année d'ingénieur aux Mines (Nancy, niveau BAC+3),
\item Travaux dirigés (Probabilités) en seconde année d'ingénieur à l'ENSEM (Nancy, niveau BAC+4).
\end{itemize}\\

& \\

\textbf{2015 - 2017} & \begin{itemize}
\item Travaux dirigés (Mécanique des fluides numérique), M1 mécanique (Nancy),
\item Travaux pratiques (Analyse numérique, Matlab), L3 génie civile (Nancy),
\item Travaux pratiques (Introduction à Matlab), L2 physique (Nancy).
\end{itemize}\\

& \\

\textbf{2014 - 2015} & \begin{itemize}
\item Travaux dirigés (Mécanique des fluides numérique), M1 mécanique (Nancy),
\item Travaux pratiques (Analyse numérique en Matlab), L3 génie civile (Nancy),
\item Travaux pratiques (Introduction à Matlab), L2 physique (Nancy).
\item Travaux pratiques (Analyse numérique non linéaire, Matlab), L3 mathématiques (Metz),
\item Projets, L3 SPI Nancy,
\item Projets, 2eme année d'école d'ingénieur à Telecom Nancy (niveau BAC+4).
\end{itemize}\\

& \\

\textbf{2011 - 2012} & Monitorat : préparation mathématiques au concours vétérinaire, L3 Biologie, Amiens.\\

& \\
\end{tabular}
\end{center}

























\sp
{\fontfamily{pnc}\selectfont
\begin{Large}
Autres
\end{Large}
\hrulefill
}
\sp

\noindent
\begin{itemize}
\item \textbf{Languages :} Français (langue maternelle), Anglais (B2),
\item \textbf{Permis de conduire :} obtenu en 2009.
\item \textbf{PESC1 :} (Formation aux premiers secours) obtenu en 2015.
\item \textbf{Loisirs :} guitare(débutant), course à pieds (trail nocturne de 15km), lecture.
\end{itemize}





\end{document}