\documentclass[11pt]{beamer}
\usetheme{CambridgeUS}
\usepackage[utf8]{inputenc}
\usepackage[french]{babel}
\usepackage[T1]{fontenc}
\usepackage{amsmath}
\usepackage{amsfonts}
\usepackage{amssymb}
\usepackage{graphicx}

\usepackage{algorithmicx}
\usepackage[ruled]{algorithm}
\usepackage{algpseudocode}
\usepackage{algpascal}
\usepackage{algc}
\alglanguage{pseudocode}

\author[M. Brachet]{{\bf M. BRACHET}\inst{1} \& J-P. CHEHAB\inst{2}}


\institute[IECL - LAMFA]{
\inst{1}
Institut Elie Cartan, Universit\'e de Lorraine - Metz, France
(matthieu.brachet@univ-lorraine.fr)\\

  
  \inst{2}
  LAMFA, UMR 7352, Universit\'e de Picardie Jules Verne, Amiens, France
(jean-paul.chehab@u-picardie.fr)\\

  }

\title[Schémas RSS et applications]{Stabilisation de schémas pour la résolution d'EDP paraboliques - applications à la mécanique des fluides et au traitement d'images}
\date[2-03-2017]{2 mars 2017} 
\begin{document}

\begin{frame}
\titlepage
\end{frame}

\begin{frame}
\tableofcontents
\end{frame}

\section{Motivations}
\subsection{Préconditionnement}
\begin{frame}{Motivations}
\begin{block}{But :}
Résoudre :
\begin{equation}
Ax=b
\end{equation}
avec $A \in \mathcal{M}_n(\mathbb{R})$ inversible et $b \in \mathbb{R}^n$.
\end{block}

Quelle est la sensibilité aux perturbations?

\end{frame}

\begin{frame}
Sensibilité aux perturbations :
\begin{block}{Théorème :}
Si :
\begin{itemize}
\item $x$ est solution de $Ax=b$,
\item $x+\delta x$ est solution de $A(x+\delta x) = b + \delta b$.
\end{itemize}
Alors :
\begin{equation}
\dfrac{\|\delta x \|}{\|x \|} \leq \underbrace{\|A\| \|A^{-1} \|}_{Cond(A)} \dfrac{\| \delta b \|}{\|b\|}
\end{equation}
\end{block}

Forte sensibilité du système au conditionnement de $A$ : $Cond(A)$.
\end{frame}

\begin{frame}
Trouver $C$ une matrice :
\begin{equation}
C^{-1}A x = C^{-1}b
\end{equation}

$C$ est un préconditionnement de $A$ :
\begin{itemize}
\item $C$ est facilement inversible,
\item $Cond(C^{-1}A) \leq Cond(A)$.
\end{itemize}

$\Rightarrow$ Diminuer l'influence des perturbations.
\end{frame}

\subsection{Stabilisation}
\begin{frame}{}
On souhaite résoudre numériquement l'EDO :
\begin{equation}
\dfrac{dU}{dt} + AU=0
\end{equation}
Avec $A \in \mathcal{M}_N ( \mathbb{R} )$ symétrique définie positive.

\begin{block}{Schéma d'Euler Explicite}
\begin{equation}
\dfrac{U^{n+1}-U^n}{\Delta t} + A U^{n} = 0
\end{equation}
\begin{itemize}
\item $U^{n+1}=\left( Id - \Delta t A \right)U^n$ facilement calculable,
\item Convergence du schéma si et seulement si $$\Delta t \leq \dfrac{2}{\rho(A)}$$
\end{itemize}
\end{block}

\end{frame}

\begin{frame}{}
On souhaite résoudre numériquement l'EDO :
\begin{equation}
\dfrac{dU}{dt} + AU=0
\end{equation}
Avec $A \in \mathcal{M}_N ( \mathbb{R} )$ symétrique définie positive.

\begin{block}{Schéma d'Euler Implicite}
\begin{equation}
\dfrac{U^{n+1}-U^n}{\Delta t} + A U^{n+1} = 0
\end{equation}
\begin{itemize}
\item Pas de problème de convergence (inconditionnellement stable),
\item $U^{n+1}=\left( Id + \Delta t A \right)^{-1}U^{n}$ : coût en calcul important (inversion d'une matrice).
\end{itemize}
\end{block}

\end{frame}

\begin{frame}{Problèmatique :}
\begin{itemize}
\item Proposer un schéma pour discrétiser des EDP Paraboliques stable et rapide,
\item Précision du schéma,
\item Stabilité,
\item Applications.
\end{itemize}

\end{frame}

% **************************************************************************************************************************

\section{Schémas RSS}
\subsection{Construction du schéma RSS}
\begin{frame}{Schéma RSS}
Résolution de : $du/dt + Au = f$.
\begin{block}{Solution: Residual
Smoothing Scheme (RSS) Schemes}
\begin{itemize}
	\item Euler Implicite :
	$$
	\dfrac{u^{(k+1)}-u^{(k)}}{\Delta t} +A(u^{(k+1)}-u^{(k)})+Au^{(k)}=f
	$$
	\pause
	\item Soit $B$ un préconditionnement de $A$, alors le nouveau schéma s'écrit :
	\begin{equation}
\begin{array}{lcl}
\dfrac{u^{(k+1)}-u^{(k)}}{\Delta t}+&\tau
\underbrace{B(u^{(k+1)}-u^{(k)})}&+Au^{(k)}=f,\\
 & \mbox{Stabilization term}& \\
 \end{array}
\end{equation}
Ici $\tau>0$ permettra d'ajuster la stabilité du schéma.
       \end{itemize}	
	\end{block}

\end{frame}

\subsection{Stabilité et précision linéaire}
\begin{frame}{}
Supposons $A$ et $B$ SPD. 
$$
({\mathcal{H}})\hskip 2.cm  \alpha <Bu,u>\le <Au,u> \le \beta <Bu,u>, \ \forall u \in {\mathbb R}^N.
$$
$\alpha$ et $\beta$ peuvent dépendre de la dimension $N$ sinon $B$ est un préconditionnement incondiionnel de $A$.
\pause
\begin{block}{Théorème}
Sous l'hypothèse ${\mathcal{H}}$, on la le résultat de stabilité suivant :
\begin{itemize}
\item Si $\tau\ge \dfrac{\beta}{2}$, le schéma est inconditionnement stable (i.e. stable $\forall \ \Delta t >0$)
\item Si $\tau < \dfrac{\beta}{2}$, alors le schéma est stable si :
$
0<\Delta t < \dfrac{2}{\left(1-\dfrac{2\tau}{\beta}\right)\rho(A)}.
$
\end{itemize} 
\label{RSS_Stab_lin}
\end{block}
\end{frame}


\begin{frame}
\begin{block}{Théorème}\label{RSSPrec}
On considère les suites :
$$
 \dfrac{u^{(k+1)}-u^{(k)}}{\Delta t} +\tau B (u^{(k+1)}-u^{(k)}) =f-Au^{(k)},
 $$
et
$$
 \dfrac{v^{(k+1)}-v^{(k)}}{\Delta t} +A v^{(k+1)} =f,
 $$
 avec  $u^{(0)}=v^{(0)}$. Soit $M=Id-\Delta t(Id+\tau \Delta t B)^{-1}A$ et supposons que $\parallel M\parallel < 1$,
 alors, il existe $\gamma \in [0,1[$ tel que :
 $$
\parallel u^{(k)}- v^{(k)}\parallel \le  \Delta t^2 \parallel  \tau B-A\parallel \dfrac{1}{1-\gamma}\parallel  f-Av^{(0)} \parallel  , \forall k \ge 0.
$$
\end{block}
On en déduit que le schéma RSS est d'ordre 1 en temps.
\end{frame}

\subsection{Stabilité pour l'équation d'Allen-Cahn}

\begin{frame}
Considérons l'équation de réaction-diffusion (de type Allen-Cahn) :
\begin{equation}
\begin{array}{rl}
\dfrac{\partial u}{\partial t} -\Delta u +\dfrac{1}{\epsilon^2}f(u)=0, & x\in \Omega, t>0,\\
\dfrac{\partial u}{\partial n}= 0 & \partial \Omega , t>0,\\
u(x,0)=u_0(x) & x\in \Omega ,
\end{array}
\end{equation}
 où $\epsilon >0$ est un paramètre donné. Le schéma semi-linéaire RSS appliqué est le suivant :
\begin{equation}
 \dfrac{u^{(k+1)}-u^{(k)}}{\Delta t} +\tau B (u^{(k+1)}-u^{(k)}) =-Au^{(k)}-\dfrac{1}{\epsilon^2}f(u^{(k)}) .
 \label{RSSAC}
 \end{equation}
  Soit $E(u)=\dfrac{1}{2}<Au,u>+\dfrac{1}{\epsilon^2}<F(u),{\bf 1}>$, où $F$ est la primitive de $f$.
 Le schéma est d'énergie décroissante si :
 $$
 E(u^{(k+1)}) < E(u^{(k)}).
 $$
 Si $F\ge 0$ (se sera le cas dans les applications) alors $E\ge 0$ et la stabilité est obtenue.
 \end{frame}
 %%%%%%%%%%%%%%%%%%%%%%%%
%
%%%%%%%%%%%%%%%%%%%%%%%%
 \begin{frame}
 \begin{block}{Théorème :}
Supposons que $f$ est ${\cal C}^1$ et $\mid f'\mid_{\infty}\le L$.Alors on a les conditions de stabilité suivantes 
(Condition de diminution de l'énergie) :
\begin{itemize}
\item Si $\tau\ge \dfrac{\beta}{2}$ alors
\begin{itemize}
\item si $\left(\dfrac{\tau}{\beta}-\dfrac{1}{2}\right)\lambda_{min} -\dfrac{L}{2\epsilon^2}\ge 0$ le schéma est inconditionnellement stable,
\item si $\left(\dfrac{\tau}{\beta}-\dfrac{1}{2}\right)\lambda_{min} -\dfrac{L}{2\epsilon^2}< 0$ alors le schéma est stable si :
$$
0<\Delta t <\dfrac{1}{\frac{L}{2\epsilon^2} -\left(\frac{\tau}{\beta}-\frac{1}{2}\right)\lambda_{min}},
$$
\end{itemize}
\item Si $\tau < \dfrac{\beta}{2}$ alors le schéma est stable si :
$$
0<\Delta t <\dfrac{1}{\frac{L}{2\epsilon^2} -\left(\frac{\tau}{\beta}-\frac{1}{2}\right)\rho(A)}.
$$
\end{itemize} 
\label{Stab_AC}
\end{block}
\end{frame}
% **************************************************************************************************************************

\section{Discrétisation}
\subsection{Extrapolation de Richardson}
\begin{frame}{Discrétisation}
Le schéma RSS est d'ordre 1, un moyen classique pour augmenter la précision est d'utiliser l'extrapolation de Richardson :
\begin{eqnarray*}
\dfrac{d u }{dt}=F(u),
\end{eqnarray*}
Après discrétisation (Euler Explicite) on a :
\begin{eqnarray*}
u^{k+1}=u^{k}+\Delta t F(u^k)=G_{\Delta t}(u^k).
\end{eqnarray*}
L'extrapolation est donnée par :
\begin{eqnarray*}
v_1=G_{\Delta t}(u^k),\\
v_{2,0}=G_{\Delta t /2}(u^k),\\
v_{2,1}=G_{\Delta t /2}(v_{2,0}),\\
u^{k+1}=2v_{2,1}-v_1.
\end{eqnarray*}
le nouveau schéma est ainsi d'ordre 2.
\end{frame}
%%%%%%%%%%%%%%%%%%%%%%%%
%
%%%%%%%%%%%%%%%%%%%%%%%%%
 \begin{frame}
 Résolution RSS de :
 \begin{equation}
 \frac{du}{dt}+Au+F(u)=0
 \end{equation}
 
 \begin{center}
\begin{minipage}[H]{12cm}
  \begin{algorithm}[H]
    \caption{: Extrapolated RSS Scheme}\label{ExtraRSS}
    \begin{algorithmic}[1]
        \State $u^{(0)}$ given
            \For $k=0,1, \cdots$ until convergence
             \State {\bf Solve} $ (Id+\tau \frac{\Delta t}{2}B) v_1=-\frac{\Delta t}{2} F(u^{(k}),$
              \State {\bf Set} $u_1=u^{(n)} +v_1,$
               \State {\bf Solve} $ (Id+\tau \frac{\Delta t}{2}B) v_2=-\frac{\Delta t}{2} F(u_1),$
              \State {\bf Set}  $u_2=u_1+v_2,$
               \State {\bf Solve} $(Id+\tau \Delta tB) v_3=-\Delta t  F(u^{(k)}),$
               \State {\bf Set} $u_3=u^{(n)}+v_3,$
               \State  {\bf Set}  $u^{(k+1)}=2u_2-u_3.$              
            \EndFor
    \end{algorithmic}
    \end{algorithm}
\end{minipage}
\end{center}
 \end{frame}

\subsection{Discrétisation spatiale}
\begin{frame}{}
Pour discrétiser :

$$\dfrac{\partial u}{\partial t} - \Delta u = 0$$

il faut discrétiser le Laplacier $-\Delta$

\end{frame}

\begin{frame}
 \begin{block}{Compact Scheme (Lele's approach, '92)}
 \begin{itemize}
 \item moyen d'obtenur un schéma aux différences finies d'haute précision
 \item Soit $U=(U_1,\cdots,U_n)^T$ le vecteur d'approximation de $u$ aux points du maillage $x_i=ih$, $i=1,\cdots, n$.  On construit l'approximation $V_i={\cal L}(u)(x_i)$ comme solution d'un système
$$
P . V= Q U,
$$
formellement la matrice d'approximation est $B=P^{-1}Q$.
\end{itemize}
 \end{block}
 \end{frame}
 \begin{frame}

Schéma d'ordre 4 pour la dérivée seconde :
$$ P= tridiag(\frac{1}{10},1,\frac{1}{10}),$$

$$ Q = \frac{1}{h^2} \begin{pmatrix}
a_1 & a_2 & a_3 & a_4 & a_5 &   &   \\ 
-\frac{6}{5} & \frac{12}{5} & -\frac{6}{5} &   &   &   &   \\ 
  & -\frac{6}{5} & \frac{12}{5} & -\frac{6}{5} &   &   &   \\ 
  &   & \ddots & \ddots & \ddots &   &   \\ 
  &   &   & -\frac{6}{5} & \frac{12}{5} & -\frac{6}{5} &   \\ 
  &   &   &   & -\frac{6}{5} & \frac{12}{5} & -\frac{6}{5} \\ 
  &   & a_{N-4} & a_{N-3} & a_{N-2} & a_{N-1} & a_N
\end{pmatrix}, $$
où  $a_1$, $a_2$, $a_3$, ... sont donnés par :
$$ 
a_1=-\frac{67}{60},   
a_2=-\frac{7}{12},   
a_3=\frac{13}{10},   
a_4=-\frac{61}{120},   
a_5=\frac{1}{12}.   
 $$

 \end{frame}
 
 \begin{frame}
 Si on pose $A_4=P^{-1}Q$, $A_4$ est préconditionné par le schéma d'ordre 2:
 
 $$A_2=\dfrac{1}{h^2} \begin{pmatrix}
2  & -1 &    &   &    &     &   \\ 
-1 & 2 & -1 &   &    &     &   \\ 
   & -1 & 1  & -1&    &     &   \\ 
   &    & \ddots & \ddots & \ddots &   &   \\ 
   &    &   & -1 & 2  & -1   &   \\ 
   &    &   &    & -1 & 2    & -1 \\ 
   &    &   &    &    & -1   & 2
\end{pmatrix}$$

\begin{itemize}
\item $A_2$ est facilement inversible,
\item Préconditionnement inconditionnel de $A_4$.
\end{itemize}
 \end{frame}

% **************************************************************************************************************************

\section{Applications}
\subsection{Equations de Navier-Stokes 2D}
\begin{frame}{Applications}

\end{frame}

\subsection{Equation d'Allen Cahn et Segmentation d'images}
\begin{frame}{}

\end{frame}

\subsection{Equation de Cahn-Hilliard et Inpainting}
\begin{frame}{}

\end{frame}
% **************************************************************************************************************************
\section{Conclusions et perspectives}
\begin{frame}{Conclusions et perspectives}

\end{frame}

\end{document}