\section{Numerical Methods}
\subsection{Discretization}

\begin{frame}{Discretization}
  \begin{block}{Finite Element Method}
    Let $\Omega_h(t)$ (resp $\widehat{\Omega}_h$) be a discretization of $\Omega(t)$ (resp $\widehat{\Omega}$) and $\mathcal{T}_h$ (resp $\widehat{\mathcal{T}}_h$) a partition of it. We use $t_n = t_0+n\Delta t$ and write $v^n=v(t_n)$.\\
    We approximate a field $v\in V_h$ by $v_h=\sum_{i=1}^N v_i\varphi_i$ where $\varphi_i$ are the basis of $V_{h,k}^d=\left\{ v \in \mathcal{C}^0 \left( \overline{\Omega} \right) \text{ s.t. } v_{|K} \in  \mathbb{P}_k^d \text{ for all } K \in \mathcal{T}_h  \right\}$
  \end{block}
  \begin{itemize}
  \item
    Taking $\widehat{b}_z\in V_{h,1}^1$, and using an integration by parts, the variational formulation for Exner reads : $\forall \phi\in V_{h,1}^1$
    \begin{equation*}
      \frac{\partial }{\partial t} \int_{\hat{\gamma}_b} \hat{b}_z(\hat{x},t)\phi(\hat{x}) \; d\hat{x}
      = \int_{\hat{\gamma}_b}  \hat{Q}(\hat{x},t) \phi'(\hat{x}) \; d\hat{x}
      + \left[  \hat{Q}(\hat{x},t) \phi(\hat{x} )\right]_{\hat{\gamma}_b}
    \end{equation*}
  \item
    For the displacement, the variational formulation is :\\
    Find $\widehat{\mathbf{d}}\in \mathbf{D}_h = \{ \mathbf{v}\in V_{h,1}^2\text{ s.t. } \mathbf{v}=0\text{ on } \Gamma_s, \mathbf{v}=\widehat{b}_z \text{ on } \Gamma_b \}$, such that
    \begin{equation*}
      \int_{\widehat{\Omega}} \nabla\widehat{\mathbf{d}} : \nabla\bm{\psi} = 0 \quad \forall \bm{\psi}\in \mathbf{D}_h
    \end{equation*}
  \end{itemize}
\end{frame}

\begin{frame}{Discretization}
  \begin{itemize}
  \item
    For Stokes, we use Taylor-Hood elements :\\
    Find $\mathbf{u}\in V_{h,2}^2$ and $p\in V_{h,1}^1$ such that : $\forall \bm{\varphi}\in V_{h,2}^2, \forall \phi\in V_{h,1}^1$
  \begin{equation*}
    \left\{
    \begin{aligned}
      \int_\Omega \frac{1}{\Delta t}\mathbf{u}^{n+1}_h\cdot\bm{\varphi}_h
      + a(\mathbf{u}^{n+1}_h,\bm{\varphi}_h) + b\left(\bm{\varphi}_h,p^{n+1}_h\right) 
      &=\int_\Omega \frac{1}{\Delta t}\mathbf{u}^n_h\cdot\bm{\varphi}_h\\
      b(\mathbf{u}^{n+1}_h,\phi_h) &= 0
    \end{aligned}
    \right.
  \end{equation*}
  with
  \begin{align*}
    a_1\left(\mathbf{u},\mathbf{v}\right) =& \int_{\Omega(t)} \mu \nabla \mathbf{u} : \nabla \mathbf{v}
    & a_2\left(\mathbf{u},\mathbf{v}\right) =&  - \int_\Omega \mathbf{w}\cdot\nabla\mathbf{u}\cdot\mathbf{v}\\
    a_3\left(\mathbf{u},\mathbf{v}\right) =& \int_{\Gamma_b} \frac{1}{\epsilon}(\mathbf{u}\cdot\mathbf{n})(\mathbf{v}\cdot\mathbf{n})
    & b\left( \mathbf{u} , q \right) =&\int_{\Omega(t)} q\; \hbox{div} (\mathbf{u})
  \end{align*}
  \begin{equation*}
    a\left(\mathbf{u},\mathbf{v}\right) = a_1\left(\mathbf{u},\mathbf{v}\right) + a_2\left(\mathbf{u},\mathbf{v}\right) + a_3\left(\mathbf{u},\mathbf{v}\right)
  \end{equation*}
  \end{itemize}
\end{frame}

\subsection{Implementation}

\begin{frame}[fragile]{Algorithm}
  \begin{figure}
    \begin{center}
      \begin{tikzpicture}
        \uncover<1-5>{\node[rounded corners=3pt,draw,fill=block body.bg,align=center] (f) at (-3,4) {$\mathbf{u}^n = $ \textbf{Stokes}$(\mathbf{w})$\\ in $\Omega(t)$};}
        \uncover<2-5>{\node[rounded corners=3pt,draw,fill=block body.bg,align=center] (b) at (3,4) {$\widehat{b}_z = $ \textbf{Exner}$(\widehat{\mathbf{u}})$\\ in $\widehat{\gamma}_b$};}
        \uncover<2-5>{\node (ua) at (0,4.5) {$\widehat{\mathbf{u}} = \mathbf{u}\circ\mathcal{A}^t$};}
        \uncover<3-5>{\node[rounded corners=3pt,draw,fill=block body.bg,align=center] (d) at (3,0) {$\widehat{\mathbf{d}} = $ \textbf{Extension}$(\widehat{b}_z)$\\ in $\widehat{\Omega}$};}
        \uncover<4-5>{\node[rounded corners=3pt,draw,fill=block body.bg,align=center] (w) at (-3,0) {$\widehat{\mathbf{w}} = $ \textbf{Derivation}$(\widehat{\mathbf{d}})$\\ in $\widehat{\Omega}$};}
        
        \uncover<2-5>{\draw[->,thick,>=latex] (f) to[bend left] (b);}
        \uncover<3-5>{\draw[->,thick,>=latex] (b) to[bend left] (d);}
        \uncover<4-5>{\draw[->,thick,>=latex] (d) to[bend left] (w);}
        \uncover<5-5>{\draw[->,thick,>=latex] (w) to[bend left] (f);}
        \uncover<5-5>{\node (wa) at (-2.2,2) {${\mathbf{w}} = \widehat{\mathbf{w}}\circ(\mathcal{A}^t)^{-1}$};}
      \end{tikzpicture}
    \end{center}
  \end{figure}
\end{frame}

\begin{frame}[fragile]{Implementation in \textsc{Feel++}}
  Solving the harmonic extension :
  \begin{lstlisting}
    auto Vh = Pchv<1>( mesh );
    auto d = Vh->element();
    auto dd = Vh->element();
    auto a = form2( _trial=Vh, _test=Vh );
    auto l = form1( _test=Vh );
    a = integrate( _range=elements(mesh),
                   _expr=inner(grad(d), gradt(dd) ) );
    a += on( _range=boundaryfaces(mesh,"b"), _rhs=l,
             _element=d, _expr=idv(b) );
    a += on( _range=boundaryfaces(mesh,"s"), _rhs=l,
             _element=d, _expr=zero<Dim,1>() );
    a.solve( _rhs=l, _solution=d );
  \end{lstlisting}
  Then to apply the ALE map and its inverse :
  \begin{lstlisting}
    meshMove( mesh, d );
  \end{lstlisting}
\end{frame}

