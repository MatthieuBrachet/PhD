\documentclass[notes]{beamer}
\usetheme{Madrid}
\usepackage{etex}

% Try the class options [notes], [notes=only], [trans], [handout],
% [red], [compress], [draft], [class=article] and see what happens!

% For a green structure color use:
%\colorlet{structure}{green!50!black}

\mode<article> % only for the article version
{
  \usepackage{beamerbasearticle}
  \usepackage{fullpage}
  \usepackage{hyperref}
}

\beamertemplateshadingbackground{white!50}{white!50}

%\usepackage{beamerthemeshadow}
\definecolor{gris25}{gray}{0.75}


\usepackage{pgf,pgfarrows,pgfnodes,pgfautomata,pgfheaps,pgfshade}
%\usepackage{amsmath,amssymb}
\usepackage[latin1]{inputenc}
\usepackage{colortbl}
%\usepackage[french]{babel}
\usepackage{lmodern}

\usepackage[T1]{fontenc}
\usefonttheme{professionalfonts}

\usepackage{graphicx}
\usepackage{tikz}
%\usepackage{pgfplots,pgfplotstable}


\usepackage{amsmath}
\usepackage{amsfonts}
\usepackage{amssymb}
\usepackage{bm}
\usepackage{empheq}
\usepackage{algorithm,algorithmic}
\usepackage{multimedia}
\usepackage{media9}
%\usepackage{movie15}
\usepackage{animate}
\usepackage[font=Times,timeinterval=10]{tdclock}

\usepackage{algorithm,algorithmic}
\usepackage{listings}
\lstset{
  basicstyle=\ttfamily\small,
  language=C++,
  frame=single,
  keywordstyle=\color{magenta}\bfseries,
  commentstyle=\color{green}\itshape,
  stringstyle=\color{blue},
  showstringspaces=false,
  breaklines=true,
  breakatwhitespace=true,
  tabsize=4,
  rangeprefix={\/\/\ [},
    rangesuffix={]},
  includerangemarker=false
}

%%%%%%%%%%%%%%%%%%%%%%%%%%%%%%%%%%%%%%%
\definecolor{blue}{rgb}{0,0,1}
\newcommand{\blue}{\color{blue}}
\newcommand{\red}{\color{red}}
\newcommand{\green}{\color{green}}
%%%%%%%%%%%%%%%%%%%%%%%%%%%%%%%%%%%%%%%
%\usepackage{times}

% Use some nice templates
\beamertemplatetransparentcovereddynamic

\setbeamertemplate{navigation symbols}{}
\setbeamertemplate{headline}{%
  \leavevmode%
  \hbox{%
    \begin{beamercolorbox}[wd=\paperwidth,ht=2.5ex,dp=1.125ex]{palette quaternary}%
      \insertsectionnavigationhorizontal{\paperwidth}{}{\hskip0pt plus1filll}
    \end{beamercolorbox}%
  }
}
%\addtobeamertemplate{footline}{\insertframenumber/\inserttotalframenumber}

\title[Hydromorpho project]
{Hydromorpho: A coupled model for unsteady Stokes/Exner equations and numerical results with Feel++ library}
\author[R. Hild]{ N. A\"issiouene, T. Amtout, M. Brachet, E. Frenod \\ \underline{R. Hild}, A. Rousseau, S. Salmon}

%\code{E-mail:}
\institute[CEMRACS2015]{CEMRACS 2015 - CIRM MARSEILLE}

%Hattiesburg, Southern Miss, May 02, 2008}
% \subject{Analyse et Contrôles des%
%Systèmes}
\date{\mbox{~}}

\begin{document}
%%%%%%%%%%%%%%%%%%%%%%%%%%%%%%%%%%
%%%%%%%%%%%%%%%%%%%%%%%%%%%%%%%%%%%%

\begin{frame}
  \maketitle
  \initclock
\end{frame}
\date{\tdtime}

\begin{frame}{Plan}
  \tableofcontents
\end{frame}


%%%%%%%%%%%%%%%%%%%%%%%%%%%%%%%%%
% SECTION :  INTRO
%%%%%%%%%%%%%%%%%%%%%%%%%%%%%%%%%
\section{Introduction}
\begin{frame}
\begin{columns}
\column{0.45\textwidth}
\begin{figure}
\includegraphics[height=3cm]{nucleaire.jpg}
\caption{Nuclear Power Plant along the Loire}
\end{figure}
\column{0.45\textwidth}
\begin{itemize}
\item Cooling nuclear plant,
\item Waterway,
\item Dam gestion.
\end{itemize}
\end{columns}
\end{frame}

\begin{frame}{Exner}
  \begin{block}{The sediment transport is modelized by Exner equation :}
    \begin{equation}
      \frac{\partial b_z}{\partial t}+\frac{\partial Q_s}{\partial x}=0
    \end{equation}
  \end{block}
  The solid transport discharge formula $Q_s$ depends on the hydrodynamical variables.
  Several expressions for this formula have been proposed.

  \begin{block}{Grass formula (1981):}
    \begin{equation}
      Q_s = \alpha u | u |^{3/2}
    \end{equation}
  \end{block}
with $\alpha$ an empirically determined constant.\\~\\

Other formula for the solid transport discharge: Meyer-Peter and Muller, Van Rijn's, Einstein, Nielson, Fernandez-Luque, ... 

\end{frame}

\begin{frame}{Shallow Water}
  \begin{block}{The fluid is modelized by Shallow-Water (Saint-Venant) equations :}
    \begin{align*}
      \frac{\partial H}{\partial t}+\frac{\partial Q}{\partial x} &= 0 \\
      \frac{\partial Q}{\partial t} +\frac{\partial}{\partial x}\left(  \frac{Q^2}{H} + g\frac{H^2}{2}\right) &= -gH\frac{\partial b_z} {\partial x} - \frac{\tau}{\rho}
    \end{align*}
  \end{block}
  with:
  \begin{itemize}
  \item
    $\tau$ the shear stress at the bottom
  \item
    $\rho$ the density of the fluid
  \item
    $H$ the water height
  \item
    $Q=H\bar{u}$ the flux
  \item
    $\bar{u}$ the flow supposed independant of the vertical dimension.
  \end{itemize}
\end{frame}

\begin{frame}
\begin{block}{First CEMRACS works :}
E. Audisse and al., Sediment transport modelling : relaxation schemes for Saint-Venant/Exner and three layer models. 2012. ESAIM Proceeding.
\end{block}

\begin{alertblock}{Problem :}
\begin{itemize}
\item develop a code to solve the morphodynamical problem,
\item bring out the influence of depth
\end{itemize}
\end{alertblock}

\begin{exampleblock}{How :}
Develop a solver in C++/FEEL++ to solve Navier-Stokes/Exner Equations.
\end{exampleblock}
\end{frame}

%%%%%%%%%%%%%%%%%%%%%%%%%%%%%%%%%
% SECTION :  MODEL
%%%%%%%%%%%%%%%%%%%%%%%%%%%%%%%%%
\section{Model}

%%%%%%%%%%%%%%%%%%%%%%
% SUBSECTION :  MODEL
%%%%%%%%%%%%%%%%%%%%%%%

\subsection{A simple Coupled Model Stokes/Exner}



\begin{frame}{Domain definition}



%In the following section, we consider the problem on a domain 
$\Omega(t)=\{ (x,z) \in \mathbb{R}^{2} \quad | \quad 0 \leq  x \leq 1, \quad b(x,t) \leq z \leq 1 \}$  
%with $b(x,t)$ 
%the bottom topography, as shown in figure \ref{notations} and we denote the velocity 
%$\mathbf{u}(x,z,t)=(v(x,z,t),w(x,z,t))^T$, and the pressure $p(x,z,t)$.\\
%The figure \ref{notations} shows the domain $\Omega(t)$.

\begin{figure}[h!]\center
% Graphic for TeX using PGF
% Title: /Users/naissiou/GoogleDrive/these/presEtRapport/ValidationModel/Diagramme1.dia
% Creator: Dia v0.97.2
% CreationDate: Wed Apr 15 14:41:28 2015
% For: naissiou
% \usepackage{tikz}
% The following commands are not supported in PSTricks at present
% We define them conditionally, so when they are implemented,
% this pgf file will use them.
\ifx\du\undefined
  \newlength{\du}
\fi
\setlength{\du}{7\unitlength}
\begin{tikzpicture}
\pgftransformxscale{1.000000}
\pgftransformyscale{-1.000000}
\definecolor{dialinecolor}{rgb}{0.000000, 0.000000, 0.000000}
\pgfsetstrokecolor{dialinecolor}
\definecolor{dialinecolor}{rgb}{1.000000, 1.000000, 1.000000}
\pgfsetfillcolor{dialinecolor}


%%%%%%%%%%%%%%%%%%%%%%%%%%%%%%%%%%%%%%%%%%%%%%%%%%%%%%%%%%%%%%%%%%d  Z axe
\pgfsetlinewidth{0.0010000\du}
\pgfsetdash{}{0pt}
\pgfsetdash{}{0pt}
\pgfsetbuttcap
{
\definecolor{dialinecolor}{rgb}{0.000000, 0.000000, 0.000000}
\pgfsetfillcolor{dialinecolor}
% was here!!!
\pgfsetarrowsstart{to}
\definecolor{dialinecolor}{rgb}{0.000000, 0.000000, 0.000000}
\pgfsetstrokecolor{dialinecolor}
\draw (18.000000\du,5.000000\du)--(18.000000\du,27.000000\du);

\node[anchor=west] at (16.5000000\du,5.500000\du){\tiny{$z$}};
}
% setfont left to latex
\definecolor{dialinecolor}{rgb}{0.000000, 0.000000, 0.000000}
\pgfsetstrokecolor{dialinecolor}
\node[anchor=west] at (20.000000\du,11.000000\du){};
% setfont left to latex
\definecolor{dialinecolor}{rgb}{0.000000, 0.000000, 0.000000}
\pgfsetstrokecolor{dialinecolor}
\node[anchor=west] at (14.000000\du,11.000000\du){};

%%%%%%%%%%%%%%%%%%%%%%%%%%%%%%%%%%%%%%%%%%%%%%%%%%%%%%%%%%%%%%%%%%d  Gamma IN 
\pgfsetlinewidth{0.050000\du}
\pgfsetdash{}{0pt}
\pgfsetdash{}{0pt}
\pgfsetbuttcap
{
\definecolor{dialinecolor}{rgb}{0.000000, 0.000000, 0.000000}
\pgfsetfillcolor{dialinecolor}
% was here!!!

\definecolor{dialinecolor}{rgb}{0.000000, 0.000000, 0.000000}
\pgfsetstrokecolor{dialinecolor}
\draw (18.000000\du,10.000000\du)--(18.000000\du,27.000000\du);

\node[anchor=west] at (14.000000\du,18.600000\du){ \tiny{$ \Gamma_{in}(t)$}};
}
% setfont left to latex
\definecolor{dialinecolor}{rgb}{0.000000, 0.000000, 0.000000}
\pgfsetstrokecolor{dialinecolor}
\node[anchor=west] at (20.000000\du,11.000000\du){};
% setfont left to latex
\definecolor{dialinecolor}{rgb}{0.000000, 0.000000, 0.000000}
\pgfsetstrokecolor{dialinecolor}
\node[anchor=west] at (14.000000\du,11.000000\du){};

%%%%%%%%%%%%%%%%%%%%%%%%%%%%%%%%%%%%%%%%%%%%%%%%%%%%%%%%%%%%%%%%%%d  Right axe
\pgfsetlinewidth{0.010000\du}
\pgfsetdash{}{0pt}
\pgfsetdash{}{0pt}
\pgfsetbuttcap
{
\definecolor{dialinecolor}{rgb}{0.000000, 0.000000, 0.000000}
\pgfsetfillcolor{dialinecolor}
% was here!!!

\definecolor{dialinecolor}{rgb}{0.000000, 0.000000, 0.000000}
\pgfsetstrokecolor{dialinecolor}
\draw (46.063156\du,10.000000\du)--(46.063156\du,27.000000\du);

\node[anchor=west] at (46.000000\du,18.60000\du){ \tiny{$ \Gamma_{out}(t)$}};
}
%%%%%%%%%%%%%%%%%%%%%%%%%%%%%%%%%%%%%%%%%%%%%%%%%%%%%%%%%%% Surface - top axe

\pgfsetlinewidth{0.050000\du}
\pgfsetdash{}{0pt}
\pgfsetdash{}{0pt}
\pgfsetbuttcap
{
\definecolor{dialinecolor}{rgb}{0.000000, 0.000000, 0.000000}
\pgfsetfillcolor{dialinecolor}
% was here!!!

\definecolor{dialinecolor}{rgb}{0.000000, 0.000000, 0.000000}
\pgfsetstrokecolor{dialinecolor}
\draw (18.019905\du,10\du)--(46.063156\du,10.0\du);


\node[anchor=west] at (27.500000\du,9.0\du){ \tiny{$ \Gamma_{s}$}};

}

%%%%%%%%%%%%%%%%%%%%%%%%%%%%%%%%%%%%%%%%%%%%%%%%%%%%%%%%%%%%%%%%%%d normal
\pgfsetlinewidth{0.012000\du}
\pgfsetdash{}{0pt}
\pgfsetdash{}{0pt}
\pgfsetbuttcap
{
\definecolor{dialinecolor}{rgb}{0.101961, 0.101961, 0.101961}
\pgfsetfillcolor{dialinecolor}
% was here!!!
\pgfsetarrowsend{to}
\definecolor{dialinecolor}{rgb}{0.101961, 0.101961, 0.101961}
\pgfsetstrokecolor{dialinecolor}
\draw (18.0\du,16\du)--(16.297535\du,16\du);
\node[anchor=west] at (15.290000\du,15.400000\du){ \tiny{$\bold{n}_{in} $}};
}



%%%%%%%%%%%%%%%%%%%%%%%%%%%%%%%%%%%%%%%%%%%%%%%%%%%%%%%%%%%%%%%%%%d  zb 
\pgfsetlinewidth{0.050000\du}
\pgfsetdash{}{0pt}
\pgfsetdash{}{0pt}
\pgfsetmiterjoin
\pgfsetbuttcap
{
\definecolor{dialinecolor}{rgb}{0.101961, 0.101961, 0.101961}
\pgfsetfillcolor{dialinecolor}
% was here!!!
\definecolor{dialinecolor}{rgb}{0.101961, 0.101961, 0.101961}
\pgfsetstrokecolor{dialinecolor}
\pgfpathmoveto{\pgfpoint{18.043891\du}{24.793349\du}}
\pgfpathcurveto{\pgfpoint{19.159633\du}{24.793349\du}}{\pgfpoint{20.288819\du}{24.993389\du}}{\pgfpoint{21.404562\du}{24.993389\du}}
\pgfusepath{stroke}
}


%%%%%%%%%%%%%%%%%%%%%%%%%%%%%%%%%%%%%%%%%%%%%%%%%%%%%%%%%%%%%%%%%%d  zb 
\pgfsetlinewidth{0.050000\du}
\pgfsetdash{}{0pt}
\pgfsetdash{}{0pt}
\pgfsetmiterjoin
\pgfsetbuttcap
{
\definecolor{dialinecolor}{rgb}{0.000000, 0.000000, 0.000000}
\pgfsetfillcolor{dialinecolor}
% was here!!!
\definecolor{dialinecolor}{rgb}{0.000000, 0.000000, 0.000000}
\pgfsetstrokecolor{dialinecolor}
\pgfpathmoveto{\pgfpoint{21.304542\du}{24.993389\du}}
\pgfpathcurveto{\pgfpoint{22.420285\du}{24.993389\du}}{\pgfpoint{24.049570\du}{24.313253\du}}{\pgfpoint{25.165313\du}{24.313253\du}}
\pgfusepath{stroke}
}


%%%%%%%%%%%%%%%%%%%%%%%%%%%%%%%%%%%%%%%%%%%%%%%%%%%%%%%%%%%%%%%%%%d  zb 
\pgfsetlinewidth{0.050000\du}
\pgfsetdash{}{0pt}
\pgfsetdash{}{0pt}
\pgfsetmiterjoin
\pgfsetbuttcap
{
\definecolor{dialinecolor}{rgb}{0.000000, 0.000000, 0.000000}
\pgfsetfillcolor{dialinecolor}
% was here!!!
\definecolor{dialinecolor}{rgb}{0.000000, 0.000000, 0.000000}
\pgfsetstrokecolor{dialinecolor}
\pgfpathmoveto{\pgfpoint{25.018278\du}{24.317088\du}}
\pgfpathcurveto{\pgfpoint{26.134021\du}{24.317088\du}}{\pgfpoint{27.270213\du}{24.388108\du}}{\pgfpoint{28.385956\du}{24.388108\du}}
\pgfusepath{stroke}
}


%%%%%%%%%%%%%%%%%%%%%%%%%%%%%%%%%%%%%%%%%%%%%%%%%%%%%%%%%%%%%%%%%%d  zb 
\pgfsetlinewidth{0.010000\du}
\pgfsetdash{}{0pt}
\pgfsetdash{}{0pt}
\pgfsetbuttcap
{
\definecolor{dialinecolor}{rgb}{0.000000, 0.000000, 0.000000}
\pgfsetfillcolor{dialinecolor}
% was here!!!
\definecolor{dialinecolor}{rgb}{0.000000, 0.000000, 0.000000}
\pgfsetstrokecolor{dialinecolor}
\draw (28.365952\du,24.387410\du)--(32.066691\du,25.347602\du);
}


%%%%%%%%%%%%%%%%%%%%%%%%%%%%%%%%%%%%%%%%%%%%%%%%%%%%%%%%%%%%%%%%%%d  zb 
\pgfsetlinewidth{0.050000\du}
\pgfsetdash{}{0pt}
\pgfsetdash{}{0pt}
\pgfsetmiterjoin
\pgfsetbuttcap
{
\definecolor{dialinecolor}{rgb}{0.000000, 0.000000, 0.000000}
\pgfsetfillcolor{dialinecolor}
% was here!!!
\definecolor{dialinecolor}{rgb}{0.000000, 0.000000, 0.000000}
\pgfsetstrokecolor{dialinecolor}
\pgfpathmoveto{\pgfpoint{32.049834\du}{25.347306\du}}
\pgfpathcurveto{\pgfpoint{33.165577\du}{25.347306\du}}{\pgfpoint{34.250602\du}{25.530491\du}}{\pgfpoint{35.366345\du}{25.530491\du}}
\pgfusepath{stroke}
}

%%%%%%%%%%%%%%%%%%%%%%%%%%%%%%%%%%%%%%%%%%%%%%%%%%%%%%%%%%%%%%%%%%d  zb 
\pgfsetlinewidth{0.050000\du}
\pgfsetdash{}{0pt}
\pgfsetdash{}{0pt}
\pgfsetmiterjoin
\pgfsetbuttcap
{
\definecolor{dialinecolor}{rgb}{0.000000, 0.000000, 0.000000}
\pgfsetfillcolor{dialinecolor}
% was here!!!
\definecolor{dialinecolor}{rgb}{0.000000, 0.000000, 0.000000}
\pgfsetstrokecolor{dialinecolor}
\pgfpathmoveto{\pgfpoint{39.120799\du}{24.851822\du}}
\pgfpathcurveto{\pgfpoint{40.158904\du}{25.397367\du}}{\pgfpoint{41.151981\du}{24.900045\du}}{\pgfpoint{42.267724\du}{24.900045\du}}
\pgfusepath{stroke}
}

%%%%%%%%%%%%%%%%%%%%%%%%%%%%%%%%%%%%%%%%%%%%%%%%%%%%%%%%%%%%%%%%%%d  zb 
\pgfsetlinewidth{0.050000\du}
\pgfsetdash{}{0pt}
\pgfsetdash{}{0pt}
\pgfsetbuttcap
{
\definecolor{dialinecolor}{rgb}{0.000000, 0.000000, 0.000000}
\pgfsetfillcolor{dialinecolor}
% was here!!!
\definecolor{dialinecolor}{rgb}{0.000000, 0.000000, 0.000000}
\pgfsetstrokecolor{dialinecolor}
\draw (42.251169\du,24.892417\du)--(46.063156\du,25.859300\du);
}





%%%%%%%%%%%%%%%%%%%%%%%%%%%%%%%%%%%%%%%%%%%%%%%%%%%%%%%%%%%%%%%%%% zb lef
\pgfsetlinewidth{0.050000\du}
\pgfsetdash{}{0pt}
\pgfsetdash{}{0pt}
\pgfsetbuttcap
{
\definecolor{dialinecolor}{rgb}{0.000000, 0.000000, 0.000000}
\pgfsetfillcolor{dialinecolor}
% was here!!!
\definecolor{dialinecolor}{rgb}{0.000000, 0.000000, 0.000000}
\pgfsetstrokecolor{dialinecolor}
\draw (18.028760\du,24.767194\du)--(18.028760\du,27.012351\du);
}

%%%%%%%%%%%%%%%%%%%%%%%%%%%%%%%%%%%%%%%%%%%%%%%%%%%%%%%%%%%%%%%%%% zb bas
\pgfsetlinewidth{0.050000\du}
\pgfsetdash{}{0pt}
\pgfsetdash{}{0pt}
\pgfsetbuttcap
{
\definecolor{dialinecolor}{rgb}{0.000000, 0.000000, 0.000000}
\pgfsetfillcolor{dialinecolor}
% was here!!!
\definecolor{dialinecolor}{rgb}{0.000000, 0.000000, 0.000000}
\pgfsetstrokecolor{dialinecolor}
\draw (46.083754\du,26.996180\du)--(18.033497\du,26.986706\du);

}

%%%%%%%%%%%%%%%%%%%%%%%%%%%%%%%%%%%%%%%%%%%%%%%%%%%%%%%%%%%%%%%%%% zb right
\pgfsetlinewidth{0.050000\du}
\pgfsetdash{}{0pt}
\pgfsetdash{}{0pt}
\pgfsetbuttcap
{
\definecolor{dialinecolor}{rgb}{0.000000, 0.000000, 0.000000}
\pgfsetfillcolor{dialinecolor}
% was here!!!
\definecolor{dialinecolor}{rgb}{0.000000, 0.000000, 0.000000}
\pgfsetstrokecolor{dialinecolor}
\draw (46.063156\du,25.839204\du)--(46.064808\du,26.983932\du);
}

%%%%%%%%%%%%%%%%%%%%%%%%%%%%%%%%%%%%%%%%%%%%%%%%%%%%%%%%%%% zb
\pgfsetlinewidth{0.050000\du}
\pgfsetdash{}{0pt}
\pgfsetdash{}{0pt}
\pgfsetmiterjoin
\pgfsetbuttcap
{
\definecolor{dialinecolor}{rgb}{0.000000, 0.000000, 0.000000}
\pgfsetfillcolor{dialinecolor}
% was here!!!
\definecolor{dialinecolor}{rgb}{0.000000, 0.000000, 0.000000}
\pgfsetstrokecolor{dialinecolor}
\pgfpathmoveto{\pgfpoint{35.266325\du}{25.530491\du}}
\pgfpathcurveto{\pgfpoint{36.382068\du}{25.530491\du}}{\pgfpoint{38.042090\du}{24.784726\du}}{\pgfpoint{39.127096\du}{24.850355\du}}
\pgfusepath{stroke}
}

\node[anchor=west] at (26.000000\du,26.000000\du){ \tiny{Bottom}};




%%%%%%%%%%%%%%%%%%%%%%%%%%%%%%%%%%%%%%%%%%%%%%%%%%%%%%%%%%% Zb fonction
\pgfsetlinewidth{0.030000\du}
\pgfsetdash{{\pgflinewidth}{0.200000\du}}{0cm}
\pgfsetdash{{\pgflinewidth}{0.200000\du}}{0cm}
\pgfsetbuttcap
{
\definecolor{dialinecolor}{rgb}{0.000000, 0.000000, 0.000000}
\pgfsetfillcolor{dialinecolor}
% was here!!!
\pgfsetarrowsstart{stealth}
\pgfsetarrowsend{stealth}
\definecolor{dialinecolor}{rgb}{0.000000, 0.000000, 0.000000}
\pgfsetstrokecolor{dialinecolor}
\draw (41.909043\du,27\du)--(41.842731\du,24.841018\du);
\node[anchor=west] at (41.900000\du,26.250000\du){ \tiny{$b(x,t)$}};
\node[anchor=west] at (32.000000\du,25.000000\du){ \tiny{$ \Gamma_b(t) $}};
}


%%%%%%%%%%%%%%%%%%%%%%%%%%%%%%%%%%%%%%%%%%%%%%%%%%%%%%%%%%% Omega Domain

\node[anchor=west] at (36.000000\du,17.000000\du){ \tiny{$\Omega(t) $}};



%%%%%%%%%%%%%%%%%%%%%%%%%%%%%%%%%%%%%%%%%%%%%%%%%%%%%%%%%%% X axe

\pgfsetlinewidth{0.0010000\du}
\pgfsetdash{}{0pt}
\pgfsetdash{}{0pt}
\pgfsetbuttcap
{
\definecolor{dialinecolor}{rgb}{0.000000, 0.000000, 0.000000}
\pgfsetfillcolor{dialinecolor}
% was here!!!
\pgfsetarrowsend{to}
\definecolor{dialinecolor}{rgb}{0.000000, 0.000000, 0.000000}
\pgfsetstrokecolor{dialinecolor}
\draw (18.019905\du,27\du)--(48.315320\du,27.0\du);


\node[anchor=west] at (48.500000\du,27.0\du){ \tiny{$ x$}};

}
\end{tikzpicture}

%\caption{Definition of the domain}
\label{notations}
\end{figure}
%
% $\Gamma(t)=\Gamma_{in}(t)\cup\Gamma_{out}(t)\cup\Gamma_{s}\cup\Gamma_{b}(t)$ with
%$\Gamma_{in}(t)=\{0\} \times [b(0,t),1 ]$
%$\Gamma_{out}(t)=\{1\} \times [b(1,t),1]$
%$\Gamma_{s}=[0,1] \times \{1\}$
%$\Gamma_{b}(t)=\{(x,z) \in \mathbb{R}^{2} / z=b(x,t) , x \in [0,1]\}$

\end{frame}


\begin{frame}{A first simple model}
\begin{itemize}
\item Steady or Unsteady Stokes equations
+ Exner equation:
\end{itemize}


%\begin{itemize}
%\item Steady or unsteady Stokes
\begin{empheq}[left={\empheqlbrace}]{alignat=2}
  \textcolor{gray}{  \frac{\partial \mathbf{u}}{\partial t}} - \mu  \triangle \mathbf{u} + \nabla{p} &= 0 &&\hbox{ on } \Omega(t)  \\
    \text{div}\left(\mathbf{u}\right) &= 0 &&\hbox{ on } \Omega(t) \\
    \frac{\partial b_z}{\partial t}+\frac{\partial Q_s}{\partial x}&=0 
\end{empheq}

  
\begin{itemize}

\item Mixed Dirichlet/Neumann/slip boundary condition :
  \begin{empheq}[left={\empheqlbrace}]{alignat=2}
    \mathbf{u} 
    &= \mathbf{g}_1 
    && \text{ on } \Gamma_{s} \nonumber \\
    p\mathbf{n} - \mu\frac{\partial\mathbf{u}}{\partial \mathbf{n}} 
    &= \mathbf{g}_2 
    && \text{ on } \Gamma_{in}(t)\cup\Gamma_{out}(t) \\
{   \mathbf{u}\cdot\mathbf{n}} 
 &  = 
 { 0}  
 &&  \text{ on } \textcolor{red}{\Gamma_{b}(t)} \\
  { p\mathbf{n} - \mu\frac{\partial\mathbf{u}_\tau}{\partial\mathbf{n}}} 
   & {=\mathbf{g}_3}  
   && 
   \text{ on }
   \textcolor{red}{ \Gamma_{b}(t) }
  \end{empheq}
  
\end{itemize}

\end{frame}

%% \begin{frame}{Results: Validation for the resolution of Stokes (Feel ++)}

%% \begin{columns}

%% \column{0.5\textwidth}

%% \begin{itemize}
%% \item  Bercovier-Engelman solution:
%% \end{itemize}

%% \begin{eqnarray*}
%% \mathbf{u}&=&
%% \begin{pmatrix}
%%     -256y(y-1)(2y-1)x^2(x-1)^2\\
%%     256x(x-1)(2x-1)y^2(y-1)^2
%%   \end{pmatrix}\\
%%     p &=&(x-0.5)(y-0.5)
%% \end{eqnarray*}


%% \begin{itemize}
%% \item With a source term $\mathbf{f}=-\mu \bigtriangleup(\mathbf{u}) + \nabla p$
%% \item Dirichlet boundary conditions 
%% \end{itemize}





%% \column{0.2\textwidth}


%% \begin{figure}

%% \includegraphics[scale=0.15]{../Images/vBE_dirichlet}

%% \caption{Exact solution $ \mathbf{u}$ }

%% \end{figure}



%% \end{columns}






%% \end{frame}

%% \begin{frame}{Results: Validation for the resolution of Stokes (Feel ++)}

%% \begin{columns}

%% \column{0.4\textwidth}

%% \begin{figure}

%% \includegraphics[scale=0.2]{../Images/vBE_dirichlet}

%% \caption{Numerical solution - Velocity field}

%% \end{figure}

%% \column{0.5\textwidth}

%% \begin{figure}
%% \includegraphics[scale=0.2]{v_errL2}
%% \caption{L2- Error}
%% \end{figure}

%% \end{columns}

%% \end{frame}

\begin{frame}{Results: Boundary interface and coupling}



\begin{columns}
\column{0.5\textwidth}
\begin{itemize}
\item Model : Steady Stokes + Exner 
\item Description of the test case

\begin{itemize}
\item $\Gamma_{in}(t)\cup \Gamma_{out}(t)$ : Dirichlet
\item $ \Gamma_s$ : Dirichlet 
\item Initial bottom : $$ \hat{b}(\hat{x},0)=A\cos(2\pi \hat{x})e^{-\hat{x}^2}$$
\end{itemize}
\end{itemize}
\column{0.5\textwidth}
\begin{figure}
\href{run:test.avi}{\includegraphics[scale=0.15]{testCaseIM}} 
\caption{Test case }
\end{figure}
\end{columns}


\end{frame}


\begin{frame}{Results: Boundary interface and coupling}


\begin{figure}
\href{run:test2.avi}{\includegraphics[scale=0.5]{testCaseIM}} 
\caption{Test case }
\end{figure}
\end{frame}



%%%%%%%%%%%%%%%%%%%%%%%
% SUBSECTION :  ALE
%%%%%%%%%%%%%%%%%%%%%%%

\subsection{Arbitrary Lagrangian Eulerian}
\begin{frame}{Problem}

\begin{block}{}
Solve a FLuid/Structure interaction (FSI) :
\end{block}


\begin{block}{Problem :}

\begin{itemize}
\item Lagrangian approach can not consider all fluid particles.

\item Eulerian approach can not be used because the boundaries are moving.

\item FEM : the mesh must move with the structure but re-mesh at each time step is too expensive.
\end{itemize}
\end{block}
\end{frame}

%****************************************************************************

\begin{frame}{ALE}

\begin{block}{Idea : }
Use Arbitrary Lagrangian Eulerian method :

$\rightarrow$ Solve equation with adapted formulation.
\end{block}

\begin{itemize}
\item Lagrangian approach is considered for structure,

\item Eulerian approach is used for fluid,

\item Move the mesh smoothly with the boundaries.
\end{itemize}
\end{frame}

%****************************************************************************

\begin{frame}{}

\begin{block}{}
$\rightarrow$ ALE map :

$\mathcal{A}^t : \left\{ 
\begin{array}{rcl}
\widehat{\Omega} & \longrightarrow  & \Omega(t)\\
\widehat{\mathbf{x}} & \longmapsto & \mathbf{x} ( \widehat{\mathbf{x}} , t)
\end{array}
\right.$
\end{block}

\begin{figure}[H]
\begin{center}
\begin{tikzpicture}[scale=1]
\draw (0,0) -- (2,0) -- (2,1) -- (0,1) -- (0,0);
\draw (1,0.5) node{$\widehat{\Omega}$} ;
\draw (1.5,0.3) node[above]{$\hat{\mathbf{x}}$} ;
\draw (1.5,0.3) node {$\bullet$} ;
\draw (3,1) -- (5,1);
\draw [domain=0:2] plot (\x+3,{(0.2*sin((0.5*pi*(\x+1)) r))}) ;
\draw [domain=0:1] plot (5,{(0.2*sin((0.5*pi*3) r)-1)*\x+1}) ;
\draw [domain=0:1] plot (3,{(0.2*sin((0.5*pi) r)-1)*\x+1}) ;
\draw (4.15,0.5) node{$\Omega(t)$} ;
\draw (4.7,0) node[above]{$\mathbf{x}$} ;
\draw (4.7,0) node {$\bullet$} ;
\draw[->,>=latex] (1.5,0.3) to[out=-20,in=-160] (4.7,0);
\draw (2.5,-0.2) node{$\mathcal{A}^t$} ;
\end{tikzpicture}
\end{center}
\end{figure}

\begin{columns}
\column{0.45\textwidth}

\begin{block}{Reference domain (Lagrangian)}
Square $\widehat{\Omega} = \left[ 0,1 \right]^2$, boundaries :
\begin{itemize}
\item $\widehat{\Gamma}_{in} = \{ 0 \} \times \left[ 0,1 \right]$,

\item $\widehat{\Gamma}_{s} = \left[ 0,1 \right] \times \{ 1 \} $,

\item $\widehat{\Gamma}_{out} = \{ 1 \} \times \left[ 0,1 \right]$,

\item $\widehat{\Gamma}_{b} = \left[ 0,1 \right] \times \{ 0 \} = \widehat{\gamma}_b \times \{ 0 \}$.
\end{itemize}
\end{block}



\column{0.45\textwidth}

\begin{block}{Physical domain (Eulerian)}
$\Omega$ fluid domain, boundaries :
\begin{itemize}
\item $\Gamma_{in} = \{ 0 \} \times \left[ b_z(0,t),1 \right]$,

\item $\Gamma_{s} = \left[ 0,1 \right] \times \{ 1 \} $,

\item $\Gamma_{out} = \{ 1 \} \times \left[ b_z(1,t),1 \right]$,

\item $\Gamma_{b} = \left[ 0,1 \right] \times \{ b_z(x,t) \}$ with $x \in \left[ 0,1 \right]$
\end{itemize}
\end{block}

\end{columns}

\end{frame}

%*************************************************************************

\begin{frame}

Then 
$$\mathbf{x} = \mathcal{A}^t ( \widehat{\mathbf{x}} ) = \widehat{\mathbf{x}} + \underbrace{\widehat{\mathbf{d}_{\delta}} ( \widehat{\mathbf{x}}, t)}_{\text{displacement}} $$

where $\widehat{\mathbf{d}_{\delta}}$ is given by a PDE to be smooth.

\begin{block}{$\widehat{\Omega} \rightarrow \Omega(t)$}
If $\widehat{u} : \widehat{\Omega} \rightarrow \mathbb{R}^d$

Then $u = \widehat{u} \circ \left( \mathcal{A}^t \right)^{-1} : \Omega(t) \overset{\mathcal{A}^t}{\longrightarrow} \widehat{\Omega} \overset{\widehat{u}}{\longrightarrow} \mathbb{R}^d$

$$\widehat{u} ( \widehat{ \mathbf{x} } ) = u \left( \underbrace{\mathcal{A}^t ( \widehat{\mathbf{x}}}_{= \mathbf{x}} ), t \right)$$
\end{block}

\end{frame}


\begin{frame}

\begin{alertblock}{}
$\mathbf{x} \in \Omega(t)$ is time dependant!
\end{alertblock}

Then, defining the mesh velocity :

$$\mathbf{\widehat{w}}( \mathbf{\widehat{x}}, t) = \dfrac{\partial \mathcal{A}^t}{\partial t}(\mathbf{\widehat{x}}) = \dfrac{\partial \mathbf{\widehat{d}_{\delta}}}{\partial t}(\mathbf{\widehat{x}},t)$$

We have the following result :

\begin{block}{}
$$\dfrac{\mathcal{D}\mathbf{u}}{\mathcal{D}t} = \left.\frac{\partial \mathbf{u}}{\partial t}\right|_x + \mathbf{w}\cdot \nabla \mathbf{u}$$
\end{block}






\end{frame}










%%%%%%%%%%%%%%%%%%%%%%%
% SUBSECTION :  Model with ALE
%%%%%%%%%%%%%%%%%%%%%%%
\subsection{Coupled Model Stokes/Exner}




\begin{frame}[fragile]{Coupled model with ALE}
\begin{figure}
  \begin{center}
    \begin{tikzpicture}
      \node[rounded corners=3pt,draw,fill=block body.bg,align=center] (b) at (-3,4) {Bottom equation defined\\ in a reference 1D domain\\ $\widehat{\gamma}_b=[0,1]$};
      \node[rounded corners=3pt,draw,fill=block body.bg,align=center] (d) at (3,4) {Equations of the deformation\\ in the reference 2D domain\\ $ \widehat{\Omega}=[0,1]\times[0,1]$};
      \node[rounded corners=3pt,draw,fill=block body.bg,align=center] (w) at (3,0) {Velocity\\ in the reference 2D domain\\ $ \widehat{\Omega}=[0,1]\times[0,1]$};
      \node[rounded corners=3pt,draw,fill=block body.bg,align=center] (f) at (-3,0) {Fluid equation in\\ the current domain\\ $\Omega(t)$};
      
      \draw[->,thick,>=latex] (b) to[bend left] (d);
      \draw[->,thick,>=latex] (d) to[bend left] (w);
      \draw[->,thick,>=latex] (w) to[bend left] (f);
      \draw[->,thick,>=latex] (f) to[bend left] (b);
    \end{tikzpicture}
  \end{center}
\end{figure}
\end{frame}

%
\begin{frame}{Bottom equation}

\begin{itemize}

  
 \item Exner equation defined in $ \hat{\gamma}_b =[0,1]$
\begin{eqnarray}
 \frac{\partial \hat{b}_z(x,t)}{\partial t}+\frac{\partial \hat{Q}(\hat{x},t)}{\partial \hat{x}}&=&0 \quad \forall  \hat{x} \in \gamma_b, t>0\\ 
 \hat{b}_z(\hat{x},0)&=&{\hat{b}_{z,0}}(\hat{x})
 \end{eqnarray}

\item Sediment law $ \hat{Q}$  written in the reference domain :
 

 \begin{eqnarray}
\hat{Q}(\hat{x},t) 
&=&
Q\left(\mathcal{A}^t\left((\hat{x},0)^T\right),t\right)\quad \forall\hat{x} \in \hat{\gamma}_b\\
&=& 
\alpha \left(u_\tau \circ \mathcal{A}^t \left(({\hat{x},0})^T\right) \right)^{3/2} 
 \end{eqnarray}


\end{itemize}
\end{frame}

\begin{frame}{ALE Equation} 


\begin{itemize}
\item Harmonic extension:

\begin{empheq}[left={\empheqlbrace}]{alignat=2}
-\bigtriangleup \mathbf{\hat{d}}_\delta 
&=\;&
 0  \quad \hbox{ on }&\hat{ \Omega} \\
\mathbf{\hat{d}}_\delta 
&=\;& 
0 \quad \hbox{ on }& {\hat{\Gamma}}_s \\
\frac{\partial \mathbf{\hat{d}}_\delta}{\partial \mathbf{n}}
&=\;&
 0 \quad \hbox{ on }& {\hat{\Gamma}}_{in}\cup \hat{{\Gamma}}_{out}\\
\mathbf{\hat{d}}_\delta  
&=\;& (0,\hat{b}_z(\hat{x},t))^T \quad \hbox{ on }& \hat{\Gamma}_b
\end{empheq}


\item $ \Gamma_s$ : fixed.
\item $ \Gamma_{in}, \Gamma_{out}$ : vertical displacement.
\item $ \Gamma_b$ : Displacement given by $ \hat{b}(\hat{x},t)$.
\end{itemize}


\end{frame}

\begin{frame}{Computation of w}

\begin{itemize}
\item $\mathbf{\hat{w}}$: velocity of the displacement
\begin{eqnarray}
\hat{\mathbf{w}} = \frac{\partial \mathbf{\hat{d}}(\mathbf{\hat{x}},t)}{\partial t}
\end{eqnarray}
\item ALE transformation $ \rightsquigarrow$ velocity $\mathbf{w}$ in the domain $ \Omega(t)$:
\begin{eqnarray}
\mathbf{w}(\mathbf{x},t)= \mathbf{\hat{w}}\left((\mathcal{A}^t)^{-1}(\mathbf{x}),t\right)
\end{eqnarray}

\end{itemize}

\end{frame}


\begin{frame}{Fluid equation}

\begin{itemize}

\item Instationary Stokes + Lagrangian derivative
 
\begin{empheq}[left={\empheqlbrace}]{alignat=2}
\frac{\mathcal{D}\mathbf{u}}{\mathcal{D}t}
-\mathbf{w}\cdot \nabla \mathbf{u}
-\mu \bigtriangleup \mathbf{u}+\nabla p 
&=\;&
 0 \quad\hbox{on } & \Omega(t)  \\
\hbox{div}(\mathbf{u}) 
&=\;&
 0 \quad \hbox{on }& \Omega(t) \\
+\hbox{\textbf{BC}} 
\end{empheq}

\item Boundary condition: 
\begin{itemize}
\item Neumann or Dirichlet on $ \Gamma_{in}(t), \Gamma_{out}(t), \Gamma_{s}$
\item Slip boundary condition on the interface 
$ \Gamma_b(t)$
\end{itemize}


\end{itemize}

\end{frame}



%%%%%%%%%%%%%%%%%%%%%%%%%%%%%%%%%
% SECTION :  ALGO
%%%%%%%%%%%%%%%%%%%%%%%%%%%%%%%%%
\section{Numerical Methods}
\subsection{Discretization}

\begin{frame}{Discretization}
  \begin{block}{Finite Element Method}
    Let $\Omega_h(t)$ (resp $\widehat{\Omega}_h$) be a discretization of $\Omega(t)$ (resp $\widehat{\Omega}$) and $\mathcal{T}_h$ (resp $\widehat{\mathcal{T}}_h$) a partition of it. We use $t_n = t_0+n\Delta t$ and write $v^n=v(t_n)$.\\
    We approximate a field $v\in V_h$ by $v_h=\sum_{i=1}^N v_i\varphi_i$ where $\varphi_i$ are the basis of $V_{h,k}^d=\left\{ v \in \mathcal{C}^0 \left( \overline{\Omega} \right) \text{ s.t. } v_{|K} \in  \mathbb{P}_k^d \text{ for all } K \in \mathcal{T}_h  \right\}$
  \end{block}
  \begin{itemize}
  \item
    Taking $\widehat{b}_z\in V_{h,1}^1$, and using an integration by parts, the variational formulation for Exner reads : $\forall \phi\in V_{h,1}^1$
    \begin{equation*}
      \frac{\partial }{\partial t} \int_{\hat{\gamma}_b} \hat{b}_z(\hat{x},t)\phi(\hat{x}) \; d\hat{x}
      = \int_{\hat{\gamma}_b}  \hat{Q}(\hat{x},t) \phi'(\hat{x}) \; d\hat{x}
      + \left[  \hat{Q}(\hat{x},t) \phi(\hat{x} )\right]_{\hat{\gamma}_b}
    \end{equation*}
  \item
    For the displacement, the variational formulation is :\\
    Find $\widehat{\mathbf{d}}\in \mathbf{D}_h = \{ \mathbf{v}\in V_{h,1}^2\text{ s.t. } \mathbf{v}=0\text{ on } \Gamma_s, \mathbf{v}=\widehat{b}_z \text{ on } \Gamma_b \}$, such that
    \begin{equation*}
      \int_{\widehat{\Omega}} \nabla\widehat{\mathbf{d}} : \nabla\bm{\psi} = 0 \quad \forall \bm{\psi}\in \mathbf{D}_h
    \end{equation*}
  \end{itemize}
\end{frame}

\begin{frame}{Discretization}
  \begin{itemize}
  \item
    For Stokes, we use Taylor-Hood elements :\\
    Find $\mathbf{u}\in V_{h,2}^2$ and $p\in V_{h,1}^1$ such that : $\forall \bm{\varphi}\in V_{h,2}^2, \forall \phi\in V_{h,1}^1$
  \begin{equation*}
    \left\{
    \begin{aligned}
      \int_\Omega \frac{1}{\Delta t}\mathbf{u}^{n+1}_h\cdot\bm{\varphi}_h
      + a(\mathbf{u}^{n+1}_h,\bm{\varphi}_h) + b\left(\bm{\varphi}_h,p^{n+1}_h\right) 
      &=\int_\Omega \frac{1}{\Delta t}\mathbf{u}^n_h\cdot\bm{\varphi}_h\\
      b(\mathbf{u}^{n+1}_h,\phi_h) &= 0
    \end{aligned}
    \right.
  \end{equation*}
  with
  \begin{align*}
    a_1\left(\mathbf{u},\mathbf{v}\right) =& \int_{\Omega(t)} \mu \nabla \mathbf{u} : \nabla \mathbf{v}
    & a_2\left(\mathbf{u},\mathbf{v}\right) =&  - \int_\Omega \mathbf{w}\cdot\nabla\mathbf{u}\cdot\mathbf{v}\\
    a_3\left(\mathbf{u},\mathbf{v}\right) =& \int_{\Gamma_b} \frac{1}{\epsilon}(\mathbf{u}\cdot\mathbf{n})(\mathbf{v}\cdot\mathbf{n})
    & b\left( \mathbf{u} , q \right) =&\int_{\Omega(t)} q\; \hbox{div} (\mathbf{u})
  \end{align*}
  \begin{equation*}
    a\left(\mathbf{u},\mathbf{v}\right) = a_1\left(\mathbf{u},\mathbf{v}\right) + a_2\left(\mathbf{u},\mathbf{v}\right) + a_3\left(\mathbf{u},\mathbf{v}\right)
  \end{equation*}
  \end{itemize}
\end{frame}

\subsection{Implementation}

\begin{frame}[fragile]{Algorithm}
  \begin{figure}
    \begin{center}
      \begin{tikzpicture}
        \uncover<1-5>{\node[rounded corners=3pt,draw,fill=block body.bg,align=center] (f) at (-3,4) {$\mathbf{u}^n = $ \textbf{Stokes}$(\mathbf{w})$\\ in $\Omega(t)$};}
        \uncover<2-5>{\node[rounded corners=3pt,draw,fill=block body.bg,align=center] (b) at (3,4) {$\widehat{b}_z = $ \textbf{Exner}$(\widehat{\mathbf{u}})$\\ in $\widehat{\gamma}_b$};}
        \uncover<2-5>{\node (ua) at (0,4.5) {$\widehat{\mathbf{u}} = \mathbf{u}\circ\mathcal{A}^t$};}
        \uncover<3-5>{\node[rounded corners=3pt,draw,fill=block body.bg,align=center] (d) at (3,0) {$\widehat{\mathbf{d}} = $ \textbf{Extension}$(\widehat{b}_z)$\\ in $\widehat{\Omega}$};}
        \uncover<4-5>{\node[rounded corners=3pt,draw,fill=block body.bg,align=center] (w) at (-3,0) {$\widehat{\mathbf{w}} = $ \textbf{Derivation}$(\widehat{\mathbf{d}})$\\ in $\widehat{\Omega}$};}
        
        \uncover<2-5>{\draw[->,thick,>=latex] (f) to[bend left] (b);}
        \uncover<3-5>{\draw[->,thick,>=latex] (b) to[bend left] (d);}
        \uncover<4-5>{\draw[->,thick,>=latex] (d) to[bend left] (w);}
        \uncover<5-5>{\draw[->,thick,>=latex] (w) to[bend left] (f);}
        \uncover<5-5>{\node (wa) at (-2.2,2) {${\mathbf{w}} = \widehat{\mathbf{w}}\circ(\mathcal{A}^t)^{-1}$};}
      \end{tikzpicture}
    \end{center}
  \end{figure}
\end{frame}

\begin{frame}[fragile]{Implementation in \textsc{Feel++}}
  Solving the harmonic extension :
  \begin{lstlisting}
    auto Vh = Pchv<1>( mesh );
    auto d = Vh->element();
    auto dd = Vh->element();
    auto a = form2( _trial=Vh, _test=Vh );
    auto l = form1( _test=Vh );
    a = integrate( _range=elements(mesh),
                   _expr=inner(grad(d), gradt(dd) ) );
    a += on( _range=boundaryfaces(mesh,"b"), _rhs=l,
             _element=d, _expr=idv(b) );
    a += on( _range=boundaryfaces(mesh,"s"), _rhs=l,
             _element=d, _expr=zero<Dim,1>() );
    a.solve( _rhs=l, _solution=d );
  \end{lstlisting}
  Then to apply the ALE map and its inverse :
  \begin{lstlisting}
    meshMove( mesh, d );
  \end{lstlisting}
\end{frame}



%%%%%%%%%%%%%%%%%%%%%%%%%%%%%%%%%
% SECTION :  RESULTS
%%%%%%%%%%%%%%%%%%%%%%%%%%%%%%%%%

\section{Results}

\begin{frame}{Results}
\begin{figure}
\href{run:animation_test9.ogv}{\includegraphics[scale=0.15]{image_test9}} 
\caption{Initial displacement : $\frac{1}{5}\exp(-5(x-2.5)^2)$ and $\mathbf{g}=(1, 0)$ on $\Gamma_{s}$\footnote{E. Kubatko, J. Westerink, \textit{Exact Discontinuous Solutions of Exner's Bed Evolution Model: Simple Theory for Sediment Bores}, Journal of Hydraulic Engineering, 2007}}
\end{figure}
\end{frame}

\begin{frame}{Results}
\begin{figure}
\href{run:animation_maree.ogv}{\includegraphics[scale=0.15]{test_maree}} 
\caption{Initial displacement : $\frac{1}{10}(1-\cos(\pi x)$ and $\mathbf{g}=(\cos(\frac{\pi}{20}t), 0)$ on $\Gamma_{s}$}
\end{figure}
\end{frame}


%%%%%%%%%%%%%%%%%%%%%%%%%%%%%%%%%
% SECTION :  CONCLUSION
%%%%%%%%%%%%%%%%%%%%%%%%%%%%%%%%%

\section{Conclusions}
\begin{frame}{Conclusion}
%\begin{columns}

%\column{0.5\textwidth}
\begin{block}{To conclude}
\begin{itemize}
\item Adapt a general Fluid Structure interaction framework to the particular problem Exner/Stokes
\item Use a library Feel++ to implement the problem
\end{itemize}
\end{block}

%\column{0.5\textwidth}
\begin{block}{Outlook}
\begin{itemize}
\item Theoretical analysis of the coupled model 
\item Have a free surface on the top
\item Compare results with existing models (SW +Exner)
\end{itemize}
\end{block}

%\end{columns}

\end{frame}

\begin{frame}
  \centering
  %% \begin{columns}
  %%   \begin{column}{0.35\textwidth}
  %%     \centering
  %%     \animategraphics[autoplay,loop,scale=0.5]{6}{vahine-}{0}{3}
  %%   \end{column}
  %%   \begin{column}{0.3\textwidth}
  %%     \begin{center}
        Thank you  :) !  
  %%     \end{center}
  %%   \end{column}
  %%   \begin{column}{0.35\textwidth}
  %%     \centering
  %%     \animategraphics[autoplay,loop,scale=0.5]{6}{vahine-}{0}{3}
  %%   \end{column}
  %% \end{columns}
  %% \includemovie{}{0.5\textheight}{maree.avi}
  %% \movie{%[loop,autostart]{
  %%   \includegraphics[height=0.5\textheight]{test_maree}
  %% }{maree.avi}
\end{frame}

\end{document}

