   
\begin{frame}{Sediment transport or bedload transport - model  }

\begin{itemize}
\item The usual Saint-Venant + Exner model [ Felix Maria Exner  1920-1925 ]
\begin{eqnarray*}
\frac{\partial H}{\partial t}+\frac{\partial H\bar{u}}{\partial x} &=& 0 \\
\frac{\partial H\bar{u}}{\partial t}
+\frac{\partial}{\partial x}\left(  H\bar{u}^2 + g\frac{H^2}{2}\right)&=&-gH\frac{\partial z_b}{\partial x} - \frac{\tau}{\rho} \\
\frac{\partial z_b}{\partial x}+\frac{\partial Q_s}{\partial x}&=&0
\end{eqnarray*}

with
\item $\tau$ is the shear stress at the bottom and $\rho$ is the density of the fluid.
 \begin{eqnarray*}
\tau = \rho g H \frac{(H\bar{u})^2 }{H^2K_s^2 R_h^{4/3}} \quad,\quad R_h=\frac{lH}{l+2H}\quad,\quad l:\hbox{lenght of the chanel}
\end{eqnarray*}

\item Closure relation : solid transport discharge $Q_s$ defined by many laws. \\
Example : Grass formula (1981):the bedload sediment transport
begins automatically when the fluid
starts to move:
 \begin{eqnarray}
Q_s = \alpha \bar{u} | \bar{u} |^{m-1}
\end{eqnarray}

with $\alpha$  usually obtained from experimental data, and $0<m<4$.

\item Other formula for the solid transport discharge: Meyer-Peter and Muller, Van Rijn's, Einstein, Nielson, Fernandez-Luque, ... 



\end{itemize}



\end{frame}


%%%%%%%%%%%%%%%%%%%%%%%%%%%%%%%%%%%%%%%%%%%%%%%%%%%%%%%%%%%%%%%%%%%%%
%                    Conclusion and outlook
%%%%%%%%%%%%%%%%%%%%%%%%%%%%%%%%%%%%%%%%%%%%%%%%%%%%%%%%%%%%%%%%%%%%%

   
\begin{frame}{Sediment transport}

\textbf{Recent models}

\begin{itemize}

\item Relaxation scheme for Saint-Venant-Exner model [E.Audusse,J.S.-M., cemracs 2011]:

\begin{itemize}
\item The hydrostatic pressure and the solid transport law $Q_s$ are relaxed 
\item Eigenvalue of the hyperbolic system are easier to compute. Solution to Riemann solver easy to compute then the Riemann solver.
\end{itemize}


\item Class of model SV + Exner including arbitrarily sloping sediment beds and associated energy [E.D. Fernandez-Nieto, T.Morales de Luna, 2015] 

\begin{itemize}
\item derived from 3D Navier-Stokes system : asymptotic analysis 
\item has an energy balance 
\item uses methods for hyperbolic system
\end{itemize}

\end{itemize}

\textbf{Numerical method}: Two main approaches: Coupled / decoupled


\begin{itemize}

\item Ex: EDF (TELEMAC) fractional time step : solving separately the shallow water system during a time step
for a topography $z_b$, and after updating the topography using the Exner equation for the same time step.
Problem: For some situations, it occurs stabilities.

\item Ex: coupled method with Godunov scheme:

\item Ex: Relaxation scheme: coupled model with Riemann solver.

\end{itemize}


\end{frame}




%%%%%%%%%%%%%%%%%%%%%%%%%%%%%%%%%%%%%%%%%%%%%%%%%%%%%%%%%%%%%%%%%%%%%
%                    Conclusion and outlook
%%%%%%%%%%%%%%%%%%%%%%%%%%%%%%%%%%%%%%%%%%%%%%%%%%%%%%%%%%%%%%%%%%%%%

   
\begin{frame}{Sediment transport - methods}

Details on the Godunov scheme [E. Audusse, Ph. Ung]:
\begin{itemize}
\item Rewriting the model under the quasilinear form:
\end{itemize}

\begin{eqnarray}
\partial X_t + A(X)\partial_x X = S(X)
\end{eqnarray}

with :
\begin{eqnarray}
X=\left(\begin{array}{ccc}
H\\H\bar{u}\\z_b
\end{array}\right)
,
S=\left(\begin{array}{ccc}
0\\H\partial_x z_b \\0
\end{array}\right)
\end{eqnarray}


A is the Jacobian matrix of $F=\left( \begin{array}{ccc}
G\bar{u}\\H\bar{u}^2+g\frac{H^2}{2}\\Q_s
\end{array}\right)$

\begin{itemize}
\item  Solve a local Riemann problem at each time step.
\end{itemize}



\end{frame}


%

   
%\begin{frame}{Sediment transport using fluid structure interaction - Model with stokes ? }
%
%\begin{itemize}
%
%\item Equation for the fluid
%Find $ \mathbf{ v} $ and $p  $   such that
%\begin{eqnarray}
%- \mu  \triangle \mathbf{v} + \nabla{p} &=& F(t)  \hbox{ on } \Omega(t) \label{stokes momentum} \\
%\hbox{div}\left(\mathbf{v}\right) &=&0 \label{stokes divu}\\
%\mathbf{\mathbf{v}}|_{\Gamma_{in}\cup \Gamma_{out} \cup \Gamma_{b}}&=&0 \\
%\mathbf{v}|_{\Gamma_{s}}&=&(1,0)^T \label{stokes BC0}
%\end{eqnarray}
%
%with $F(t)$  a forcing term that will allow to modeling the flow. 
%\\ We denote by $ \Gamma = \Gamma_{in}\cup\Gamma_{out}\cup \Gamma_{b} \cup \Gamma_s$ the boundary of $\Omega$.
%
%
%\item Equation for the bottom
%\begin{eqnarray}
%\mathbf{v} \cdot \mathbf{n_b}=0 \\
%\mathbf{v}_{\tau} + \beta \frac{\partial \mathbf{v}_{\tau}}{\partial \mathbf{n}_b} = 0
%\end{eqnarray}
%
%\end{itemize}
%
%
%
%\end{frame}

\begin{frame}{Objectives}
\begin{itemize}
\item Stokes with Feel++
\item Stokes with boundary conditions $ b(x,t)$ fixed
\item Add ALE method to make the mesh moving
\item Coupled the model with the Exner equation
\end{itemize}
\end{frame}