\begin{frame}
\begin{columns}
\column{0.45\textwidth}
\begin{figure}
\includegraphics[height=3cm]{nucleaire.jpg}
\caption{Nuclear Power Plant along the Loire}
\end{figure}
\column{0.45\textwidth}
\begin{itemize}
\item Cooling nuclear plant,
\item Waterway,
\item Dam gestion.
\end{itemize}
\end{columns}
\end{frame}

\begin{frame}{Exner}
  \begin{block}{The sediment transport is modelized by Exner equation :}
    \begin{equation}
      \frac{\partial b_z}{\partial t}+\frac{\partial Q_s}{\partial x}=0
    \end{equation}
  \end{block}
  The solid transport discharge formula $Q_s$ depends on the hydrodynamical variables.
  Several expressions for this formula have been proposed.

  \begin{block}{Grass formula (1981):}
    \begin{equation}
      Q_s = \alpha u | u |^{3/2}
    \end{equation}
  \end{block}
with $\alpha$ an empirically determined constant.\\~\\

Other formula for the solid transport discharge: Meyer-Peter and Muller, Van Rijn's, Einstein, Nielson, Fernandez-Luque, ... 

\end{frame}

\begin{frame}{Shallow Water}
  \begin{block}{The fluid is modelized by Shallow-Water (Saint-Venant) equations :}
    \begin{align*}
      \frac{\partial H}{\partial t}+\frac{\partial Q}{\partial x} &= 0 \\
      \frac{\partial Q}{\partial t} +\frac{\partial}{\partial x}\left(  \frac{Q^2}{H} + g\frac{H^2}{2}\right) &= -gH\frac{\partial b_z} {\partial x} - \frac{\tau}{\rho}
    \end{align*}
  \end{block}
  with:
  \begin{itemize}
  \item
    $\tau$ the shear stress at the bottom
  \item
    $\rho$ the density of the fluid
  \item
    $H$ the water height
  \item
    $Q=H\bar{u}$ the flux
  \item
    $\bar{u}$ the flow supposed independant of the vertical dimension.
  \end{itemize}
\end{frame}

\begin{frame}
\begin{block}{First CEMRACS works :}
E. Audisse and al., Sediment transport modelling : relaxation schemes for Saint-Venant/Exner and three layer models. 2012. ESAIM Proceeding.
\end{block}

\begin{alertblock}{Problem :}
\begin{itemize}
\item develop a code to solve the morphodynamical problem,
\item bring out the influence of depth
\end{itemize}
\end{alertblock}

\begin{exampleblock}{How :}
Develop a solver in C++/FEEL++ to solve Navier-Stokes/Exner Equations.
\end{exampleblock}
\end{frame}