\documentclass[10pt,a4paper]{article}
\usepackage[utf8]{inputenc}
\usepackage[francais]{babel}
\usepackage[T1]{fontenc}
\usepackage{amsmath}
\usepackage{amsfonts}
\usepackage{amssymb}
\usepackage{graphicx}
\usepackage[left=2cm,right=2cm,top=2cm,bottom=2cm]{geometry}
\author{Brachet Matthieu}
\title{Filtrage adaptatif}

\newtheorem{remarque}{Remarque}

\begin{document}
\maketitle

\section{Problèmatique}

La résolution d'équations aux dérivées partielles hyperboliques par des schémas aux différences finies centrées présente deux difficultés :

\begin{itemize}
\item la mauvaise représentation des hautes fréquences par le schéma numérique entraine des perturbations de type phénomène de Gibbs pouvant aboutir à des perturbations numériques,
\item l'apparition de chocs difficiles à représenter par le schéma discrétisé.
\end{itemize}

L'objectif de ce document est de présenter deux manières de filtrer en espace une donnée spatiale en ajoutant le moins d'erreur possible.

Dans la suite de ce document, on considère une fonction $u : \left[ 0, 1 \right] \rightarrow \mathbb{R}$ destinée à être filtrée. Le segment $\left[ 0, 1 \right]$ est discrétisé aux point $(x_i)_i$ équirépartis avec comme pas de maillage $h$.

\section{Filtrage classique}

On travaille dans cette partie sur un filtrage systèmique pouvant, par exemple, être appliqué à chaque itération du calcul en temps.
La donnée filtrée, notée $u_{f,k}$, approchant $u(x_k)$ est cherchée comme une combinaison linéaire de ses voisins :

\begin{equation}
u_{f,k} = \sum_j f_j u(x_{k+j}) 
\label{eq:filtrage}
\end{equation}

Dans cette partie, on ne s'intéresse qu'à des filtrage locaux (i.e. $f_j= 0$ pour $|j|$ assez grand) et symétriques ($f_j = f_{-j}$).

Partant de \eqref{eq:filtrage}, le coefficiant d'amplification (en Fourier) est donné par :

\begin{equation}
F(\theta) = \sum_{j \in \mathbb{Z}} f_j e^{ij \theta} = f_0 + 2 \sum_{j \geq 1} cos ( j \theta )
\label{eq:ampli}
\end{equation}

avec $\theta = h \xi$, $\xi$ étant le paramètre de Fourier.

Pour que les coefficients $(f_j)_j$ donnent un filtrage passe bas à l'ordre $P$, $F$ doit vérifier les conditions suivantes :

\begin{itemize}
\item \textbf{Consistance} :
\begin{equation}
F(0) = 1
\label{cd:consistance}
\end{equation}

\item \textbf{Précision} : pour tout $p$ entre $1$ et $P-1$ inclus :
\begin{equation}
F^{(p)}(0) = 0
\label{cd:precision}
\end{equation}

\item \textbf{Filtre passe bas} : 
\begin{equation}
F(\pi) = 0
\label{cd:passe bas}
\end{equation}
\end{itemize}

Ce qui, en exploitant la forme de $F$, s'écrit de manière équivalente :

\begin{equation}
\left\lbrace
\begin{matrix}
f_0 + 2 \sum_{j \geq 1} f_j & = & 1 \\
\sum_{j\in\mathbb{Z}} f_j (ij)^p & = & 0 \\
f_0 + 2 \sum_{j \geq 1} (-1)^j f_j & = & 0 \\
\end{matrix}
\right.
\end{equation}

pour tout $p$ entre $1$ et $P-1$.

\begin{remarque}
\begin{itemize}
\item $P$ est l'ordre du filtre satisfesant \eqref{cd:precision}, \eqref{cd:consistance} et \eqref{cd:passe bas},
\item la condition \eqref{cd:precision} est automatiquement satisfaite lorsque $p$ est impair. Un filtre centré et symétrique comme présenté ici est donc d'ordre $P$ pair,
\item Il existe des filtres implicites \cite{VG2002} prenant en compte l'ensemble des points du maillage,
\item Le filtrage peut être vue comme la transformée de Fourier inverse d'une fonction créneau (filtre passe bas exacte) : une fonction sinus cardinale.
\end{itemize}
\end{remarque}

Voici quelques exemples de filtres locaux symétriques centrés issus de \cite{Redonnet2001}


\begin{table}[ht]
\begin{center}
\begin{tabular}{c|cccccc}
& $f_0$ & $f_1$ & $f_2$ & $f_3$ & $f_4$ & $f_5$ \\
\hline
\hline
ordre 2 & $1/2$ & $1/4$ & $0$ & $0$ & $0$ & $0$\\
\hline
ordre 4 & $10/16$ & $4/16$ & $-1/16$ & $0$ & $0$ & $0$ \\
\hline
ordre 6 & $44/64$ & $15/64$ & $-6/64$ & $1/64$ & $0$ & $0$ \\
\hline
ordre 8 & $186/256$ & $56/256$ & $-28/256$ & $8/256$ & $-1/256$ & $0$ \\
\hline
ordre 10 & $772/1024$ & $210/1024$ & $-120/1024$ & $45/1024$ & $-10/1024$ & $1/1024$ \\
\end{tabular}
\caption{Exemples de filtrages explicites centrés}
\label{tab:redonnet}
\end{center}
\end{table}

La figure \ref{fig:ampli} représente les filtrages donnés par la table \ref{tab:redonnet}. On observe que plus l'ordre est élevé mieu sont représentées les basses fréquences.

\begin{figure}
\begin{center}
\includegraphics[scale=0.2]{ampli.jpg}
\caption{Coefficient d'amplifications pour différents filtres explicites}
\label{fig:ampli}
\end{center}
\end{figure}

On peut aussi chercher la fréquence $\theta_m$ au dela de laquelle les fréquences sont représentées avec moins de $95 \%$ de qualité par le filtrage (\textit{i.e.} $F(\theta_m)=0.95$). Les résultats sont données dans la table \ref{tab:qualite freq}. Les filtres utilisés sont issus de la table \ref{tab:redonnet}.

\begin{table}[ht]
\begin{center}
\begin{tabular}{c|c}
Ordre & $\theta_m$ \\
\hline
\hline
$10$ & $1.6695$ \\
$8$  & $1.5165$ \\
$6$  & $1.3045$ \\
$4$  & $0.9851$ \\
$2$  & $0.4510$ 
\end{tabular}
\caption{Fréquence à $95 \%$ de qualité pour les filtres de la table \ref{tab:redonnet}}
\label{tab:qualite freq}
\end{center}
\end{table}


\section{Filtrage adaptatif}




\bibliographystyle{abbrv}
\bibliography{PhDbiblio}
\end{document}