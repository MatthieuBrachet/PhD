%% Conclusion générale

Dans cette thèse, nous présentons un nouveau schéma aux différences finies pour la résolution d'équations aux dérivées partielles d'évolution sur la sphère.

Après semi-discrétisation en espace, un schéma explicite en temps de Runge-Kutta d'ordre 4 couplé à un opérateur de filtrage donne d'excellents résultats sur des équations classiques telles que l'équation de transport, l'équation des ondes et l'équation de Burgers. Malgré le caractère hyperbolique de ces équations, l'ajout de l'opérateur de filtrage nous permet de conserver des schémas centrés en espace sans que les ondes parasites deviennent néfastes pour le bon déroulement des algorithmes. Les opérateurs de filtrage permettent en effet d'améliorer la stabilité en ajoutant de la dissipation numérique au schéma. Ces derniers, n'affectent pas la conservation de la masse.

Le maillage utilisé est celui de la Cubed-Sphere. Il est construit à partir de sections de grands cercles. Cela permet de définir un schéma hermitien centré de nature périodique. On obtient ainsi une version discrète naturelle des opérateurs gradient, divergence et vorticité. Les opérateurs obtenus sont consistants au moins à l'ordre 3. Lors des tests numériques effectués, un ordre 4 ou supérieur est observé.

D'autres part, nous avons considéré des systèmes d'équations du type Shallow Water sphériques. Les tests effectués sur l'équation Shallow Water linéarisée et l'équation Shallow Water donnent des résultats comparables à ceux obtenus par des méthodes de Galerkin ou de volumes finis d'ordres élevés. Les niveaux d'erreurs sont très faibles. Bien que le schéma ne soit pas a priori conservatif, l'erreur de conservation est très faible. Pour la masse le comportement est satisfaisant. Pour l'énergie et l'enstrophie potentielle, les erreurs sont similaires à celles obtenues par d'autres méthodes y compris sur des tests difficiles tels que le test de la montagne isolée ou le cas tests barotropique avec instabilité. Le schéma aux différences finis considéré est centré et l'opérateur de filtrage utilisé affecte peu les calculs. Bien que les équations soient hyperboliques, ce schéma est suffisant sur les tests effectués.

Les perspectives de ce travail concernent des simulations en temps long. En effet, nous nous sommes restreints ici à des schémas explicites. Ces derniers imposent des restrictions importantes sur le pas de temps. Il serait intéressant de développer un schéma implicite de manière à pouvoir considérer des pas de temps plus grands et ainsi des simulations sur des temps plus longs. Une autre perspective est de considérer des méthodes de zoom de type "Local Defect Correction" de manière à obtenir une meilleure représentation de phénomènes locaux tels que les tourbillons. Au niveau équations, un objectif futur est de travailler avec les équations de Navier-Stokes en dimension 3 plus proche de la réalité physique.

